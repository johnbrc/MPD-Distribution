\def\version{0.2.4 --- Post Review Draft}
\def\change{---     Post Review Draft (0.2.0) ---
---     Author's ammendments (0.2.x) ---
--- Relocated paragraph in Maize Plant Metaphor ---}
\input{mpd-header}
\def\mytitle{Facilitators' Handbook}
\def\subtitle{Maize Plant Discipleship Handbook}
\def\myauthor{Dr John B Clements}
\def\twitter{@johnbrc, @mpdresource}
\def\email{clements.jb@gmail.com}
\def\web{http:/\slash maizeplantdiscipleship.wordpress.com}
\def\mycopyright{John B Clements, 2014.  Maize Plant Discipleship by John B Clements is licensed under a Creative Commons Attribution-NonCommercial-ShareAlike 4.0 International License. Based on a work at http:/\slash maizeplantdiscipleship.wordpress.com}
\def\keywords{discipleship, mission, messianic, community}
\def\latexmode{memoir}
\input{mpd-document-facilitator}

\chapter{In this Handbook}
\label{inthishandbook}

Maize Plant Discipleship has been designed in consultation with missional leaders and learners in Burkina Faso, to be a practical, relevant and accessible learning resource, appropriate for use in African contexts.

It is being published as a series of low-cost, short, modular textbooks that incorporate reflective learning and group discussions, suitable for formal and informal modes of study.

It's goal is to facilitate biblical learning that moves constantly outwards, drawing whole communities into patterns of scripturally-based discipleship, in living dialogue with contextual culture.

This \emph{Facilitators' Handbook} provides guidelines for people facilitating \emph{Maize Plant Discipleship}. 

\begin{itemize}
\item \textbf{Discipleship and MPD} (\autoref{discipleshipandmpd}) explores biblical and missional perspectives of Messianic discipleship.

\item \textbf{Africa and MPD} (\autoref{africaandmpd}) explains the background to the development of Maize Plant Discipleship as a resource for use in African contexts.

\item \textbf{MPD Syllabus} (\autoref{mpdsyllabus}) introduces the metaphor of the maize plant and the sixteen modules of the MPD \emph{Syllabus}.

\item \textbf{Facilitating MPD} (\autoref{facilitatingmpd}) provides practical insights and suggestions for facilitating group learning.

\item [\textbf{Publishing MPD}][] examines practical issues relating to publication, including licensing, translation, printing and distribution.

\end{itemize}

\chapter{Facilitating MPD}
\label{facilitatingmpd}

\begin{quote}

{\ldots}the word of God increased; and the number of disciples multiplied greatly in Jerusalem, and a great many of the priests were obedient to the faith—\emph{Acts 1.6--7}

I have received a valuable teaching which I would like to use to impact the life of people in my neighbourhood and in my church. I would like to see this teaching be the flame of the Spirit of God which will help the ministry to grow more—\emph{Pastor Benao, 2010}
\end{quote}

\section{Discipleship group format}
\label{discipleshipgroupformat}

MPD modules are structured to provide a \emph{reflective learning process}, which incorporates practical, educational principles married to practical, spiritual principles, drawn from Scripture.

\subsection{Reflective Learning}
\label{reflectivelearning}

\autoref{learning-cycle.png} illustrates a reflective learning process of the kind that might be encountered within progressive educational or training contexts, including these four elements: Hear — Reflect — Discuss — Act.

\begin{figure}[htbp]
\centering
\includegraphics[width=225pt,height=214pt]{learning-cycle.png}
\caption{Reflective Learning Process}
\label{learning-cycle.png}
\end{figure}

\begin{itemize}
\item We learn by \emph{hearing} about other's experiences: practice informing theory and theory informing practice. 

\item As we listen, we \emph{reflect} upon both fresh ideas and current understanding: old and new sometimes conflicting, sometimes harmonising. 

\item In \emph{discussion} with others, we broaden and deepen our understanding and sharpen our convictions about the most appropriate way to{\ldots}

\item \emph{act}, when we disperse and return to our contexts of work and vocation. Later, we gather again, so the reflective learning process can be repeated {\ldots}again and again.

\end{itemize}

\subsection{Adapting the learning cycle}
\label{adaptingthelearningcycle}

I want to suggest how this reflective learning process, or cycle, may be adapted and applied to Maize Plant Discipleship groups, by combining it with a simple pattern, drawn from the example of the early church, which reports how

\begin{quote}

They continued faithfully in the teaching of the apostles, in fellowship, in breaking bread and in prayer—\emph{Acts 2.42}
\end{quote}

The combined \emph{MPD learning process} that I am proposing thus incorporates six elements: Hear — Receive — Open — Share — Pray — Act

\subsection{Hear{\ldots}what the Spirit is saying}
\label{hearwhatthespiritissaying}

When we gather together as disciples of the Messiah, to hear biblical teaching, we are opening ourselves not only to human ideas or wisdom, as Paul explains to the Corinthians:

\begin{quote}

Now we have not received spirit that belongs to the world, But the Holy Spirit Who is from God, given to us that we might realise and comprehend and appreciate the gifts of divine favour blessing so freely and lavishly bestowed on us by God. And we're setting these truths forth in words not taught by human Wisdom but taught by the Holy Spirit, combining and interpreting spiritual truths with spiritual language to those who possess the Holy Spirit—\emph{1 Corinthians 2:12--13 TAB}
\end{quote}

\begin{figure}[htbp]
\centering
\includegraphics[width=119pt,height=120pt]{hear.png}
\caption{Hear}
\label{hear.png}
\end{figure}

Our goal, therefore, is not only to listen with our minds, but also with our spirit—our heart; not to become `puffed up' with knowledge, but to \emph{hear what the Spirit is saying to his people}.\footnote{Revelation 2:29, 3:6,13,23; Matthew 11:15, Mark 4:9 etc.} And we listen in order to live more faithfully. This kind of listening is called \emph{heeding}: listening with the intention to learn and follow, or obey.

\pagebreak 

\subsection{Receive{\ldots}God's message inwardly}
\label{receivegodsmessageinwardly}

Hearing God's message to us, either as a body of people, or as individuals and families, is a start. However, it is not enough only to \emph{hear}, we must learn to \emph{receive} God's message.

\begin{quote}

The one who received the seed that fell on rocky places is the man who hears the word and at once receives it with joy. But since he has no root, he lasts only a short time{\ldots} The one who received the seed that fell among the thorns is the man who hears the word, but the worries of this life and the deceitfulness of wealth choke it, making it unfruitful. 

The seed on good soil stands for those with a noble and good heart, who hear the word and retain it, and by persevering produce a crop{\ldots} yielding a hundred, sixty or thirty times what was sown — \emph{Matthew 13:18--23; Luke 8:15}
\end{quote}

\begin{figure}[htbp]
\centering
\includegraphics[width=120pt,height=123pt]{receive.png}
\caption{Receive}
\label{receive.png}
\end{figure}

The key is receiving God's word inwardly, allowing the word to begin to ``\emph{dwell richly within us}'' (Collossians 3:16). Think about how we receive food: chewing, enjoying the taste, swallowing and digesting, enabling the body to receive the food's vitality and goodness. \emph{It's the same with God's word}. We must chew it over, meditating and reflecting upon its meaning and application to our lives. Then we allow it to settle in our spirit, forming and shaping our convictions, or holding it in our hearts, even before we fully understand all that it means to us. 

\pagebreak 

\subsection{Open{\ldots}heart to other's perspectives}
\label{openhearttoothersperspectives}

In messianic community, we must learn to open our hearts to the perspectives and experiences of those around us and those who see things differently to ourselves.

\begin{figure}[htbp]
\centering
\includegraphics[width=119pt,height=121pt]{open.png}
\caption{Open}
\label{open.png}
\end{figure}

The following suggestions may help group discussions to be lively and spiritually satisfying.

\begin{itemize}
\item \textbf{Invite free-moving debate relating to the study topic}. 

Ensure women and both young and old people are included. Allow discussions to ebb and flow, as people consider their responses and return to earlier discussions \emph{(see also: Facilitating openness (\autoref{facilitatingopenness}))}. 

\item \textbf{Discuss traditional proverbs that relate to study topics}. 

Use of the \emph{sweet talk} of proverbs will provide fresh insight. This can also help to provide opportunities to discuss aspects of the teaching with elders and others, including non-believers.

\item \textbf{Discuss practical, vocational applications of the topical study}. 

Vocation is more than simply our job, or employment: it incorporates all the responsibilities towards which God calls us, including workplaces and practices, families, communities and networks, as well as personal vision and service.

\end{itemize}

\pagebreak 

\subsection{Share{\ldots}break bread together}
\label{sharebreakbreadtogether}

The celebratory breaking of bread, in order to remember the Lord Jesus and his sacrificial obedience, is a sign and symbol of the new covenant and a profound way for discipleship groups to visibly proclaim their shared devotion to the Messiah.

\begin{figure}[htbp]
\centering
\includegraphics[width=119pt,height=120pt]{share.png}
\caption{Share}
\label{share.png}
\end{figure}

Although breaking and sharing bread has typically become ceremonial in modern Christianity, it is possible to incorporate a simple meal into times of meeting together and to identify this prayerfully as a form of breaking bread together. 

\begin{center}\rule{3in}{0.4pt}\end{center}


\begin{quote}

\emph{Breaking bread was originally fashioned, by the early church, upon the weekly Sh'bat (Sabbath) meal of the Jews -- a time of thanksgiving for God's deliverance of Israel from the oppression of Egypt}.
\end{quote}

\begin{center}\rule{3in}{0.4pt}\end{center}


Sharing food together is a vital part of human fellowship and a practical way of celebrating and proclaiming God's daily provision and blessing. Where a meal is not practical or possible, consider sharing a small amount of bread together, in a symbolic act of shared hospitality and commitment to membership of the body of the Messiah.

\pagebreak 

\subsection{Pray{\ldots}let God's kingdom come}
\label{prayletgodskingdomcome}

After discussion has taken place, invite the group to pray together, including intercession on behalf of neighbours, networks and communities and local and national rulers and governors.

\begin{figure}[htbp]
\centering
\includegraphics[width=119pt,height=121pt]{pray.png}
\caption{Pray}
\label{pray.png}
\end{figure}

Use MPD teaching to infuse prayer with fresh confidence concerning God's will and purpose, including personal and vocational concerns and challenges faced by group members. 

\begin{itemize}
\item \textbf{Pray for the gospel} to deeply impact and transform individuals, communities, cultures and societies throughout your nation, Africa, Europe, Asia and the Americas; pray for unreached people groups.

\item \textbf{Pray for the Maize Plant Discipleship Project}: that it will be used by God to edify and strengthen and bless the Messianic Community, within Africa {\ldots} and beyond!

\item \textbf{Bless one another} with a blessing from Scripture, or as led by the Holy Spirit; speak a blessing over your community or nation, or with regards to a specific problem.

\item \textbf{Expect the power of God} and the blessing of his life in us will overcome all opposition!

\end{itemize}

\pagebreak 

\subsection{Act{\ldots}in light of God's message}
\label{actinlightofgodsmessage}

The purpose of our gathering to hear God's message is not simply to hear it, but to act upon it. As the epistle of Ya'akov (James) explains, we deceive ourselves when we listen to God's word, but do not do what it says:

\begin{quote}

Don't deceive yourselves by only hearing what the Word says, but do it! For whoever hears the Word but doesn't do what it says is like someone who looks at his face in a mirror, who looks as himself, goes away and immediately forgets what he looks like.

But if a person looks closely into the perfect \emph{Torah}, which gives freedom, and continues, becoming not a forgetful hearer but a doer of the work it requires, then he will be blessed in what he does — \emph{James 1:22--25}
\end{quote}

\begin{figure}[htbp]
\centering
\includegraphics[width=119pt,height=120pt]{act.png}
\caption{Act}
\label{act.png}
\end{figure}

The goal of discipleship is to be transformed ourselves and to become a transforming influence in our homes and workplaces and communities. As we learn to do this as part of a growing, dynamic movement of disciples, we begin to fulfil our corporate vocation: \emph{a messianic community blessed{\ldots}to be a blessing!}

\pagebreak 

\subsection{MPD learning cycle}
\label{mpdlearningcycle}

Bringing together the six elements of reflective learning and messianic community together — Hear, Receive, Open, Share, Pray, Act — provides us with the MPD learning cycle (\autoref{mpd-learning-cycle.png})

\begin{figure}[htbp]
\centering
\includegraphics[width=307pt,height=340pt]{mpd-learning-cycle.png}
\caption{MPD Learning Cycle}
\label{mpd-learning-cycle.png}
\end{figure}

\pagebreak 

\section{Facilitating role}
\label{facilitatingrole}

MPD is an intentionally open, reflective, group-orientated learning process, in which all kinds of leaders and learners are invited to participate together in discovering what the Spirit is saying.

\subsection{Facilitating Spirit-led discipleship}
\label{facilitatingspirit-leddiscipleship}

Holy Spirit-led discipleship is about participating in a process through which the Spirit of Christ is able to lead, warn, direct, encourage, teach, challenge and exhort us personally and corporately.

\begin{itemize}
\item Discipleship is never a matter of pouring out information from a teacher or from text books, in order to fill others with the same knowledge. This may be how we learned in school, but Spirit-led discipleship is different.

\item The goal of discipleship is not to establish shared dogmatic belief or conformity to the convictions of a leader, mentor or facilitator, nor to church traditions—and certainly not to every aspect of MPD. 

\item The goal is conformity to the Spirit of the Messiah, Jesus, and obedience to the will of the Father.

\item Since, every person differs in gift and personality and stage of development, each needs be empowered to learn in their own particular way and at their own pace. 

\item The aim therefore is to allow MPD to be a vehicle for God's Spirit to speak to, educate and disciple people — recognising that MPD itself represents only one part of a whole-life process through which the Messiah may be guiding and directing his people.

\end{itemize}

\subsection{Facilitating openness}
\label{facilitatingopenness}

The aim is to foster an environment of learning and discovery, where debate and discussion is lively, yet relaxed and uncompetitive. An environment in which all members, even the youngest and least experienced, are comfortable to openly contribute their views, burdens and questions. 

\begin{itemize}
\item Openness, difference and diversity of views will not stifle the true unity of the Spirit, which is based upon love and mutual respect. 

\item Forced conformity, will however, typically erupt, sooner or later, because people need to feel that they have been heard and listened to—especially those with a different perspective. 

\end{itemize}

By encouraging open debate, it is possible for members to learn how to express and assert strong and diverse views, without creating conflict or forcing conformity. However, openness can be particularly challenging to foster in cultures where where traditional authority flows downwards from the top and conformity is highly valued. 

Here are some practical suggestions that may help establish openness and diversity of opinion.

\begin{itemize}
\item Facilitators should contribute to discussions as regular group members; they should never abuse their position by dominating a discussion or belittling other views—indeed, they should encourage them.

\item A facilitator should normally resist the temptation to rehearse teaching, affirm particular views or summarise discussions. Doing these things can undermine the discovery and learning taking place.

\item If discussion becomes harsh or factious, facilitators should resist furthering the confrontation. Instead, quieten the group and then consider inviting a member with a harmonious or gentle personality to summarise (rather than resolve) the tension.

\end{itemize}

\subsection{Facilitating relationship}
\label{facilitatingrelationship}

Discipleship represents a form of learning that is \emph{called out} of students or disciples, under the guidance and direction of a facilitator, mentor, educator or teacher. People in these roles come alongside motivated learners, not to direct them forcefully, but to assist, encourage, facilitate and draw out the \emph{learning-through-living} that is continuously taking place within those being discipled.

Regular discipleship groups are a good way to begin fermenting interpersonal interaction, however they usually need to be supplemented by informal exchanges that allow mutual encounter, openness and listening. Time for companionship and friendship, during which it is possible to share both painful realities stretching or testing our faith and hopes and dreams sustaining us. 

Without this, learning groups may risk feeling too impersonal, theoretical or removed from daily life. However, unlike formal, classroom learning, this sort of practical, interpersonal, \emph{open-to-Others} kind of learning is rarely neat and tidy and may even seem chaotic at times. It is, though, potentially highly effective.

\subsection{Who can facilitate?}
\label{whocanfacilitate}

A facilitator needs to be someone who senses a calling to help others become faithful Christian disciples. This must motivate them to be humble, patient, flexible, warm, open and secure enough to allow others to explore personal boundaries of vocational understanding, experience and creativity, at their own pace.

A facilitator, or mentor, does not hold a position of authority over people. They simply facilitate the gathering of people into groups, for learning and discussion. Accordingly, a facilitator:

\begin{itemize}
\item may be a lay-leader;

\item may be relatively young;

\item may be a woman;

\item need not have gone to bible college;

\item need not be an established church leader;

\item need not be an experienced mentor.

\end{itemize}

Of course, MPD can be facilitated by established leaders, mentors or disciplers—providing they are willing and comfortable to facilitate group discussions that are genuinely reflective and exploratory. 

\section{Practical considerations}
\label{practicalconsiderations}

Facilitating a discipleship group will be most effective when practicalities are considered and appropriate planning takes place. This should be done in advance, in order to grain the most from group interaction.

\subsection{Getting started}
\label{gettingstarted}

Maize Plant Discipleship is probably ideal for discipleship groups of between eight to ten people. This is small enough to grow together with a degree of intimacy; large enough for group members to explore discipleship commitment at their own pace.

\begin{itemize}
\item \emph{More than ten?} Consider forming more than one group and begin forming disciples right away, by helping others to facilitate a group.

\end{itemize}

\subsection{Involving others}
\label{involvingothers}

Although a facilitator is responsible for convening gatherings, they may delegate responsibilty for hosting, presenting the teaching, or moderating group discussions. Sharing responsibilities avoids one person dominating or carrying too much responsibility. Ideally, as many group members as possible should be employed, over time, in sharing responsibilities.

Whoever is presenting the topical study should read through it carefully, in advance: absorbing, familiarising and reflecting upon the teaching and its lessons. If areas are unfamiliar or unclear, don't ignore them. Instead, allow a lively discussion about that area of the study, encouraging others to bring forward their interpretation of the topic.

\subsection{Timetable}
\label{timetable}

Plan a timetable for the whole syllabus of — approximately 64 — studies to be completed, taking into consideration the nature of the group and availability of time. For example, are members effected by the agricultural or academic calender? 

\subsection{Location}
\label{location}

Meeting together can take place in any appropriate location that can comfortably accommodate a group. A large room in someone's home, or a community hall is ideal. 

Consider how the seating can be used to form a more intimate space for the group. It is important that people are able to see one another, in order to provide a practical forum for discussions and a more intimate space in general.

Experiment with a change of location occasionally. Outside environments may be helpful in producing a liberating interaction; find out what works well for your group.

\subsection{Adaption}
\label{adaption}

Be prepared to adapt both the teaching and the method of presentation, to create a helpful and culturally-appropriate learning environment. Take into account the abilities and capacities of each particular discipleship group. Here are some suggestions to consider.

\begin{itemize}
\item Ensure literature is a helpful servant, not a hard task-master, to oral learners in particular. 

\item Avoid overcomplicating matters; try to keep things simple and lighthearted.

\item Consider inviting creative people to contribute by interpreting or celebrating the message of the teaching using drama, art or song.

\item What other ways might it be appropriate to organise the time and the group in order to ensure the learning is culturally appropriate, relevant and practical? 

\end{itemize}

\begin{quote}

\emph{Think about these things as you progress throughout this discipleship experience}.
\end{quote}

\chapter{Republishing MPD}
\label{republishingmpd}

\begin{quote}

I THINK you are touching something that is not already existing{\ldots} If we talk about evangelism, it may well be a new way of approaching evangelism, but we already have many methods of evangelism. But discipleship is something that is really (an) innovative thing.—\emph{Burkinabé theological educator, 2010}
\end{quote}

\section{Books for African contexts}
\label{booksforafricancontexts}

Over the past three-to-four decades, the African Christian community has multiplied and expanded massively, such that the global Christian community's statistical centre of gravity is now located in Africa. 

While this extraordinary numerical growth has taken place, there has been a relative shortage of book publishing within Africa. Resources serving to edify, disciple and inspire Christian communities have been particularly sparse within Francophone Africa. 

Furthermore, many books are imports or translations of books written for Christians in different, generally Western, cultural contexts—wherein quite different questions, cultural values and theological imperatives are considered significant. In short, many Christian textbooks used in African leadership training are contextually inappropriate.

Added to this, books are published and distributed in a manner principally designed to protect the commercial interests of authors, publishers, distributers and retailers. Consequently, many books are practically unattainable to those who are economically poor—including many leaders hungry for resources to enrich their vocational work of making Christian disciples. These leaders typically mediate between the two cultural worlds of orality and literacy respectively and consequently need textbooks that

\begin{itemize}
\item facilitate, rather than hinder, mediation between these cultures;

\item are readily translatable into mother-tongue languages;

\item can be republished in a low-cost way that energises mission.

\end{itemize}

\section{A new publishing philosophy}
\label{anewpublishingphilosophy}

\emph{Maize Plant Discipleship} has been formulated and published to meet these kind of contextual needs. With a missional, rather than commercial, philosophy, intended to practically foster and facilitate a widespread adoption of Christian discipleship praxis, led by African leaders and learners.

Accordingly, MPD resources are being licenced for reproduction under the terms of a \emph{Creative Commons} licence. This allows MPD resources to be reproduced, re-published, including translations, and re-distributed within a particular context--without breaching copyrigh, providing any and all reproduction identifies the original author \emph{and} retains the licence as an integral element of the republication.

This respects and maintains the personal copyright of the author and translators, while also releasing the traditionally tight, commercial control of copyright owners and publishers.

The licence permits textbooks reproduced under licence to be sold or otherwise distributed by agencies that organise \emph{translation, printing and distribution}, according to their own priorities and budgetary constraints. 

Assuming textbooks of 25 A4 pages, double-sided printing, folded to produce 50 page-length, A5-size booklets, stapled and trimmed, with a single-colour printed cover, estimated production costs suggests a cost per book of \$2 is achievable. An agency that organises printing and distribution on behalf of a number of other organisations may be the most appropriate way to economise on cost.

\pagebreak 

\section{License}
\label{license}

Maize Plant Discipleship Facilitator's Handbook (this textbook) by John Clements is licensed for distribution within Burkina Faso, as follows:

Note: This is a human-readable summary of (and not a substitute for) the license, which may be accessed at: http:/\slash creativecommons.org\slash licenses\slash by-sa\slash 4.0\slash legalcode

\begin{center}\rule{3in}{0.4pt}\end{center}


\textbf{Creative Commons Attribution-ShareAlike 4.0 REGIONAL License}. 

\textbf{You are free to}:

\begin{itemize}
\item \textbf{Share} — copy and re-distribute the material in any medium or format

\item \textbf{Adapt} — remix, transform and build upon the material 

\end{itemize}

\begin{quote}

for any purpose, even commercially. The licensor cannot revoke these freedoms as long as you follow the license terms.
\end{quote}

\textbf{Under the following terms}:

\begin{itemize}
\item \textbf{Attribution} — You must give appropriate credit,\footnote{If supplied, you must provide the name of the creator and attribution parties, a copyright notice, a license notice, a disclaimer notice, and a link to the material.} provide a link to the license, and indicate if changes were made. You may do so in any reasonable manner, but not in any way that suggests the licensor endorses you or your use.

\item \textbf{ShareAlike} — If you remix, transform, or build upon the material, you must distribute your contributions under the same license\footnote{You may also use a license listed as compatible at https:/\slash creativecommons.org\slash compatiblelicenses} as the original.

\item \textbf{No additional restrictions} — You may not apply legal terms or technological measures that legally restrict others from doing anything the license permits.

\end{itemize}

\begin{center}\rule{3in}{0.4pt}\end{center}


\subsection{Notice}
\label{notice}

\begin{itemize}
\item You do not have to comply with the license for elements of the material in the public domain or where your use is permitted by an applicable exception or limitation. 

\item No warranties are given. The license may not give you all of the permissions necessary for your intended use. For example, other rights such as publicity, privacy, or moral rights may limit how you use the material.

\item Permissions beyond the scope of the license may be available at http:/\slash maizeplantdiscipleship.wordpress.com\slash contact.

\end{itemize}

\chapter{Appendices}
\label{appendices}

\section{MPD Syllabus}
\label{mpdsyllabus}

\begin{quote}

I TELL you that unless a grain of wheat that falls to the ground dies, it stays just a grain; but if it dies, it produces a big harvest — \emph{Jesus, John 12:24}

May he who supplies seed to the sower and bread for food, supply and multiply the seed you have sown and increase the fruits of your righteousness — \emph{Paul, 2 Corinthians 9:10}
\end{quote}

\subsection{Maize Plant Metaphor}
\label{maizeplantmetaphor}

In the gospel of John, Jesus refers to his own mission using the metaphor of a seed that enters the ground and dies, in order to produce a large harvest. Grown throughout sub-Saharan Africa, the maize plant provides a similar, highly recognisable metaphor. 

The Maize Plant Discipleship Syllabus is structured to symbolically reflect the metaphor of a maize plant. There are three main components to the metaphor (see \autoref{mp-metaphor.png}).

\begin{enumerate}
\item \textbf{Soil and roots}

Representing the biblical, covenantal community, context and vocation of Israel.

\item \textbf{Maize plant}

Representing the dynamic spiritual growth and multiplication of messianic community.

\item \textbf{Sunlight and rainfall}

Representing essential spiritual disciplines that stimulate and sustain the growth of messianic community.

\end{enumerate}

\begin{figure}[htbp]
\centering
\includegraphics[width=150pt,height=245pt]{mp-metaphor.png}
\caption{Maize Plant Metaphor}
\label{mp-metaphor.png}
\end{figure}

Maize produce effectively represents \emph{daily bread} for millions of African people. The messianic community is called, like the Messiah, to become a kind of \emph{life-giving bread} to the peoples of the world—see John 6 \& 20:21.

\subsubsection{Soil and roots}
\label{soilandroots}

Only seed sown into good soil produces a good harvest. Discipleship communities also need to be rooted in good soil, allowing us to draw on essential, spiritual nutrients and the living water of the Messiah's Spirit, anchoring us against winds of false teaching and sustaining us amidst the heat of trials, temptations and vocational responsibility.

\paragraph{MPD-M1-Eternal Purpose}
\label{mpd-m1-eternalpurpose}

The foundational \emph{Module 1} (MPD-M1) provides a study of the biblical, Hebraic soil and messianic roots of new-covenant faith, in order to unfold a panoramic overview of the whole story of scripture, representing God's unchanging, eternal purpose.

\begin{quote}

In this foundational teaching (see \autoref{mp-roots.png}), the Messiah represent \emph{the Seed} that enters the soil—of the biblical, covenantal history of the people of Israel—to die and ultimately produce a rich harvest of people—a messianic covenant community—from amongst all the peoples of the earth. 
\end{quote}

\begin{figure}[htbp]
\centering
\includegraphics[width=250pt,height=154pt]{mp-roots.png}
\caption{Roots}
\label{mp-roots.png}
\end{figure}

\subsubsection{Maize plant}
\label{maizeplant}

\textbf{Modules MPD-M2 to MPD-M9} explore eight characteristic \emph{dynamics} of messianic community (see \autoref{mp-dynamics.png}). These dynamics refer to the unique, spiritual forces\footnote{\emph{Dynamic} derives from a Greek word, \emph{dunamis}, meaning power and refers to forces stimulating change or progress within a system or process.} that stimulate the characteristic development, growth, structure, shape and fruit of messianic community.

\begin{figure}[htbp]
\centering
\includegraphics[width=251pt,height=425pt]{mp-dynamics.png}
\caption{Maize plant}
\label{mp-dynamics.png}
\end{figure}

\paragraph{MPD-M2-Dynamics of Vocation-The Nations}
\label{mpd-m2-dynamicsofvocation-thenations}

\begin{quote}

Examines the historical development the Messianic Community's vocational mission to bless the nations of the world.
\end{quote}

\paragraph{MPD-M3-Dynamics of Vocation-The Jews}
\label{mpd-m3-dynamicsofvocation-thejews}

\begin{quote}

Examines the significant responsibilities of the Messianic Community towards the Jewish people.
\end{quote}

\paragraph{MPD-M4-Dynamics of Commissioning}
\label{mpd-m4-dynamicsofcommissioning}

\begin{quote}

Explores the dynamic of messianic commissioning, including a focus upon strategy, structures and expansion.
\end{quote}

\paragraph{MPD-M5-Dynamics of Body Membership}
\label{mpd-m5-dynamicsofbodymembership}

\begin{quote}

Explores basic patterns of membership, commitment and spiritual maturity within the body of the Messiah.
\end{quote}

\paragraph{MPD-M6-Dynamics of Revival}
\label{mpd-m6-dynamicsofrevival}

\begin{quote}

Reveals a cycle of spiritual activities representing the dynamics of revival, leading to a spiritual harvest.
\end{quote}

\paragraph{MPD-M7-Dynamics of Truth}
\label{mpd-m7-dynamicsoftruth}

\begin{quote}

Explores our encounter with truth, as we learn to walk in practical, covenantal faithfulness towards God's revelation, wisdom and direction.
\end{quote}

\paragraph{MPD-M8-Dynamics of Intercession}
\label{mpd-m8-dynamicsofintercession}

\begin{quote}

Explores the priestly vocation of the Messianic Community to be \emph{a house of prayer for all nations}.
\end{quote}

\paragraph{MPD-M9-Dynamics of Cultural Transformation}
\label{mpd-m9-dynamicsofculturaltransformation}

\begin{quote}

Examines our call to work amongst and on behalf of the nations, towards cultural transformations that signal the presence of God's kingdom.
\end{quote}

\subsubsection{Sunlight and rainfall}
\label{sunlightandrainfall}

\textbf{Modules MPD-M10 to MPD-M16} examine seven characteristic \emph{disciplines} of messianic community (see \autoref{mp-disciplines.png}). In the metaphor of the maize plant, sunlight and rainfall are representative of messianic disciplines, which open us to receive the revelatory Light and sustaining Living Water of God's Spirit—without which we become spiritually weak and unable to produce good fruit or a plentiful harvest.

\begin{figure}[htbp]
\centering
\includegraphics[width=310pt,height=425pt]{mp-disciplines.png}
\caption{Sunlight and rainfall}
\label{mp-disciplines.png}
\end{figure}

\paragraph{MPD-M10-Disciplines of Spiritual Maturity}
\label{mpd-m10-disciplinesofspiritualmaturity}

\begin{quote}

Reviews three stages of encounter, discovery and growth in spiritual maturity: pastoral—devotional; missional—vocational and apostolic—intercessory.
\end{quote}

\paragraph{MPD-M11-Disciplines of Running the Race}
\label{mpd-m11-disciplinesofrunningtherace}

\begin{quote}

Uses the metaphor of an athlete to examine motivations, qualities and disciplines for living an enduring life of service—\emph{completing the race marked out for us}.
\end{quote}

\paragraph{MPD-M12-Disciplines of Pressing Toward Our Vocation}
\label{mpd-m12-disciplinesofpressingtowardourvocation}

\begin{quote}

Explores how we can be sustained us in our personal vocations through a deepening of our relationship with the Messiah, including \emph{prayer without ceasing}.
\end{quote}

\paragraph{MPD-M13-Disciplines of Economic Faithfulness}
\label{mpd-m13-disciplinesofeconomicfaithfulness}

\begin{quote}

Explores economic faithfulness, highlighting biblical perspectives of wealth and poverty radically different to those of the world. 
\end{quote}

\paragraph{MPD-M14-Disciplines of Messianic Leadership}
\label{mpd-m14-disciplinesofmessianicleadership}

\begin{quote}

Explores qualifications, motivations and characteristics of faithful messianic leadership \emph{and} the balancing of task, team and individual.
\end{quote}

\paragraph{MPD-M15-Disciplines of Living by Faith}
\label{mpd-m15-disciplinesoflivingbyfaith}

\begin{quote}

Explores the testing and purification of messianic faith, seeing with eyes of faith and embracing challenges as opportunities to experience God's faithfulness.
\end{quote}

\paragraph{MPD-M16-Disciplines of Overcoming}
\label{mpd-m16-disciplinesofovercoming}

\begin{quote}

Employs the analogy of a sporting arena to represent spiritual confrontation and conflict with cultural idols and strongholds, discerning strategies that make room for a transformative encounter with God's overcoming power.
\end{quote}

\subsection{MPD Handbooks}
\label{mpdhandbooks}

Each MPD module \emph{Handbook} contains four interrelated \emph{Studies}, each of which incorporates:

\begin{itemize}
\item Scripture Readings

\item Topical Sections and Summary

\item Questions for Group Discussion

\end{itemize}

The format is designed to facilitate reflective group learning—so that Maize Plant Discipleship resources rely not upon the presence of a qualified leader, teacher or theologian, but simply upon facilitators willing to co-ordinate study groups and discussions (see Facilitating MPD (\autoref{facilitatingmpd})).

\subsection{Terms and translations}
\label{termsandtranslations}

A number of significant terms are used throughout the MPD syllabus. They are defined here, so that Facilitators can refer to them at any time.

\subsubsection{Messiah}
\label{messiah}

the promised Jewish deliverer{\ldots}
Jesus, the Christ{\ldots}the Messiah 

\subsubsection{Messianic}
\label{messianic}

\begin{itemize}
\item Messianic
Cf. ``Christian''

\item Messianic Community

\item Messianic, New-Covenant Community

\item messianic community or communities

\end{itemize}

\subsubsection{Bible translation}
\label{bibletranslation}

Wherever possible facilitators should use and encourage the use of mother-tongue translations of Scripture. Throughout the MPD Syllabus Handbooks, scripture references—rather than quotations—are generally provided with this in mind.

Where a quotation is incorporated within the teaching, the following abbreviations are used to indicate the version:

\begin{itemize}
\item CJB — The Complete Jewish Bible

\item NIV — The New International Version

\item TAB — The Amplified Bible

\end{itemize}

\subsection{Questions for Reflection and Discussion}
\label{questionsforreflectionanddiscussion}

\begin{itemize}
\item How would you describe the significance of the maize plant to this syllabus and learning resource?

\item How would you describe a dynamic?

\end{itemize}

\section{Africa and MPD}
\label{africaandmpd}

\begin{quote}

Africanness and (theological) correctness should not be measured in either dissimilarity or similarity to the West. The way forward is to measure the Africanness of any theology purporting to be African by the degree to which it speaks to the needs of African's in their total context. Quite naturally, the needs of African Christians should be taken seriously when determining these needs — \emph{Tite Tienou, The Uphill Road: Indigenous African Christian Theologies, 1990}
\end{quote}

\subsection{Introduction}
\label{introduction}

Tite Tienou, a Malian who grew up in Burkina Faso, encapsulates (above) a highly significant aspect of the philosophy that has driven the development of Maize Plant Discipleship: African voices must determine the theology that is of practical relevance to African contexts.

\subsubsection{The Africa connection}
\label{theafricaconnection}

The origins of Maize Plant Discipleship are rooted in my personal experiences amongst an international, Pentecostal missionary community. Over time, a stream of African leaders, training or studying in the UK, joined with our prayer fellowship, reporting how spiritually \emph{at home} they felt with us. On returning to Africa, memories of shared experiences in prayer prompted them to invite me to visit and teach in their contexts.

The first opportunity to do so came in 2000, through a former prayer team member, then working in Nakuru, Kenya. There, I presented five days of seminars on prayer, intercession and mission. The considerable enthusiasm and appreciation expressed by the attendants encouraged me to continue investing in the seminal training resource, which has since developed into Maize Plant Discipleship.

\subsubsection{Burkina Faso}
\label{burkinafaso}

The next stage of development came through a series of visits to Burkina Faso, between 2003 and 2008, during which I taught various conferences of leaders and learners:

\begin{itemize}
\item In Léo: Bible school students, pastors, church-planters and trainees, many of whom were living in missionary contexts, within Muslim-dominant villages. 

\item In Ouagadougou: associates of a vibrant, national youth movement, incorporating university students, graduates, post-graduates, office workers and other young people.

\end{itemize}

At each point, the feedback of participants and organisers was sought. It became clear from their responses that a significant spiritual dynamic was taking place. The teaching was reported to be motivational and spriritually challenging. To understand this better, I asked Burkinabé associates: how could the discipleship teaching be made more accessible and useful within Burkinabé contexts?

Their response was to propose the development of a ``Train the Trainers'' program, incorporating Maize Plant Discipleship teaching in a series of accessible handbooks. The books would be designed to accompany teaching presentations \emph{and} to be studied and used by trainees in their own contexts.

\subsection{Doctoral research}
\label{doctoralresearch}

In order to maximise effectiveness, from an intercultural perspective, I took the opportunity incorporate further development of the books within a program of doctoral research, directed by \emph{Fuller School of Intercultural Studies}, between 2009 and 2013. 

This enabled me to consult Burkinabé leaders and learners on a wide range of concerns and questions relating to the appropriate biblical and practical formulation of discipleship and leadership training resources. 

In total, over seventy Burkinabé participants were consulted, drawn from members of the \emph{Assemblée Evangélique de Pentecôte, Assemblée de Deus, Association Nationale pour la Traduction de la Bible et de l'Alphabetisation, Mouvement des Jeunes Serviteurs de Dieu} and \emph{SIL}.

The research involved collecting data—opinions, perspectives, information and insight—following prototypical seminars delivered to conferences in Léo and Ouagadougou. This was done through multiple survey questionnaires and group interviews and a number in-depth interviews, conducted with national and, or denominational leaders—most of whom held responsibility for leadership training and development. This allowed a rich seam of data to be gathered for analysis.

\subsubsection{Findings}
\label{findings}

Applying qualitative data-analysis to the collected data revealed a number of significant \emph{findings}, or themes, relating to issues of discipleship, leadership training, theology, culture, intercultural relations and literacy. In particular, participants:

\begin{itemize}
\item identified strongly with the concept of discipleship and a need to freshly embrace \emph{holistic, disciple-forming practices};

\item consistently validated the missiological content (\autoref{missiologicalcontent}) of a prototype MPD training resource as \emph{appropriate to Burkinabé contexts}, highlighting the value in terms of biblical content, practical focus and cultural sensitivity;

\item affirmed the importance of incorporating graphical and analogical \emph{illustrations} and alloted time for \emph{reflection} and \emph{discussion};

\item expressed desire for appropriate \emph{literature}, tailored to facilitate leaders working in oral contexts;

\item identified the value of translating appropriate textbooks into minority African languages;

\item highlighted significant logistical challenges relating to publication—e.g. licensing, promotion, cost and distribution.

\end{itemize}

These analytical conclusions were used to define a set of \emph{practical, relevant and accessible} attributes that could be applied to the formulation of training resources appropriate to Burkinabé leaders and learners.\footnote{Visit: http:/\slash \textbf{jbclements.wordpress.com\slash missiology} for further details of doctoral dissertation, \emph{Facilitating A Renewal of Discipleship Praxis Amongst Burkinabé Leaders and Learners}.}

\subsubsection{Practical}
\label{practical}

Encouraging and facilitating a focus upon intentional action, rather than classroom theorising.

\begin{itemize}
\item Awaken or strengthen contextual ownership of the call to serve God's eternal purpose amongst their generation.

\item Promoting lifelong commitment to missional discipline and action.

\item Envisioning personal, communal, cultural transformation.

\item Emphasising generational formation of disciples.

\end{itemize}

\subsubsection{Relevant}
\label{relevant}

Incorporating qualities and characteristics appropriate to contextual culture.

\begin{itemize}
\item Providing a biblically faithful missiology, integrated with Pentecostal spirituality.

\item Publishing short, topical, illustrative handbooks.

\item Facilitating group discussion and reflective learning. 

\item Encouraging contextual adaption.

\end{itemize}

\subsubsection{Accessible}
\label{accessible}

Lowering or removing barriers that could hinder access to the curriculum.

\begin{itemize}
\item Linguistically and conceptually comprehensible.

\item Facilitating vernacular language translation.

\item Licencing republication and redistribution.

\item Distribution models that favour the economically poor.

\end{itemize}

\subsubsection{Missiological content}
\label{missiologicalcontent}

My research concluded that a biblically faithful missiology, integrated with Pentecostal spirituality and appropriate to African contexts, should encompass not less than the following content or characteristics:

\begin{enumerate}
\item A holistic worldview;

\item A communal orientation;

\item An historical, covenantal, missionary interpretation of Scripture;

\item A missiology of

\begin{itemize}
\item biblical discipleship;

\item suffering and overcoming;

\item spiritual revival;

\item intercessory prayer and spiritual power;

\item poverty and prosperity;

\item personal and corporate vocation;

\item Christ-centred servant-leadership;

\item cultural transformation.

\end{itemize}

\end{enumerate}

\subsubsection{Continuous improvement}
\label{continuousimprovement}

These \emph{practical, relevant} and \emph{accessible} attributes, extracted from data gathered from Burkinabé participants, are those which have defined the Maize Plant Discipleship Resource.

\begin{quote}

After you have examined and used MPD resources, if you think we can improve upon these attributes, or the application of them, please contact us with your ideas, via the MPD website: 
\end{quote}

\begin{itemize}
\item http:/\slash maizeplantdiscipleship.wordpress.com\slash contact

\end{itemize}

\subsection{Questions for Reflection and Discussion}
\label{questionsforreflectionanddiscussion}

\begin{itemize}
\item Do you think it's important that African Christians decide upon what is theologically appropriate to African contexts?

\item Has this happened historically in your contexts? If not, why not? If so, what has changed?

\item What theological issues are important to you and others in your context?

\item If they are not covered by this syllabus, what will you do?

\end{itemize}

\section{Discipleship and MPD}
\label{discipleshipandmpd}

\begin{quote}

Go and make people from all nations into disciples, immersing them into the reality of the Father, the Son and the Holy Spirit and teaching them to obey everything that I commanded you---\emph{Jesus, Matthew 28:19--20}
\end{quote}

\begin{center}\rule{3in}{0.4pt}\end{center}


\begin{quote}

Discipleship is the process of making disciples who themselves will also make other disciples. In a family where there is no birth there will be no continuity, so a church without disciples will not live for a long time. However, all church members are not disciples{\ldots}a mere believer is far from a disciple---\emph{Pastor Ayoro, Léo, 2010}
\end{quote}

\subsection{What is discipleship?}
\label{whatisdiscipleship}

\textbf{Messianic discipleship is a dynamic, generational process, empowered by the Holy Spirit}. 

To fully illustrate what is involved in this process, I want to link together two crucial statements made by the apostle, Paul, in his second letter to his disciple, Timothy:\footnote{There is good reason to link these two statements, seperated as they are only by Paul's emotional description of two disciples who failed to ``keep the treasure safe,'' by not standing with Paul at a critical time.}

\begin{quote}

Keep safe the great treasure that has been entrusted to you, with the help of the Holy Spirit, who lives in us{\ldots} and the things you heard from me, which were supported by many witnesses, these commit to faithful people, such as will be competent to teach others---\emph{2 Timothy 1.14 and 2.2}
\end{quote}

Together they reveal three vital components of messianic discipleship.

\begin{enumerate}
\item \textbf{The ``great treasure'' of the knowledge of the Messiah, Jesus Christ}.

The personal, experiential knowledge of the Messiah is more than human knowledge or philosophy: it is ``a great treasure,'' a spiritual reality, a divine relationship, mediated by the Holy Spirit{\ldots}

\item \textbf{The vitality of the Holy Spirit}.

The Holy Spirit provides an intimate source of divine help to messianic disciples.\footnote{John 16:7--15} It is he who mediates the reality of the Gospel and the Presence of the Messiah amongst his people—and it is he who helps them to safekeep this reality{\ldots}

\item \textbf{The neccessity of generational formation}.

Timothy has received an impartation of the reality of the Messiah through Paul—something of great worth that he must now safeguard by \emph{committing it to the stewardship and safekeeping of other faithful people}. This is generational discipleship (see \autoref{timothy.png}). 

\end{enumerate}

\begin{figure}[htbp]
\centering
\includegraphics[width=275pt,height=45pt]{timothy.png}
\caption{Generational discipleship}
\label{timothy.png}
\end{figure}

In the kingdom of God, treasure is kept safe, kept from becoming unproductive,\footnote{Matthew 25:14--30} by sharing it with others. Let's try to understand this a little more.

\subsubsection{Seed and harvest}
\label{seedandharvest}

In farming contexts, seeds are a form of wealth—a type of treasure. Yet seed is generally stored only for a short time before being used. Whatever is not required for food—for \emph{daily bread}—must soon be sown to produce another harvest.\footnote{see 2 Corinthians 9:6--12}

In the same way, we discover that God is able to supply spiritual life to us: the \emph{treasure} of knowing the Messiah. This experience of being alive to God, experiencing the grace of the Messiah and the love, joy, peace, patience, kindness, goodness, faithfulness, gentleness and self-control of the Holy Spirit,\footnote{Galatians 5:22} is the spiritual equivalent of receiving daily sustenance—\emph{daily bread}. 

Enjoying God's life ourselves, however, is not the whole purpose of our relationship with him. In fact, as we will explore in this resource, the Messianic, New Covenant Community (the whole body of the Messiah's people) has been called to know God in order to become his \emph{Servant Community}. This means that we are called to give ourselves, our lives \emph{to serve his purposes}. This requires discipline and sacrifice—that is what it means to be a disciple.

This sacrificial, disciplined giving of ourselves in service to God is the equivalent of taking precious seed that could be used for food, for ourselves, and instead sowing it into the ground to produce another harvest.

\subsubsection{Sharing treasure}
\label{sharingtreasure}

This is one of the secrets to living a truly \emph{messianic} life, which many people seem never to properly discover or experience: it is not in hoarding the treasure of our knowledge, relationship and communion with God, but in sharing and \emph{sowing} it with others, both within and beyond our own communities, that we discover and realise our vocation.\footnote{Matthew 10:38--39}

Our spiritual treasure, however, is not to be wasted or cast away carelessly. Even though some seeds inevitably fall onto unreceptive ground,\footnote{Matthew 13:1--23} like a farmer who never intentionally wastes his seed, our treasure is too precious to be deliberately squandered on people who spurn its value.\footnote{Matthew 7:6} It must be shared with people who recognise its worth and who make room for its transformative power to change them.

This is what the parable calls \emph{good soil}. People willing to be transformed through a personal knowledge and experience of the Messiah, who will share it with other faithful people{\ldots} who share it with other faithful people{\ldots} and so on and so forth.

\subsection{Discipleship movements}
\label{discipleshipmovements}

The formation of faithful disciples was at the heart of the Messiah Jesus' life and work. The entire historical and now-worldwide Christian movement began with one small, core group of disciples, formed around Jesus. 

Discipling movements have significant potential to impact and transform. Leaders, money, books and power all have their places within radical, popular movements. But, people-movements are most capable of producing deep, wide, enduring change.

Two things are essential to transformative people movements: \emph{vision} and \emph{the formation of disciples}. Visionary leaders must impart a hope that is powerful, challenging and instrumental. A vision capable of gripping the hearts of others and forming them into faithful, active disciples:

\begin{itemize}
\item committed to significant transformation;

\item persistent and determined to turn vision into reality;

\item operating as co-workers—not selfishly-ambitious individuals;

\item actively forming other faithful disciples.

\end{itemize}

In this way a visionary, discipling movement is developed. A movement of focussed, inspired, dynamic people. People deeply and profoundly allied to a vision, a cause and a purpose larger than themselves.

\subsubsection{Intercultural movement}
\label{interculturalmovement}

During the past two-thousand years, the messianic, new-covenant community has grown and developed through precisely this sort of dynamism. From its origins as an obscure, tiny, Jewish sect it has grown into an international, intercultural, multi-ethnic community, now existing, in some form or another, in practically every nation of the world.

The scripture citations below, from the book of Acts, illustrate a pattern. It is worth examining the context of these verses in your own Bibles, reflecting upon how each statement represents a conclusion to a significant period of church expansion.

\begin{itemize}
\item 2:46--47 — day after day, the Lord adding to them

\item 6:7 — the word of God continued to spread

\item 9.31 — their numbers kept multiplying

\item 12.24 — word of the Lord went on growing and being multiplied

\item 16.5 — the congregations{\ldots}increased in number day by day

\item 19.20 — the message about the Lord continued in a powerful way to grow in influence

\end{itemize}

This dynamic movement then spreads geographically and, more importantly, across ethnic and cultural boundaries. From its beginnings in Jerusalem, the movement expands throughout Palestine, Asia Minor, Greece and, finally, on to Rome—the very centre and seat of worldly power, at that time.

\begin{quote}

Sh'aul remained two whole years in a place he rented for himself and continued receiving all who came to see him, openly and without hindrance proclaiming the Kingdom of God and teaching about the Lord Yeshua the Messiah—\emph{Acts 28.30--31}
\end{quote}

Since then, it has continued to grow and to effect and impact many new ethnicities and peoples, stories and cultures, throughout the world. It will surely effect many more in the days and years that are ahead. 

This \emph{dynamism} can be traced to the Messianic Community operating as \emph{a movement of disciples}, continually spreading out, across geographical, social, ethnic, linguistic and cultural boundaries. It is this kind of dynamism that I hope will be stimulated by Maize Plant Discipleship, wherever it is adopted and utilised.

\subsubsection{Renewal of movement}
\label{renewalofmovement}

Any messianic movement or community—national, regional or local—that earnestly desires spiritual revival or renewal must place visionary, messianic discipleship at the core of its spirituality and its practical formation.

Disciples must be invited, formed and sent forth as part of a \emph{world-facing} movement. The goal is much more than the maintenance of the church-as-an-organisation, greater even than serving-one-another, as part of messianic community. The goal is to be part of a movement of disciples called to serve God's eternal purpose, amongst a world of lost, hurting, confused, oppressed, fear-filled, idol-bound populations.

Amongst the corruption in society, besides its filthy gutters and behind its social, political, economic and religious walls, the Messiah is at work by his Spirit. That is where he calls his co-workers to come alongside him in his work of redemption and transformation of individuals, families, marriages, partnerships, communities, organisations, structures, workplaces and working practices.

\subsubsection{Anointed community}
\label{anointedcommunity}

To make possible such an otherwise impossibly-high calling, messianic discipleship provides a unique ingredient that no other philosophy, ideology or faith can provide: the dynamic of the indwelling Spirit of the Messiah. 

\emph{Through the Spirit}, the new-covenant community is transformed into a charismatic community. A group of people endowed with spiritual gifts that are profoundly shaped to liberate human beings from idolatry and every other allegiance and falsehood that competes and sets itself against God and the knowledge of him\footnote{2 Corinthians 10:3--5}. 

\begin{center}\rule{3in}{0.4pt}\end{center}


\begin{quote}

The term \emph{charismatic} derives from \emph{charism} (Greek), meaning \emph{gift}. The \emph{charisma} of the Christian community comes from its spiritual anointing. \emph{Messiah} essentially means \emph{Anointed One}: the One Anointed with the Spirit. \footnote{Matthew 3:13--17; Mark 1:9--11; Luke 3:21--22; John 1:32--34} 
\end{quote}

\begin{center}\rule{3in}{0.4pt}\end{center}


Thus, the Messianic Community is a body of people anointed with the fragrant oil, or presence, of the Holy Spirit, having been brought under God's authority, through baptism into the Messiah. It is a body learning to walk in the footsteps of Jesus: learning to exercise its God-appointed mediatory, intercessory role, under the direction of the Spirit of God.\footnote{Romans 6:3--4; Galatians 3:26--29; Hebrews 6:4}

\begin{quote}

\textbf{This community of disciples is a messianic, charismatic people called into covenant relationship with the Father, through the Son and sent into the world to bless the nations in the power of the Spirit!}
\end{quote}

\subsubsection{Life{\ldots}through death}
\label{lifethroughdeath}

God's intention is that this messianic, charismatic, covenant community co-works in partnership with him, using the strength, the power, the spiritual life, the anointing that he provides. 

Too often though, that power, that anointed-life-of-Christ-within-us, seems to elude us. It seems out of our reach. Beyond our grasp. Indeed, it is not something that can be \emph{grasped}. Instead, the pathway to life is through dying. Yielding ourselves to God the Father, through the Messiah, by the Spirit. That is the message of the cross. As we \emph{die to self}, we become \emph{alive to God}.\footnote{Romans 6:4--13}

\subsection{The heart of discipleship}
\label{theheartofdiscipleship}

Thus we end where we began. With the foundational principle of transformative discipleship: seed sown into the ground, in order to produce a harvest.

\begin{quote}

I tell you that unless a grain of wheat that falls to the ground dies, it stays just a grain; but if it dies, it produces a big harvest—\emph{John 12.24}
\end{quote}

This life-giving spiritual reality is at the heart of Jesus' own life, mission, ministry and pain-filled death. And this same principle forms the foundation and wellspring of Maize Plant Discipleship: 

\begin{quote}

As we embrace a practical form of discipleship, incorporating a daily dying-to-self, we learn how to truly become alive-to-God and equipped to serve his eternal purpose.
\end{quote}

\emph{That is the heart of Maize Plant Discipleship}.

\subsection{Questions for Reflection and Discussion}
\label{questionsforreflectionanddiscussion}

\begin{itemize}
\item How is discipleship valued in your context?

\item How faithfully is it practised?

\end{itemize}

If there is a gap between what is believed and valued and what is practised, discuss why you think that is.

\chapter{Further information}
\label{furtherinformation}

\section{Translation}
\label{translation}

The text books of the Maize Plant Discipleship Curriculum have been intentionally formulated to be readily translatable. 

\begin{quote}

If you would like to discuss translating MPD resources, for printing and distribution in another language, please get in touch with the author, or via the MPD website.
\end{quote}

\section{Internet}
\label{internet}

Additional information about Maize Plant Discipleship:

\begin{itemize}
\item \href{http://maizeplantdiscipleship.wordpress.com/}{http:/\slash maizeplantdiscipleship.wordpress.com\slash }\footnote{\href{http://maizeplantdiscipleship.wordpress.com/}{http:/\slash maizeplantdiscipleship.wordpress.com\slash }}

\item @MPDResource

\end{itemize}

\pagebreak 

\section{Author}
\label{author}

Dr John B Clements is a missiological educator, having received a doctorate from Fuller Theological Seminary School of Intercultural Studies, in 2012. 

\begin{itemize}
\item Vita — http:/\slash jbclements.wordpress.com

\item Linked-in — http:/\slash uk.linkedin.com\slash in\slash jbclements

\end{itemize}

John is married to Sarah; they have three boys and one girl and presently live in a delightful corner of South West Wales, UK. John is an avid bird-watcher and casual photographer, pastimes that he combines with his enjoyment of countryside and coastal walking.

\begin{itemize}
\item Social networking — http:/\slash about.me\slash jbclements

\item Twitter — @johnbrc

\item Email — jbclements@icloud.com

\end{itemize}

\input{mpd-footer}

\end{document}
