\def\version{0.3.11 --- Copyediting Draft}
\def\change{updated figures, ch2; 
discipleship movements, ch4}
\input{mpd-header}
\def\mytitle{Facilitators' Handbook}
\def\subtitle{Maize Plant Discipleship Handbook}
\def\myauthor{Dr John B Clements}
\def\twitter{@johnbrc, @mpdresource}
\def\email{clements.jb@gmail.com}
\def\web{http:/\slash maizeplantdiscipleship.info}
\def\mycopyright{John B Clements, 2014.  Maize Plant Discipleship by John B Clements is licensed under a Creative Commons Attribution-NonCommercial-ShareAlike 4.0 International Licence. Based on a work at http:/\slash johnbrc.github.io\slash MPD-Distribution}
\def\keywords{discipleship, mission, messianic, community}
\def\latexmode{memoir}
\input{mpd-document-fh}

\chapter{Maize Plant Discipleship}
\label{maizeplantdiscipleship}

THIS CHAPTER PROVIDES practical insights and suggestions for facilitating Maize Plant Discipleship learning groups.

\section{What is Maize Plant Discipleship?}
\label{whatismaizeplantdiscipleship}

Maize Plant Discipleship is a learning resource, designed to be practical, relevant and accessible for use in African and other majority world contexts. It has been derived and road-tested in collaboration with Africans, formulated in response to contextual doctoral research in Burkina Faso and is being published as a series of short, modular, low-cost handbooks:

\begin{itemize}
\item suitable for formal and informal modes of study

\item incorporating reflective learning and group discussions

\item reliant simply upon facilitators co-ordinating small learning groups

\item easily replicable, in terms of both republication and training.

\end{itemize}

Its goal is to facilitate biblical learning that continuously moves outwards, drawing whole communities into patterns of scripturally-based discipleship, in living dialogue with contextual culture.

\section{Maize Plant Discipleship Learning Groups}
\label{maizeplantdiscipleshiplearninggroups}

Maize Plant Discipleship is intended to be an open, reflective, group learning process, in which leaders and learners alike participate together in discovering what the Spirit is saying, as Scripture is studied and related to contextual \emph{signs of the times.}

\subsection{Why learning groups?}
\label{whylearninggroups}

There are many reasons to bring together a group of people to learn together. Some people would point to Jesus' gathering of twelve disciples. For most people, groups represent a natural and lively place in which to learn. They bring together people with different experiences, gifts, capacities and perspectives. 

When we share our lives, we learn together and \emph{learning groups} mirror this reality. Together, we experience and learn quite differently to when we study alone. Reflective discussion with others, in particular, provides a highly stimulating forum for learning through exploration, listening and discovery. For further discussion about reflective learning, see \emph{Maize Plant Discipleship Learning Process} (\autoref{maizeplantdiscipleshiplearningprocess}).

\subsection{Learning, not teaching?}
\label{learningnotteaching}

Yes. Maize Plant Discipleship is principally a \emph{learning}, rather than a teaching resource. Learning depends on many factors, in addition to the presentation of topical information—most of which relate to the student, rather than the teacher, or teaching. Such factors include desire, temperament, experience, talent, time, energy, environment and so forth. Thus the decision to learn something new must always begin with the student themselves. 

Accordingly, discipleship should be recognised as a form of learning that is \emph{called out} of students or disciples, through the guidance and direction of a facilitator, mentor, educator or teacher. People in those roles come alongside \emph{motivated learners} to assist, encourage, facilitate and call out the learning taking place within those being discipled.

\section{Facilitating Maize Plant Discipleship Learning Groups}
\label{facilitatingmaizeplantdiscipleshiplearninggroups}

A Maize Plant Discipleship Facilitator can help a group open up to the message of Scripture, to one another and, above all, to the leadership of the Holy Spirit. This section explores how.

\subsection{Facilitating Spirit-led discipleship}
\label{facilitatingspirit-leddiscipleship}

Discipleship cannot be reduced to a mere replication of information, from teacher, or textbooks, into students. We may have been taught this way in school; Spirit-led discipleship is different. It \emph{forms}, as well as informs.

The intention is that through teaching, reflection and discussion, God's Spirit is able to speak to, lead, warn, direct, encourage, educate, challenge and exhort us, personally and corporately. Since each person differs in gift, personality and development, at any time, each person may be learning something different from the Spirit. 

\begin {pause}

\begin{description}

\item[The goal of discipleship —]

is not to establish shared dogmatic belief or conformity to the convictions of a leader, mentor or facilitator, nor to church traditions, certainly not to every aspect of Maize Plant Discipleship.

\item[The goal —]

is conformity to the Spirit of the Messiah, Jesus, and obedience to the will of the Father.
\end{description}

\end {pause}

\subsection{Facilitating openness}
\label{facilitatingopenness}

Fostering an environment of learning and discovery, where debate and discussion are lively, yet relaxed and uncompetitive, is essential. An ideal environment will allow strong and diverse views to be expressed, without creating either conflict or conformity, so that all present feel comfortable to contribute their views, questions and burdens. 

It can be particularly challenging to foster openness in cultures where authority traditionally flows downwards and conformity is highly valued. Thus, facilitators should typically contribute to discussions as regular group members, never dominating or belittling the views of others.

\begin{itemize}
\item Allow discussion to ebb and flow, as people consider their responses and return to earlier discussions. Encourage others to contribute, especially quieter members, women, youth and elders.

\item If discussion becomes harsh or factious, quieten the group, then invite a member with a harmonious or gentle spirit to summarise (not resolve) the tension, then move on.

\end{itemize}

\subsection{Who can facilitate?}
\label{whocanfacilitate}

A facilitator needs to be someone who senses a calling to help others become faithful Christian disciples. This must motivate them to be humble, patient, warm, flexible, open and secure enough to allow others to explore personal boundaries of vocational understanding, experience and creativity, at their own pace.

A facilitator does not hold a position of authority over people. They simply facilitate the gathering of people into groups, for learning and discussion. Accordingly, a facilitator:

\begin{itemize}
\item may be a lay-leader;

\item may be relatively young;

\item may be a woman;

\item need not have gone to bible college;

\item need not be an established church leader;

\item need not be an experienced mentor.

\end{itemize}

Of course, Maize Plant Discipleship can be facilitated by established leaders, mentors or disciplers, providing they are willing and comfortable to facilitate group discussions that are genuinely \emph{reflective and exploratory}.

\subsection{Practical Considerations}
\label{practicalconsiderations}

Facilitating a discipleship group will be most effective when practicalities are considered in advance and appropriate planning takes place. 

\subsubsection{Getting started}
\label{gettingstarted}

Maize Plant Discipleship is ideal for learning groups of 8--10 people. This is small enough to allow group members to grow together with a degree of intimacy and large enough for group members to explore discipleship commitments at their own pace.

\begin {pause}

\begin{description}

\item[More than ten? —]

Think about helping others to facilitate additional learning groups: what problems might you face?
\end{description}

\end {pause}

\subsubsection{Involving others}
\label{involvingothers}

Although a facilitator is responsible for convening gatherings, they may delegate responsibility for hosting, presenting the teaching, or moderating group discussions. Ideally, over time, all group members will carry some responsibility, according to their gift and capability. This avoids one person carrying too much and gives everyone some experience of the responsibilities of the facilitating role.

The person responsible for presenting a topical study should read through it carefully, in advance: absorbing, familiarising and reflecting upon the teaching and its lessons. If anything is unfamiliar or unclear, invite discussion about that area of the study, encouraging group members to bring forward their perspectives.

\subsubsection{Replication}
\label{replication}

Replication is an important goal of Maize Plant Discipleship. See \emph{Maize Plant Philosophy of Discipleship} (\autoref{maizeplantphilosophyofdiscipleship}) for a detailed exhortation about its significance and importance. It may not be necessary for a group member to complete the entire syllabus before branching out to facilitate another learning group; be led by the Spirit.

\subsubsection{Location}
\label{location}

Meeting together can take place in any appropriate location that can comfortably accommodate a learning group. For example, a large room in someone's home, or a communal building, such as a church.

\begin {pause}

\begin{description}

\item[Arrange seating —]

in order to create intimate and practical spaces, whereby everyone can hear and see the presenter and each other during discussions.

\item[Find out what works for your group —]

consider using different locations, maybe even meeting outside sometimes.
\end{description}

\end {pause}

\subsubsection{Adaption}
\label{adaption}

Be prepared to adapt the teaching and the method of presentation, in order to create a helpful and culturally-appropriate learning environment. Take account of the abilities and capacity of each particular discipleship group. 

\begin {pause}

\begin{description}

\item[Ensure literature is a helpful servant —]

not a hard task-master, especially to oral learners. As far as possible, keep things simple and lighthearted.

\item[Invite creative people to contribute —]

by interpreting or celebrating an aspect of the teaching through songs, drama or art.
\end{description}

\end {pause}

\subsubsection{Timetable}
\label{timetable}

The entire Maize Plant Discipleship syllabus incorporates approximately 64 studies. When considering a timetable, take into consideration the nature of the group availability. \emph{For example, are members affected by the demands of agricultural seasons, or academic terms?}

\subsection{Other learning forums}
\label{otherlearningforums}

Maize Plant Discipleship handbooks can easily lend themselves to personal study, theological education and other forms of guided learning. In particular,

\begin{itemize}
\item Scripture references, within the footnotes, provide a rich treasury of additional material for in-depth biblical study

\item discussion questions can be adopted as the basis for written answers, or even short essays. 

\end{itemize}

\subsubsection{Personal study}
\label{personalstudy}

Students using Maize Plant Discipleship handbooks for personal study should seek to incorporate reflective learning methods by either submitting the fruit of their study to the oversight of a mentor, or sharing it with a fellow student, for reflection, discussion and critical consideration.

\subsubsection{Theological education}
\label{theologicaleducation}

Bible school students should be encouraged to form and facilitate small learning groups, either within or alongside classroom contexts, and to reflect on their experiences together. This will provide highly valuable experience and momentum for facilitating Maize Plant Discipleship learning groups within their own vocational contexts.

\subsubsection{Congregational teaching}
\label{congregationalteaching}

Maize Plant Discipleship studies can be adapted for use in local congregations. For example, the congregation, after receiving a teaching study, could divide into small discussion groups. Alternatively, learning groups could meet on a separate occasion, to reflect upon and discuss the teaching and to pray together.

\begin{center}\rule{3in}{0.4pt}\end{center}


\begin {questions}

\begin{itemize}
\item What are the characteristics of a good facilitator?

\item What is the goal of discipleship?

\item How do Maize Plant Discipleship learning groups differ from other learning forums, such as schools?

\end{itemize}

\end {questions}

\chapter{Maize Plant Discipleship Learning Process}
\label{maizeplantdiscipleshiplearningprocess}

THIS CHAPTER PRESCRIBES the reflective learning process intended to underlie presentation of the Maize Plant Discipleship syllabus.

\section{Reflective Learning}
\label{reflectivelearning}

Maize Plant Discipleship modules are structured to provide a \emph{reflective learning process} incorporating practical, educational principles combined with practical, spiritual principles, drawn from Scripture. Reflective group learning must minimally incorporate these four components:

\begin{itemize}
\item \textbf{Hearing} — about others' experiences and perspectives; practice informing theory and theory informing practice. 

\item \textbf{Reflecting} — upon the interaction of old and new ideas, concepts and perspectives. 

\item \textbf{Discussing} — understanding and sharpening conviction in conversation with others.

\item \textbf{Acting} — integrating learning into practice, in vocational contexts.

\end{itemize}

As this learning process is repeated, it becomes a \emph{cycle}, which can be illustrated figuratively.

\begin{figure}[htbp]
\centering
\includegraphics[width=208pt,height=200pt]{learning-cycle.png}
\caption{}
\label{learning-cycle.png}
\end{figure}

The \emph{Maize Plant Discipleship Learning Process} builds upon this basic cycle, by incorporating a simple pattern, drawn from Acts 2:42:

\begin{quote}

They continued faithfully in the teaching of the apostles, in fellowship, in breaking bread and in prayer.
\end{quote}

The passage inspires some adaption of the reflective learning elements, whilst prayer and the breaking of bread provide two unique, additional elements. Thus the \emph{Maize Plant Discipleship Learning Process} has \emph{six} elements altogether:

\begin{enumerate}
\item \textbf{HEAR} {\ldots} what the Spirit is saying

\item \textbf{RETAIN} {\ldots} God's message inwardly

\item \textbf{OPEN} {\ldots} hearts to others

\item \textbf{SHARE} {\ldots} daily bread together

\item \textbf{PRAY} {\ldots} for God's kingdom to come

\item \textbf{ACT} {\ldots} in the light of God's message

\end{enumerate}

\pagebreak 

\section{1. HEAR What the Spirit is Saying}
\label{hearwhatthespiritissaying}

\begin{figure}[htbp]
\centering
\includegraphics[width=108pt,height=108pt]{hear.png}
\caption{}
\label{hear.png}
\end{figure}

When we gather together as disciples of the Messiah, to hear biblical teaching, we are opening ourselves not simply to human ideas or wisdom, but to spiritual words and truths, taught by the Spirit of God.

\begin{quote}

Now we have not received the spirit that belongs to the world, but the Holy Spirit Who is from God, given to us that we might realise and comprehend and appreciate the gifts of divine favour and blessing so freely and lavishly bestowed on us by God. And we're setting these truths forth in words not taught by human wisdom but taught by the Holy Spirit, combining and interpreting spiritual truths with spiritual language to those who possess the Holy Spirit—\emph{1 Corinthians 2:12--13 TAB}
\end{quote}

\begin {pause}

\begin{description}

\item[We listen in order to live more faithfully —]

this type of listening is called \emph{heeding}: listening with the intention to learn and obey, or follow.

\item[We listen with our mind, but also with our heart —]

in order to \emph{hear what the Spirit is saying to his people} (Revelation 2:29, 3:6,13,23; Matthew 11:15, Mark 4:9 etc)—never in order to become \emph{puffed up} by knowledge.
\end{description}

\end {pause}

\pagebreak 

\section{2. RETAIN God's Message Inwardly}
\label{retaingodsmessageinwardly}

\begin{figure}[htbp]
\centering
\includegraphics[width=108pt,height=108pt]{retain.png}
\caption{}
\label{retain.png}
\end{figure}

It is not enough only to hear God's message: we must learn to \emph{retain} God's word inwardly, where it can begin to \emph{dwell richly within us.} (Colossians 3:16)

\begin{quote}

The one who received the seed that fell on rocky places is the man who hears the word and at once receives it with joy. But since he has no root, he lasts only a short time{\ldots} The one who received the seed that fell among the thorns is the man who hears the word, but the worries of this life and the deceitfulness of wealth choke it, making it unfruitful. The seed on good soil stands for those with a noble and good heart, who hear the word and \emph{retain} it, and by persevering produce a crop {\ldots} yielding a hundred, sixty or thirty times what was sown — \emph{Matthew 13:18--23; Luke 8:15}
\end{quote}

\begin {pause}

\begin{description}

\item[Think about how we receive and retain food —]

chewing it, enjoying the taste, swallowing, digesting, inwardly retaining its vitality and goodness.

\item[It's the same with God's word —]

we must \emph{chew it over,} meditating and reflecting upon its meaning and application to our lives, both as individuals and as communities, allowing it to settle in our spirit, where it can form and shape our convictions and renew our hope.
\end{description}

\end {pause}

\pagebreak 

\section{3. OPEN Hearts to Others}
\label{openheartstoothers}

\begin{figure}[htbp]
\centering
\includegraphics[width=108pt,height=108pt]{open.png}
\caption{}
\label{open.png}
\end{figure}

Discussion and debate is an opportunity to open our hearts to the perspectives and experiences of those around us and those who see things differently to ourselves. 

\begin{itemize}
\item This requires listening with the heart, as well as the head, in order to appreciate what others are sharing, rather than to win an argument. 

\item Discussion of practical, \emph{vocational} applications of topical study is vital; think about how Maize Plant Discipleship teachings relate to the cultural contexts amongst which group members live.

\item Allow plenty of time for this aspect of Maize Plant Discipleship learning!

\end{itemize}

\begin {pause}

\begin{description}

\item[Vocation is more than simply our job, or employment —]

it incorporates all the responsibilities towards which God calls us, including families, workplaces (and practices), communities and networks.

\item[Consider traditional proverbs that relate to study topics —]

the \emph{sweet talk} of proverbs can provide fresh insight and be helpful in discussing Maize Plant Discipleship with others, such as elders or non-believers.
\end{description}

\end {pause}

\pagebreak 

\section{4. SHARE Daily Bread}
\label{sharedailybread}

\begin{figure}[htbp]
\centering
\includegraphics[width=108pt,height=108pt]{share.png}
\caption{}
\label{share.png}
\end{figure}

The celebratory breaking of bread, in order to remember the sacrificial obedience of Jesus, is a significant symbol of the New Covenant and a profound way for discipleship groups to proclaim their shared devotion to the Messiah.

\begin{itemize}
\item In modern forms of Christianity, breaking and sharing bread is typically ceremonial \emph{(Eucharist, Holy Communion, Mass)}. Yet, the earliest messianic communities based it simply upon the Passover meal, like the one Jesus shared with his disciples, immediately prior to his arrest by the Jerusalem authorities.

\item Sharing food together is therefore both a vital part of human fellowship \emph{and} a practical way of celebrating and proclaiming God's covenantal provision and blessing.

\end{itemize}

\begin {pause}

\begin{description}

\item[Consider incorporating a simple meal —]

perhaps once a month, into times of meeting together and prayerfully identifying it as a form of breaking bread.

\item[If a meal is not a practical possibility —]

consider sharing a small amount of bread together, as a symbolic act of hospitality and shared commitment to membership of the body of the Messiah.
\end{description}

\end {pause}

\pagebreak 

\section{5. PRAY for God's Kingdom to Come}
\label{prayforgodskingdomtocome}

\begin{figure}[htbp]
\centering
\includegraphics[width=108pt,height=108pt]{pray.png}
\caption{}
\label{pray.png}
\end{figure}

After a period of discussion, invite the group to pray together, including intercession on behalf of neighbours, networks and communities, local and national rulers and governors. 

\begin{itemize}
\item Allow the teaching to infuse prayer with fresh confidence concerning God's will and purpose, including personal and vocational concerns and challenges faced by group members. 

\item Allow the Holy Spirit to lead you in speaking blessings, rooted in Scripture, over one another's lives and over your community or nation, or with regards to a specific problem.

\item Expect the power of God to overcome all opposition, through the blessing of his overcoming life at work in and through his people!

\end{itemize}

\begin {pause}

\begin{description}

\item[Pray for the gospel —]

to deeply impact and transform individuals, communities, cultures and societies throughout your nation, Africa, Europe, Asia and the Americas; pray for unreached people groups.

\item[Pray for the Maize Plant Discipleship project —]

that it will be used by God to edify, strengthen and bless the Messianic Community, within Africa {\ldots} and beyond!
\end{description}

\end {pause}

\pagebreak 

\section{6. ACT in the Light of God's Message}
\label{actinthelightofgodsmessage}

\begin{figure}[htbp]
\centering
\includegraphics[width=108pt,height=108pt]{act.png}
\caption{}
\label{act.png}
\end{figure}

The purpose of our gathering to hear God's message is not simply to hear it, but to act upon it. As the epistle of \emph{James} explains, we deceive ourselves when we listen to God's word, yet do not do what it says:

\begin{quote}

Don't deceive yourselves by only hearing what the Word says, but do it! For whoever hears the Word but doesn't do what it says is like someone who looks at his face in a mirror, who looks as himself, goes away and immediately forgets what he looks like. But if a person looks closely into the perfect \emph{Torah (Teaching)}, which gives freedom, and continues, becoming not a forgetful hearer but a doer of the work it requires, then he will be blessed in what he does — \emph{James 1:22--25}
\end{quote}

\begin {pause}

\begin{description}

\item[The goal of discipleship —]

is to be transformed ourselves and thus a transformative influence, in our homes and amongst our workplaces and social networks.

\item[As we are transformed —]

as part of a growing, dynamic movement of disciples, we begin to fulfil our corporate vocation: \emph{to be a Messianic Community blessed to be a blessing to the families of the earth!}
\end{description}

\end {pause}

\pagebreak 

\section{Maize Plant Discipleship Learning Cycle}
\label{maizeplantdiscipleshiplearningcycle}

Combining together the six elements of the Maize Plant Discipleship learning process produces the Maize Plant Discipleship learning cycle.

\begin{figure}[htbp]
\centering
\includegraphics[width=268pt,height=300pt]{mpd-learning-cycle.png}
\caption{}
\label{mpd-learning-cycle.png}
\end{figure}

\begin {pause}

\begin{description}

\item[The Maize Plant Discipleship learning cycle is a tool —]

its purpose is to serve facilitators and learning groups. It may not always be possible to incorporate all the elements, each and every time. Allow it to stretch, but not to limit your learning. \emph{Where appropriate, adapt it}.
\end{description}

\end {pause}

\chapter{Maize Plant Discipleship Syllabus}
\label{maizeplantdiscipleshipsyllabus}

THIS CHAPTER INTRODUCES the metaphor of the maize plant and the sixteen modules of the Maize Plant Discipleship Syllabus.

\section{The Maize Plant Metaphor}
\label{themaizeplantmetaphor}

Jesus refers to his own mission using the metaphor of a seed that enters the ground and dies, in order to produce a large harvest. 

\begin{quote}

I tell you that unless a grain of wheat that falls to the ground dies, it stays just a grain; but if it dies, it produces a big harvest — \emph{Jesus, John 12:24}
\end{quote}

Maize is grown throughout sub-Saharan Africa and provides a similar, highly recognisable metaphor relating to the essential dynamics of life, death, sustenance and growth. The Maize Plant Discipleship Syllabus is symbolically structured to reflect this fundamental metaphor:

\pagebreak 

\textbf{Good seed sown in good soil, stimulated by sunshine and refreshed by rainfall produces dynamic growth and a good harvest}.

\begin{figure}[htbp]
\centering
\includegraphics[width=174pt,height=250pt]{mp-metaphor.png}
\caption{}
\label{mp-metaphor.png}
\end{figure}

Like maize plants:

\begin{itemize}
\item messianic communities need to be rooted in good ground that allows them to draw on vital, spiritual nutrients, stimulated by revelatory light and refreshed by the living water of the Spirit;

\item strong roots will anchor messianic communities against destructive winds of false teaching and sustain them amidst the withering heat of trials, temptations and vocational responsibility.

\end{itemize}

Based on this metaphor, the Maize Plant Discipleship Syllabus is divided into three parts:

\begin{enumerate}
\item Soil and roots (Module 1)

\item Maize plant (Modules 2--9)

\item Sunlight and rainfall (Modules 10--16)

\end{enumerate}

\subsection{Soil and roots}
\label{soilandroots}

Metaphorically, the roots of the maize plant represent the biblical community of Israel. The soil in which the roots grow equates to the historical, cultural and geo-political contexts of Israel's covenant vocation (such as Egypt, Canaan, Babylon and the Roman occupation).

\begin{itemize}
\item \textbf{Module 1, The Eternal Purpose of God}

Establishes a panoramic overview of Scripture, revealing God's unchanging, eternal purpose. In this perspective, the Messiah is \emph{the Seed} that sprouts from the soil of the biblical, covenantal history of the people of Israel, in order to die and ultimately produce a rich harvest of people—a Messianic Covenant Community—from amongst all the peoples of the earth. \emph{Module 1, thus forms the foundation of the entire syllabus}.

\end{itemize}

\begin{figure}[htbp]
\centering
\includegraphics[width=270pt,height=150pt]{mpd-roots.png}
\caption{}
\label{mpd-roots.png}
\end{figure}

\begin {pause}

\begin{description}

\item[Maize produce —]

effectively represents \emph{daily bread} for millions of African people. Like the Messiah, the Messianic Community is called to become a kind of \emph{life-giving bread} to the peoples of the world—see John 6 (\& 20:21)
\end{description}

\end {pause}

\pagebreak 

\subsection{Maize plant}
\label{maizeplant}

\textbf{Modules 2 to 9} explore seven messianic \emph{dynamics} that represent the characteristic development, growth, structure, shape and fruit of the Messianic Community.\footnote{\emph{Dynamic} derives from a Greek word, \emph{dunamis}, meaning power and refers to forces stimulating change within a process or system (such as a plant or a body).}

\begin{figure}[htbp]
\centering
\includegraphics[width=258pt,height=374pt]{mp-dynamics.png}
\caption{}
\label{mp-dynamics.png}
\end{figure}

\begin{itemize}
\item \textbf{Module 2, Dynamics of Vocation, The Nations}

The historical development of the Messianic Community's vocational mission to bless the peoples of the world.

\item \textbf{Module 3, Dynamics of Vocation, The Jews}

The special responsibility of the Messianic Community towards the Jewish people.

\item \textbf{Module 4, Dynamics of Commissioning}

Strategic and structural dynamics of messianic commissioning and community growth.

\item \textbf{Module 5, Dynamics of Body Membership}

Membership, commitment and spiritual maturity within the body of the Messiah.

\item \textbf{Module 6, Dynamics of Revival}

The dynamics of revival and a spiritual harvest of covenant faithfulness.

\item \textbf{Module 7, Dynamics of Truth}

Encountering truth as we walk in practical, covenantal faithfulness towards God's revelation, wisdom and direction.

\item \textbf{Module 8, Dynamics of Intercession}

The priestly vocation of the Messianic Community: to be \emph{a house of prayer for all nations}.

\item \textbf{Module 9, Dynamics of Cultural Transformation}

The call to work amongst and on behalf of the nations, towards cultural transformations that signal the presence of God's kingdom.

\end{itemize}

\pagebreak 

\subsection{Sunlight and rainfall}
\label{sunlightandrainfall}

\textbf{Modules 10 to 16} examine seven characteristic \emph{disciplines} that enable messianic communities to receive the revelatory \emph{light} and sustaining \emph{living water} of God's Spirit, without which we become spiritually weak and incapable of producing good fruit or a plentiful harvest.

\begin{figure}[htbp]
\centering
\includegraphics[width=305pt,height=391pt]{mp-disciplines.png}
\caption{}
\label{mp-disciplines.png}
\end{figure}

\begin{itemize}
\item \textbf{Module 10, Disciplines of Spiritual Maturity}

Three stages of encounter and growth in spiritual maturity of messianic disciples and communities.

\item \textbf{Module 11, Disciplines of Running the Race}

Motivations, qualities and disciplines for living an enduring life of service.

\item \textbf{Module 12, Disciplines of Pressing Toward Our Vocation}

Identifying and excelling in our personal vocation, through a deepening of our relationship with the Messiah.

\item \textbf{Module 13, Disciplines of Economic Faithfulness}

A biblical perspective upon economic faithfulness, wealth and poverty—radically different to that of the world. 

\item \textbf{Module 14, Disciplines of Messianic Leadership}

Qualifications, motivations and characteristics of faithful messianic leadership.

\item \textbf{Module 15, Disciplines of Living by Faith}

Seeing with eyes of faith enables us to endure times of testing and purification and to embrace challenge as an opportunity for experiencing God's faithfulness.

\item \textbf{Module 16, Disciplines of Overcoming}

Confronting idolatrous, cultural strongholds in the power of the Spirit and discerning strategies that make room for a transformative encounter with God's overcoming power.

\end{itemize}

\pagebreak 

\section{Module Handbooks}
\label{modulehandbooks}

Each syllabus module is incorporated in a handbook, containing four interrelated \emph{studies}, each incorporating:

\begin{itemize}
\item Scripture readings

\item topical sections, illustrations and summaries

\item discussion questions.

\end{itemize}

\begin {pause}

\begin{description}

\item[Illustrations —]

If insufficient books are available, diagrams and illustrations should be reproduced, using blackboards or other aids, even drawing on the ground if necessary, in order to allow all group members to appreciate the relevance of the topical illustration. \emph{If you have an artist in the group, try giving them this task}.
\end{description}

\end {pause}

\subsection{Scripture Versions}
\label{scriptureversions}

Scripture references are typically provided, rather than quotations. Where a quotation is provided, these abbreviations indicate the version:

\begin{itemize}
\item CJB — The Complete Jewish Bible

\item NIV — The New International Version

\item TAB — The Amplified Bible

\end{itemize}

\begin {pause}

\begin{description}

\item[Mother-tongue is the natural language of the heart —]

Mother-tongue Scripture translations should be encouraged and utilised as much as possible, including during group discussions and particularly for reading and memorisation.
\end{description}

\end {pause}

\section{Significant Terms}
\label{significantterms}

Important definitions provided below explain how these significant terms are used in the the Maize Plant Discipleship Syllabus.

\begin{description}

\item[Messiah —]

a mediator or saviour, acting with God's authority to deliver a people from the grip of their enemies and, or to govern over and keep them safe (experiencing \emph{shalom}). In the biblical history of Israel, deliverance came through a variety of mediators, such as prophets, priests, judges and kings. 

The root meaning of messiah is \emph{anointed} or \emph{poured on}, referring to the anointing oil poured onto Israel's kings and priests. Anointing oil symbolises the pouring out, or placing of God's Spirit upon a leader, as they were invested with their authority, usually by prophets.\footnote{E.g. Exodus 30:22--25} Thus the idea of God's priestly, kingly and prophetic authority is inherent in the concept of \emph{messiah}. 

The New Testament identifies Jesus as the Jewish Messiah,\footnote{\emph{Christ} is the Greek translation of \emph{Moshiach} (Hebrew, Messiah); Jesus Christ is the Greek rendering of \emph{Yeshua Moshiach}} anointed by the Spirit to fulfil three mediatory roles—prophet, priest and king—on behalf of God's people.\footnote{Matthew 3:13--17; Mark 1:9--11; Luke 3:21--22 \& 4:16--19; John 1:32--34} After his ascension to the \emph{Right Hand of God}, Jesus becomes the \emph{one Mediator between God and humanity}.\footnote{Hebrews 8:1--2; 1 Timothy 2:5}

\item[Messianic —]

relating to the Messiah; primarily used in the syllabus to refer to \emph{Messianic Community} and \emph{messianic communities}.

\item[Messianic Community (capitalised) —]

the whole, worldwide and historical body of people belonging to the Messiah. In the New Testament this community is referred to as \emph{the body of Messiah (Christ)}. The reference is broadly equivalent to \emph{Worldwide Christian Community} or \emph{Church}. Those terms, however, are generally avoided because of their association with particular historical expressions of Christianity that are not inclusive.

\item[messianic communities (un-capitalised) —]

localised congregations of the Messianic Community. The term is used in preference to \emph{churches} in order to emphasise the biblical link to the whole body of the Messiah, the Messianic Community.

\item[Vocation —]

a calling, life's work, mission, purpose, function; profession, occupation, career, job, employment, trade, craft, business, line, line of work, métier.

In the Maize Plant Discipleship syllabus, \emph{vocation} and \emph{vocational} may refer to both personal and communal calling. Within the syllabus, the terms are used to emphasise that both an individual and a local community's sense of vocation flow from the divine calling to serve God's eternal purpose, in union with the Messiah.

Vocation also represents an umbrella term that incorporates and dignifies all forms of work and ministry. It looks beyond traditional divides of laity and clergy, male and female, pointing towards the reality that all followers of the Messiah are \emph{called to faithfully serve God's purposes, within homes, workplaces and communities}. 
\end{description}

\begin{center}\rule{3in}{0.4pt}\end{center}


\begin {questions}

\begin{itemize}
\item What is the significance of the metaphor of seed to the ministry of Jesus?

\item How would you describe the significance of the maize plant to Maize Plant Discipleship?

\item How would you describe a \emph{dynamic}?

\item Consider the various ways that maize is important in your context. What does this suggest to you about the importance of the messianic community to a social or cultural context?

\end{itemize}

\end {questions}

\chapter{Maize Plant Philosophy of Discipleship}
\label{maizeplantphilosophyofdiscipleship}

THIS CHAPTER EXPLORES the foundational biblical perspectives of messianic discipleship, upon which Maize Plant Discipleship is established.

\section{What is Messianic Discipleship?}
\label{whatismessianicdiscipleship}

Maize Plant Discipleship approaches messianic discipleship as a dynamic, generational process, empowered by the Holy Spirit. Two crucial statements made by the apostle, Paul, in his second letter to his disciple, Timothy, reveal the essence of this process:\footnote{The two statements are separated only by Paul's emotional response to two disciples who failed to stand with him at a crucial time, which he contrasts with Onesiphorus' loyalty.}

\begin{quote}

Keep safe the great treasure that has been entrusted to you, with the help of the Holy Spirit, who lives in us {\ldots} and the things you heard from me, which were supported by many witnesses, these commit to faithful people, such as will be competent to teach others---\emph{2 Timothy 1.14, 2.2}
\end{quote}

Together, these two Scriptures establish three key components of messianic discipleship:

\begin{enumerate}
\item \textbf{The great treasure of knowing the Messiah, Jesus Christ}

The real, personal, experiential knowledge of the Messiah is more than human knowledge or philosophy: it is \emph{a great treasure}, a divine relationship, mediated by the Holy Spirit.

\item \textbf{The vitality of the Holy Spirit}

The Holy Spirit provides an intimate source of divine help to messianic disciples,\footnote{John 16:7--15} mediating and helping to safeguard the reality of the Gospel and the presence of the Messiah amongst his people.

\item \textbf{The necessity of generational formation}

Having received through Paul an impartation of the reality of the Messiah, Timothy is called upon by his mentor to safeguard this treasure by committing it to the stewardship and safekeeping of other faithful people. \emph{This is generational discipleship in action}. 

\begin{figure}[htbp]
\centering
\includegraphics[width=268pt,height=45pt]{Generational-discipleship.png}
\caption{}
\label{Generational-discipleship.png}
\end{figure}

Generational discipleship is how treasure is kept safe in the kingdom of God. The significance of the principle of generational discipleship can be further illustrated by considering the metaphor of \emph{seeds} and the harvest that comes from sowing seeds into good soil.

\end{enumerate}

\subsection{Seed and harvest}
\label{seedandharvest}

In farming contexts, seeds are a form of wealth. They are a type of treasure. Yet seed is generally stored for only a short time before being used. Whatever is not required for food, \emph{for daily bread}, must soon be sown to produce another harvest.\footnote{2 Corinthians 9:6--12}

In a similar way, God supplies spiritual life to us. This is what Paul refers to as the treasure of knowing the Messiah. Being alive to God and experiencing the grace of the Messiah and the love, joy, peace, patience, kindness, goodness, faithfulness, gentleness and self-control of the Holy Spirit,\footnote{Galatians 5:22} is the spiritual equivalent of receiving daily bread.

This personal aspect of knowing the Messiah, however, is not the whole purpose of our relationship with him. In fact, the Messianic New Covenant Community (the whole body of the Messiah's people) has been called to know God \emph{in order to become his servant community}. This means that we are called to give ourselves, our lives, to serve his purposes. This requires sacrifice and discipline—which is what it means to be a disciple.

This sacrificial, disciplined giving of ourselves, in service to God, is the equivalent of taking precious seed that could be consumed by ourselves and instead sowing it within the ground, in order to produce a harvest.

\subsection{Sharing treasure}
\label{sharingtreasure}

Discipline and sacrifice are amongst the most significant secrets to living a truly messianic life. Sadly, they are secrets that many people never properly discover, let alone realise as life-giving principles. Yet the illustration of seeds teaches us that hoarding the treasure of our knowledge, relationship and communion with God is not the way to a rich harvest. It is only in sacrificially sharing our spiritual treasure, both within and beyond our own communities, that we discover and realise our vocation and, in due time, reap a harvest of faithfulness.\footnote{Matthew 10:38--39, 13:23; Galatians 6:6--10; Hebrews 12:11; James 3:18}

Nevertheless, spiritual treasure should never be wasted or cast away carelessly. Whilst some seed inevitably falls upon unreceptive ground, a farmer never intentionally wastes seed. Similarly, spiritual treasure is too precious to be deliberately squandered on people who spurn its value.\footnote{Matthew 7:6} Investment must be made with people who recognise the worth of this treasure and who make room for its transformative power.

Such people are those whom Jesus, in the parable of the sower, refers to as \emph{good soil}:\footnote{Matthew 13:1--23} people willing to be transformed through a personal knowledge and experience of the Messiah, who will share their treasure with other faithful people {\ldots} who will share it with other faithful people and so on and so forth.

\pagebreak 

\section{Discipleship Movements}
\label{discipleshipmovements}

The formation of faithful disciples was at the heart of Jesus' life and work. Today the entire historical and now worldwide Messianic movement testifies to the significance of that small, core group of disciples, formed around Jesus.

From its first-century origins, as an obscure, tiny, Jewish sect, the Messianic Community has grown and developed across two millennia. Today it is an international, intercultural, multi-ethnic community that exists, in some form or another, in practically every nation of the world. 

On its journey, the Messianic Community and the message it has proclaimed has impacted innumerable peoples, stories and cultures, throughout the world. Its dynamism can be traced directly to its functioning faithfully as \emph{a generational discipling movement}, spreading across social, ethnic, linguistic, geographical and cultural boundaries.

\begin{pause}

\begin{description}

\item[The book of Acts provides a powerful illustration —]

of the dynamic growth of the earliest messianic discipling movement, as it spread across the ancient world. 

From its beginnings in Jerusalem, it expands rapidly throughout Israel, into Asia Minor, across Greece and finally to Rome, the seat of imperial power.

\emph{Witness how this growth took place, by examining the context of these verses, in your own Bibles:}
\end{description}

\begin{itemize}
\item Acts 2:46--47, 6:7, 9:31, 12:24, 16:5, 19:20, 28:30--31

\end{itemize}

\end{pause}

\subsection{Loss of generational momentum}
\label{lossofgenerationalmomentum}

History also teaches us that momentum, once gained, does not continue inevitably. Many messianic movements began well, but are today, sadly, only historical footnotes. Others continue institutionally, yet without any real sense of spiritual renewal, generational momentum or confidence in challenging cultural and social idols.

It may be surmised then, that maintaining the momentum of generational discipleship requires a persistent focus upon personal, social and cultural transformation. Vision must be imparted that is capable of gripping the hearts of others and forming them into faithful, active disciples:

\begin{itemize}
\item committed to significant personal and social renewal

\item persistent and determined to turn vision into reality

\item operating as co-workers—not selfishly ambitious individuals

\item active in forming other faithful disciples.

\end{itemize}

Only in this way can a visionary, discipling movement become authentically established.

\subsection{Renewal of movement}
\label{renewalofmovement}

Accordingly, any messianic movement or community that earnestly desires spiritual renewal needs to place visionary, generational discipleship at the heart of its spirituality and practical formation.

Disciples must be invited, formed and sent forth as part of a \emph{world-facing} movement. The goal is much more than the maintenance of congregational activity. It is more than learning to serve one another, within messianic communities. The goal is to be part of a movement of disciples responding to God's call to serve his eternal purpose, amongst a world of lost, hurting, confused, oppressed, fear-filled and idol-bound populations.

Amongst the corruption in society, behind its social, political, economic and religious walls and beside its filthy gutters, there the Messiah is at work, by his Spirit. From there he calls his co-workers to come alongside him in his work of redemption and transformation of individuals, families, marriages, partnerships, communities, organisations, structures, workplaces and working practices.

\section{Anointed Community}
\label{anointedcommunity}

To make possible such an otherwise impossibly-high calling, messianic discipleship provides a unique ingredient that no other philosophy, ideology or faith can provide: the dynamic of the indwelling Spirit of the Messiah. 

Through the Spirit, the covenant community is transformed into a \emph{charismatic community}—a group of people endowed with spiritual gifts profoundly shaped to liberate human beings from idolatry and the allegiances and falsehoods that compete against the knowledge of God.\footnote{2 Corinthians 10:3--5}

The Messianic Community is a body of people anointed with the fragrant presence of the Holy Spirit, having been brought under God's authority, through baptism into the Messiah. It is a body learning to walk in the footsteps of Jesus: learning to exercise its God-appointed mediatory, intercessory role, under the direction of the Spirit of God.\footnote{Romans 6:3--4; Galatians 3:26--29; Hebrews 6:4}

\begin{quote}

\textbf{This community of disciples is a messianic, charismatic people called into covenant relationship with the Father, through the Son and sent into the world to bless the nations in the power of the Spirit!}
\end{quote}

\begin{pause}

\begin{description}

\item[Charismatic —]

from the Greek, \emph{charism}, meaning \emph{gift}; the \emph{charisma} of the Messianic Community derives from its anointing with the Spirit of the Messiah

\item[Messianic —]

essentially means \emph{anointed to bring deliverance}, anointed with the Spirit to mediate on behalf of God's people and the nations
\end{description}

\end{pause}

\subsection{Dying to live}
\label{dyingtolive}

God's intention is that this messianic, charismatic, covenant community co-works in partnership with him—using the strength, the power, the spiritual life, the anointing that he provides. Yet too often, the power of the anointed-life-of-Christ-within seems to elude us. It seems out of our reach. Beyond our grasp. 

Indeed, it is not something that can be grasped. Rather, the pathway to life is through dying. Yielding ourselves to God the Father, through the Messiah, by the Spirit. That is the message of the cross. As we \emph{die to self}, we become \emph{alive to God}.\footnote{Romans 6:4--13}

\subsection{The heart of discipleship}
\label{theheartofdiscipleship}

Thus we end, as we began, with the foundational principle of transformative discipleship: seed sown into the ground, in order to produce a rich harvest.

\begin{quote}

I tell you that unless a grain of wheat that falls to the ground dies, it stays just a grain; but if it dies, it produces a big harvest—\emph{John 12:24}
\end{quote}

This life-giving spiritual reality represents the heart of Jesus' own life, mission, ministry and pain-filled death. And this same principle forms the foundation and wellspring of Maize Plant Discipleship:

\begin{quote}

\textbf{As we embrace a practical form of discipleship, incorporating a daily dying-to-self, we learn how to truly become alive-to-God and equipped to serve his eternal purpose}
\end{quote}

\emph{That is the heart of Maize Plant Discipleship}.

\begin{center}\rule{3in}{0.4pt}\end{center}


\begin {questions}

\begin{itemize}
\item How is discipleship valued in your context?

\item How faithfully is it practised?

\item If there is a gap between what is believed and valued and what is practised, why do you think that is?

\end{itemize}

\end {questions}

\chapter{Maize Plant Discipleship and Africa}
\label{maizeplantdiscipleshipandafrica}

THIS CHAPTER EXPLAINS the development and potential of Maize Plant Discipleship as a resource for use in African contexts.

\section{The Africa Factor}
\label{theafricafactor}

Tite Tienou, a Malian who grew up in Burkina Faso, encapsulates a significant aspect of the philosophy that has guided the development of Maize Plant Discipleship: African voices should determine the theology that is of practical relevance to African contexts. Tienou states:

\begin{quote}

Africanness and (theological) correctness should not be measured in either dissimilarity or similarity to the West. The way forward is to measure the Africanness of any theology purporting to be African by the degree to which it speaks to the needs of Africans in their total context. Quite naturally, the needs of African Christians should be taken seriously when determining these needs — \emph{Tite Tienou, The Uphill Road: Indigenous African Christian Theologies, 1990}
\end{quote}

Although Maize Plant Discipleship is authored by an outsider to Africa, it could and would not have developed without the critical input, as well as missional energy, encouragement, friendship and culture of African people. At each stage of its formation, it has been:

\begin{itemize}
\item authored exclusively in response to African leaders and learners

\item road-tested in collaboration with African leaders and learners

\end{itemize}

In this final stage of development, it has been formulated for publication as a series of low-cost, practical, relevant and accessible modular handbooks, in response to missiological doctoral research carried out amongst leaders and learners in Burkina Faso.

\begin {pause}

\begin{description}

\item[Romans 4:17 —]

Abraham is described as one whose faith \emph{called into being things that did not exist}. In essence, Maize Plant Discipleship has been \emph{called into being} by the faith of Africans, acting in response to African contexts.
\end{description}

\end {pause}

\section{Doctoral Research}
\label{doctoralresearch}

Between 2003 and 2010, multiple prototypes of Maize Plant Discipleship seminars were delivered to conferences in Léo and Ouagadougou. In 2010, the author conducted doctoral research using a series of survey questionnaires, focus groups and personal interviews. This enabled the gathering of a wide range of reflective opinions, perspectives, insights, information, questions and concerns relating to discipleship, theology, leadership training, methodologies, literature and intercultural dynamics. 

\textbf{Over 70 Burkinabé participants contributed to the research.} Drawn from over five organisational groupings, a significant number of participants held responsibility for local, regional, national and denominational leadership training and discipleship. Analysis of their responses revealed a number of crucial findings that have directly shaped the formulation of Maize Plant Discipleship, particularly in relation to: 

\begin{itemize}
\item Discipleship

\item Theology

\item Literature. 

\end{itemize}

\subsection{Discipleship}
\label{discipleship}

Research participants identified strongly with the concept of discipleship and, in particular, the need to freshly embrace holistic, generational discipleship practices. Accordingly, Maize Plant Discipleship focuses upon:

\begin{enumerate}
\item Awakening or strengthening contextual ownership of the call to serve God's eternal purpose

\item Promoting lifelong commitment to missional action and disciplines, including the generational formation of disciples

\item Envisioning personal, communal and cultural transformation.

\end{enumerate}

\subsection{Theology}
\label{theology}

Participants consistently validated the theological content of a prototype learning resource. Accordingly, Maize Plant Discipleship incorporates:

\begin{enumerate}
\item A holistic worldview, communal orientation, charismatic spirituality

\item A historical, covenantal, missionary interpretation of Scripture

\item Biblical theologies of discipleship, suffering and overcoming; spiritual revival, intercessory prayer and spiritual power; poverty and prosperity; personal and corporate vocation; Christ-centred servant-leadership and cultural transformation.

\end{enumerate}

\begin {pause}

\begin{description}

\item[One Burkinabé theological educator —]

examining a prototype Maize Plant Discipleship booklet, stated:
\end{description}

\begin{quote}

\emph{You are touching something that is not already existing. If we talk about evangelism, it may well be a new way of approaching evangelism, but we already have many methods of evangelism. But (a series of books focusing on) discipleship is {\ldots} really an innovative thing.}
\end{quote}

\end {pause}

\subsection{Literature}
\label{literature}

Research participants strongly expressed a desire for \emph{contextually appropriate literature}. In Burkina Faso, as in Africa and the Majority World generally, leaders and learners typically mediate between two cultural worlds of orality and literacy respectively. Accordingly Maize Plant Discipleship handbooks are: 

\begin{enumerate}
\item Formulated to cross boundaries of orality and literacy

\begin{itemize}
\item Focused on practical discipleship—not theoretical ideas

\item Devised for use amongst cohort groups

\item Incorporating reflective learning, group discussion and scripture memorisation

\item Containing numerous graphical and metaphorical illustrations.

\end{itemize}

\item Intended to be translatable into mother-tongue languages

\begin{itemize}
\item Thematic, modular structure of studies and topics

\item Avoids of academic, philosophical language

\item Encourages contextual adaption.

\end{itemize}

\item Published according to a missional philosophy

\begin{itemize}
\item Allows republication, translation and low cost sales

\item Honours copyright of author and translators

\item Releases tight, commercial control over republication.

\end{itemize}

\end{enumerate}

\begin {pause}

\begin{description}

\item[A 52 page handbook —]

printed double-sided on A4 paper, folded, stapled and trimmed to produce A5 booklets, of approximately 48 pages, with a single-colour printed cover, can be produced for the equivalent of approximately \$2.

\item[An agency organising republication —]

of handbooks on behalf of a number of other organisations may be the most effective way to achieve low costs. A creative economic model could then potentially allow the agency to subsidise handbook costs for the economically poorest.
\end{description}

\end {pause}

\section{Translation and Republication}
\label{translationandrepublication}

It is hoped that Maize Plant Discipleship will be widely translated, republished, trialled and adopted for use in numerous African and other majority world contexts. After you have explored and trialled Maize Plant Discipleship, we would welcome hearing from you. You can contact us via the Maize Plant Discipleship website, (www.maizeplantdiscipleship.info). Please tell us how and where you are using the resource and with what effect, including any improvements that you want to suggest.

\begin {pause}

\begin{description}

\item[Are you aware of stories, proverbs or cultural analogies —]

that could help to contextually illustrate and culturally illuminate teaching concepts contained in the Maize Plant Discipleship syllabus?

\emph{If you send them to us, we may incorporate them in future editions. Alternatively, under the terms of the Maize Plant Discipleship Licence, you could publish an adaption}.

\emph{The Creative Commons Licence, reproduced at the end of each handbook, permits republication and adaption, such as translation or inclusion of contextualised elements. Simply follow the terms of the licence appropriately}.
\end{description}

\end {pause}

\begin{center}\rule{3in}{0.4pt}\end{center}


\begin {questions}

\begin{itemize}
\item Is it important that African Christians decide upon what is missiologically appropriate to African contexts?

\item Has this happened historically in your contexts? 

\begin{itemize}
\item If not, why might that be?

\item If so, what has changed as a result?

\end{itemize}

\item What missiological issues are important to you and others in your context? 

\end{itemize}

\end {questions}

\pagebreak 

\section{The Maize Plant Discipleship Licence}
\label{themaizeplantdiscipleshiplicence}

Maize Plant Discipleship by John Clements is licensed under a \textbf{Creative Commons Attribution-ShareAlike 4.0 International License}. Based on a work at http:/\slash johnbrc.github.io\slash MPD-Distribution\slash . Permissions beyond the scope of this licence may be available at http:/\slash maizeplantdiscipleship.info\slash . 

\begin{quote}

\emph{Note: What follows is a simplified summary of (and not a substitute for) the licence, which may be accessed at: http:/\slash creativecommons.org\slash licenses\slash by-sa\slash 4.0\slash legalcode\slash  }
\end{quote}

\begin{summary}

\textbf{You are free to}:

\begin{itemize}
\item \textbf{Share} — copy and re-distribute the material in any medium or format

\item \textbf{Adapt} — remix, transform and build upon the material 

\end{itemize}

\begin{quote}

for any purpose, even commercially. The licensor cannot revoke these freedoms as long as you follow the licence terms.
\end{quote}

\textbf{Under the following terms}:

\begin{itemize}
\item \textbf{Attribution} — You must give appropriate credit, provide a link to the licence, and indicate if changes were made. You may do so in any reasonable manner, but not in any way that suggests the licensor endorses you or your use.

\item \textbf{ShareAlike} — If you remix, transform, or build upon the material, you must distribute your contributions under the same licence as the original.

\item \textbf{No additional restrictions} — You may not apply legal terms or technological measures that legally restrict others from doing anything the licence permits.

\end{itemize}

\end{summary}

\chapter*{Author}\label{author}

\textbf{Dr John B Clements} is a missiological educator, having received a Doctorate of Missiology (Contextual Missiology) from Fuller Theological Seminary School of Intercultural Studies, in 2012. 

\begin{figure}[htbp]
\centering
\includegraphics[width=133pt,height=100pt]{john-rhossilli-bw.jpg}
\caption{}
\label{john-rhossilli-bw.jpg}
\end{figure}

John is married to Sarah; they have three boys and a girl and presently live in a delightful corner of South West Wales, UK. John is an avid bird-watcher and casual photographer, pastimes that he combines with his enjoyment of countryside and coastal walking.

\begin{wsite}

\begin{description}

\item[Vita]

jbclements.wordpress.com

\item[Linked-In]

www.linkedin.com\slash in\slash jbclements

\item[Social]

about.me\slash jbclements
\end{description}

\end{wsite}

\begin{itemize}
\item Interested in translating Maize Plant Discipleship resources? 

\item Using Maize Plant Discipleship resources? \emph{Get in touch!}

\end{itemize}

\input{mpd-footer}

\end{document}
