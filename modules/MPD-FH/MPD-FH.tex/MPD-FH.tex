\def\version{0.3.0 --- Copyediting Draft}
\def\change{Finalised for copyediting
(Significant terms outstanding)}
\input{mpd-header}
\def\mytitle{Facilitators' Handbook}
\def\subtitle{Maize Plant Discipleship Handbook}
\def\myauthor{Dr John B Clements}
\def\twitter{@johnbrc, @mpdresource}
\def\email{clements.jb@gmail.com}
\def\web{http:/\slash maizeplantdiscipleship.wordpress.com}
\def\mycopyright{John B Clements, 2014.  Maize Plant Discipleship by John B Clements is licensed under a Creative Commons Attribution-NonCommercial-ShareAlike 4.0 International License. Based on a work at http:/\slash johnbrc.github.io\slash MPD-Distribution}
\def\keywords{discipleship, mission, messianic, community}
\def\latexmode{memoir}
\input{mpd-document-facilitator}

\chapter{Facilitating Maize Plant Discipleship Learning Groups}
\label{facilitatingmaizeplantdiscipleshiplearninggroups}

\section{What is Maize Plant Discipleship?}
\label{whatismaizeplantdiscipleship}

Maize Plant Discipleship is a learning resource, derived and road-tested in collaboration with Africans and formulated in response to contextual doctoral research in Burkina Faso, to be practical, relevant and accessible for use in African and other post-colonial contexts.

It is being published as a series of short, modular, low-cost handbooks:

\begin{itemize}
\item suitable for formal and informal modes of study,

\item incorporating reflective learning and group discussions,

\item reliant simply upon facilitators co-ordinating small learning groups,

\item easily replicable, in terms of both republication and training.

\end{itemize}

It's goal is to facilitate biblical learning that continuously moves outwards, drawing whole communities into patterns of scripturally-based discipleship, in living dialogue with contextual culture.

\section{Maize Plant Discipleship Learning Groups}
\label{maizeplantdiscipleshiplearninggroups}

Maize Plant Discipleship is intended to an open, reflective, group learning process, in which leaders and learners alike participate together in discovering what the Spirit is saying, as Scripture is studied and related to contextual \emph{``signs of the times.''}

\subsection{Why Learning Groups?}
\label{whylearninggroups}

There are many reasons to bring together a group of people to learn together. Some people would point to Jesus' example and his gathering of twelve disciples. Certainly, for most people, groups represent a natural and lively place in which to learn. They bring together people with different experience, gifts, capacities and perspectives. 

When we share our lives, we learn together and \emph{learning groups} mirror this reality. Together, we experience and learn quite differently to when studying alone. Reflective discussion with others, in particular, provides a highly stimulating forum for learning. \emph{For further discussion, see Maize Plant Discipleship Learning Process (\autoref{maizeplantdiscipleshiplearningprocess})}.

\subsection{Learning, not teaching?}
\label{learningnotteaching}

Yes. Maize Plant Discipleship is principally a learning, not a teaching resource. Facilitating Maize Plant Discipleship provides an \emph{opportunity} for learning, but the decision to actually learn rests with the student and depends upon many things, including desire, temperament, experience, opportunity, talent, time, energy, environment and so forth.

Accordingly, discipleship should be recognised as a form of learning that is \emph{called out} of students or disciples, through the guidance and direction of a facilitator, mentor, educator or teacher. People in those roles come alongside motivated learners to assist, encourage, facilitate and call out the learning that is taking place within those being discipled.

\section{Understanding the Facilitator Role}
\label{understandingthefacilitatorrole}

A Maize Plant Discipleship Facilitator can help a group open up to the message of Scripture, to one another and, above all, to the leadership of the Holy Spirit. This section explores how.

\subsection{Facilitating Spirit-led discipleship}
\label{facilitatingspirit-leddiscipleship}

Discipleship can never be reduced to an replication of information, from teacher or text books into students. We may have been taught this way in school; Spirit-led discipleship is different. It \emph{forms}, as well as informs.

The intention is that through teaching, reflection and discussion, God's Spirit is able to speak to, lead, warn, direct, encourage, educate, challenge and exhort us personally and corporately. Since each person differs in gift, personality and development, at any time, each person may be learning something different from the Spirit. 

\begin {pause}

\begin{description}

\item[The goal of discipleship]

is not to establish shared dogmatic belief or conformity to the convictions of a leader, mentor or facilitator, nor to church traditions, certainly not to every aspect of Maize Plant Discipleship.

\item[The goal is]

conformity to the Spirit of the Messiah, Jesus, and obedience to the will of the Father.
\end{description}

\end {pause}

\subsection{Facilitating openness}
\label{facilitatingopenness}

Fostering an environment of learning and discovery, where debate and discussion is lively, yet relaxed and uncompetitive is essential. An ideal environment will allow strong and diverse views to be expressed, without creating conflict or conformity, so all present are comfortable to contribute views, burdens and questions. 

Openness can be particularly challenging to foster in cultures where authority traditionally flows downwards and conformity is highly valued. Thus, facilitators should typically contribute to discussions as regular group members, never dominating or belittling the views of others.

\begin{itemize}
\item Allow discussion to ebb and flow, as people consider their responses and return to earlier discussions. Encourage others to contribute, especially quieter members, women, youth and elders.

\item If discussion becomes harsh or factious, quieten the group, then invite a member with a harmonious or gentle spirit to summarise (not resolve) the tension, then move on.

\end{itemize}

\subsection{Who can facilitate?}
\label{whocanfacilitate}

A facilitator needs to be someone who senses a calling to help others become faithful Christian disciples. This must motivate them to be humble, patient, flexible, warm, open and secure enough to allow others to explore personal boundaries of vocational understanding, experience and creativity, at their own pace.

A facilitator does not hold a position of authority over people. They simply facilitate the gathering of people into groups, for learning and discussion. Accordingly, a facilitator:

\begin{itemize}
\item may be a lay-leader;

\item may be relatively young;

\item may be a woman;

\item need not have gone to bible college;

\item need not be an established church leader;

\item need not be an experienced mentor.

\end{itemize}

Of course, Maize Plant Discipleship can be facilitated by established leaders, mentors or disciplers, providing they are willing and comfortable to facilitate group discussions that are genuinely \emph{reflective and exploratory}.

\section{Maize Plant Discipleship Learning Process}
\label{maizeplantdiscipleshiplearningprocess}

Maize Plant Discipleship modules are structured to provide a \emph{reflective learning process} incorporating practical, educational principles combined with practical, spiritual principles, drawn from Scripture. Reflective, group learning minimally incorporates four components:

\begin{itemize}
\item \textbf{HEAR}—hearing about other's experiences and perspectives; practice informing theory and theory informing practice. 

\item \textbf{REFLECT}—reflecting upon the interaction of old and new ideas and understanding. 

\item \textbf{DISCUSS}—testing understanding and sharpening conviction in conversation with others.

\item \textbf{ACT}—integrating learning into praxis, within vocational contexts.

\end{itemize}

As this learning process is repeated, it becomes a \emph{cycle}, which can be illustrated figuratively.

\begin{figure}[htbp]
\centering
\includegraphics[width=208pt,height=200pt]{learning-cycle.png}
\caption{}
\label{learning-cycle.png}
\end{figure}

The \emph{Maize Plant Discipleship Learning Process} builds upon this basic cycle, by incorporating a simple pattern, drawn from Acts 2:42:

\begin{quote}

They continued faithfully in the teaching of the apostles, in fellowship, in breaking bread and in prayer.
\end{quote}

This provides The \emph{Maize Plant Discipleship Learning Process} with \emph{six} elements (in practice, it may not be possible to incorporate all the elements, each time a learning group meets):

\begin{enumerate}
\item \textbf{Hear} (listen, heed)

\item \textbf{Retain} (meditate, reflect)

\item \textbf{Open} (discuss, debate)

\item \textbf{Share} (celebrate, proclaim)

\item \textbf{Pray} (intercede, bless)

\item \textbf{Act} (work, serve).

\end{enumerate}

\pagebreak 

\subsection{HEAR{\ldots}what the Spirit is saying}
\label{hearwhatthespiritissaying}

\begin{figure}[htbp]
\centering
\includegraphics[width=119pt,height=120pt]{hear.png}
\caption{}
\label{hear.png}
\end{figure}

When we gather together as disciples of the Messiah, to hear biblical teaching, we are opening ourselves not only to human ideas or wisdom, but to spiritual words and truths, taught by the Spirit of God.

\begin{quote}

Now we have not received spirit that belongs to the world, But the Holy Spirit Who is from God, given to us that we might realise and comprehend and appreciate the gifts of divine favour blessing so freely and lavishly bestowed on us by God. And we're setting these truths forth in words not taught by human wisdom but taught by the Holy Spirit, combining and interpreting spiritual truths with spiritual language to those who possess the Holy Spirit—\emph{1 Corinthians 2:12--13 TAB}
\end{quote}

\begin {pause}

\begin{description}

\item[We listen in order to live more faithfully]

This type of listening is called \emph{heeding}: listening with the intention to learn and follow, or obey.

\item[We listen with our mind, but also with our heart]

in order to \emph{hear what the Spirit is saying to his people} (Revelation 2:29, 3:6,13,23; Matthew 11:15, Mark 4:9 etc)—not to become \emph{puffed up} with knowledge.
\end{description}

\end {pause}

\pagebreak 

\subsection{RETAIN{\ldots}God's message inwardly}
\label{retaingodsmessageinwardly}

\begin{figure}[htbp]
\centering
\includegraphics[width=120pt,height=123pt]{receive.png}
\caption{}
\label{receive.png}
\end{figure}

It is not enough only to hear God's message: we must learn to \emph{retain} God's word inwardly, where it can begin to \emph{dwell richly within us.} (Colossians 3:16)

\begin{quote}

The one who received the seed that fell on rocky places is the man who hears the word and at once receives it with joy. But since he has no root, he lasts only a short time{\ldots} The one who received the seed that fell among the thorns is the man who hears the word, but the worries of this life and the deceitfulness of wealth choke it, making it unfruitful. The seed on good soil stands for those with a noble and good heart, who hear the word and \emph{retain} it, and by persevering produce a crop{\ldots} yielding a hundred, sixty or thirty times what was sown — \emph{Matthew 13:18--23; Luke 8:15}
\end{quote}

\begin {pause}

\begin{description}

\item[Think about how we receive and retain food]

chewing it, enjoying the taste, swallowing, digesting, inwardly retaining its vitality and goodness.

\item[It's the same with God's word]

we must ``chew it over,'' meditating and reflecting upon its meaning and application to our lives, both as individuals and as communities, allowing it to settle in our spirit, where it can form and shape our convictions and renew our hope.
\end{description}

\end {pause}

\pagebreak 

\subsection{OPEN{\ldots}hearts to other's perspectives}
\label{openheartstoothersperspectives}

\begin{figure}[htbp]
\centering
\includegraphics[width=119pt,height=121pt]{open.png}
\caption{}
\label{open.png}
\end{figure}

Discussion and debate is an opportunity to open our hearts to the perspectives and experiences of those around us and those who see things differently to ourselves. 

\begin{itemize}
\item This requires learning to listen with the heart, as well as the head, in order to appreciate what others are sharing, rather than to win an argument. 

\item Discuss practical, \emph{vocational} applications of the topical study. Think about how Maize Plant Discipleship teachings relate to the cultural contexts amongst which group members live.

\item Allow plenty of time for this aspect of Maize Plant Discipleship learning!

\end{itemize}

\begin {pause}

\begin{description}

\item[Vocation is more than simply our job, or employment]

incorporating all the responsibilities towards which God calls us, including workplaces and practices, families, communities and networks.

\item[Consider traditional proverbs that relate to study topics]

the \emph{sweet talk} of proverbs can provide fresh insight and be helpful in discussing Maize Plant Discipleship with others, including elders and non-believers.
\end{description}

\end {pause}

\pagebreak 

\subsection{SHARE{\ldots}our daily bread}
\label{shareourdailybread}

\begin{figure}[htbp]
\centering
\includegraphics[width=119pt,height=120pt]{share.png}
\caption{}
\label{share.png}
\end{figure}

The celebratory breaking of bread, in order to remember the Lord Jesus and his sacrificial obedience, is a significant symbol of new covenant and a profound way for discipleship groups to proclaim their shared devotion to the Messiah.

\begin{itemize}
\item Breaking and sharing bread is typically ceremonial, in modern forms of Christianity \emph{(Eucharist, Holy Communion, Mass)}. The early church, however, based it simply upon the Passover meal, like the one Jesus shared with his disciples, prior to his death.

\item Sharing food together is therefore both a vital part of human fellowship and a practical way of celebrating and proclaiming God's covenantal provision and blessing.

\end{itemize}

\begin {pause}

\begin{description}

\item[Consider incorporating a simple meal]

perhaps once each month, into times of meeting together and prayerfully identifying it as a form of breaking bread.

\item[If a meal is not a practical possibility]

consider sharing a small amount of bread together, as a symbolic act of shared hospitality and commitment to membership of the body of the Messiah.
\end{description}

\end {pause}

\pagebreak 

\subsection{PRAY{\ldots}for God's kingdom to come}
\label{prayforgodskingdomtocome}

\begin{figure}[htbp]
\centering
\includegraphics[width=119pt,height=121pt]{pray.png}
\caption{}
\label{pray.png}
\end{figure}

After discussion has taken place, invite the group to pray together, including intercession on behalf of neighbours, networks and communities and local and national rulers and governors. 

\begin{itemize}
\item Allow the teaching to infuse prayer with fresh confidence concerning God's will and purpose, including personal and vocational concerns and challenges faced by group members. 

\item Bless one another, with blessings from Scripture, or as led by the Holy Spirit; speak a blessing over your community or nation, or with regards to a specific problem.

\item Expect the power of God, to overcome all opposition, through the blessing of his overcoming life in us!

\end{itemize}

\begin {pause}

\begin{description}

\item[Pray for the gospel]

to deeply impact and transform individuals, communities, cultures and societies throughout your nation, Africa, Europe, Asia and the Americas; pray for unreached people groups.

\item[Pray for the Maize Plant Discipleship Project]

that it will be used by God to edify and strengthen and bless the Messianic Community, within Africa {\ldots} and beyond!
\end{description}

\end {pause}

\pagebreak 

\subsection{ACT{\ldots}in light of God's message}
\label{actinlightofgodsmessage}

\begin{figure}[htbp]
\centering
\includegraphics[width=119pt,height=120pt]{act.png}
\caption{}
\label{act.png}
\end{figure}

The purpose of our gathering to hear God's message is not simply to hear it, but to act upon it. As the epistle of Ya'akov (James) explains, we deceive ourselves when we listen to God's word, but do not do what it says:

\begin{quote}

Don't deceive yourselves by only hearing what the Word says, but do it! For whoever hears the Word but doesn't do what it says is like someone who looks at his face in a mirror, who looks as himself, goes away and immediately forgets what he looks like. But if a person looks closely into the perfect \emph{Torah}, which gives freedom, and continues, becoming not a forgetful hearer but a doer of the work it requires, then he will be blessed in what he does — \emph{James 1:22--25}
\end{quote}

\begin {pause}

\begin{description}

\item[The goal of discipleship]

is to be transformed ourselves and to become a transforming influence in our homes and workplaces and communities.

\item[As we are transformed]

as part of a growing, dynamic movement of disciples, we begin to fulfil our corporate vocation: to be \emph{a messianic community blessed to be a blessing to the families of the earth!}
\end{description}

\end {pause}

\pagebreak 

\subsection{Maize Plant Discipleship learning cycle}
\label{maizeplantdiscipleshiplearningcycle}

Combining the six elements of our learning process together produces the Maize Plant Discipleship learning cycle.

\begin{figure}[htbp]
\centering
\includegraphics[width=271pt,height=299pt]{mpd-learning-cycle.png}
\caption{}
\label{mpd-learning-cycle.png}
\end{figure}

\begin {pause}

\begin{description}

\item[The Maize Plant Discipleship learning cycle is a tool]

like everything else in Maize Plant Discipleship, intended to serve facilitators and learning groups. Allow it to stretch, but not to limit your learning and adapt it, where appropriate.
\end{description}

\end {pause}

\section{Practical Considerations}
\label{practicalconsiderations}

Facilitating a discipleship group will be most effective when practicalities are considered in advance and appropriate planning takes place. 

\subsection{Getting started}
\label{gettingstarted}

Maize Plant Discipleship is ideal for learning groups of 8--10 people. This is small enough to allow group members to grow together with a degree of intimacy and large enough for group members to explore discipleship commitments at their own pace.

\begin {pause}

\emph{More than ten?} How could you help others to facilitate additional learning groups? What problems might you face? 

\end {pause}

\subsection{Involving others}
\label{involvingothers}

Although a facilitator is responsible for convening gatherings, they may delegate responsibility for hosting, presenting the teaching, or moderating group discussions. Ideally, over time, all group members will carry some responsibility, according to their gift and capability. This avoids one person carrying too much and gives everyone some experience of the facilitating role.

Whoever is presenting a topical study should read through it carefully, in advance: absorbing, familiarising and reflecting upon the teaching and its lessons. If anything is unfamiliar or unclear, invite discussion about that area of the study, encouraging group members to bring forward their perspectives.

\subsection{Replication}
\label{replication}

Replication is an important goal of Maize Plant Discipleship. see the Maize Plant Philosophy of Discipleship (\autoref{maizeplantphilosophyofdiscipleship}) for encouragement about its importance. Note: it may not be necessary for a group member to complete the entire syllabus before branching out to facilitate another learning group. Be led by the Spirit.

\subsection{Location}
\label{location}

Meeting together can take place in any appropriate location that can comfortably accommodate a group. For example, a large room in someone's home, or a communal building, such as a church.

\begin {pause}

\begin{description}

\item[Arrange seating]

to create an intimate, yet practical space, in which everyone can see everyone else during discussions.

\item[Find out what works for your group]

by experimenting with changes of location; try meeting outside occasionally.
\end{description}

\end {pause}

\subsection{Adaption}
\label{adaption}

Be prepared to adapt the teaching and the method of presentation, in order to create a helpful and culturally-appropriate learning environment. Take account of the abilities and capacity of each particular discipleship group. 

\begin {pause}

\begin{description}

\item[Ensure literature is a helpful \emph{servant}]

not a hard \emph{task-master}, especially to oral learners. As far as possible, keep things simple and lighthearted.

\item[Invite creative people to contribute]

by interpreting or celebrating the message of the teaching in song, drama or art.
\end{description}

\end {pause}

\subsection{Timetable}
\label{timetable}

The entire Maize Plant Discipleship syllabus incorporates approximately 64 studies. When considering a timetable take into consideration the nature of the group availability. \emph{For example, are members effected by the demands of agricultural seasons or academic terms?}

\subsection{Other learning forums}
\label{otherlearningforums}

Maize Plant Discipleship handbooks can easily lend themselves to personal study, theological education or other forms of guided learning. 

In particular, 

\begin{itemize}
\item Scripture references, within in the footnotes, provide a rich treasury of addition material for in-depth biblical study;

\item discussion questions can be adapted for use as the basis for written answers, or even short essays. 

\end{itemize}

\subsubsection{Personal study}
\label{personalstudy}

Students using Maize Plant Discipleship handbooks for personal study should seek to incorporate reflective learning methods by either submitting the fruit of their study to the oversight of a mentor or sharing it with a fellow student, for reflection, discussion and critical consideration.

\subsubsection{Theological education}
\label{theologicaleducation}

Bible school students should be encouraged to form and facilitate small learning groups, either within or alongside classroom contexts, and to reflect on their experiences together. This will provide highly valuable experience and momentum for facilitating Maize Plant Discipleship learning groups in their own vocational contexts.

\subsubsection{Congregational teaching}
\label{congregationalteaching}

With appropriate adaption, Maize Plant Discipleship studies can be used for congregational teaching. For example, if teaching is delivered to a whole congregation, it could afterwards divide into small discussion groups. Alternatively, small learning groups could meet on a separate occasion, to reflect upon and discuss the teaching and to pray together.

\begin {pause}

\begin{description}

\item[In what other ways]

could aspects of the Maize Plant Discipleship Learning Cycle be adapted and incorporated into personal study, theological education and even congregational contexts?
\end{description}

\end {pause}

\chapter{The Maize Plant Discipleship Syllabus}
\label{themaizeplantdiscipleshipsyllabus}

THIS CHAPTER INTRODUCES the metaphor of the maize plant and the sixteen modules of the Maize Plant Discipleship Syllabus.

\section{Maize Plant Metaphor}
\label{maizeplantmetaphor}

Jesus refers to his own mission using the metaphor of a seed that enters the ground and dies, in order to produce a large harvest. 

\begin{quote}

I tell you that unless a grain of wheat that falls to the ground dies, it stays just a grain; but if it dies, it produces a big harvest — \emph{Jesus, John 12:24}
\end{quote}

Grown throughout sub-Saharan Africa, the maize plant provides a similar, highly recognisable metaphor. The Maize Plant Discipleship Syllabus is structured to symbolically reflect this metaphor of a maize plant:

\begin{itemize}
\item Good seed sown in good soil, stimulated by sunshine and refreshed by rainfall produces dynamic growth and a good harvest. 

\item Likewise, Messianic communities need to be rooted in good ground that allows them to draw on vital, spiritual nutrients, stimulated by the revelatory light and refreshed by the living water of the Messiah's Spirit. 

\item Strong growth anchors messianic communities against withering winds of false teaching and sustains them amidst the withering heat of trials, temptations and vocational responsibility.

\end{itemize}

\begin{figure}[htbp]
\centering
\includegraphics[width=168pt,height=275pt]{mp-metaphor.png}
\caption{}
\label{mp-metaphor.png}
\end{figure}

Drawing on this metaphor, the Maize Plant Discipleship Syllabus is divided into three parts:

\begin{enumerate}
\item Soil and roots

\item Maize plant

\item Sunlight and rainfall

\end{enumerate}

\subsection{Soil and roots}
\label{soilandroots}

Metaphorically, the roots of the maize plant represent the biblical community of Israel. The soil in which the roots grow equates to the historical, cultural and geo-political contexts of Israel's covenant vocation (such as Egypt, Canaan, Babylon and the Roman occupation).

\begin{itemize}
\item \textbf{Module 1, The Eternal Purpose of God}

A panoramic overview of Scripture, revealing God's unchanging, eternal purpose, in the scheme of which the Messiah is \emph{the Seed} that enters the soil of the biblical, covenantal history of the people of Israel, in order to die and ultimately produce a rich harvest of people—a Messianic Covenant Community—from amongst all the peoples of the earth. Module 1, thus forms the foundation of the entire syllabus.

\end{itemize}

\begin{figure}[htbp]
\centering
\includegraphics[width=250pt,height=154pt]{mp-roots.png}
\caption{}
\label{mp-roots.png}
\end{figure}

\begin {pause}

Maize produce effectively represents \emph{daily bread} for millions of African people. Like the Messiah, the Messianic Community is called to become a kind of \emph{life-giving bread} to the peoples of the world—see John 6 \& 20:21.

\end {pause}

\pagebreak 

\subsection{Maize plant}
\label{maizeplant}

\textbf{Modules 2 to 9} explore eight messianic \emph{dynamics},\footnote{\emph{Dynamic} derives from a Greek word, \emph{dunamis}, meaning power and refers to forces stimulating change or progress within a system or process.} which are responsible for the characteristic development, growth, structure, shape and fruit of messianic community. 

\begin{figure}[htbp]
\centering
\includegraphics[width=221pt,height=375pt]{mp-dynamics.png}
\caption{}
\label{mp-dynamics.png}
\end{figure}

\begin{itemize}
\item \textbf{Module 2, Dynamics of Vocation, The Nations}

The historical development the Messianic Community's vocational mission to bless the peoples of the world.

\item \textbf{Module 3, Dynamics of Vocation, The Jews}

The special responsibility of the Messianic Community towards the Jewish people.

\item \textbf{Module 4, Dynamics of Commissioning}

Strategic and structural dynamics of messianic commissioning and community growth.

\item \textbf{Module 5, Dynamics of Body Membership}

Membership, commitment and spiritual maturity within the body of the Messiah.

\item \textbf{Module 6, Dynamics of Revival}

The dynamics of revival and a spiritual harvest of covenant faithfulness.

\item \textbf{Module 7, Dynamics of Truth}

Encountering truth, as we walk in practical, covenantal faithfulness towards God's revelation, wisdom and direction.

\item \textbf{Module 8, Dynamics of Intercession}

The priestly vocation of the Messianic Community: to be \emph{a house of prayer for all nations}.

\item \textbf{Module 9, Dynamics of Cultural Transformation}

The call to work amongst and on behalf of the nations, towards cultural transformations that signal the presence of God's kingdom.

\end{itemize}

\pagebreak 

\subsection{Sunlight and rainfall}
\label{sunlightandrainfall}

\textbf{Modules 10 to 16} examine seven characteristic \emph{disciplines} that enable messianic communities to receive the revelatory \emph{light} and sustaining \emph{living water} of God's Spirit, without which we become spiritually weak and incapable of producing good fruit or a plentiful harvest.

\begin{figure}[htbp]
\centering
\includegraphics[width=291pt,height=399pt]{mp-disciplines.png}
\caption{}
\label{mp-disciplines.png}
\end{figure}

\begin{itemize}
\item \textbf{Module 10, Disciplines of Spiritual Maturity copy}

Three stages of encounter and growth in spiritual maturity of messianic disciples and communities.

\item \textbf{Module 11, Disciplines of Running the Race}

Motivations, qualities and disciplines for living an enduring life of service.

\item \textbf{Module 12, Disciplines of Pressing Toward Our Vocation}

Identifying and excelling in our personal vocation, through a deepening of our relationship with the Messiah.

\item \textbf{Module 13, Disciplines of Economic Faithfulness}

A biblical perspective upon economic faithfulness, wealth and poverty, radically different to that of the world. 

\item \textbf{Module 14, Disciplines of Messianic Leadership}

Qualifications, motivations and characteristics of faithful messianic leadership.

\item \textbf{Module 15, Disciplines of Living by Faith}

Seeing with eyes of faith enables us to endure times of testing and purification and to embrace challenge as an opportunity to experience God's faithfulness.

\item \textbf{Module 16, Disciplines of Overcoming}

Confronting idolatrous, cultural strongholds in the power of the Spirit and discerning strategies that make room for a transformative encounter with God's overcoming power.

\end{itemize}

\pagebreak 

\section{Module handbooks}
\label{modulehandbooks}

Each syllabus module is incorporated into a Maize Plant Discipleship \emph{Handbook}, containing four interrelated \emph{Studies}, each of which incorporates:

\begin{itemize}
\item Scripture readings

\item Topical sections and Summary

\item Discussion questions

\end{itemize}

Illustrations are provided throughout the Maize Plant Discipleship handbooks. Where possible, facilitators should reproduce diagrams on blackboards or in other ways, in order to allow all group members to view them and appreciate their relevance.

\section{Scripture versions}
\label{scriptureversions}

Throughout the handbooks, Scripture references are typically provided, rather than quotations. Where a quotation is incorporated within the teaching, the following abbreviations indicate the version from which the quote is taken:

\begin{itemize}
\item CJB — The Complete Jewish Bible

\item NIV — The New International Version

\item TAB — The Amplified Bible

\end{itemize}

\begin {pause}

\begin{description}

\item[Mother-tongue is the natural language of the heart]

thus, mother-tongue Scripture translations may be encouraged and utilised as much as possible, including during group discussions and particularly for reading and memorisation.
\end{description}

\end {pause}

\section{Significant terms}
\label{significantterms}

The following terms are used throughout the The Maize Plant Discipleship Syllabus (\autoref{themaizeplantdiscipleshipsyllabus}). 

\subsection{Messiah}
\label{messiah}

\begin{itemize}
\item The promised Jewish deliverer{\ldots}

\item Jesus, the Christ{\ldots}the Messiah

\end{itemize}

\subsection{Messianic}
\label{messianic}

\begin{itemize}
\item Messianic
Cf. ``Christian''

\item Messianic Community

\item Messianic, New-Covenant Community

\item messianic community or communities

\end{itemize}

Explain principles governing capitalisation.

\subsection{Vocation}
\label{vocation}

- 

\chapter{Africa and Maize Plant Discipleship}
\label{africaandmaizeplantdiscipleship}

THIS CHAPTER EXPLAINS the development and potential of Maize Plant Discipleship as a resource for use in African contexts.

\section{Books and African contexts}
\label{booksandafricancontexts}

Over the past three-to-four decades, the African Christian community has multiplied and expanded hugely, such that the global Christian community's statistical centre of gravity is now located in Africa. Yet, while this extraordinary numerical growth has taken place, there has been a relative shortage of book publishing within Africa. Resources serving to edify, disciple and inspire Christian communities have been particularly sparse within Francophone Africa.

Books imported into Africa for theological education have typically been written for Christians in western cultural contexts and translated only as an afterthought. As a result, they are concerned with quite different questions, cultural values and missiological convictions to those concerning African communities. In short, many textbooks utilised in African leadership training are contextually inappropriate.

Added to this, books are generally published and distributed in a manner designed to protect the commercial interests of authors, publishers, distributers and retailers. Thus, many books are practically unattainable to those who are economically poor—including many leaders hungry for resources to enrich their vocational work of forming Christian disciples.

\subsection{The Africa factor}
\label{theafricafactor}

Tite Tienou, a Malian who grew up in Burkina Faso, encapsulates, below, a highly significant aspect of the philosophy that has driven the development of Maize Plant Discipleship: African voices need to determine the theology that is of practical relevance to African contexts. 

\begin{quote}

\emph{Africanness and (theological) correctness should not be measured in either dissimilarity or similarity to the West. The way forward is to measure the Africanness of any theology purporting to be African by the degree to which it speaks to the needs of African's in their total context. Quite naturally, the needs of African Christians should be taken seriously when determining these needs}, Tite Tienou, The Uphill Road: Indigenous African Christian Theologies, 1990
\end{quote}

Although Maize Plant Discipleship is authored by an outsider to Africa, it could and would not have developed without the missional energy, encouragement, friendship, culture and input of African people, at each stage of its formation.

\begin{itemize}
\item Authored exclusively in response to African leaders and learners.

\item Road-tested in collaboration with African leaders and learners.

\item Formulated in response to contextual \emph{doctoral research} carried out in Burkina Faso.

\end{itemize}

\begin {pause}

\begin{description}

\item[Romans 4:17]

describes Abraham as a man whose faith \emph{called into being things that did not exist}. In essence, Maize Plant Discipleship has been \emph{called into being} by the trusting faithfulness of Africans, acting in response to the needs of African contexts.
\end{description}

\end {pause}

\section{Doctoral research}
\label{doctoralresearch}

Between 2009--13, doctoral research was carried out,\footnote{http:/\slash maizeplantdiscipleship.info\slash research} incorporating a qualitative consultation of Burkinabé leaders and learners, using survey questionnaires, focus groups and personal interviews, which took place \emph{following} prototypical Maize Plant Discipleship seminars delivered to conferences in Léo and Ouagadougou, which:

\begin{itemize}
\item enabled the gathering of a wide range of reflective opinions, perspectives, information, insights, questions and concerns, relating to discipleship, theology, leadership training, methodologies, literature and intercultural dynamics;

\item revealed a number of significant findings, particularly in relation to \emph{discipleship, theology} and training \emph{literature}, which have directly shaped the formulation of Maize Plant Discipleship.

\end{itemize}

\subsection{Discipleship}
\label{discipleship}

Research participants identified strongly with the concept of discipleship and a need to freshly embrace holistic, disciple-forming practices. Accordingly, Maize Plant Discipleship focuses upon

\begin{enumerate}
\item Awakening or strengthening contextual ownership of the call to serve God's eternal purpose, amongst their generation.

\item Promoting lifelong commitment to missional action and disciplines.

\item Emphasising the generational formation of disciples.

\item Envisioning personal, communal, cultural transformation.

\end{enumerate}

\subsection{Theology}
\label{theology}

Participants consistently validated the missiological and theological content of a prototypical Maize Plant Discipleship resource. Accordingly, Maize Plant Discipleship incorporates:

\begin{enumerate}
\item A holistic worldview; communal orientation; charismatic spirituality.

\item A historical, covenantal, missionary interpretation of Scripture.

\item A biblical theology of discipleship, suffering and overcoming, spiritual revival, intercessory prayer and spiritual power, poverty and prosperity, personal and corporate vocation, Christ-centred servant-leadership and cultural transformation.

\end{enumerate}

\subsection{Literature}
\label{literature}

Participants strongly expressed a desire for \emph{appropriate literature}. In Burkina, as in Africa and the Majority World generally, leaders and learners typically \emph{mediate} between two cultural worlds of orality and literacy respectively. Accordingly Maize Plant Discipleship handbooks are: 

\begin{enumerate}
\item Formulated to cross boundaries of orality and literacy.

\begin{itemize}
\item Focussed on practical discipleship, not theoretical ideas.

\item Studies worked through in relationship with co-learners.

\item Incorporating reflective learning and group discussion.

\item Including scripture memorisation.

\item Containing numerous graphical and metaphorical illustrations.

\end{itemize}

\item Translatable into mother-tongue languages.\footnote{To discuss translating Maize Plant Discipleship resources, for printing and distribution in another language, visit http:/\slash maizeplantdiscipleship.info}

\begin{itemize}
\item Thematic, modular structure of studies and topics.

\item Absence of academic, philosophical language.

\item Encouraging contextual adaption.

\end{itemize}

\item Published according to a missional philosophy.

\begin{itemize}
\item Licensed for reproduction under the terms of a \emph{Creative Commons} licence, allowing Maize Plant Discipleship handbooks to be republished, translated, distributed and sold, at a low cost.

\item The Licence maintains the copyright of the author and translators, yet releases the traditionally tight, commercial control of copyright owners and publishers, by allowing sharing and adapting of the resource, providing licensing terms are respected. \emph{The licence is reproduced on the ensuing pages, in human-readable form, with a link to the full licence.}

\end{itemize}

\end{enumerate}

\pagebreak 

\section{Licence}
\label{licence}

Maize Plant Discipleship by John Clements is licensed under a Creative Commons Attribution-ShareAlike 4.0 International License. Based on a work at http:/\slash maizeplantdiscipleship.info. Permissions beyond the scope of this license may be available at http:/\slash maizeplantdiscipleship.info\slash contact. 

\begin{quote}

\textbf{Note}: What follows is a human-readable summary of (and not a substitute for) the license, which may be accessed at: http:/\slash creativecommons.org\slash licenses\slash by-sa\slash 4.0\slash legalcode
\end{quote}

\begin{summary}

\textbf{You are free to}:

\begin{itemize}
\item \textbf{Share} — copy and re-distribute the material in any medium or format

\item \textbf{Adapt} — remix, transform and build upon the material 

\end{itemize}

\begin{quote}

for any purpose, even commercially. The licensor cannot revoke these freedoms as long as you follow the license terms.
\end{quote}

\textbf{Under the following terms}:

\begin{itemize}
\item \textbf{Attribution} — You must give appropriate credit, provide a link to the license, and indicate if changes were made. You may do so in any reasonable manner, but not in any way that suggests the licensor endorses you or your use.

\item \textbf{ShareAlike} — If you remix, transform, or build upon the material, you must distribute your contributions under the same license as the original.

\item \textbf{No additional restrictions} — You may not apply legal terms or technological measures that legally restrict others from doing anything the license permits.

\end{itemize}

\end{summary}

\begin {pause}

\begin{description}

\item[Over 70 Burkinabé participants]

contributed data to the research, a significant proportion of whom held local, regional, national and denominational responsibilities for leadership training and the development of discipleship praxis.

This included members of \emph{Assemblée Evangélique de Pentecôte}, \emph{Mouvement des Jeunes Serviteurs de Dieu}, \emph{Association Nationale pour la Traduction de la Bible et de l'Alphabetisation}, \emph{Assemblée de Deus} and \emph{SIL}.
\end{description}

\begin{center}\rule{3in}{0.4pt}\end{center}


One Burkinabé theological educator, examining a prototype Maize Plant Discipleship booklet, stated:

\begin{quote}

\emph{You are touching something that is not already existing. If we talk about evangelism, it may well be a new way of approaching evangelism, but we already have many methods of evangelism. But (a series of books on) discipleship is something that is really an innovative thing.}
\end{quote}

\begin{center}\rule{3in}{0.4pt}\end{center}


\begin{description}

\item[A 52 page handbook]

Printed double-sided, on A4 paper, folded, stapled and trimmed to produce A5 booklets, with a single-colour printed cover, can be produced for the equivalent of approximately \$2.

An agency organising republication, on behalf of a number of other organisations, may be the most appropriate way to achieve low costs. A creative economic model could subsidise costs for some individuals and, or groups.
\end{description}

\end {pause}

\section{Improvement and republication}
\label{improvementandrepublication}

After you have explored and used Maize Plant Discipleship resources, if you have ideas about how it might be improved, contact us. 

\begin {pause}

\begin{description}

\item[Are you African? Or working in African contexts?]

Can you think of stories, proverbs and examples that illustrate and express the principles contained in this teaching, in a way that would be especially suited to your local context?

If you send then to us, we may be able to incorporate them into future editions. However{\ldots}

\item[Are you aware]

that the Creative Commons license permits you to rewrite parts of the Maize Plant Discipleship Syllabus and republish it? 

All that is required is that you follow the terms of the license precisely, including \emph{attribution} and \emph{sharing-alike}, so that others can access and use the material.
\end{description}

\end {pause}

\section{Questions for Reflection and Discussion}
\label{questionsforreflectionanddiscussion}

\begin{itemize}
\item Do you think it's important that African Christians decide upon what is theologically appropriate to African contexts?

\item Has this happened historically in your contexts? If not, why not? If so, what has changed?

\item What theological issues are important to you and others in your context?

\item If they are not covered by this syllabus, what will you do?

\end{itemize}

\chapter{Maize Plant Philosophy of Discipleship}
\label{maizeplantphilosophyofdiscipleship}

THIS CHAPTER EXPLORES the biblical and missional perspectives of messianic discipleship that undergird Maize Plant Discipleship.

\section{What is messianic discipleship?}
\label{whatismessianicdiscipleship}

Maize Plant Discipleship approaches messianic discipleship as a dynamic, generational process, empowered by the Holy Spirit. Two crucial statements made by the apostle, Paul, in his second letter to his disciple, Timothy, reveal the essence of this process:\footnote{The two statements are separated only by Paul's emotional description of two disciples who failed to stand with him at a crucial time.}

\begin{quote}

Keep safe the great treasure that has been entrusted to you, with the help of the Holy Spirit, who lives in us{\ldots} and the things you heard from me, which were supported by many witnesses, these commit to faithful people, such as will be competent to teach others---\emph{2 Timothy 1.14 and 2.2}
\end{quote}

Together, these two Scriptures establish three key components of messianic discipleship.

\begin{enumerate}
\item \textbf{The great treasure of knowing the Messiah, Jesus Christ}.

The real, personal, experiential knowledge of the Messiah is more than human knowledge or philosophy: it is \emph{a great treasure}, a divine relationship, mediated by the Holy Spirit.

\item \textbf{The vitality of the Holy Spirit}.

The Holy Spirit provides an intimate source of divine help to messianic disciples,\footnote{John 16:7--15} mediating and helping to safeguard the reality of the Gospel and the Presence of the Messiah, amongst his people.

\item \textbf{The necessity of generational formation}.

Having received an impartation of the reality of the Messiah through Paul, his mentor calls upon Timothy to safeguard the treasure \emph{by committing it to the stewardship and safekeeping of other faithful people}. This is generational discipleship in action. 

\begin{figure}[htbp]
\centering
\includegraphics[width=275pt,height=45pt]{Generationaldiscipleship.png}
\caption{}
\label{Generationaldiscipleship.png}
\end{figure}

Generational discipleship is how treasure is kept safe in the kingdom of God. The significance of this principle can be further illustrated by \emph{seeds}.

\end{enumerate}

\subsection{Seed and harvest}
\label{seedandharvest}

In farming contexts, seeds are a form of wealth. They are a type of treasure. Yet seed is generally stored only for a short time before being used. Whatever is not required for food, \emph{for daily bread}, must soon be sown to produce another harvest.\footnote{2 Corinthians 9:6--12}

In a similar way, God supplies spiritual life to us. This is what Paul refers to as the \emph{treasure} of knowing the Messiah. This experience of being alive to God, experiencing the grace of the Messiah and the love, joy, peace, patience, kindness, goodness, faithfulness, gentleness and self-control of the Holy Spirit,\footnote{Galatians 5:22} is the spiritual equivalent of receiving daily sustenance—\emph{daily bread}. 

Enjoying God's life ourselves, however, is not the whole purpose of our relationship with him. In fact, as we will explore in this resource, the Messianic, New Covenant Community (the whole body of the Messiah's people) has been called to know God in order to become his \emph{Servant Community}. This means that we are called to give ourselves, our lives, \emph{to serve his purposes}. This requires sacrifice and discipline—that is what it means to be a disciple.

This sacrificial, disciplined giving of ourselves in service to God is the equivalent of taking precious seed that could be consumed by ourselves and instead sowing it into the ground, in order to produce another harvest.

\subsection{Sharing treasure}
\label{sharingtreasure}

Sadly, sacrifice is one of the secrets to living a truly \emph{messianic} life that many people seem never to properly discover or experience.
Yet the illustration of seeds teaches us that hoarding the treasure of our knowledge, relationship and communion with God is not the way to a rich harvest. It is only in \emph{sacrificially sharing our spiritual treasure} in with others, both within and beyond our own communities, that we discover and realise our vocation and, in due time, reap a harvest of faithfulness.\footnote{Matthew 10:38--39, 13:23; Galatians 6:6--10; Hebrews 12:11; James 3:18}

Spiritual treasure, though, should never be wasted or cast away carelessly. Whilst inevitably some seeds fall onto unreceptive ground, a farmer never intentionally wastes his seed. Likewise, our treasure is too precious to be deliberately squandered on people who spurn its value.\footnote{Matthew 7:6} It must be shared with people who recognise its worth and who make room for its transformative power to change them.

This is what Jesus, in the parable of the sower, calls \emph{good soil}.\footnote{Matthew 13:1--23} People willing to be transformed through a personal knowledge and experience of the Messiah, who will share it with other faithful people {\ldots}who share it with other faithful people {\ldots}and so on and so forth.

\section{Discipleship movements}
\label{discipleshipmovements}

The formation of faithful disciples was at the heart of the Messiah Jesus' life and work. The entire historical and now-worldwide Christian movement began with one small, core group of disciples, formed around Jesus. 

Discipling movements have significant potential to impact and transform. Leaders, money, books and power all have their places within radical, popular movements. But socially-impacting people movements have the greatest capacity to produce deep, wide, enduring change.

Two things are essential to transformative people movements: \emph{vision} and \emph{the formation of disciples}. Visionary leaders must impart a hope that is powerful, challenging and instrumental. A vision capable of gripping the hearts of others and forming them into faithful, active disciples:

\begin{itemize}
\item committed to significant transformation;

\item persistent and determined to turn vision into reality;

\item operating as co-workers—not selfishly-ambitious individuals;

\item actively forming other faithful disciples.

\end{itemize}

Thus develops a visionary, discipling movement of focussed, inspired, dynamic people, who are deeply and profoundly allied to a vision and a purpose greater than themselves.

\begin{pause}

The book of Acts provides a powerful illustration of the dynamic growth of a messianic, discipling movement. Examine the context of these verses, in your own Bibles, to see how this growth takes place.

\begin{itemize}
\item 2:46--47, 6:7, 9.31, 12.24, 16.5, 19.20, 28.30--31

\end{itemize}

This dynamic movement spreads geographically and, significantly, across ethnic and cultural boundaries. From its beginnings in Jerusalem, the movement expands throughout Palestine, Asia Minor, Greece and finally to Rome, the seat of imperial power.

\end{pause}

\subsection{Intercultural movement}
\label{interculturalmovement}

Since it's birth in the first century, over the course of two millennia, the messianic, new-covenant community has continued to grow and develop dynamically. From its origins as an obscure, tiny, Jewish sect it has grown into an international, intercultural, multi-ethnic community, now existing, in some form or another, in practically every nation of the world, impacting peoples, stories and cultures, throughout the world. It will surely effect many more in the days and years that are ahead. Its dynamism can be directly traced to its operation as \emph{a movement of disciples}, continually spreading out, across geographical, social, ethnic, linguistic and cultural boundaries. 

\subsection{Renewal of movement}
\label{renewalofmovement}

Thus, any messianic movement or community that earnestly desires spiritual revival or renewal must place visionary, messianic discipleship at the core of its spirituality and its practical formation.

Disciples must be invited, formed and sent forth as part of a \emph{world-facing} movement. The goal is much more than the maintenance of congregational activity. It is more than learning to serve one another, within messianic community. The goal is to be part of a movement of disciples responding to God's call to serve his eternal purpose, amongst a world of lost, hurting, confused, oppressed, fear-filled, idol-bound populations.

It is there, amongst the corruption in society, behind its social, political, economic and religious walls and besides its filthy gutters, that the Messiah is at work, by his Spirit. From there he calls his co-workers to come alongside him in his work of redemption and transformation of individuals, families, marriages, partnerships, communities, organisations, structures, workplaces and working practices.

\subsection{Anointed community}
\label{anointedcommunity}

To make possible such an otherwise impossibly-high calling, messianic discipleship provides a unique ingredient that no other philosophy, ideology or faith can provide: the dynamic of the indwelling Spirit of the Messiah. 

\emph{Through the Spirit}, the new-covenant community is transformed into a charismatic community. A group of people endowed with spiritual gifts that are profoundly shaped to liberate human beings from idolatry and every other allegiance and falsehood that competes and sets itself against God and the knowledge of him.\footnote{2 Corinthians 10:3--5}

Thus, the Messianic Community is a body of people anointed with the fragrant oil, or presence, of the Holy Spirit, having been brought under God's authority, through baptism into the Messiah. It is a body learning to walk in the footsteps of Jesus: learning to exercise its God-appointed mediatory, intercessory role, under the direction of the Spirit of God.\footnote{Romans 6:3--4; Galatians 3:26--29; Hebrews 6:4}

\begin{quote}

\textbf{This community of disciples is a messianic, charismatic people called into covenant relationship with the Father, through the Son and sent into the world to bless the nations in the power of the Spirit!}
\end{quote}

\begin{pause}

\begin{description}

\item[Charismatic]

From the Greek, \emph{charism}, meaning \emph{gift}. The \emph{charisma} of the Christian community derives from its spiritual anointing.

\item[Messiah]

essentially means \emph{Anointed One}; the One anointed with the Spirit to rule God's people—see Matthew 3:13--17; Mark 1:9--11; Luke 3:21--22; John 1:32--34.
\end{description}

\end{pause}

\subsection{Life. Through death}
\label{life.throughdeath}

God's intention is that this messianic, charismatic, covenant community co-works in partnership with him, using the strength, the power, the spiritual life, the anointing that he provides. 

Too often though, the power of the anointed-life-of-Christ-within seems to elude us. It seems out of our reach. Beyond our grasp. 

Indeed, it is not something that can be \emph{grasped}. Rather, the pathway to life is through dying. Yielding ourselves to God the Father, through the Messiah, by the Spirit. That is the message of the cross. As we \emph{die to self}, we become \emph{alive to God}.\footnote{Romans 6:4--13}

\section{The heart of discipleship}
\label{theheartofdiscipleship}

Thus we end where we began. With the foundational principle of transformative discipleship: seed sown into the ground, in order to produce a harvest.

\begin{quote}

I tell you that unless a grain of wheat that falls to the ground dies, it stays just a grain; but if it dies, it produces a big harvest—\emph{John 12.24}
\end{quote}

This life-giving spiritual reality is at the heart of Jesus' own life, mission, ministry and pain-filled death. And this same principle forms the foundation and wellspring of Maize Plant Discipleship: 

\begin{quote}

As we embrace a practical form of discipleship, incorporating a daily dying-to-self, we learn how to truly become alive-to-God and equipped to serve his eternal purpose.
\end{quote}

\emph{That is the heart of Maize Plant Discipleship}.

\section{Questions for Reflection and Discussion}
\label{questionsforreflectionanddiscussion}

\begin{itemize}
\item How is discipleship valued in your context?

\item How faithfully is it practised?

\item If there is a gap between what is believed and valued and what is practised, discuss why you think that is?

\end{itemize}

\chapter*{Author}\label{author}

\textbf{Dr John B Clements} is a missiological educator, having received a Doctorate of Missiology (Contextual Missiology) from Fuller Theological Seminary School of Intercultural Studies, in 2012. 

\begin{figure}[htbp]
\centering
\includegraphics[width=135pt,height=99pt]{john-rhossilli.jpg}
\caption{}
\label{john-rhossilli.jpg}
\end{figure}

John is married to Sarah; they have three boys and one girl and presently live in a delightful corner of South West Wales, UK. John is an avid bird-watcher and casual photographer, pastimes that he combines with his enjoyment of countryside and coastal walking.

\begin{wsite}

\begin{itemize}
\item Vita http:/\slash jbclements.wordpress.com

\item Linked-In http:/\slash uk.linkedin.com\slash in\slash jbclements

\item Social http:/\slash about.me\slash jbclements

\end{itemize}

\end{wsite}

\input{mpd-footer}

\end{document}
