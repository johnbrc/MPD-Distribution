\def\version{0.5.0 --- Post-translation draft}
\def\change{Initial typesetting; initial parallel changes with MPD-FH}
\input{mpd-header}
\def\mytitle{Livret du Facilitateur}
\def\subtitle{Discipolat Plante de Maïs}
\def\module{fh-fr}
\def\translation{Gérémie Yaldia}
\def\latexmode{memoir}
\input{mpd-document-fh-fr}
\chapter{Discipolat Plante de Maïs}
\label{discipolatplantedemas}

CE CHAPITRE FOURNIT des approches et des suggestions pratiques pour la facilitation des groupes d'apprentissage du Discipolat Plante de Maïs.

\section{Qu'est-ce que c'est que le Discipolat Plante de Maïs?}
\label{quest-cequecestquelediscipolatplantedemas}

Le Discipolat Plante de Maïs est une ressource d'apprentissage, pratique, utile et accessible pour être utilisé aussi bien en Afrique que dans d'autres contextes du monde. Ce document a été élaboré et expérimenté avec la collaboration des Africains, formulé en guise de réponse aux recherches doctorales contextuelles et publié en séries de livrets modulaires courts et à bon marché:

\begin{itemize}
\item approprié pour des études formelles et informelles

\item incorpore l'apprentissage réfléchi et des discussions de groupes

\item s'appuie simplement sur la coordination de petits groupes d'apprentissage des facilitateurs

\item facilement reproductible, en termes de re-publication et de formation.

\end{itemize}

Son but est de faciliter l'apprentissage biblique qui continuellement s'exteriorise incorporant des communautés entières avec des comportements de discipolat basés sur les écritures, en dialogue vivant avec la culture contextuelle.

\section{Groupes d'apprentissage}
\label{groupesdapprentissage}

Le Discipolat Plante de Maïs se veut un processus de groupe d'apprentissage, réfléchi et ouvert, dans lequel les leaders et les apprenants participent ensemble à découvrir ce que dit l'esprit, tandis que l'écriture est étudiée et liée aux \emph{signes des temps} contextuels.

\subsection{Pourquoi des groupes d'apprentissage?}
\label{pourquoidesgroupesdapprentissage}

Il ya plusieurs raisons qui justifient que l'on regroupe des gens afin qu'ils apprennent ensemble. On pourrait s'appuyer sur l'exemple de Jésus qui rassemble douze disciples. Pour beaucoup de gens, les groupes représentent un lieu naturel et vivant où apprendre. Les groupes rassemblent des gens avec des expériences, des dons, des capacités et des perspectives différents. 

Lorsque nous partageons nos vies, nous apprenons ensemble et les \emph{groupes d'apprentissage} reflètent cette réalité. Quand nous étudions ensemble, nous faisons des expériences et apprenons de manière différente que lorsque nous le faisons individuellement. Des discussions réfléchies avec d'autres personnes, en particulier, pourvoit un forum hautement stimulants d'apprentissage à travers l'exploration, l'écoute et la découverte. Pour plus d'analyses sur l'apprentissage réfléchi, voir le [\emph{Processus d'apprentissage du Discipolat Plante de Maïs}].

\subsection{L'apprentissage, pas l'enseignement?}
\label{lapprentissagepaslenseignement}

Oui. Le Discipolat Plante de Maïs est plutôt une ressource \emph{d'apprentissage} que d'enseignement. L'apprentissage dépend de plusieurs facteurs. En plus de la présentation de l'information d'actualité—la plupart de ces facteurs dépendent des apprenants, et non de l'enseignant, ou de l'enseignement. Parmi ces facteurs, on peut citer le désir, le tempérament, l'expérience, le talent, le temps, l'énergie, l'environnement etc. Ainsi la décision d'apprendre quelque chose de nouveau doit toujours provenir des apprenants eux-mêmes. 

\begin{quote}

\textbf{En conséquence, le discipolat doit être perçu comme une forme d'apprentissage \emph{composé} d'étudiants ou de disciples, à travers les conseils et la direction d'un facilitateur, mentor, éducateur ou enseignant. Les gens qui joueront ces rôles cotoient des \emph{apprenants motivés} pour assister, encourager, faciliter et annoncer l'apprentissage qui a lieu parmi ceux qui se font disciples.}
\end{quote}

\section{Facilitation des Groupes d'Apprentissage}
\label{facilitationdesgroupesdapprentissage}

Un facilitateur du Discipolat Plante de Maïs peut aider un groupe à introduire le message de l'Ecriture, les uns aux autres et, par dessus tout, sous la conduite du Saint-Esprit. Cette section examine comment.

\subsection{Facilitation d'ouverture}
\label{facilitationdouverture}

Promouvoir un environnement d'apprentissage et de découverte, où le débat et les discussions chaudes ont lieu, mais relaxes et non compétitifs est essentiel. Un environnement idéal favorisera l'expression de divers points de vue opposés sans créer de conflit ni de conformité, de sorte que tous ceux qui sont présents se sentent à l'aise pour donner leur opinion, poser des questions ou partager leur fardeau. 

Il peut être particulièrement difficile de promouvoir une ouverture au sein des cultures où l'autorité est traditionnellement centralisée et où la conformité est hautement valorisée. Si tel est le cas, les facilitateurs devront contribuer typiquement aux discussions comme des membres réguliers de groupes, sans chercher à dominer encore moins négliger les points de vue des autres.

Permettre à la discussion de monter et descendre, pendant que les gens sondent leurs réponses et font référence aux discussions antérieures. Encourager les autres à contribuer, particulièrement ceux qui sont timides, les femmes, la jeunesse et les personnes âgées. Si la discussion devient rude et factieux, interrompre le groupe, inviter ensuite un membre du groupe ayant une position modérée et harmonieuse de résumer la discussion (pas de résoudre) la tension, après quoi les échanges peuvent reprendre.

\subsection{Facilitation de discipolat conduit par l'Esprit}
\label{facilitationdediscipolatconduitparlesprit}

Le discipolat ne peut pas se résumer en une simple transposition d'informations, de l'enseignant,ou des manuels aux étudiants. Peut-être que c'est de cette manière que nous avons été enseignés à l'école; un discipolat conduit par l'esprit est different. Il \emph{forme}, en même tant qu'il informe.

L'intention est de voir à travers l'enseignement la reflexion et la discussion l'esprit de Dieu nous parler, conduire, prévenir, diriger, encourager, éduquer, défier et exhorter individuellement et collectivement. Comme nous avons des dons, personnalités et developments différents, à tout moment, chaque personne peut apprendre quelque chose de particulier de la part de l'Esprit.

\begin{quote}

\textbf{Le but du discipolat n'est pas d'établir une croyance dogmatique partagée ou une conformité aux convictions du leader, mentor ou facilitateur, ni aux traditions de l'église, pas même à tous les aspects du Discipolat Plante de Maïs. Le but c'est la conformité à l'esprit du Messi, Jésus, et l'obéissance à la volonté du Père.}
\end{quote}

\subsection{Qui peut être facilitateur?}
\label{quipeuttrefacilitateur}

Le facilitateur doit être quelqu'un qui se sent appelé pour aider d'autres personnes à devenir des disciples fidèles dans la foi. De ce fait, les facilitateurs devront être humbles, patients, bouillants, flexibles, ouverts et suffisamment sécurisés afin de permettre aux autres d'explorer les limites personnelles de l'entendement de leur vocation, expérimenter et créer à leur propre rythme.

Le facilitateur n'use pas d'autorité sur les autres. Son rôle est de tout simplement faciliter le rassemblement des gens en groupe pour apprendre et échanger. En conséquence, un facilitateur:

\begin{itemize}
\item peut être un simple leader;

\item peut être relativement jeune;

\item peut être une femme;

\item n'a pas besoin d'avoir été à l'école biblique;

\item n'a pas besoin d'être leader établi d'une église;

\item n'a pas besoin d'être un mentor expérimenté.

\end{itemize}

Naturellement que le Discipolat Plante de Maïs peut être facilité par des leaders établis, des mentors ou des faiseurs de disciples, pourvu simplement qu'ils veuillent et souhaitent faciliter des discussions de groupes qui donnent vraiment à \emph{réfléchir et sont exploratoire}.

\section{Considérations pratiques}
\label{considrationspratiques}

La facilitation du groupe de discipolat sera plus efficace s'il ya une prise en compte des questions pratiques à l'avance et la mise en place d'une planification appropriée. 

\subsection{Le démarrage}
\label{ledmarrage}

Le discipolat Plante de Maïs sied à des groupes d'apprentissage de 8--10 personnes. Ce nombre est suffisamment réduit pour permettre aux membres du groupe de grandir ensemble avec un certain degré d'intimité, mais en même temps ce nombre est assez grand pour permettre aux membres du groupe d'explorer leurs engagements de disciples à leur propre rythme.

\begin {pause}\begin{description}

\item[Plus de dix?]

Penser à aider les autres à modérer les groupes d'apprentissage supplémentaires: Quels problèmes pourriez-vous rencontrer?
\end{description}

\end {pause}

\subsection{Le lieu de rencontre}
\label{lelieuderencontre}

Les rencontres peuvent avoir lieu n'importe où pourvu que le cadre soit approprié et commode pour une rencontre d'apprentissage. Par exemple, une salle spacieuse chez quelqu'un, un bâtiment commun comme une église.

\begin {pause}\begin{description}

\item[Le programme comprend environ 64 études]

— pour donc établir un programme d'étude, il faut tenir compte de la disponibilité des membres du groupe. \emph{Par exemple la saison agricole ou l'année scolaire ont-elles une influence sur la participation des uns et des autres?}

\item[Rechercher ce qui convient à votre groupe —]

Songer à utiliser plusieurs cadres pour les rencontres, peut-être même des rencontres souvent au dehors. Apprêter les sièges de sorte, à créer un espace intime et pratique pour les discussions, afin que les participants puissent se voir et s'entendre.
\end{description}

\end {pause}

\subsection{Inclure les autres}
\label{inclurelesautres}

Même si le modérateur est la personne chargée de convoquer les rencontres, il peut laisser la responsabilité à un membre du groupe d'abriter la rencontre, de présenter l'enseignement du jour ou de modérer les discussions du groupe. Ce qui sera idéal, c'est qu'avec le temps chaque membre du groupe ait une tâche au sein du groupe chacun selon ses dons et capacités. Cela empêche qu'une seule personne ait beaucoup trop de responsabilité et permet à chacun d'avoir de l'expérience quant au rôle de modérateur.

La personne chargée de présenter l'étude du jour doit la lire attentivement, à l'avance: Absorber, se familiarise et réfléchir sur l'enseignement et ses leçons à tirer. S'il ya quelque chose qui vous parait étrange ou peu évident, invitez les autres à se prononcer la-dessus, tout en encourageant les autres à exposer sa pensée.

\begin {pause}\begin{description}

\item[La re-production est un but important du discipolat]

— \emph{voir La Philosophie de Discipolat Plante de Maïs (\autoref{laphilosophiedediscipolatplantedemas})} — il n'est pas impératif qu'un membre du groupe achève tout le programme pour être apte à modérer un groupe; soyez simplement conduit par l'esprit.
\end{description}

\end {pause}

\subsection{Adaptation}
\label{adaptation}

Soyez préparés à adapter l'enseignement et la méthode de présentation, afin de créer un environnement d'apprentissage culturel approprié et qui puisse aider tout le monde. Prenez en compte les aptitudes et les capacités de chaque groupe de discipolat. 

\begin {pause}\begin{description}

\item[Assurez que le manuel est un domestique utile]


\begin{itemize}
\item pas un task-master dur, particulièrement aux étudiants oraux.

\end{itemize}

\item[Inviter des personnes ressources à contribuer —]

en traduisant ou transformant par exemple un aspect de l'enseignement en chanson or théâtre ou en faire des représentations artistiques.
\end{description}

\end {pause}

\subsection{D'autres forums d'apprentissage}
\label{dautresforumsdapprentissage}

Les livrets du Discipolat Plante de Maïs peuvent facilement s'adapter à des études personnelles, une éducation théologique ainsi que d'autres formes d'apprentissages guidés. En particulier, les références de bas de page constituent un riche trésor d'outil d'apprentissage supplémentaire pour des études bibliques approfondies{\ldots} et les questions de débat peuvent être adoptées comme socle de réponses écrites ou même de courtes dissertations:

\begin{itemize}
\item L'éducation théologique — les étudiants doivent à former et gérer de petits groupes d'apprentissage soit à l'intérieur ou hors des classes afin de partager et réfléchir sur leurs expériences. Cela leur procurera une expérience considérable et une plus-value dans la modération des groupes d'apprentissage dans leurs propres zones d'intervention

\item Etude Personnelle — les étudiants qui utilisent les livrets du Discipolat Plante de Maïs pour des études personnelles doivent chercher à incorporer des méthodes d'apprentissage réfléchies soit en soumettant le résultat de leurs études à l'appréciation d'un mentor avisé soit en le partageant avec un autre étudiant en vue d'y réfléchir, d'en discuter de façon critique.

\item Enseignement à l'église — les études du Discipolat Plante de Maïs peuvent être adaptées pour un usage dans des églises locales. Par exemple, après une étude, l'église peut se scinder en de petits groupes pour en discuter. De façon alternative, les groupes d'apprentissage peuvent se rencontrer de temps à autre pour échanger, méditer sur l'enseignement et prier ensemble.

\end{itemize}

\vspace*{\fill}\begin {questions}

\begin{itemize}
\item En quoi les groupes d'apprentissage du Discipolat Plante de Maïs different-ils des autres forums d'apprentissage comme les écoles par exemple ?

\item Quelles sont les caractéristiques d'un bon modérateur\slash facilitateur ?

\end{itemize}

\end {questions}

\chapter{Le Processus d'Apprentissage}
\label{leprocessusdapprentissage}

CE CHAPITRE S'INTÉRESSE au processus de l'apprentissage réfléchi dont l'objet est de sous-tendre la présentation du programme (syllabaire) du Discipolat Plante de Maïs.

\section{L'apprentissage réfléchi}
\label{lapprentissagerflchi}

Les modules du Discipolat Plante de Maïs sont structurés de manière à offrir \emph{un processus d'apprentissage réfléchi}. Le groupe d'apprentissage réfléchi doit avoir en son sein au moins les éléments suivants:

\begin{itemize}
\item \textbf{Entendre} — l'expérience et des perpectives de d'autres.

\item \textbf{Réfléchir} — sur les idées et concepts, anciens et nouveaux.

\item \textbf{Discuter} — affilez la compréhension, dans le dialogue avec d'autres.

\item \textbf{Agir} — intégrer l'apprentissage dans les pratiques professionnelles.

\end{itemize}

Ce faisant, la pratique informe la théorie et la théorie informe la pratique. Comme ce processus d'apprentissage se répète,il forme un \emph{cycle}, qui peut être représenter de façon figurative.

\begin{figure}[htbp]
\centering
\includegraphics[width=170pt,height=165pt]{learning-cycle.png}
\label{learning-cycle.png}
\end{figure}



Le \emph{Processus d'Apprentissage du Discipolat Plante de Maïs} comprend non seulement ce cycle d'apprentissage réfléchi mais aussi bien des principes bibliques pratiques tirés des écritures:

\begin{quote}

Ils persévéraient dans l'enseignement des apôtres, dans la communion fraternelle, dans la fraction du pain et dans les prières — Actes 2:42
\end{quote}

Ce passage nous inspire quant à l'adaptation des éléments de l'apprentissage réfléchi alors que la prière et la fraction du pain dont il est question ici fournissent deux éléments uniques supplémentaires. Ainsi, dans son ensemble \emph{Le Processus d'Apprentissage du Discipolat Plante de Maïs} contient \emph{six} éléments:

\begin{enumerate}
\item \textbf{ENTENDRE} {\ldots} ce que dit l'Esprit

\item \textbf{GARDER} {\ldots} la parole de Dieu au dedans de vous

\item \textbf{OUVRIR} {\ldots} les coeurs aux autres

\item \textbf{PARTAGER} {\ldots} le pain quotidien ensemble

\item \textbf{PRIER} {\ldots} pour que le royaume de Dieu vienne

\item \textbf{AGIR} {\ldots} dans la lumière du message de Dieu

\end{enumerate}

\pagebreak 

\section{1. ENTENDRE ce que dit l'Esprit}
\label{entendrecequeditlesprit}

\begin{figure}[htbp]
\centering
\includegraphics[width=108pt,height=108pt]{hear.png}
\label{hear.png}
\end{figure}


Quand nous nous assemblons comme disciples du Messie, pour entendre l'enseignement biblique, nous nous ouvrons pas simplement aux idées et à la sagesse humaines, mais aussi et surtout aux paroles spirituelles vraies enseignées par l'Esprit de Dieu.

\begin{quote}

Or nous, nous n'avons pas reçu l'esprit du monde, afin que nous connaissions les choses que Dieu nous a données par sa grace. Et nous en parlons, non avec des discours qu'enseigne la sagesse humaine, mais avec ceux qu'enseigne l'Esprit, employant un langage spirituel pour les choses spirituelles—\emph{1 Corinthiens 2:12--13}
\end{quote}

\begin {pause}\begin{description}

\item[Nous écoutons afin de vivre plus fidèlement]

— ce type d'écoute s'appelle \emph{faire attention à}: écouter avec l'intention d'apprendre et d'obéïr, ou de suivre.

\item[Nous écoutons]

avec notre esprit, mais également notre coeur — afin \emph{d'entendre ce que l'esprit dit au peuple} (Apocalypse 2:29, 3:6,13,23; Matthieu 11:15, Marc 4:9 etc)—jamais pour être \emph{rassasié} de connaissance.
\end{description}

\end {pause}

\pagebreak 

\section{2. GARDER la parole de Dieu au dedans de soi}
\label{garderlaparolededieuaudedansdesoi}

\begin{figure}[htbp]
\centering
\includegraphics[width=108pt,height=108pt]{retain.png}
\label{retain.png}
\end{figure}


Ce n'est pas suffisant d'entendre simplement la parole de Dieu: il est important pour nous d'apprendre à \emph{garder} la parole de Dieu au dedans de nous, car là est le lieu où elle pourra \emph{habiter abondamment parmi nous.} (Colossiens 3:16) 

\begin{quote}

Celui qui a reçu la semence dans les endroits pierreux, c'est celui qui entend la parole et la reçoit aussitôt avec joie; mais il n'a pas de racines en lui-même, il manque de persévérance et ne dure qu'un instant{\ldots} Celui qui a reçu la semence parmi les épines est cet homme qui entend la parole, mais les soucis de ce monde et la séduction des richesses étouffent cette parole la rendant infructueuse. La semence sur la bonne terre représente ceux qui ont un coeur noble et bon, qui entendent la parole et la \emph{garde}, et du fait de leur persévérance produisent du fruit {\ldots} portant cent, soixante ou trente — \emph{Matthieu 13:18--23; Luc 8:15}
\end{quote}

\begin{pause}\begin{description}

\item[Pensons à la nourriture]

— la mâcher, apprécier son gout, l'avaler, la digérer, et au dedans de nous ses éléments nutritifs sont retenus. Il en est aussi ainsi de la parole de Dieu: nous devons la \emph{mâcher,} la méditer tout en réfléchissant sur son sens et sur comment elle s'applique à nos vies, aussi bien au niveau individuel que collectif en tant que communauté la permettant de s'implanter dans notre esprit ou elle peut nous forger nos convictions et renouveler notre espérance.
\end{description}

\end{pause}

\pagebreak 

\section{3. OUVRIR les coeurs aux autres}
\label{ouvrirlescoeursauxautres}

\begin{figure}[htbp]
\centering
\includegraphics[width=108pt,height=108pt]{open.png}
\label{open.png}
\end{figure}


La discussion et le débat sont une opportunité d'ouvrir nos coeurs aux perspectives et expériences des gens autour de nous et ceux qui perçoivent les choses différemment que nous. 

\begin{itemize}
\item Cela nécessite que l'on écoute avec le coeur, mais aussi avec la tête afin d'apprécier ce que les autres partagent plutôt que de remporter un argument. 

\item Les applications de la discussion pratique, \emph{professionnelle} de l'étude d'actualité est vital; penser à comment les enseignements du Discipolat Plante de Maïs ont un lien culturel au milieu duquel les membres du groupe habitent.

\item Accorder suffisamment de temps à cet aspect d'apprentissage du Discipolat Plante de Maïs!

\end{itemize}

\begin{pause}\begin{description}

\item[La Vocation]

— dépasse le simple aspect de notre travail ou de notre emploi, elle incorpore toutes les responsabilités pour lesquelles Dieu nous a appelés, y compris les familles, lieux de travail (et de pratiques), les communautés et les réseaux.

\item[Pensez la \emph{parole mielleuse} des proverbes traditionnels]

— procurent un aperçu et peuvent être d'une grande utilité lors des échanges avec les autres, tels les anciens ou les non-croyants.
\end{description}

\end{pause}

\pagebreak 

\section{4. PARTAGER le pain quotidien}
\label{partagerlepainquotidien}

\begin{figure}[htbp]
\centering
\includegraphics[width=108pt,height=108pt]{share.png}
\label{share.png}
\end{figure}


La célébration de la fraction du pain afin de se remémorer le sacrifice d'obéissance de Jésus, est un symbole significatif de la nouvelle alliance et un moyen puissant aux groupes de discipolat de proclamer leur commun attachement au Messie.

\begin{itemize}
\item La fraction du pain dans le christianisme moderne revêt un caractère typiquement cérémonial \emph{(L'Eucharie, Communion Saine, Messe)}. Pourtant, les communautés messianiques primitives l'ont concentré seulement autour du repas de pâques comme celui que Jésus a partagé avec ses disciples, juste avant son arrestation par les autorités de Jérusalem.

\item Le partage de la nourriture est donc un élément important de la communion humaine \emph{et} un moyen pratique de proclamer la bénédiction et provision liées à l'alliance de Dieu.

\end{itemize}

\begin {pause}\begin{description}

\item[Songer à y associer un simple repas]

— peut-être une fois par mois, pendant les temps de rencontre et dans la prière, partager un repas comme symbole de fraction de pain.

\item[Si le partage d'un repas n'est pas pratique]

— songer à partager entre vous de petites quantités de pain comme un acte symbolique d'hospitalité et un engagement partagé de l'appartenance au corps du Messie.
\end{description}

\end {pause}

\pagebreak 

\section{5. PRIER pour que le royaume de Dieu vienne}
\label{prierpourqueleroyaumededieuvienne}

\begin{figure}[htbp]
\centering
\includegraphics[width=108pt,height=108pt]{pray.png}
\label{pray.png}
\end{figure}


Après une période de discussion, inviter le groupe à prier ensemble, y compris l'intercession en faveur des voisins, réseaux et communautés, les gouvernants nationaux et locaux. 

\begin{itemize}
\item Permettre à l'enseignement d'infuser la prière avec une confiance fraîche concernant la volonté et le but de Dieu, y compris les préoccupations et défis personnels auxquels les membres du groupe font face. 

\item Permettre au Saint-Esprit de vous conduire à proférer des paroles de bénédiction les uns sur les autres, enracinés dans l'écriture, ainsi que sur votre communauté et nation ou bien pour un problème spécifique.

\item Vous attendre à ce que la puissance de Dieu vainque toute opposition, à travers la bénédiction de sa vie triomphante à l'oeuvre dans la vie de son peuple!

\end{itemize}

\begin {pause}\begin{description}

\item[Prier pour l'évangile]

— afin d'impacter et transformer profondément des individus, des communautés, des cultures et sociétés à travers votre nation, à travers l'Afrique, l'Europe, l'Asie et les Amériques; prier pour les peuples non atteints.

\item[Prier pour le projet du Discipolat Plante de Maïs]

— afin qu'il soit utilisé par Dieu pour édifier, fortifier et bénir la Communauté Messianique à l'Afrique{\ldots}et au delà!
\end{description}

\end {pause}

\pagebreak 

\section{6. AGIR à la lumière de la Parole de Dieu}
\label{agirlalumiredelaparolededieu}

\begin{figure}[htbp]
\centering
\includegraphics[width=108pt,height=108pt]{act.png}
\label{act.png}
\end{figure}


Le but de notre rassemblement pour entendre le message de Dieu n'est pas simplement pour l'entendre, mais d'agir en fonction de ce message. Comme l'épitre de \emph{Jacques} l'explique, nous nous trompons nous-même quand nous écoutons la parole de Dieu sans la mettre en pratique:

\begin{quote}

Ne vous trompez pas en vous bornant seulement d'écouter la parole sans pour autant la mettre en pratique, pratiquez-là! Car quiconque écoute la parole et ne la met pas en pratique est semblable à quelqu'un qui se regarde dans un miroir, et dès qu'il quitte le miroir, oublie ce à quoi il ressemble. Mais si quelqu'un regarde attentivement à la parfaite \emph{Torah (Enseignement)},la loi de la liberté, et qui persévère, n'étant pas un auditeur oublieux , mais se mettant à l'oeuvre , celui-là sera heureux dans son activité. — \emph{Jacques 1:22--25}
\end{quote}

\begin {pause}\begin{description}

\item[Le but du discipolat]

— est d'être transformé nous-même et devenir une influence transformatrice dans nos maisons, nos lieux de travail et les réseaux sociaux.

\item[Comme nous sommes transformés]

— en tant que mouvement de disciples dynamiques et croissants, nous commençons à réaliser notre vision conjoint: \emph{d'être une Communauté Messianique bénie afin de devenir des bénédictions pour les familles de la terre!}
\end{description}

\end {pause}

\pagebreak 

\section{Le cycle d'apprentissage}
\label{lecycledapprentissage}

La combinaison des six éléments du processus d'apprentissage du Discipolat Plante de Maïs produit le cycle d'apprentissage du Discipolat Plante de Maïs.

\begin{figure}[htbp]
\centering
\includegraphics[width=228pt,height=260pt]{mpd-learning-cycle.png}
\label{mpd-learning-cycle.png}
\end{figure}



\begin {pause}\begin{description}

\item[Un cycle d'apprentissage est un outil]

— son but est de faciliter et non de contrôler l'apprentissage et il n'est pas essentiel d'incorporer tous les éléments chaque fois que le groupe se rencontre. Permettez au cycle de s'étendre au lieu d'être limité. \emph{Là où c'est approprié, adoptez-le}. Par exemple, les groupes peuvent incorporer deux (ou plusieurs) études en une seule rencontre, répétant les trois premiers éléments du cycle, avant d'achever par les trois derniers.
\end{description}

\end {pause}

\chapter{Le Programme du Discipolat Plante de Maïs}
\label{leprogrammedudiscipolatplantedemas}

CE CHAPITRE INTRODUIT la métaphore de la plante de mais et les seize modules du syllabaire (programme) du Discipolat Plante de Maïs.

\section{La Métaphore de la Plante du Maïs}
\label{lamtaphoredelaplantedumas}

Jésus se réfère à sa propre mission en utilisant la métaphore d'une graine qui enfouit dans le sol et meurt, afin de produire une grande récolte. 

\begin{quote}

Je vous dis que à moins que le grain de blé qui tombe sur la terre ne meurt, il restera un seul grain, mais s'il meurt, il produira du fruit en abondance — \emph{Jésus, Jean 12:24}
\end{quote}

On produit le maïs à travers l'Afrique sub-saharienne et pourvoit une métaphore similaire et hautement reconnaissable relatif aux dynamiques de la vie, de la mort,de la nourriture et de la croissance. Le syllabaire du Discipolat Plante de Maïs est symboliquement structuré pour refléter cette métaphore fondamentale:

\begin{summary}

\pagebreak 

\textbf{De la bonne semence, ensemencée sur du bon sol, stimulée par les rayons solaires arrosée par la pluie produit une croissance dynamique, ayant pour résultat une bonne moisson.}

\end{summary}

\begin{figure}[htbp]
\centering
\includegraphics[width=290pt,height=265pt]{mp-metaphor.png}
\label{mp-metaphor.png}
\end{figure}



Tout comme les plantes de maïs, les communautés messianiques ont besoin d'être enracinées sur de la bonne terre où elles bénéficient des nutriments (nourriture) spirituels issus de ces milieux, être rafraichies par l'eau vivante de l'Esprit et être stimulées par la lumière révélatrice.

Les racines profondes non seulement protègeront les communautés messianiques des vents destructrices de faux enseignements mais aussi les soutiendront pendant les moments difficiles d'épreuves, de tentations et de responsabilités professionnelles.

Au regard de cette métaphore, le syllabaire du Discipolat Plante de Maïs est divisé en trois parties:

\begin{enumerate}
\item Le sol et les racines (Module 1)

\item La tige de maïs (Modules 2--9)

\item Les rayons solaires et la pluie (Modules 10--16)

\end{enumerate}

\subsection{Le sol et les racines}
\label{lesoletlesracines}

De façon métaphorique, les racines de la plante de maïs représentent la communauté biblique d'Israel. Le sol sur lequel les racines poussent correspond aux contextes historiques, culturels et géopolitiques de la vocation de l'alliance d'Israel (tel que l'Egypte, Canaan, Babylone et l'occupation romaine).

\subsubsection{Module 1, Le But Eternel de Dieu}
\label{module1lebuteterneldedieu}

\begin{itemize}
\item Etablit un aperçu panoramique de l'Ecriture, révélant le but éternel inchangé de Dieu. Dans cette perspective, le Messie est la \emph{semence} qui germe du sol de l'histoire biblique et l'alliance du peuple d'Israel, pour mourrir afin de produire une moisson abondante d'âmes-Une Communauté de l'Alliance Messianique- parmi tous les peuples de la terre.

\end{itemize}

\emph{Le Module 1, constitue de ce fait le fondement du syllabaire tout entier}.

\begin{figure}[htbp]
\centering
\includegraphics[width=250pt,height=125pt]{roots.png}
\label{roots.png}
\end{figure}



\begin {pause}\begin{description}

\item[Le maïs]

— représente effectivement le \emph{pain quotidien} pour des millions d'Africains. A l'image du Messie, la Communauté Messianique est appelée à devenir une sorte de \emph{pain de vie} aux peuples de la terre—voir Jean 6 (\& 20:21)
\end{description}

\end {pause}

\subsection{La tige de maïs}
\label{latigedemas}

Les \textbf{Modules 2 à 9} explorent sept \emph{dynamiques} messianiques qui représentent le development caractéristique, la croissance, la structure, la forme et le fruit de la Communauté Messianique.\footnote{\emph{Dynamique} provient du mot Grec \emph{dunamis}, ce qui signifie puissance et renvoit aux forces qui stimulent le changement à l'intérieur d'un système ou d'un processus (comme une plante ou un corps).}

\begin{figure}[htbp]
\centering
\includegraphics[width=266pt,height=374pt]{mp-dynamics.png}
\label{mp-dynamics.png}
\end{figure}



\subsubsection{Module 2 : Vocation, les Nations}
\label{module2:vocationlesnations}

\begin{itemize}
\item Le development historique de la mission vocationnelle de la Communauté Messianique de bénir les races de la terre.

\end{itemize}

\subsubsection{Module 3 : Vocation, les Juifs}
\label{module3:vocationlesjuifs}

\begin{itemize}
\item La responsabilité spéciale de la Communauté Messianique envers le peuple Juif.

\end{itemize}

\subsubsection{Module 4 : La Commission}
\label{module4:lacommission}

\begin{itemize}
\item Strategic and structural dynamics of messianic commissioning and community growth.

\end{itemize}

\subsubsection{Module 5 : L'Appartenance au Corps}
\label{module5:lappartenanceaucorps}

\begin{itemize}
\item Les dynamiques structurelles et stratégiques de commission messianique et la croissance communautaire

\end{itemize}

\subsubsection{Module 6 : Reveil}
\label{module6:reveil}

\begin{itemize}
\item La dynamique du réveil et une moisson spirituelle de la fidélité de l'alliance.

\end{itemize}

\subsubsection{Module 7 : La Vérité}
\label{module7:lavrit}

\begin{itemize}
\item Rencontrer la vérité alors que nous marchons dans une alliance fidèle pratique vers la révélation, sagesse et direction de Dieu.

\end{itemize}

\subsubsection{Module 8 : L'Intercession}
\label{module8:lintercession}

\begin{itemize}
\item La vocation sacerdotale de la Communauté Messianique: être \emph{une maison de prière pour toutes les nations}.

\end{itemize}

\subsubsection{Module 9 : La transformation culturelle}
\label{module9:latransformationculturelle}

\begin{itemize}
\item L'appel à oeuvrer parmi les peuples et pour les peuples vers des transformations culturelles qui signalent la présence du royaume de Dieu.

\end{itemize}

\subsection{Les rayons solaires et la pluie}
\label{lesrayonssolairesetlapluie}

Les Modules 10 à 16 examinent sept \emph{disciplines} caractéristiques qui permettent aux communautés messianiques de recevoir la \emph{lumière} révélatrice et \emph{l'eau vivante} nourricière de l'esprit de Dieu sans lequel nous devenons spirituellement faible et incapables de produire du bon fruit ou une moisson abondante.

\begin{figure}[htbp]
\centering
\includegraphics[width=283pt,height=397pt]{mp-disciplines.png}
\label{mp-disciplines.png}
\end{figure}



\subsubsection{Module 10 : La Maturité Spirituelle}
\label{module10:lamaturitspirituelle}

-Trois étapes de rencontre et croissance dans la maturité spirituelle des disciples et communautés messianiques.

\subsubsection{Module 11 : La Course}
\label{module11:lacourse}

\begin{itemize}
\item Les motivations, qualités et disciplines pour vivre une vie de service durable.

\end{itemize}

\subsubsection{Module 12 : Pression vers notre Vocation}
\label{module12:pressionversnotrevocation}

\begin{itemize}
\item Identifier et exceller dans notre vocation personnelle à travers un approfondissement de notre relation avec le Messie.

\end{itemize}

\subsubsection{Module 13 : La Fidélité Economique}
\label{module13:lafidliteconomique}

\begin{itemize}
\item Une perspective biblique basée sur la fidélité économique, la richesse et la pauvreté-foncièrement différents de celles du monde.

\end{itemize}

\subsubsection{Module 14 : Leadership Messianique}
\label{module14:leadershipmessianique}

\begin{itemize}
\item Les qualifications, motivations et caractéristiques du leadership messianique fidèle.

\end{itemize}

\subsubsection{Module 15 : Vivre par la foi}
\label{module15:vivreparlafoi}

\begin{itemize}
\item Regarder avec les yeux de la foi nous permet d'endurer les temps difficiles et de purification et d'embrasser les défis comme une opportunité pour expérimenter la fidélité de Dieu

\end{itemize}

\subsubsection{Module 16 : La victoire}
\label{module16:lavictoire}

\begin{itemize}
\item Affronter les idoles, les forteresses culturelles avec la puissance de l'esprit et discerner les stratégies qui donnent lieu à une rencontre transformatrice par la puissance victorieuse de Dieu.

\end{itemize}

\pagebreak 

\section{Les livrets de Module}
\label{leslivretsdemodule}

Chaque module syllabaire est consigné dans un livret,contenant quatre \emph{études} interalliées, dont chacune comprend: lecture des écritures, sections thématiques, illustrations et résumés, questions de discussion.

\begin {pause}\begin{description}

\item[Illustrations —]

S'il ya une insuffisance de livres, des schémas et illustrations doivent être reproduits, à l'aide de tableaux ou d'autres supports, même faire des représentations caricaturales sur le sol afin de permettre à tous les membres du groupe de comprendre à juste titre la pertinence du sujet illustré. \emph{Si vous avez un artiste dans votre groupe, confiez-lui alors cette tache}.

\item[La langue maternelle est la langue naturelle du coeur]

— les traductions des écritures en langues maternelles devront être encouragées et utilisées autant que faire se peut, y compris lors des discussions de groupe mais particulièrement pour la lecture et la mémorisation.
\end{description}

\end {pause}

\section{Termes significatifs}
\label{termessignificatifs}

Des définitions Importantes données ci-dessous expliquent comment ces termes importants sont utilisés dans le syllabaire du Discipolat plante de Maïs.\begin{description}

\item[Messie]

— un médiateur ou sauveur, agissant avec l'autorité de Dieu pour délivrer un peuple de la main de ses ennemis et, ou pour régner sur ou garder en sécurité ce peuple en question (afin qu'il expérimente \emph{shalom}). Dans l'histoire biblique d'Israel, la délivrance leur est parvenu par plusieurs médiateurs, tels que les prophètes, les sacrificateurs, les juges et les rois. 

Le vrai sens du mot messi est \emph{oint} ou \emph{couler dessus}, faisant référence à l'huile d'onction qu'on versait sur les rois et sacrificateurs d'Israel. L'huile d'onction symbolise le couler à travers, ou le fait de placer l'esprit de Dieu sur un leader, comme ils étaient investis de leur autorité, habituellement par les prophètes.\footnote{Ex. Exode 30:22–25} Ainsi, l'idée de l'autorité sacerdotale, royale et prophétique de Dieu est inhérente au concept du \emph{messie}. 

Le Nouveau Testament parle de Jésus comme étant le Messie Juifh,\footnote{\emph{Christ} est la traduction Grecque du mot \emph{Moshiach} (Hébreu, Messi); Jésus Christ est la traduction du grec \emph{Yeshua Moshiach}} oint par l'esprit pour accomplir trois rôles de médiations—prophète, sacrificateur et roi—pour le peuple de Dieu.\footnote{Matthieu 3:13--17; Marc 1:9--11; Luc 3:21--22 \& 4:16--19; Jean 1:32--34} Après son ascecion à la \emph{Droite de Dieu}, Jésus devient le \emph{seul Médiateur entre Dieu et l'humanité}.\footnote{Hebrews 8:1--2; 1 Timothy 2:5}

\item[Messianique —]

relatif au Messi; utilisé principalement dans le syllabaire pour fair référence à la \emph{Communauté Messianique} et aux \emph{communautés messianiques}.

\item[Communauté messianique]

(capitalisé) — l'ensemble du corps des gens constitué à travers le monde appartenant au Messie. Dans le Nouveau Testament cette communauté est appelée \emph{le corps du Messie (Christ)}. Cette appellation équivaut aussi à \emph{l'ensemble de la communauté chrétienne} ou \emph{église}. Ces termes sont toutefois généralement évités à cause de leur association avec des expressions historiques particulières de Christianisme (Chrétienté) qui ne sont pas inclusives.

\item[les communautés messianiques]

(non-capitalisé) — congregations localisées de Communauté Messianique. Ce terme est préférable au mot \emph{églises} afin de mettre l'accent sur le lien biblique qui lie l'ensemble du corps du Messie, la Communauté Messianique.

\item[la vocation —]

un appel, l'oeuvre d'une vie, mission, but, fonction; profession, occupation, carrière, travail, emploi, affaire, art, business, ligne de travail, métier.

Dans le syllabaire du Discipolat Plante de Mais, \emph{vocation} et \emph{professionnel} s'appliquent aussi bien à l'appel personnel que collectif. Dans le cadre de ce syllabaire, ces termes sont utilisés pour insister que la vocation d'un individu et d'une communauté locale provient d'un appel divin à servir le but éternel de Dieu, en union avec le Messi. 
Le terme vocation constitute ainsi une couverture qui incorpore et dignifie toute forme d'oeuvre et ministère. Ce terme va au delà des divisions traditionnelles de laïcité et du clergé, homme et femme, rendant à l'évidence que tous ceux qui suivent le Messie sont appelés \emph{à servir les buts de Dieu, à l'intérieur des maisons, des lieux de travail et à l'intérieur des communautés}. 
\end{description}

\chapter{La Philosophie de Discipolat Plante de Maïs}
\label{laphilosophiedediscipolatplantedemas}

CE CHAPITRE EXPLORE les perspectives de fondements bibliques de discipolat messianique, sur lesquelles est établi le Discipolat Plante de Maïs.

\section{Qu'est-ce que le Discipolat Messianique?}
\label{quest-cequelediscipolatmessianique}

Le Discipolat Plante de Maïs aborde le leadership messianique comme un processus dynamique générationnel, rendu possible par le Saint-Esprit. Deux déclarations importantes faites par l'apôtre Paul, dans sa deuxième épître à son disciple Timothée révèle l'essence de ce processus:\footnote{Les deux déclarations sont séparées seulement par la réponse émotionnelle de Paul à deux disciples qui ne l'ont pas soutenu à un moment crucial, ce qu'il a comparé avec la loyauté d'Onesiphore.}

\begin{quote}

Garde avec soin le grand trésor qui t'a été donné avec l'aide du Saint-Esprit qui habite en nous {\ldots} et ce que tu as entendu de moi, qui était soutenu par plusieurs témoins, des gens fidèles, compétents pour s'enseigner les uns les autres---\emph{2 Timothée 1.14, 2.2}
\end{quote}

Ces deux écritures établissent ensemble trois composantes essentielles du discipolat messianique:

\begin{enumerate}
\item \textbf{Le grand trésor de la connaissance du Messie, Jésus}

La connaissance véritable, personnelle et expérimentale du Messie dépasse la simple connaissance humaine ou la philosophie: c'est un \emph{immense trésor}, une relation divine, dont le médiateur est le Saint-Esprit.

\item \textbf{La vitalité du Saint-Esprit}

Le Saint Esprit garantit une source intime de soutien divin aux disciples messianiques,\footnote{Jean 16:7--15} servant de médiateur et aidant à sauvegarder la réalité de l'Evangile et la présence du Messie parmi son peuple.

\item \textbf{La nécessité de la formation générationnelle}

Ayant reçu de Paul l'attribution vraie de la réalité du Messie, Paul interpelle son filleul Timothée à sauvegarder ce trésor en le confiant aux bons soins d'autres fidèles. \emph{ça, c'est le discipolat générationnel en action}. 

\begin{figure}[htbp]
\centering
\includegraphics[width=268pt,height=45pt]{Generational-discipleship.png}
\label{generational-discipleship.png}
\end{figure}



\end{enumerate}

\begin{quote}

\textbf{Le discipolat générationnel c'est comment un trésor et se garde avec soin dans le royaume de Dieu. La signification du principe de discipolat générationnel peut être davantage illustré par la métaphore de la \emph{semence} et de la moisson qui s'en suit lorsque la semence tombe sur de la bonne terre.}
\end{quote}

\subsection{La semence et la moisson}
\label{lasemenceetlamoisson}

En agriculture, les semences représentent une forme de richesse. C'est une sorte de trésor. Pourtant on ne garde la semence que pour un bout de temps avant d'en faire usage. Ce qui n'est pas utilisé pour la nourriture, \emph{pour le pain quotidien}, doit être bientôt ensemencé pour produire une autre récolte.\footnote{2 Corinthiens 9:6--12}

De la même façon, Dieu nous donne une vie spirituelle. C'est ce que Paul désigne par le trésor de la connaissance du Messie. Vivant en Dieu et expérimentant la grace du Messie et l'amour, la joie, la paix, la patience, la gentillesse, la bonté, la fidélité, la douceur et le contrôle de soi que donne le Saint-esprit,\footnote{Galates 5:22} est l'équivalent spirituel de recevoir le pain quotidien.

Cet aspect personnel de la connaissance du Messie,n'est cependant pas le seul objectif de notre relation avec lui. En effet, la Communauté de la Nouvelle Alliance Messianique (l'ensemble du corps du peuple du Messie) est appelé à connaître Dieu \emph{afin de devenir sa communauté de serviteurs}. Cela signifie que nous sommes appelés à nous donner nous-même, nos vies, pour servir ses objectifs. Cela requiert du sacrifice et de la discipline toute chose qui est nécessaire pour être un vrai disciple.

\begin{quote}

\textbf{Ce don en sacrifice de nous-même, et cette discipline personnelle au service de Dieu équivaut pour nous à prendre une graine précieuse et plutôt que de la consommer, nous l'ensemençons afin qu'elle produise de la récolte.}
\end{quote}

\subsection{Partager le trésor}
\label{partagerletrsor}

La discipline et le sacrifice font partie des secrets les plus importants quant à comment vivre une vraie vie messianique. Malheureusement, beaucoup de gens ne découvrent pas ces secrets encore moins faire de ces secrets des principes de vie. Pourtant l'illustration des graines nous enseigne que lorsque le trésor de notre connaissance, relation et communion avec Dieu sont inactifs alors cela ne produit aucune moisson. C'est seulement en partageant notre trésor spirituel dans le sacrifice, aussi bien à l'intérieur qu'à l'extérieur de nos communautés que nous découvrons et réalisons notre vocation, et au moment opportun, nous moissonnerons le fruit de la fidélité.\footnote{Matthieu 10:38--39, 13:23; Galates 6:6--10; Hébreux 12:11; Jacques 3:18}

Toutefois le trésor spirituel ne devra aucunement être vandangé ou traité avec négligence. Quoique de façon inévitable des graines tombent sur de la terre inappropriée, aucun cultivateur ne jette intentionnellement sa semence. De même, le trésor spirituel est trop précieux pour être volontairement gaspillé pour des personnes qui n'en connaissent pas la valeur.\footnote{Matthieu 7:6} On doit plutôt investir sur des gens qui connaissent la valeur de ce trésor et qui sont prêts à subir une transformation sous ses effets.

Ce type de personnes sont ceux que Jésus dans la parobole du semeur appelle la \emph{bonne terre}:\footnote{Matthieu 13:1--23} 

\begin{quote}

\textbf{Des gens désireuses d'être transformés à travers une connaissance et expérience personnelle du Messie, qui partageront leur trésor avec d'autres personnes fidèles {\ldots} qui en feront de même, ainsi de suite ainsi de suite.}
\end{quote}

\section{Les mouvements du Discipolat}
\label{lesmouvementsdudiscipolat}

La formation de disciples fidèles était au coeur de de la vie et du ministère de Jésus. Aujourd'hui, le mouvement messianique historique et mondial rend témoignage de l'importance de ce petit groupe de disciples qui s'étaient constitués autour de Jésus.

Depuis ses origines au premier siècle, indexée comme une petite secte juive obscure, la communauté messianique a grandi et s'est développé à travers deux millénaires. Aujourd'hui, cette communauté est devenue internationale, interculturelle, multi-ethnique d'une manière ou d'une autre dans pratiquement chaque nation dans le monde. 

Dans ce parcours, la Communauté Messianique et le message qu'elle proclame a impacté un nombre innombrable de gens, histoires et cultures à travers le monde. Son dynamisme peut se voir directement par son fonctionnement fidelé comme \emph{un mouvement générationnel de disciples}, se répandant à travers des frontières sociales, ethniques, linguistiques, géographiques et culturelles.

\begin{pause}

\textbf{Le livre des Actes fournit une puissante illustration} de la croissance dynamique mouvement primitif de disciples messianiques, et comment il s'est répandu à travers le monde ancien. Dépuis ses origines à Jerusalem, il s'étend rapidement à travers Israel, jusqu'en Asie Mineure, à travers la Grèce et finalement jusqu'à Rome, le siège de la puissance impériale.

\emph{Soyez témoins de comment cette croissance a été possible en examinant dans vos propres Bibles les contextes des versets suivants:}

\begin{itemize}
\item Actes 2:46--47, 6:7, 9:31, 12:24, 16:5, 19:20, 28:30--31

\end{itemize}

\end{pause}

\subsection{L'engouement générationnel}
\label{lengouementgnrationnel}

L'histoire nous enseigne que l'élan, une fois acquis, ne demeure pas indéfiniment. Beaucoup de Mouvements Messianiques ont très bien commencé, mais aujourd'hui sont malheureusement dans les oubliettes. D'autres existent seulement sous une forme institutionnelle, mais sans aucun vrai sens de renouvellement spirituel, engouement générationnel ou même de la confiance pour affronter les défis à eux posés par les idoles culturelles ou sociales.

Le maintien de l'élan de discipolat générationnel requiert un focus persistent personnel sur une transformation personnelle, sociale et culturelle. Le Messie est à l'oeuvre, par la présence de son esprit dans une société corrompue aussi bien au plan social, politique, économique que religieux. De cette société corrompue, il a appelé ses collaborateurs à se joindre à lui dans son oeuvre pour le salut et la transformation des individus, des familles, des foyers, des partenariats, des communautés, des organisations, des structures, des services et des pratiques en matière de travail.

\subsection{Le renouvellement du mouvement}
\label{lerenouvellementdumouvement}

En conséquence, tout mouvement ou communauté messianique qui aspire vraiment au renouvellement spirituel doit accorder une place de choix au discipolat visionnaire et générationnel qu'il placera au centre des sa spiritualité et de sa formation pratique.

Les disciples doivent être invités, formés et envoyés au dehors comme un mouvement pour \emph{affronter le monde}. Un tel but requiert beaucoup plus que le simple fait d'être limité au maintien des activités de l'église. C'est aussi plus que d'apprendre à s'occuper les uns des autres, à l'intérieur de communautés messianiques. Le but est plutôt de faire partie de mouvement de disciples répondant à l'appel de Dieu afin de \emph{servir son but éternel}, parmi des gens perdus, bléssés, confus, opprimés, remplis de crainte et dominés par des idoles.

La mise en place d'un tel mouvement transformationnel nécessité une vision capable d'attirer le coeur des autres les unissant en un groupe de disciples fidèles:

\begin{itemize}
\item Dévoué quant à l'importance du renouvellement personnel et social

\item Persévérant et déterminé à transformer la vision en réalité

\item opérant et collaborant pour le bien de tous et non des individus aux ambitions égoïstes

\item Actif et prompt quant à la formation d'autres disciples.

\end{itemize}

\section{Communauté ointe}
\label{communautointe}

Afin de rendre possible l'accomplissement d'un si grand appel, le discipolat messianique doit susciter un ingrédient unique en son genre qu'aucune philosophie, idéologie ou toute autre foi ne peut avoir: La dynamique de la conservation en soi de l'esprit du Messie. 

\begin{itemize}
\item A travers l'esprit, la communauté de l'alliance est transformée en \emph{communauté charismatique}—un groupe de gens doté de dons spirituels oeuvrant à libérer les gens de l'idolâtrie, des pactes et autres faussetés qui s'érigent contre la connaissance de Dieu. \footnote{2 Corinthiens 10:3--5}

\item Cette communauté charismatique est un corps de gens unis sous l'autorité de Dieu, à travers le baptême dans le Messie, et oint de la présence aromatique du Saint-Esprit, à travers le baptême de l'Esprit 

\item C'est un corps qui apprend à marcher sur les pas de Jésus: apprenant à exercer son rôle de messianique de médiateur et d'intercesseur à lui assigné par Dieu lui-même sous la direction de l'esprit de Dieu.\footnote{Romains 6:3--4; Galates 3:26--29; Hébreuxs 6:4}

\end{itemize}

\begin{quote}

\textbf{Cette communauté de disciples
est un peuple messianique, charismatique
appelée à une alliance avec le Père,
unie avec le Fils,
envoyée dans le monde pour bénir les nations,
dans la puissance de l'Esprit!}
\end{quote}

\begin{pause}\begin{description}

\item[Charismatique]

— étymologiquement du Grec, \emph{charisme}, signifiant \emph{don}; le \emph{charisme} de la communauté messianique provient de son onction avec l'esprit du Messie

\item[Messianique]

— signifie principalement \emph{oint pour apporter la délivrance}; la communauté messianique est ointe d'esprit pour servir de médiateur au peuple de Dieu et aux nations, apportant la délivrance du mal et une nouvelle vie dans le Messie
\end{description}

\end{pause}

\subsection{Mourir pour vivre}
\label{mourirpourvivre}

La volonté de Dieu est que cette communauté messianique, charismatique et d'alliance travaille de concert avec lui en se servant de la force, puissance, la vie spirituelle, et l'onction qu'elle procure. 

Pourtant très souvent, la puissance de la vie-ointe-de Christ à l'intérieur semble nous échapper. Elle semble hors de notre portée. Au delà de ce que nous pouvons atteindre. En effet c'est quelque chose que nous pouvons belle et bien atteindre. 

Au contraire, le chemin idéal vers la vie, c'est à travers la mort. Nous abandonnant à Dieu le Père, à travers l'unité avec le Messi,par la puissance de l'Esprit. tel est le message de la croix. Alors que nous \emph{mourrons en nous-même}, nous devenons \emph{vivants en Dieu}.\footnote{Romains 6:4--13}

\subsection{Le coeur du discipolat}
\label{lecoeurdudiscipolat}

Ainsi nous terminons comme nous avons commencé par les principes fondateurs du discipolat transformateur: de la semence enfouie dans le sol afin de produire une moisson abondante.\footnote{Je vous déclare qu'à moins que le grain de blé qui tombe sur la terre meurt, il restera toujours un; mais s'il meurt, il produira beaucoup de fruit—\emph{Jean 12:24}}
Cette réalité spirituelle qui donne la vie représente le coeur de la vie personnelle de Jésus, sa mission, son ministère et sa mort douloureuse. Et ce même principe sert de fondement et de base pour le Discipolat Plante de Maïs:

\begin{quote}

\textbf{Alors que nous embrassons une forme pratique de discipolat, qui nous fait renoncer à nous-même quotidiennement, nous apprenons à devenir véritablement vivants et authentiquement équipés pour servir son but éternel}
\end{quote}

\emph{Tel est le coeur du Discipolat Plante de Maïs}.

\vspace*{\fill}\begin {questions}

\begin{itemize}
\item Quelle est la valeur du discipolat dans ton contexte et est-il fidèlement pratiqué?

\item Pourquoi pensez-vous qu'il en est ainsi?

\end{itemize}

\end {questions}

\chapter{Maize Plant Discipleship and Africa}
\label{maizeplantdiscipleshipandafrica}

CE CHAPITRE EXPLIQUE le development et potentiel du Discipolat Plante de Maïs comme ressource à utiliser dans le contextes africains.

\section{Le Facteur Afrique}
\label{lefacteurafrique}

Tite Tienou, un Malien qui a grandi au Burkina Faso, est celui qui a dévoilé l'aspect important de la philosophie qui guidé la mise en place du Discipolat Plante de Maïs: les voies Africaines sont celles qui doivent déterminer le type de théologie qui correspond aux contextes Africains. déclare Tienou :

\begin{quote}

L'Africanité et la justesse (théologique) ne doivent pas être mesurées en terme de similarité ou de différence par rapport à l'Occident. L'issue est d'évaluer l'africanité de toute théologie la rendant africaine au point où elle s'occupe des besoins des Africains dans leur contexte total. De façon naturelle, les besoins des chrétiens africains devront donc être sérieusement pris en compte lors de détermination de ces besoins — \emph{Tite Tienou, The Uphill Road: Indigenous African Christian Theologies, 1990}
\end{quote}

L'auteur du Discipolat Plante de Maïs n'est pas un Africain. Malgré cela son contenu répond entièrement aux contextes Africains. A chaque étape, l'énergie missionnaire et la perspective culturelle du peuple Africain a toujours été pris en compte. C'est la réponse des pasteurs, étudiants, missionnaires et jeunes Africains, qui ont permis que le Discipolat Plante de Maïs existe.

En particulier, entre 2003 et 2010, des leaders de l'\emph{Assemblée Evangélique de Pentecôte} et du \emph{Mouvement des Jeunes Serviteurs de Dieu} ont régulièrement organisé des conférences de formation sur le leadership à Léo et Ouagadougou. Cette collaboration a engendré un cadre propice et facilité la transmission de l'enseignement qui constitue maintenant le Discipolat Plante de Maïs.

Dans cette dernière étape de développement, des \emph{recherches doctorales}, entreprises parmi des leaders et apprenants Burkinabé, ont conduit à l'élaboration de series de livrets modulaire à bas prix du Discipolat Plante de Maïs. Ce sont des livrets pratiques, pertinents et accessibles pour un usage dans le contexte Africain.

\begin {pause}\begin{description}

\item[Romains 4:17 —]

Abraham est décrit comme celui dont la foi \emph{a appelé à l'existence des choses qui n'existaient pas}. En termes claires le Discipolat Plante de Maïs a \emph{été appelé à l'existence} par la foi des Africains, agissant en réponse aux contextes Africains.
\end{description}

\end {pause}

\section{Recherches Doctorales}
\label{recherchesdoctorales}

En 2010, l'auteur a mené des recherches doctorales au Burkina Faso. Les recherches ont porté sur des thèmes de discipolat, de théologie, de formation en leadership, de littérature et dynamiques interculturelles. Un séminaire typique du Discipolat Plante de Maïs a constitué une partie intégrale de la recherche.

Des questionnaires de sondage, des groupes d'échange et des interviews individuelles ont généré de la substance de plus de 70 participants Burkinabé. Parmi eux, un bon nombre ont déjà occupé des postes de responsabilité au niveau dénominationel dans le cadre de la formation en leadership et discipolat. Une analyse des réponses a révélé une série de trouvailles critiques liées: au Discipolat, à la Théologie, à la Littérature. 

\subsection{Le Discipolat}
\label{lediscipolat}

Les participants à la recherche se sont clairement identifiés au concept du discipolat et, en particulier, la nécessité d'embrasser des pratiques de discipolat holistique et générationnel. En conséquence, Le Discipolat Plante de Maïs met l'accent sur:

\begin{enumerate}
\item L'éveil et le renforcement de l'appel individuel contextuel à servir le but éternel de Dieu

\item Promouvoir l'engagement à l'action missionnaire et aux disciplines, y compris la formation générationnelle des disciples

\item Envisager une transformation personnelle, communale et culturelle.

\end{enumerate}

\subsection{La théologie}
\label{lathologie}

Les participants ont constamment validé le contenu théologique d'une ressource prototype d'apprentissage. De ce fait, le Discipolat Plante de Maïs incorpore:

\begin{enumerate}
\item Une vision holistique du monde, une orientation communale, une spiritualité charismatique

\item Une interprétation historique, missionnaire et alliance de l'Ecriture

\item Les théologies bibliques du discipolat, souffrir et vaincre; le réveil spirituel, la prière d'intercession et la puissance spirituelle; la pauvreté et la prospérité; la vocation individuelle et collective; le leadership-serviteur centré sur Christ et la transformation culturelle.

\end{enumerate}

\begin {pause}\begin{description}

\item[Un éducateur théologique Burkinabé —]

s'étant imprégné du contenu d'un livret prototype du Discipolat Plante de Maïs a déclaré:
\end{description}

\begin{quote}

\emph{Tu touches à quelque chose qui n'existe pas encore. Si nous parlons d'évangélisation, cela peut bien être une nouvelle approche à l'évangélisation, mais nous avons déjà beaucoup de méthodes d'évangélisation. Mais (une série de livres sur) le discipolat est {\ldots} vraiment quelque chose d'innovateur.}
\end{quote}

\end {pause}

\subsection{La littérature}
\label{lalittrature}

Les participants au séminaire ont exprimé un profond désir de voir la mise en place d'une \emph{littérature contextuellement adaptée}. Au Burkina Faso, tout comme en Afrique et une grande partie du monde,leaders et apprenants naviguent généralement entre deux mondes culturels; celui de l'oralité et celui de l'écriture (alphabétisation) respectivement. En conséquence, les livrets du Discipolat Plante de Maïs sont conçus pour servir de pont entre l'oralité et l'écriture

\begin{itemize}
\item Sont focalisés sur le discipolat pratique et non sur des idées théoriques

\item Evitant le langage académique et philosophique

\item Structure thématique et modulaire des études et des sujets abordés

\item Conçus pour être utilisés parmi des groupes cohortes

\item Incorporent de l'apprentissage qui donne matière à refléchir, favorisant des échanges dans le groupe et la mémorisation des écritures

\item Contiennent plusieurs illustrations graphiques et métaphoriques.

\item Publiés selon une philosophie missionnaire, non-commercial

\item Encourageant l'adaptation contextuelle vernaculaire

\item Favorisant la re-publication, la traduction et la vente à bas prix

\end{itemize}

\begin {pause}\begin{description}

\item[Un livret de 52 pages —]

imprimé au recto et au verso sur un format A4, compilé, relié et bien entretenu pour en faire des livrets sur un format A5, d'environ 48 pages, avec une couverture à couleur unique, ne dépassant pas \$2.

\item[Une agence de re-publication —]

de livrets au nom d'un certain nombre d'organisations pourrait être le moyen efficace pour les avoir à moindre coût. Un model économique créatif pourrait donc potentiellement permettre à l'agence de subventionner les livrets pour les personnes les plus démunies.
\end{description}

\end {pause}

\section{La traduction et la re-publication}
\label{latraductionetlare-publication}

Nous espérons voir le Discipolat Plante de Maïs largement traduit, re-publié et adopté pour un usage en Afrique ainsi que dans d'autres parties du monde. Nous sommes réceptifs aux feedback de ceux qui l'ont étudié et l'ont essayé. Dites-nous où, comment et quel effet son usage a produit pour vous. N'hésitez pas non plus à nous faire parvenir vos suggestions par \href{http://maizeplantdiscipleship.info}{http:/\slash maizeplantdiscipleship.info}\begin{description}

\item[Connais-tu des histoires, proverbes et métaphores —]

qui pourront aider à illustrer contextuellement et illuminer sur le plan culturel les concepts de l'enseignement contenu dans le syllabaire du Discipolat Plante de Maïs?

\emph{Si vous nous les envoyez, nous pourrions les introduire dans les prochaines éditions. Il est aussi possible de publier une copie adaptée pourvu que vous respectiez les droits réservés au Discipolat Plante de Maïs}.

\item[La Licence Commune Créative]

reproduit à la fin de ce livret et de chaque livret permet la re-publication et l'adaptation telle que la traduction ou l'inclusion d'éléments contextuels.
\end{description}

\vspace*{\fill}\begin {questions}

\begin{itemize}
\item Est-il important que que les Chrétiens Africains décident de ce qui est approprié sur le plan missiologique pour le contexte Africain?

\item Cela s'est-il déjà produit dans ton contexte? 

\begin{itemize}
\item Si non, pourquoi cela ne l'est-il pas?

\item Si oui, quel changement cela a-t-il produit?

\end{itemize}

\item Quels problèmes missiologiques te semblent-ils importants dans ton contexte? 

\end{itemize}

\end {questions}

\pagebreak 

\section{La Licence du Discipolat Plante de Mais}
\label{lalicencedudiscipolatplantedemais}

Le Discipolat Plante de Maïs de John Clements est enrégistré sous \textbf{Creative Commons Attribution-ShareAlike 4.0 International License}. Basé sur un travail à \href{http://johnbrc.github.io/MPD-Distribution/}{http:/\slash johnbrc.github.io\slash MPD-Distribution\slash }

\begin{quote}

\emph{Note: Ce qui suit est un résumé simplifié (non un substitut) de la licence, que vous pourrez retrouver sur: \href{http://creativecommons.org/licenses/by-nc-sa/4.0/}{http:/\slash creativecommons.org\slash licenses\slash by-nc-sa\slash 4.0\slash } or via \href{http://maizeplantdiscipleship.info/}{http:/\slash maizeplantdiscipleship.info\slash }}
\end{quote}

\begin{summary}

\textbf{Vous êtes libre de}:

\begin{itemize}
\item \textbf{partager} — reproduire et le revendre sous n'importe quel format

\item \textbf{Adapter} — remixer, transformer et l'utiliser comme base de réflexion 

\end{itemize}

\begin{quote}

Le licencié ne peut pas révoquer ces droits tant que vous respectez les termes de la licence.
\end{quote}

\textbf{Sous les termes suivants}:

\begin{itemize}
\item \textbf{Attribution} — Vous devez vous montrer crédible, donnez à la licence un lien, et indiquer si vous avez introduit des changements. Vous pouvez le faire de toute manière raisonnable, mais pas de façon à laisser croire que vous agissez sous la coupe du détenteur de licence.

\item \textbf{Non Commerciale} - Non Commerciale — Vous ne devez pas utiliser le document pour des raisons commerciales, (c.a.d. principalement à but lucratif.)

\item \textbf{Copie conforme} — Quelque soit les transformations, ajouts ou ajustements que vous faites du document, il devra toujours être vendu suivant la même licence que l'original.

\item \textbf{Pas de restrictions supplémentaires} — Vous ne devez pas appliquer des termes légaux ou mesures technologiques qui légalement empêchent les autres de faire ce que la licence leur permet de faire.

\end{itemize}

\end{summary}

\input{mpd-footer}

\end{document}
