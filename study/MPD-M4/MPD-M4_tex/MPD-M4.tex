\def\version{0.5.1 Translation draft}
\def\change{Tweaked, in parallel with FR translation}
\input{mpd-header}
\def\mytitle{Dynamics of Commissioning}
\def\subtitle{Maize Plant Discipleship, Module 4}
\def\myauthor{Dr J.B. Clements}
\def\translation{See translations}
\def\twitter{@johnbrc, @mpdresource}
\def\email{clements.jb@gmail.com}
\def\web{http:/\slash maizeplantdiscipleship.info}
\def\mycopyright{Maize Plant Discipleship by John B Clements is licensed under the Creative Commons Attribution-NonCommercial-ShareAlike 4.0 International License.  To view a copy of this license, visit http:/\slash creativecommons.org\slash licenses\slash by-nc-sa\slash 4.0\slash . Based on a work at http:/\slash johnbrc.github.io\slash MPD-Distribution\slash }
\def\keywords{discipleship, mission, messianic, community}
\def\latexmode{memoir}
\input{mpd-document}
\chapter*{Using this handbook}\label{Usingthishandbook}

This module, no. 4 of 16, explores \emph{The Commissioning of Messianic Community}, in four studies:

\begin{studies}
{\Small \makeatletter\@starttoc{toc}\makeatother}
\end{studies}

\textbf{Maize Plant Discipleship} is an open educational resource, derived and road-tested in collaboration with Africans. It has been formulated in response to contextual doctoral research in Burkina Faso to be practical, relevant and accessible for use in majority-world contexts and is being published as a series of short, modular, low-cost handbooks:

\begin{itemize}
\item suitable for formal and informal modes of study

\item incorporate reflective learning and group discussions,

\item reliant simply upon facilitators co-ordinating small learning groups

\item easily replicable, in terms of both republication and translation.

\end{itemize}

It's goal is to facilitate biblical learning that continuously moves outwards, drawing whole communities into patterns of scripturally-based discipleship, in living dialogue with contextual culture.

\pagebreak 

The \emph{Maize Plant Discipleship Facilitators Handbook}—available from the same source as this handbook or via www.maizeplantdiscipleship.info—contains comprehensive guidelines on appropriately facilitating Maize Plant Discipleship learning groups and discussions.

\begin{syllabus}

Maize Plant Discipleship Syllabus incorporates 16 handbooks.

\begin{enumerate}
\item The Eternal Purpose of God

\item Dynamics of Vocation, The Nations

\item Dynamics of Vocation, The Jews

\item \emph{Dynamics of Commissioning}

\item Dynamics of Body Membership

\item Dynamics of Revival

\item Dynamics of Truth

\item Dynamics of Intercession

\item Dynamics of Cultural Transformation

\item Disciplines of Spiritual Maturity

\item Disciplines of Running the Race

\item Disciplines of Pressing Towards our Vocation

\item Disciplines of Economic Faithfulness

\item Disciplines of Messianic Leadership

\item Disciplines of Living By Faith

\item Disciplines of Overcoming

\end{enumerate}

\end{syllabus}

\chapter{Commissioning of Messianic Community}
\label{commissioningofmessianiccommunity}

\begin{synopsis}

\textbf{The Messianic Covenant Community has been commissioned to work alongside the Messiah in his mission}

\end{synopsis}\begin{topics}

\begin{enumerate}
\item Military commissioning

\item God's commissioning

\item Commissioned as disciples

\end{enumerate}

\pagebreak 

\end{topics}\osection{Terms used in this study}\begin{description}

\item[Commission]

To charge with responsibility for a task or duty, as when a military officer is given a specific rank and responsibility, e.g. \emph{he was commissioned after attending the training academy}

To grant authority to undertake a task or function, as when an architect is authorised to build something, e.g. \emph{the architect was commissioned to manage the project}
\end{description}

\osection{Scripture}\bible

Read these passages aloud; memorise the \textbf{bold} passages.

\begin{itemize}
\item Numbers 27:23

\item \textbf{Matthew 22:14}

\item \textbf{1 Corinthians 9:17}

\item 2 Corinthians 2:17

\item Galatians 1:1

\item Colossians 1:25

\end{itemize}

\section{Military commissioning}
\label{militarycommissioning}

This topic examines the experiences of a military solider, in order to enlarge our understanding of what it means to be \emph{commissioned}. It highlights three principal stages of becoming a soldier: calling, training and commissioning.

\subsection{Calling}
\label{calling}

A soldier's life starts when they perceive a \emph{calling} to a life of military service. A calling is an awareness that a particular occupation represents a desirable, compelling, or appropriate vocation to pursue. There may be many reasons why someone enlists in an army, but at some point they sense a calling towards it.

\subsection{Training}
\label{training}

A soldier's calling is tested with a period of intensive \emph{basic training}, which potentially equips them for a lifestyle of military service. 

Trainee soldiers experience trials and hardships intended to test their discipline, teamwork, communication, competence, obedience, initiative, determination, loyalty and resolve. During this time, each soldier's capabilities and characteristics are either affirmed, enhanced or exposed as inadequate. At the end of this period, training officers assess whether each soldier has satisfactorily completed basic training. \begin{description}

\item[Failure]

means a soldier must undergo basic training again, until passing satisfactorily—or leaving military service altogether.

\item[Success]

means a soldier is equipped and authorised for \emph{active service}: capable, ready and trusted to fulfil their duties, which include general soldiering \emph{and} vocational duties—e.g. infantry, driver, engineer, medic, officer, chef{\ldots}
\end{description}

\subsection{Commissioning}
\label{commissioning}

A soldier's commissioning represents the beginning of their active service, as a member of a specific regiment of the army. Historically, soldiers are commissioned ceremonially by the King or Queen, providing a vivid illustration of the calling to ``serve the crown,'' as a loyal subject.

Once commissioned, soldiers continue to lead highly disciplined lives and to experience demanding trials and tests. However, there are significant differences to the training phase. Failure is no longer an option. Instead, the entire military apparatus upholds and supports commissioned soldiers in their work, providing every opportunity for success. Furthermore, significant achievements are typically rewarded by receiving enhanced or additional responsibilities.

\begin{discuss}[\currentsectiontitle]

\begin{itemize}
\item How do the demands of military soldiering differ from the challenges of civilian life?

\item How helpful is the metaphor of soldiering in your cultural context? In your own life?

\end{itemize}

\end{discuss}

\section{God's commissioning}
\label{godscommissioning}

This topic explores the origins of the word \emph{mission} and its connection with \emph{commissioning}, in order to understand how the messianic, new covenant community is united with God's mission.

\subsection{The sending of God}
\label{thesendingofgod}

The English word, \emph{mission\footnote{From Latin \emph{missio}, meaning \emph{sending}}} was originally used exclusively to refer to God's active sending of himself, into the world, to restore it from the effects of human wickedness, idolatry, chaos, spiritual darkness, oppression, injustice and evil.\footnote{\emph{Missio Dei}---an increasingly widespread theological concept; closely equivalent with \emph{God's eternal purpose} (Module 1).} This is seen in three particular ways. \begin{description}

\item[The Old Testament reveals God's \emph{Shekinah} Presence]

most vividly as a pillar of cloud and fire during the exodus from Egypt, within the Tabernacle of Meeting, above the Ark of the Covenant and at Solomon's dedication of the temple.\footnote{\emph{Shekinah} (Hebrew) refers to the ``Glorious Presence of God Dwelling Amongst His People''—see Exodus 13:17--14:29; Exodus 40; 2 Chronicles 7; also Luke 2:9, Hebrews 1:3; 1 John 1:14, 2 Peter 1:17; Matthew 17:6 etc.}

\item[The New Testament reveals the Messiah, Jesus]

as \emph{Lamb of God, apostle, high priest\footnote{John 1:29; Hebrews 3:1 (4:14--16)}} and \emph{the radiance of the (Shekinah) glory, the very expression of God's essence{\ldots}the visible image of the invisible God}—in every way Jesus the Messiah reflects the reality that God the Father is a missional god.\footnote{Hebrews 1:1--4; Colossians 1:15,19}

\item[The New Testament reveals the Holy Spirit]

as the \emph{Breath\footnote{Hebrew: \emph{ruach}, means breath, or spirit}} of the Messiah, sent by the Messiah, as the Messiah was sent by the Father, the Holy Spirit empowers and \emph{sends} the Messianic Community, anointing it to do the works of God.\footnote{See John 6.28, 14:12--17 and 16:7--11} and to become his living temple.\footnote{1 Corinthians 3:16--17; 1 Peter 2:4--5}
\end{description}

\subsection{The sending of God's people}
\label{thesendingofgodspeople}

Having begun with God's \emph{sending} of himself, mission is enlarged through the biblical covenants, as God's people are united with God and his sending of himself into the world, to restore it.\begin{description}

\item[Commission]

\emph{co} means joining, uniting or sharing, hence, \emph{commission}: to join, or unite with a particular mission.
The covenant community is \emph{commissioned} by God: united with him in his mission to reconcile and restore creation to himself, through the Messiah. This commissioning begins with the covenant community of Israel and is renewed through the Messiah.
\end{description}

\subsection{Joining the Messiah's mission}
\label{joiningthemessiahsmission}

Jesus' mission is the purpose for which he was sent into the world: to represent the Father and to do the works of God, forming and sending disciples, in his name.\footnote{John 16:5, for example} He then calls and commissions (sends) his disciples to complete his mission:

\begin{quote}

\emph{As the Father sent me, so I now send you}—John 20:21
\end{quote}\begin{description}

\item[Messianic commission]

therefore, refers joining, being united with and sharing in the \emph{mission} of the Messiah: representing the Father, joining him in doing the works of God, forming and sending disciples in his name. In this way, the messianic community is prepared, set apart, blessed, anointed and \emph{sent} towards the world, for the sake of the world, \emph{commissioned} to bless the peoples of the world, in God's name.\footnote{See \emph{Module 1, The Eternal Purpose of God}.}
\end{description}

\begin{discuss}[\currentsectiontitle]

\begin{itemize}
\item What is significant about God's sending of himself into the world?

\item What does it mean to be sent as the Messiah was sent?

\end{itemize}

\end{discuss}

\section{Commissioned as disciples}
\label{commissionedasdisciples}

This topic explores parallels between soldiering and messianic discipleship.

\subsection{The metaphor of a soldier}
\label{themetaphorofasoldier}

\begin{quote}

\emph{Take your share of the hardships and suffering which you are called to endure as a good, first-class soldier of Jesus Christ. No soldier when in service gets entangled in the enterprises of civilian life; his aim is to satisfy and please the one who enlisted him}—2 Timothy 2:3--4 ABV\footnote{Modules 10--16 explore the disciplines, hardships and suffering required, as disciples of the Messiah progress in their calling to serve God's eternal purpose.}
\end{quote}

\subsection{Called to serve}
\label{calledtoserve}

\emph{Military commissioning} (\autoref{militarycommissioning}) explored how soldiering begins with discerning a calling to military service. Messianic discipleship similarly begins with a calling. This happens as we personally or corporately discern a calling to serve God's mission, in some way—to serve God's eternal purpose.\footnote{See modules 1--3, incorporating \emph{The Eternal Purpose of God} and \emph{Dynamics of Vocation}.}

\subsection{Many are called, few are chosen}
\label{manyarecalledfewarechosen}

Those who hear and respond to a calling to military service are commissioned only after satisfactorily completing basic training. Jesus' words, \emph{Many are called, but few are chosen}\footnote{Matthew 22:1--14} likewise implies that discerning the call to serve God's mission is only the start.

Like soldiers, disciples of the Messiah also need to learn basic disciplines and specialised skills. Like soldiers, progressing from calling to commissioning requires yielding ourselves fully to the demands of vocational service, in order to become disciplined, skilled, and faithful in our work. 

In this way, the process of \emph{calling, training} and \emph{commissioning} represents the method by which the Messiah prepares his disciples for useful service.\footnote{2 Timothy 2:20--21} It is how we become \emph{co-workers} with the Messiah in his mission. It is how we become \emph{chosen} by him, which means to be appointed and anointed to work alongside him, reaping a \emph{harvest of faithfulness,\footnote{Hebrews 12:11}} in the power of his Spirit.

\begin{discuss}[\currentsectiontitle]

\begin{itemize}
\item What messianic disciplines might be compared to basic military training?

\item How might a messianic disciple fail the equivalent of basic training?

\end{itemize}

\end{discuss}

\ssection{The Commissioning of Messianic Community}

This concludes Study 1, which:

\begin{itemize}
\item explored various aspects of soldiering, including calling, training and commissioning;

\item explained that \emph{commissioning} essentially means joining together in mission, illustrating how the new-covenant community joins with the Messiah's mission;

\item compared military service with messianic discipleship, highlighting similarities relating to obedience, faithfulness, duty and reward.

\end{itemize}

In summary, the study revealed how:

\begin{summary}

\textbf{The Messianic Covenant Community has been commissioned to work alongside the Messiah in his mission: as a community of disciples, called, equipped and chosen to serve God's eternal purpose}

\end{summary}

\vspace*{\fill}

\begin{bonus}

\begin{itemize}
\item How does the \emph{duty} of a commission differ from the \emph{discipline} of basic training?

\item Describe the responsibility implied by the giving and receiving of a commission.

\end{itemize}

\end{bonus}

\chapter{Strategies of Messianic Community}
\label{strategiesofmessianiccommunity}

\begin{synopsis}

\textbf{Five primary strategies empower the mission of the Messianic Community:}

Pray - Reconcile - Disciple - Teach - Send

\end{synopsis}\begin{topics}

\begin{enumerate}
\item The strategy of prayer

\item The strategy of reconciliation

\item The strategy of discipleship

\item The strategy of teaching

\item The strategy of sending

\end{enumerate}

\end{topics}\osection{Terms used in this study}\begin{description}

\item[Strategy]

a plan of action or policy designed to achieve a major or overall aim; military operations and movements in a war or battle.
\end{description}

\osection{Scripture}\bible

Read these passages aloud; memorise the \textbf{bold} passages.

\begin{itemize}
\item Matthew 9:35--38

\item Mark 16:15--18

\item Luke 24:44--49

\item \textbf{John 20:21--23}

\item \textbf{Matthew 28:18--20}

\item Acts 26:15--18

\item Hebrews 5:11--14

\end{itemize}

\section{The strategy of prayer}
\label{thestrategyofprayer}

The foundational strategy of messianic mission is prayer.

\begin{figure}[htbp]
\centering
\includegraphics[width=72pt,height=30pt]{pray.png}
\label{pray.png}
\end{figure}



\subsection{Strategic prayer}
\label{strategicprayer}

There are many kinds of prayer that are not \emph{strategic}. Strategic prayer is a forward-thinking, planned priority, focussed upon a clear purpose. It represents a fundamental commitment to being involved in what God is doing.

\subsection{A plentiful harvest}
\label{aplentifulharvest}

When Jesus proclaimed the kingdom of God in Israel, he compared a ripe, abundant harvest of crops to crowds of distressed and dejected people, who were to him \emph{like sheep without a shepherd.} He said to his disciples:

\begin{quote}

\emph{The harvest is indeed plentiful, but the labourers are few. So pray to the Lord of the harvest to force and thrust out labourers into his harvest}—Matthew 9:35--38 ABV\footnote{Amplified Bible amplifies meaning based on original Hebrew and Greek text}
\end{quote}

By speaking of \emph{the Lord of the harvest} Jesus affirms how significant the harvest is to the Lord. Just as a crop of maize is vitally important to the farmer. The harvest belongs to the Lord!

\subsection{A harvesting problem}
\label{aharvestingproblem}

Although the harvest is plentiful, Jesus identifies a problem: a shortage of workers ready and willing to ingather it. There is a shortage of trained disciples, ready to be commissioned into service. 

How does Jesus teach his disciples to respond to this challenge? Jesus immediately points them towards the most fundamental strategy of mission, instructing them:

\begin{quote}

\textbf{\emph{Pray to the Lord of the harvest!}}

\emph{Pray to the Lord of the harvest{\ldots}to force and thrust out labourers into the harvest field}!
\end{quote}

This is the starting point of strategic mission: focussed prayer, interceding for workers, disciples loyal to Lord of the Harvest to be empowered by the Spirit into wholeheartedly serving God's eternal purpose.

\begin{discuss}[\currentsectiontitle]

\begin{itemize}
\item What makes prayer strategic?

\item What hinders strategic prayer?

\end{itemize}

\end{discuss}

\section{The strategy of reconciliation}
\label{thestrategyofreconciliation}

The second strategic step of messianic mission is reaching people, in order to encourage reconciliation with God, through the Messiah.

\begin{figure}[htbp]
\centering
\includegraphics[width=135pt,height=30pt]{pray-reach.png}
\label{pray-reach.png}
\end{figure}



\subsection{Reconciliation with God}
\label{reconciliationwithgod}

Reaching people implies a holistic process of reconciliation with God and his eternal purpose.\footnote{Acts 26:17--18} This process of reconciliation incorporates: 

\begin{itemize}
\item God's forgiveness of and cleansing from our wrongdoings;

\item forgiving others for wrongs inflicted upon us;

\item deliverance from dominant sinful behaviour;

\item cleansing from spiritual and practical impurity;

\item renunciation of idols and idolatry;

\item establishing a wholehearted devotion to the Messiah

\item inviting the Messiah to rule in our daily lives.

\end{itemize}

Without experiencing the transformative power of reconciliation with God, through the Messiah, people cannot be liberated to serve him as disciples.

\subsection{Personal reconciliation}
\label{personalreconciliation}

Reaching others with a message of reconciliation challenges our own lifestyle and faithfulness. Failing to exhibit kingdom values and priorities corrupts our personal testimony and that of the Messianic Community as a whole. Before reconciling others, we ourselves must first become fully reconciled and submitted to God and his purposes.

\begin{discuss}[\currentsectiontitle]

\begin{itemize}
\item What does it means to be fully reconciled to God and his kingdom purposes?

\item What issues hinder you, your household or family, from personally experiencing peace with God?

\end{itemize}

\end{discuss}

\section{The strategy of discipleship}
\label{thestrategyofdiscipleship}

The third strategy is the formation of disciples.

\begin{figure}[htbp]
\centering
\includegraphics[width=199pt,height=30pt]{pray-reach-disciple.png}
\label{pray-reach-disciple.png}
\end{figure}



\subsection{Discipleship is at the heart of messianic community}
\label{discipleshipisattheheartofmessianiccommunity}

The forming of disciples is the very heart and centre of messianic community and mission. We are called and commissioned to make people from all nations into disciples, by immersing (baptising) them into the reality of God's life, through the Messiah, by the Spirit.\footnote{Matthew 28:19--20}

\subsection{Discipleship deals with our hearts}
\label{discipleshipdealswithourhearts}

Discipleship deals with something deeper than the mind. It deals with our \emph{hearts}: the very centre of our being, the seat of our motivation, our willpower, our commitment. Through discipleship, we are challenged to become \emph{wholeheartedly} aligned with God's eternal and vocational purposes and to serve him as our Lord and Master. Unless our hearts are challenged and changed in this way, we remain mere \emph{religious converts}, engaging in devotional, religious activity, whilst our will, character, allegiances, loyalties and lifestyle remain practically unchanged.

This renewal of our hearts is not a singular, instantaneous event. It is a continual process of encounter, challenge, testing and yielding to the Spirit of the Messiah. Through it our hearts are first opened and then enlarged by the demands of Messianic mission. As we experience this process of formation into disciples of Jesus, we repeatedly face the choice of either resisting and hardening our hearts towards God, or yielding and making more of our hearts available to him. When we allow our hearts to be transformed by the Holy Spirit in this way, we begin to live a new life: as a disciplined co-worker of the Messiah.\footnote{1 Corinthians 3:9}

\subsection{Discipleship is the priority}
\label{discipleshipisthepriority}

Because discipleship deals with the heart, strategically and practically it should \emph{precede} concentrated biblical teaching (\autoref{thestrategyofteaching}), which is for committed disciples who have progressed beyond the elementary teachings.

\begin{discuss}[\currentsectiontitle]

\begin{itemize}
\item To what is your heart devoted? Be honest.

\item What compromises your expression of whole-hearted commitment to serving God's purposes?

\end{itemize}

\end{discuss}

\section{The strategy of teaching}
\label{thestrategyofteaching}

The fourth strategy is teaching.

\begin{figure}[htbp]
\centering
\includegraphics[width=262pt,height=29pt]{pray-reach-disciple-teach.png}
\label{pray-reach-disciple-teach.png}
\end{figure}



\subsection{Solid food}
\label{solidfood}

\begin{quote}

By this time you ought to be teachers, yet you need someone to teach you the very first principles of God's word all over again. You need milk, not solid food! Anyone who has to drink milk is still a baby, lacking experience in applying the Word about covenantal faithfulness. But solid food is for the mature, for those whose faculties have been trained by continuous exercise to distinguish good from evil—\emph{Hebrews 5:12--14 (author's paraphrase)}
\end{quote}

The writer of Hebrews talks about \emph{solid food}, or \emph{strong meat,\footnote{The KJV phrase, \emph{strong meat}, captures well the sense of maturity required to ingest and digest challenging scriptural teaching.}} as a metaphor for the challenging teaching of God's word. He chastises them that they should be teaching others by this time, yet instead they still need \emph{milk}—a metaphor for elementary teaching.\footnote{Hebrews 6:1--3}

Strong meat is for spiritual, mature, committed disciples. For them, biblical teaching represents a fruitful source of insight, conviction, trust, knowledge, wisdom and understanding. It develops and deepens appreciation of messianic life, spirituality and vocational service.

\subsection{The whole counsel of God}
\label{thewholecounselofgod}

Paul refers to \emph{the word of God in its fullness} or \emph{the whole counsel of God.\footnote{Colossians 1:25; also Acts 20:27 ff.}} Experiencing this fullness requires holding together, in appropriate tension, three complementary strands of biblical teaching: pastoral-evangelistic, prophetic and apostolic.

\subsubsection{Pastoral-evangelistic teaching}
\label{pastoral-evangelisticteaching}

Pastoral-evangelistic teaching is directed towards appreciating and applying the accumulated wisdom, knowledge, understanding and traditions of messianic community.\footnote{1 Timothy 3:15} 

\begin{itemize}
\item Calling disciples into \emph{The Way} of the Messiah.\footnote{Acts 9:2; 18:25--26; 19:9,23; 22:4; 24:14,22; cf. John 14:4--6}

\item Systematically studying and interpreting scripture.

\item Exploring church history and denominational identity.

\end{itemize}

Historically, pastoral-evangelistic teaching has dominated Western church traditions—with mixed results. Often, systematic pastoral-evangelistic teaching has produced strong, rooted and active messianic communities. At times, however, dogmatic beliefs have been vigorously, sometimes even violently, used to protect denominational traditions and to justify political power.\footnote{Christian belief has even been used to justify the tyranny of anti-Semitism, colonialism, racism and racial apartheid.} Balance is needed with the spontaneity and cultural relevance of prophetic insight \emph{and} the intercultural outlook of apostolic missionary vision.

\subsubsection{Prophetic teaching}
\label{propheticteaching}

Prophetic teaching is directed towards interpreting historical, contextual \emph{signs of the times,\footnote{Matthew 16:3; see also 1 Chronicles 12:32}} calling for an appropriate response from messianic communities.

\begin{itemize}
\item Responding to God's historical, contextual, missional purposes, amongst the peoples, societies and cultures\footnote{We may think of culture as including the \emph{activities, institutions, knowledge, traditions, values, motivations and thought-processes} of a particular nation or people group.} of the nations.

\item Critiquing the inner culture of messianic communities, organisations (denominations) and structures, in the light of God's missional purpose and historic timing.

\end{itemize}

Historically, prophetic teaching has tended to represent a threat to the status quo of Christian tradition. This can lead to a marginalisation of fresh, prophetic movements and a corresponding deepening of mainstream inertia. At the same time, prophetic movements that cut themselves off from other parts of the Messianic Community can become extreme in their views and practices. Balance is need with both the predictable steadiness of pastoral-evangelistic teaching and the intercultural vision of apostolic teaching.

\subsubsection{Apostolic teaching}
\label{apostolicteaching}

Apostolic teaching is directed towards interpreting the responsibility of messianic communities to look beyond their own sphere of activity and towards other regions, where there are different needs to be served.

\begin{itemize}
\item Offering and relating \emph{the word of life\footnote{Philippians 2:16}} to those outside of messianic community.

\item Preparing personnel for intercultural missionary activity.

\item Appropriately critiquing human culture, in the light of God's Word.

\end{itemize}

The peculiar demands of intercultural ministry can mean that it does not rest easily alongside parochial concerns. As a result pastoral-evangelistic communities and apostolic, missionary agencies can end up operating separately, to the detriment of both. Prophetic insight and teaching can help to bridge the gap between them.

\subsection{Spiritual revelation}
\label{spiritualrevelation}

The \emph{spirit of wisdom and revelation} plays a vital role in messianic teaching and understanding.\footnote{Ephesians 1:16--19} 

\begin{quote}

\emph{This is what we speak, not in words taught us by human wisdom but in words taught by the Spirit, explaining spiritual realities with Spirit-taught words. The person without the Spirit does not accept the things that come from the Spirit of God but considers them foolishness, and cannot understand them because they are discerned only through the Spirit}—1 Corinthians 2:6--16 NIV
\end{quote}

Revelation is a gift of the Spirit that opens our understanding to spiritual truths and realities. The Spirit of God expands our human understanding to incorporate spiritual truths that are not obtained by, or received within, our natural, rational minds. Instead, they are revealed to our spirit.\footnote{1 Corinthians 2:6--16}

A vital key to experiencing wisdom and revelation is the desire to do God's will: Jesus promised that \emph{Anyone who chooses to do the will of God will find out whether my teaching comes from God{\ldots}\footnote{John 7:17}} and \emph{Everyone who asks receives; the one who seeks finds and to the one who knocks, the door will be opened.\footnote{Luke 11:10}}

\begin{discuss}[\currentsectiontitle]

\begin{itemize}
\item How have you responded to God's \emph{strong meat}?

\item How have your responses effected your life?

\end{itemize}

\end{discuss}

\section{The strategy of sending}
\label{thestrategyofsending}

The fifth strategy is sending disciples.

\begin{figure}[htbp]
\centering
\includegraphics[width=292pt,height=26pt]{pray-reach-disciple-teach-send.png}
\label{pray-reach-disciple-teach-send.png}
\end{figure}



\subsection{Culmination of strategies}
\label{culminationofstrategies}

\emph{Sending} represents a culmination of the four strategies that have preceded it: prayer, reconciliation with God, formation of disciples and teaching about the kingdom. God sends the whole messianic community towards all the peoples and cultures of the world both \emph{inter-culturally} and \emph{intra-culturally}.

\subsubsection{Inter-cultural contexts}
\label{inter-culturalcontexts}

Inter-cultural mission implies being sent to significantly different cultural, ethnic and geographical contexts. 

\begin{itemize}
\item Cultural context profoundly affects the translation into practice of messianic identity and vocation, requiring specialist intercultural training, experience and understanding.

\item The responsibility for intercultural mission belongs to the whole messianic community, whilst being the vocation of \emph{apostolic missionary teams} (\autoref{apostolicmissionaryteams}) in particular.

\end{itemize}

\subsubsection{Intra-cultural contexts}
\label{intra-culturalcontexts}

Intra-cultural mission represents the sending of the messianic community towards the people of its resident cultural context.

\begin{itemize}
\item Intra-cultural sending implies identifying and co-operating with the Messiah's purposes \emph{wherever} we are, whatever our vocational role—including participation in industry, commerce, arts, sports, media, health, civil and other government services. 

\item Every cultural contexts need messianic disciples capable of bringing God's blessing into the personal and vocational contexts of home, family, community, workplace and organisation.

\end{itemize}

\subsection{Sent by the Spirit}
\label{sentbythespirit}

Whether inter- or intra-culturally, God is the one who equips, empowers and sends workers, by his Spirit. When a particular community sends workers into any context, it needs to do so in harmony with the Holy Spirit, upholding the Lord's sending of those people.\footnote{Acts 13:1--4 provides an example of the Messianic Community and the Holy Spirit acting in harmony together, in sending Paul and Barnabas on a missionary journey.}

\subsection{A cyclical process}
\label{acyclicalprocess}

As each generation of disciples is prepared, taught and sent, they carry the responsibility to renew the whole process in their own context. This generational impetus is integral to the entire strategic process. It generates a cyclical process that is capable of establishing an expanding, missional movement.

\begin{figure}[htbp]
\centering
\includegraphics[width=239pt,height=250pt]{commissioncycle.png}
\label{commissioncycle.png}
\end{figure}



\begin{discuss}[\currentsectiontitle]

\begin{itemize}
\item What does it mean to be sent? Is it something that happens once, regularly, or continuously?

\item Discuss the role of the Holy Spirit in sending the whole Messianic Community.

\end{itemize}

\end{discuss}

\ssection{The Strategies of Messianic Community}

This concludes Study 2, which explored five foundational, missional strategies: strategic prayer, reconciliation with God, formation of disciples, teaching the spiritually mature and sending people into mission. 

In summary, the study revealed how:

\begin{summary}

\textbf{Scripture reveals a cycle of five major strategies empowering the mission of the Messianic community that may be summarised: Pray, Reach, Disciple, Teach, Send}

\end{summary}

\vspace*{\fill}

\begin{bonus}

\begin{itemize}
\item Which of the five strategies in this study are being enacted most effectively in your context?

\item Which strategies are the least effective in your context? Why might that be?

\end{itemize}

\end{bonus}

\chapter{Structures of Messianic Community}
\label{structuresofmessianiccommunity}

\begin{synopsis}

\textbf{The messianic community has a God-ordained structure that uniquely equips it to fulfil the messianic commission.}

\end{synopsis}\begin{topics}

\begin{enumerate}
\item Pastoral, evangelistic communities

\item Teaching and training centres

\item Apostolic missionary teams

\item Prophets, priests, mediators

\item Structures and strategies

\end{enumerate}

\end{topics}\osection{Terms used in this study}\begin{description}

\item[Structure]

referring to vocational communities structured (organised) according to their purpose and function. 

\emph{Note: In the following sections, various messianic structures are represented by circles—and a combination of overlapping circles.}

\item[Pastoral]

from \emph{pastor,} meaning shepherd—implying care, protection, provision, discipline and guidance—as a good shepherd with his sheep.

\item[Evangelistic]

from \emph{evangel,} meaning \emph{Good News}—the proclamation that the Messiah, Jesus, is Lord, especially of his covenant community.

\item[Prophetic]

from \emph{prophet}—those appointed by God, to speak to human beings on behalf of God \emph{and} to God, on behalf of human beings.

\item[Apostolic]

from \emph{apostle,} meaning \emph{sent one}; referring to those sent as intercultural, missionary pioneers.
\end{description}

\osection{Scripture}\bible

Read these passages aloud; memorise the \textbf{bold} passages.

\begin{itemize}
\item Exodus 26:30

\item 2 Corinthians 5:20--21

\item Acts 13:1--4

\item Acts 19:8--11

\item \textbf{Ephesians 4:11--13}

\item \textbf{Hebrews 8:5}

\end{itemize}

\pagebreak 

\section{Pastoral, evangelistic communities}
\label{pastoralevangelisticcommunities}

\begin{figure}[htbp]
\centering
\includegraphics[width=125pt,height=125pt]{pastoral-evangelistic-community.png}
\label{pastoral-evangelistic-community.png}
\end{figure}


Pastoral evangelistic communities are defined by their dual pastoral \emph{and} evangelistic character.\footnote{Pastoral, evangelistic community is expressed contextually as \emph{local church}.} \begin{description}

\item[Pastoral]

reflects the affirming, selfless, humble, protective, overseeing care of the Good Shepherd, Jesus.

\item[Evangelistic]

reflects the Lordship of Jesus over the community itself \emph{and} the spiritual and natural powers that influence and shape human society.
\end{description}

This pastoral-evangelistic character is most visibly expressed through \emph{faithfulness}, \emph{hospitality} and \emph{celebration}.

\subsection{Community faithfulness}
\label{communityfaithfulness}

Community faithfulness reflects a unity of trust in and faithfulness towards the Messiah: a shared expression of confidence in the \emph{evangel}—the Good News—of what God has done for all peoples, through his Messiah.

\begin{itemize}
\item Messianic faith is most powerfully demonstrated through love.\footnote{See John 13:35; Galatians 5:6; 1 John 3:14 \& 4:20} Sharing one another's concerns and burdens is a vital aspect of trusting in and proclaiming our faithfulness to the Messiah. Love-in-action provides a profound, living demonstration of the Messiah's victory over human self-centredness.\footnote{Galatians 5:6}

\end{itemize}

\subsection{Community hospitality}
\label{communityhospitality}

Community hospitality offers friendliness, kindness, warmth, welcoming, care, openness, acceptance and concern, most especially for those who are aliens, strangers, outsiders.\footnote{Deuteronomy 10:19; cf. 1 Peter 2:11--12} The root meaning of hospitality is to host others.

\begin{itemize}
\item \emph{Being an inviting, hospitable community}, however, implies more than simply offering invitations to events. On a practical level, it means opening our hearts and our homes to one another. Above all, it means expressing messianic faith and love in a manner that \emph{invites the interest and involvement} of outsiders.\footnote{Colossians 4:6; Philippians 2:4} 

\item The ultimate invitation to outsiders is join the messianic community in giving the Messiah wholehearted allegiance, which incorporates worship, service and faithfulness.

\end{itemize}

\subsection{Community celebration}
\label{communitycelebration}

Community celebrations express honour, gratitude and commitment towards God \emph{and} the public commemoration of his goodness and love towards his creation.

\begin{itemize}
\item Covenant feasts, prescribed in the \emph{Torah,\footnote{The first five books of the Bible, accredited to Moses and forming the covenantal foundation of the nation of Israel.}} incorporate prophetic signs, pointing towards both the good things that God has done for his people \emph{and} his call to faithful service. The annual \emph{Passover} is the most significant Hebraic celebration. 

\item The new covenant, inaugurated by Jesus, fulfils the Passover: the sharing of bread and wine, representing the body and blood of the Messiah, speaks of the Passover \emph{Lamb of God}\footnote{John 1:29}, who sacrificed his life to serve God's eternal purpose.

\item Celebrating this sacrifice, by sharing \emph{daily bread} with one another, reminds us of the devoted, sacrificial service to which we are called and commissioned, as members of the messianic community.

\end{itemize}

\subsection{Dual characteristics, one community}
\label{dualcharacteristicsonecommunity}

Clearly, pastoral and evangelistic characteristics of messianic communities overlap each other. It is true to say that the two things cannot be separated: when a community is living a faithful, devoted, celebratory lifestyle that is hospitable, open, welcoming and inviting towards outsiders, its existence represents both a practical \emph{embodiment} and a living \emph{proclamation} of the Good News.\footnote{1 Peter 2:12}\begin{description}

\item[Pastors and evangelists]

are responsible for equipping messianic communities to express their pastoral and evangelistic character. They are principally called to be inspirers, facilitators, catalysts and equippers, so that \emph{whole communities}, of all ages, abilities and types, are equipped to mutually support one another with hospitality and pastoral care \emph{and} to share faith confidently with others, beyond the Messianic Community.
\end{description}

\tsection{Discipling into allegiance}

\begin{figure}[htbp]
\centering
\includegraphics[width=125pt,height=125pt]{disciplingallegiance.png}
\label{disciplingallegiance.png}
\end{figure}



Pastoral, evangelistic communities are called to be a fellowship of people learning to share their lives and values in ways that practically express both the \emph{Shepherding} and the \emph{Lordship} of Jesus, so that the whole community is working together towards a shared, primary goal of:

\begin{summary}

\textbf{Discipling people into faithful allegiance to God's Messiah, facilitating and encouraging significant spiritual and practical expressions of faithfulness, in homes, workplaces and communal places}

\end{summary}

\begin{discuss}

\begin{itemize}
\item How do people express and experience alliances and allegiance to families, tribes, sports teams, nation? 

\item How is being \emph{in alliance} with the Messiah significant?

\end{itemize}

\end{discuss}

\section{Teaching and training centres}
\label{teachingandtrainingcentres}

\begin{figure}[htbp]
\centering
\includegraphics[width=125pt,height=125pt]{teachingtrainingcentres.png}
\label{teachingtrainingcentres.png}
\end{figure}


Teaching and training centres supplement the formation of disciples taking place within pastoral, evangelistic communities (\autoref{pastoralevangelisticcommunities}). Their function is to equip mature messianic disciples for missional, vocational service, in both \emph{intra-cultural} and \emph{inter-cultural} contexts, between which there are significant differences, yet which together signify the full scope of messianic mission: God's servant community, blessed to be a blessing to \emph{all the peoples} of the world.

\subsection{Intra cultural contexts}
\label{intraculturalcontexts}

Intra-cultural teaching and training equips messianic disciples and communities to live faithfully within their own culture, amongst their own people. 

\begin{itemize}
\item Researching, understanding, presenting and explaining the content of God's Word, in order to equip disciples with a \emph{messianic worldview}—a way of understanding and relating to the world with a biblical, messianic perspective.

\item Edifying—encouraging, strengthening and correcting—the practices and self-understanding of messianic communities, enabling them to become \emph{pillars and foundations of truth},\footnote{1 Timothy 3:15} in the context of cultures shaped by different spiritual and moral values.

\item Typical examples: Bible schools, conferences, seminars, workshops.

\end{itemize}

\subsection{Intercultural contexts}
\label{interculturalcontexts}

Intercultural teaching and training equips disciples to live faithfully and effectively in relationship with people of a different culture. 

\begin{itemize}
\item Preparing disciples with spiritual confidence and equipping them with practical resources to undertake \emph{apostolic missionary work} in non-native contexts.

\item Researching, presenting, explaining, understanding the \emph{worldviews} of people from other cultures and religions.

\item Typical examples: Scripture translation, language learning, cross-cultural training, missionary trips.

\end{itemize}

\subsection{Teachers and trainers}
\label{teachersandtrainers}

The role of messianic teachers and trainers is critical to the formation of mature disciples with a faithful understanding of Scripture, appropriately equipping them for works of service.\begin{description}

\item[Training]

tends to emphasise learning from the experience of others, encouraging learners to be responsive, effective and accountable to cultures and systems encountered in \emph{specific} contexts.

\item[Teaching]

tends to emphasise the value of knowledge and understanding, making learners responsible for evaluating, internalising and utilising knowledge appropriately, in \emph{variable} or \emph{multiple} contexts.
\end{description}

\tsection{Equipping with truth}

\begin{figure}[htbp]
\centering
\includegraphics[width=125pt,height=125pt]{equippingtruth.png}
\label{equippingtruth.png}
\end{figure}



The characteristic role of messianic teaching and training centres is to supplement the formation of disciples taking place within pastoral, evangelistic communities, by:

\begin{summary}

\textbf{Equipping mature disciples with biblical truth, enabling them to fulfil personal, vocational callings, in a manner that expresses faithful allegiance to the Messiah, in a range of cultural contexts}.

\end{summary}

\begin{discuss}

\begin{itemize}
\item How does context effect our understanding of mission?

\item How are you passing on your experience and knowledge?

\end{itemize}

\end{discuss}

\pagebreak 

\section{Apostolic missionary teams}
\label{apostolicmissionaryteams}

\begin{figure}[htbp]
\centering
\includegraphics[width=125pt,height=125pt]{apostolicmissionaryteams.png}
\label{apostolicmissionaryteams.png}
\end{figure}


New Testament missionary apostle, Paul, employs two significant metaphors to illustrate the function of apostles and their teams: \emph{architect} and \emph{ambassador}.\begin{description}

\item[Architects]

are responsible for both designing and supervising the construction of buildings, which others will inhabit and use.

\item[Ambassadors]

are internationally-accredited diplomats, or emissaries, sent by a country as its official representative to a foreign country.
\end{description}

\subsection{Architect : laying a messianic foundation}
\label{architect:layingamessianicfoundation}

Paul compares the apostolic missionary role to that of a skilful architect, or \emph{master builder}, laying a Messianic foundation.\footnote{I Corinthians 3:10--15} Thus, apostles and apostolic missionary teams are responsible for pioneering the formation and establishment of pastoral, evangelistic communities (\autoref{pastoralevangelisticcommunities}) and messianic teaching and training centres (\autoref{teachingandtrainingcentres}). These messianic structures are then handed over to local, contextual leaders, as the apostle moves on to other fields of work.

Because apostolic work is laying a foundation upon which others will build, its quality is crucial to the future of the messianic community in those settings. Apostolic work is especially critical to intercultural contexts.

\subsection{Ambassador : acting as the Messiah's representative}
\label{ambassador:actingasthemessiahsrepresentative}

Paul compares to the responsibility of representing the Messiah amongst people of other nations or cultures, as that of \emph{ambassadors of the Messiah}.\footnote{2 Corinthians 5:20--21} Thus, apostles and apostolic missionary teams cross geographical and cultural boundaries in order to represent and to pioneer messianic community amongst people and places where there are no, few or waning gospel communities.

Living and working inter-culturally, in non-native contexts, places significant additional demands upon workers, because of differences encountered in a whole range of experiences, including language, climate and food; political, economic and bureaucratic systems; customs, social expectations and religious sensibilities.

\subsection{First in the church}
\label{firstinthechurch}

Through the \emph{pioneering} roles of ambassador and architect, apostles lay a crucial foundation for the building of the whole messianic community. For this reason the apostolic ministry is considered \emph{first} in the Messianic Community.\footnote{1 Corinthians 12:28} 

Yet for all the reasons discussed, apostolic work is especially demanding for practitioners. Thus Paul, who personally experienced many trials and tribulations, identifies his apostolic, missionary service with being \emph{put on display at the end of the procession}.\footnote{2 Corinthians 4:7--12 \& 6:3--10; 1 Corinthians 4:9--13} He is referring to the apostolic missionary's need to risk humiliation and all manner of difficult challenges, for the sake of furthering the message of the Messiah.

Through their sacrificial dedication, endurance of suffering, embrace of humility and deep-seated reliance upon the power of the Holy Spirit, apostolic workers typically provide a profound example to the whole Messianic Community, of our \emph{shared calling} to faithful, missional service.

\tsection{Pioneering with power}

\begin{figure}[htbp]
\centering
\includegraphics[width=125pt,height=125pt]{pioneeringpower.png}
\label{pioneeringpower.png}
\end{figure}



The characteristic role of \emph{apostolic missionary teams} is to lay a messianic foundation, by:

\begin{summary}

\textbf{Skilfully, resolutely, purposefully, sacrificially pioneering in the power of the Spirit: planting and establishing pastoral, evangelistic communities and teaching and training centres, in new geographical and cultural contexts}

\end{summary}

\begin{discuss}

\begin{itemize}
\item What is your most significant experience of intercultural work?

\item Which qualifications—gifts, talents, characteristics—are \emph{most} needed by apostolic missionary workers?

\end{itemize}

\end{discuss}

\section{Prophets, priests, mediators}
\label{prophetspriestsmediators}

Of the five leadership gifts,\footnote{Ephesians 4:11--12} pastor, evangelist, teacher and apostle are each represented by one or other of the three principal structures examined in Topics 1--3. This topic explores how the \emph{prophetic} ministry functions in relation to other messianic structures.

\subsection{Heart of Messianic Community}
\label{heartofmessianiccommunity}

Overlaying the three circles representing each of the three foundational messianic structures, creates a central region where all three overlap. This region may be identified as representing the \emph{prophetic heart} of messianic community.

\begin{figure}[htbp]
\centering
\includegraphics[width=225pt,height=250pt]{propheticmediatoryrole.png}
\label{propheticmediatoryrole.png}
\end{figure}



The idea of a central heart, interacting with and influencing each of the other ministry structures thus provides a profound metaphor for the prophetic ministry. This suggests the prophetic role is vital, influential and central, even though it functions in a somewhat hidden, non-structural manner.

Interestingly, Paul describes the foundational significance of the prophet as second only to that of the apostle.\footnote{1 Corinthians 12:27--31} When embraced, it has a profound capacity to spearhead deep, spiritual renewal and cultural impact. Yet, perhaps partly because of its somewhat hidden way of functioning, the prophetic role and ministry can easily become overlooked or misunderstood.

\subsection{God's heartbeat}
\label{godsheartbeat}

The prophetic role may be thought of as \emph{listening to and discerning God's heart}.\footnote{See \emph{Module 8: The Dynamic of Intercession}, for further discussion of this concept.} This implies a significant level of spiritual intimacy, enabling prophets to:

\begin{itemize}
\item walk closely with God, in order to discern the thoughts, feelings and intentions upon his heart;

\item share in the intercessory ministry of the Spirit;

\item discern and share God's \emph{kairos\footnote{\emph{Kairos} (Greek) refers to a particularly opportune, favourable, suitable or appropriate moment--e.g. see John 7:6--8, 12:23; Luke 21:13; Mark 13:4; Acts 1:6--7; 1 Timothy 2:6; \emph{kairos} contrasts with \emph{chronos}, which refers to fixed, measurable units of time.}} word for a particular context.

\end{itemize}

Accordingly, those gifted to walk prophetically with God are often sensitive personalities—including poets, artists, writers, visionaries and other kinds of imaginative, inventive or creative individuals. This factor, combined with its relatively obscure, non-structural mode of functioning, may lead to such people being misunderstood, reacted to, dismissed and under-appreciated—with a corresponding loss to the functioning of the whole body.

\subsection{Dual mediatory role}
\label{dualmediatoryrole}

The prophetic ministry fulfils a dual mediatory role, incorporating both \emph{priestly} and \emph{prophetic} mediation. \begin{description}

\item[Priestly mediation]

involves speaking with God, as a representative of the Messianic Community.

It is based upon a purity of heart and faithfulness of life that enables practitioners to offer effective intercession on behalf of others.\footnote{James 5:13--20; Psalm 24:3--4; also \emph{Module 8: The Dynamic of Intercession}.} It involves (i) sharing the intercession of the Holy Spirit, through a lifestyle of intercessory prayer\footnote{Romans 8:26--27} and (ii) advocating and encouraging the Messianic Community to fulfil its calling as \emph{a house of prayer for all nations}.\footnote{Matthew 21:13, c.f. Isaiah 56:7} 

\item[Prophetic mediation]

involves speaking with the Messianic Community, as a representative of God.

It is based upon the prophetic gift's capacity for \emph{hearing what the Spirit is saying to the Messianic Community}.\footnote{Revelation 2:7,11,17,29,3:13,22; also Matthew 11:15, Mark 4:9, Luke 8:8 etc.} It involves (i) providing discernment, direction and insight, regarding contextual and historical challenge, opportunity and responsibility and (ii) providing exhortation, clarification, even rebuke, when messianic communities are failing in their vocational calling.
\end{description}

\tsection{Picking up God's heartbeat}

\begin{figure}[htbp]
\centering
\includegraphics[width=200pt,height=233pt]{pickingupgodsheartbeat.png}
\label{pickingupgodsheartbeat.png}
\end{figure}



The prophetic role is essentially a mediatory role, requiring a sensitive, intimate, faithful walk with God that enables prophets to:

\begin{summary}

\textbf{Pick up God's heartbeat for the peoples of the world, interceding with the Spirit for the purposes of God and hearing and conveying what the Spirit is saying to the Messianic Community}

\end{summary}

\begin{discuss}

\begin{itemize}
\item How do you respond to the idea that it is possible to discern what is upon God's heart?

\item How might the \emph{non-structural} aspect of the prophetic ministry effect the confidence of people called to function prophetically?

\end{itemize}

\end{discuss}

\section{Whole body working together}
\label{wholebodyworkingtogether}

This topic explores structural and spiritual interaction between pastoral-evangelistic community, teaching and training centres, apostolic missionary teams and prophetic mediators.

\subsection{Structures and strategies}
\label{structuresandstrategies}

\emph{Strategies of Messianic Community} (\autoref{strategiesofmessianiccommunity}) illustrated and explored five foundational strategies: pray, reach, disciple, teach, send. These five strategies correspond closely with the messianic structures explored in this study. This correspondence is tabulated, below.

\begin{table}[htbp]
\begin{minipage}{\linewidth}
\setlength{\tymax}{0.5\linewidth}
\centering
\small
\caption{}
\label{table1}
\begin{tabulary}{\textwidth}{@{}RCL@{}} \toprule
\textbf{Strategy}&\textbf{Messianic structure}&\textbf{Function}\\
\midrule
pray&prophetic mediatory role&picking up God's heartbeat\\

\midrule
reach, disciple&pastoral, evangelistic communities&discipling into allegiance\\

\midrule
teach&teaching and training centres&equipping with truth\\

\midrule
send&apostolic missionary teams&pioneering in power\\

\bottomrule

\end{tabulary}
\end{minipage}
\end{table}


These correspondences are not intended to be definitive or exclusive. They simply illustrate how different parts of the body contribute significantly to the diverse functioning of the whole body, so that the Messianic Community is able to function according to God's unique plan, purpose and vocation:

\begin{quote}

You are{\ldots}members of the household of God, built upon the foundation of the apostles and prophets, with Christ himself as the cornerstone. In him the whole structure is joined together and grows into a holy temple in the Lord, in whom you also are being built spiritually into a dwelling place for God—\emph{Ephesians 2:19b--22 ABV}
\end{quote}

\subsection{Structural interactions}
\label{structuralinteractions}

\begin{figure}[htbp]
\centering
\includegraphics[width=207pt,height=200pt]{structuralinteractions.png}
\label{structuralinteractions.png}
\end{figure}


This figure illustrates how the various messianic structures can interact with the other parts. For example, apostolic missionary teams planting pastoral-evangelistic communities and establishing teaching centres \emph{and} pastoral-evangelistic communities sending apostolic missionary teams and partnering with teaching and training centres.

As messianic structures learn to function faithfully and to interact dynamically with one another in accordance with God's strategic design and purpose an effective missional, discipleship movement becomes established.

\begin{discuss}

\begin{itemize}
\item Briefly discuss each of the identified interactions, between the three foundational structures. Which of these structural dynamics are most significant? Why?

\item Which is presently \emph{least evident} in your contexts?

\end{itemize}

\end{discuss}

\ssection{The Structures of Messianic Community}

This concludes Study 3, which examined the foundational structures of messianic community, including their relation to the five foundational gifts of pastor, evangelist, teacher, apostle and prophet and the five foundational strategies of pray, reach, disciple, teach, send.

In summary, the study revealed how:

\begin{summary}

\textbf{The messianic community has a God-ordained structure that uniquely equips it to fulfil the messianic commission}

\end{summary}

\vspace*{\fill}

\begin{bonus}

\begin{itemize}
\item How does the prophetic ministry complement the interactions between the three messianic structures?

\item How might the three foundational structures explore responding towards prophetic gifts and insights?

\end{itemize}

\end{bonus}

\chapter{Expansion of Messianic Community}
\label{expansionofmessianiccommunity}

\begin{synopsis}

\textbf{The Messianic Community is intended to be a continually multiplying movement of disciples, constantly expanding into all the world}

\end{synopsis}\begin{topics}

\begin{enumerate}
\item Honey bees

\item Maize plant

\item Commercial organisations

\item Missional movements

\end{enumerate}

\pagebreak 

\end{topics}\osection{Terms used in this study}\begin{description}

\item[Increase]

referring to numerical growth

\item[Expansion]

referring to geographical enlargement
\end{description}

\osection{Scripture}\bible

Read these passages aloud; memorise the \textbf{bold} passages. \emph{Notice how each passage summarises a significant period of missional activity, during which early messianic community established an expanding movement of disciples.}

\begin{itemize}
\item \textbf{Acts 1:8}

\item Acts 2:42--47

\item Acts 6:7

\item \textbf{Acts 9:31}

\item Acts 12.24

\item Acts 16.5

\item Acts 19.20

\end{itemize}

\section{Honey bees}
\label{honeybees}

A bee colony is a kind of \emph{collective organism}, because individual bees cannot survive very long outside of a colony and, although there are clearly defined \emph{roles}, there is no hierarchy or leadership. A bee colony may contain between 2,000 and 60,000 bees, including: 

\begin{itemize}
\item a single fertile \emph{queen bee}

\item a few thousand fertile male \emph{drone bees}

\item several thousand non-fertile female \emph{worker bees}.

\end{itemize}

\begin{figure}[htbp]
\centering
\includegraphics[width=99pt,height=135pt]{swarm-bw.png}
\caption{Bee swarm}
\label{swarm-bw.png}
\end{figure}



\subsection{Increase}
\label{increase-bees}

A colony grows as workers raise thousands of new bees, born to the queen. Usually, around a queen's second springtime, a colony will prepare to \emph{swarm}. In readiness for swarming, worker bees begin preparing new virgin queen bees, one of which will take over the existing hive, by killing all of the other virgin queens, after the old queen leaves with the swarm.

\subsection{Division}
\label{division}

When it is time for the swarm to leave the hive, \emph{scout} bees will find a suitable place for the swarm to gather initially and report this location to the colony. Shortly afterwards, about 6 out of every 10 worker bees in the colony—usually the the most vigorous ones—swarm around the queen bee. The swarm then leaves the hive altogether, moving directly to the scouted location.

\subsection{Expansion}
\label{expansion-bees}

Scout bees must swiftly identify a suitable, permanent hive location, so the swarm can form a new colony. This step is critical since swarming bees can survive on the honey in their stomachs for only 2--3 days. Once a new hive is settled, the cycle of growth begins again. The old queen may not live long and must quickly start the process of repopulating the colony, including producing new virgin queens, ready to take on her role.

\tsection{Honey bees}

\begin{figure}[htbp]
\centering
\includegraphics[width=271pt,height=45pt]{division.png}
\label{division.png}
\end{figure}



\begin{summary}

Organic, steady, cyclical \emph{increase} in bee numbers leads to swarming and eventually a \textbf{division} of the colony, which then \emph{expands} into a different geographical region.\begin{description}

\item[Risks]

— swarming divides the colony, temporarily weakening both groups.

\item[Benefits]

— bees produce, from plant pollen, an abundance of honey that is used to feed the growing colony; honey provides a rich food source for animals and a harvestable crop for humans.
\end{description}

\end{summary}

\begin{discuss}

\begin{itemize}
\item What is a collective organism?

\item If bee colonies are a metaphor for messianic community, what can we learn from their characteristics?

\end{itemize}

\end{discuss}

\section{Maize plant}
\label{maizeplant}

The maize plant is a rapidly-reproducing cereal crop, with a leafy stalk, typically growing two or more metres high. Maize is grown all over the world, feeding millions of people every day. Because of its shallow roots, maize is susceptible to drought, intolerant of poor soils and prone to be uprooted by severe winds.

\begin{figure}[htbp]
\centering
\includegraphics[width=53pt,height=200pt]{mp-sketch-bw.png}
\label{mp-sketch-bw.png}
\end{figure}



\subsection{Increase}
\label{increase-maize}

Maize plants grow from seeds sown into the ground. Growth begins when the seed's hard, outer shell breaks open, allowing the soft, inner kernel to access the moisture and nutrients within the soil. It immediately sprouts roots and a single stem that moves upwards, towards the surface of the soil. 

Once through the surface, the plants leaves can begin photosynthesising sunlight, while its roots continue drawing on soil nutrients and moisture. In arid locations, typical of Africa, the most significant growth factor is a sufficiency of rainfall.

\subsection{Multiplication}
\label{multiplication}

Each maize plant produces a number of ears, each of which typically contains 600--800 seeds. Thus, a single season of growth can produce thousands of seeds from a single plant.

\subsection{Expansion}
\label{expansion-maize}

Maize plants expand into new areas through redistribution of their seeds. Because of the way the plant has been cultivated, over hundreds of years, the intervention of farmers is required to effectively distribute seeds.

\tsection{Maize plant}

\begin{figure}[htbp]
\centering
\includegraphics[width=267pt,height=45pt]{multiplication.png}
\label{multiplication.png}
\end{figure}



\begin{summary}

Organic \emph{increase} in plant growth potentially produces a thousand-fold \textbf{multiplication} of seeds, often twice a year, allowing rapid crop \emph{expansion}, depending upon climate and soil conditions.\begin{description}

\item[Risks]

— shallow roots make plants susceptible to poor soils, drought and severe winds.

\item[Benefits]

— maize is a valuable crop, suitable for use in a variety of foodstuffs.
\end{description}

\end{summary}

\begin{discuss}

\begin{itemize}
\item How are maize plants dependent on farmers for reproduction?

\item If maize plants are a metaphor for messianic community, what can we learn from their characteristics?

\end{itemize}

\end{discuss}

\section{Commercial organisations}
\label{commercialorganisations}

Commercial, business organisations, including many charitable enterprises, seek to achieve \emph{economic growth}, using profit and loss accounts as a primary indicator of success and failure. 

Rewards associated with commercial growth typically produces highly competitive, demanding environments. This can lead to extraordinary performance, yet also to a loss of human dignity.

\begin{figure}[htbp]
\centering
\includegraphics[width=157pt,height=160pt]{commerce.png}
\label{commerce.png}
\end{figure}



\subsection{Increase}
\label{increase-orgns}

Organisational growth is marked by an increase in commercial measurements such as turnover, profit, employees, management experience, production capacity, market size and share and so on. At some point, most organisations reach a limit to their growth in one or other of these areas. Further growth is (usually) sought by duplicating the number of outlets (shops, offices, factories etc.) owned or managed by the organisation.

\subsection{Duplication}
\label{duplication}

Duplication essentially reproduces, in a new location, a \emph{copy} of an existing, successful model. This builds upon proven characteristics of the original concept, increases production and establishes an identifiable brand. Buildings, budgets, payrolls and competition typically play a significant role in decision-making processes, usually controlled closely by a centralised, hierarchical, \emph{command-and-control} structure of management.

\subsection{Expansion}
\label{expansion-orgns}

Regional location, form, speed and costs of expansion are usually determined by a central, organisational strategy. If a new structure does not function according to expectations, it may be closed down—without reference to local, contextual concerns.

\tsection{Commercial organisations}

\begin{figure}[htbp]
\centering
\includegraphics[width=267pt,height=45pt]{duplication.png}
\label{duplication.png}
\end{figure}



\begin{summary}

Logistical growth, measured by \emph{increase} in profits, personnel and management experience, leading to \textbf{duplication} of structures (premises and management hierarchy) and geographical \emph{expansion} of the market being served.\begin{description}

\item[Risks]

— duplication typically ignores or suppresses local insight and initiative and the effect of contextual differences on the establishment of new structures.

\item[Benefits]

— successful duplication leads to increased profits and more managers.
\end{description}

\end{summary}

\begin{discuss}

\begin{itemize}
\item What benefits and risks are associated with the principles of commercial organisations?

\item If commercial organisations are a metaphor for messianic community, what can we learn from their characteristics?

\end{itemize}

\end{discuss}

\section{Missional movements}
\label{missionalmovements}
\begin{description}

\item[A people movement]

is an informally-organised grouping of people and organisations, dedicated to achieving shared political, social, or artistic ideas, ideals and goals.

\item[A missional movement]

is an informally-organised grouping of people and organisations, dedicated to serving God's eternal purpose.
\end{description}

\subsection{Increase}
\label{mvmts}

Missional movements grow through the forming of disciples who make disciples, who make disciples—as discussed in [\emph{The Commissioning of Messianic Community}] and [\emph{The Structures of Messianic Community}]. Disciples do more than simply \emph{believe} in missional values: they \emph{embody} core beliefs and values by realigning their lives, in order to affect their own contexts and futures—including making disciples {\ldots} who make disciples {\ldots} who make disciples and so on.

\subsection{Reproduction}
\label{reproduction}

Within missional movements, as in nature, not all reproduction succeeds. Some individuals and groups fail to mature. Others reach maturity, but don't reproduce. Some start slowly, others rapidly. Some groups evolve a different sense of identity, purpose or form to that of their originating contexts. 

Reproduction in a missional movement is never merely a numerical eventuality. Discipleship is ultimately about a reproduction of the life of Jesus, with a community of people, by the power of his Spirit.

\subsection{Expansion}
\label{mvmts}

Ultimately, the Holy Spirit is responsible for governing the wild, haphazard growth of a missional, discipling movement capable, over time, of impacting communities, socities and cultures.\footnote{The wind breathes where it wills; and though you hear its sound, yet you neither know where it comes from, nor where it is going. So it is with everyone born of the Spirit—John 3:8 ABV} It is the Holy Spirit who takes hold of a community, implanting a missional vision larger than that of any individual or single organisation. It is the Holy Spirit who is able to direct where and how our energies are applied.\footnote{Witness the role of the Holy Spirit, throughout the book of Acts, e.g. see Scripture readings, at start of Study.}

Throughout the world, within every people group, wherever willing hearts and available hands are found, the Holy Spirit may be found working to establish a community that manifests God's heart. The \emph{dynamics} and \emph{disciplines} set out in the Maize Plant Discipleship syllabus illustrate and demonstrate how he works amongst us, continuously moving us onwards towards this goal.

\pagebreak 

\tsection{Missional movements}

\begin{figure}[htbp]
\centering
\includegraphics[width=267pt,height=45pt]{reproduction.png}
\label{reproduction.png}
\end{figure}



\begin{summary}

An \emph{increase} in fruitfulness amongst missional movements is measured in terms of disciples within whom there is a faithful \textbf{reproduction} of the life of the Spirit, including the missional vision to make disciples amongst people of all nations. This leads to a \emph{expansion} in the Gospel's social and cultural sphere of influence.\begin{description}

\item[Risks]

— missional movements are unpredictable; results are difficult to measure and assess accurately; some groups that lose focus and faithfulness towards foundational values, yet are difficult to identify and reform.

\item[Benefits]

— members exhibits wide variety of charisma, talent and contextual influence. As the Good News is accepted amongst people, it leads to gradual transformation amongst families, households, communities, societies, cultures and people groups.
\end{description}

\end{summary}

\begin{discuss}

\begin{itemize}
\item Considering both the \emph{risks} and \emph{benefits} of missional movements, are they a worthwhile investment?

\item What would you be prepared to invest in establishing a disciple-forming, missional movement?

\end{itemize}

\end{discuss}

\pagebreak 

\ssection{The Expansion of Messianic Community}

This concludes Study 4, which explored three examples of either organic or organisational reproduction, in order to compare and contrast them with the organised expansion of messianic, discipleship movements. 

The study highlighted characteristics associated with the growth, reproduction, relocation, harvest and risks of swarming bees, maize plants, commercial organisations and missional movements. 

In summary, the study revealed that:

\begin{summary}

\textbf{The Messianic Community is intended to be a rapidly multiplying movement of disciples, constantly expanding into all the world, in order to bless the peoples of the world.}

\end{summary}

\vspace*{\fill}

\vspace*{\fill}

\begin{conclusion}

You have completed Module 4 of the Maize Plant Discipleship Syllabus, which explored the dynamics of \emph{God's commissioning} of the Messianic Community, in four inter-related studies:

\begin{enumerate}
\item The Commissioning of Messianic Community

\item The Strategies of Messianic Community

\item The Structures of Messianic Community

\item The Expansion of Messianic Community

\end{enumerate}

Module 5 explores \emph{Dynamics of Body Membership}. For further information see: \emph{Using this Handbook}.

\end{conclusion}

\chapter*{Author}\label{author}
\osection{Dr John B Clements}

\begin{quote}

John was awarded a Doctorate of Missiology (Contextual Missiology), by \emph{Fuller Theological Seminary School of Intercultural Studies}, in 2013.
\end{quote}

\begin{figure}[htbp]
\centering
\includegraphics[keepaspectratio,width=\textwidth,height=0.75\textheight]{john-rhossilli-sm.jpg}
\label{john-rhossilli-sm.jpg}
\end{figure}



\begin{quote}

John is married to Sarah, with four children. He is an avid bird-watcher and casual photographer, pastimes he combines with his enjoyment of countryside and coastal walking in South West Wales.
\end{quote}

\begin{wsite}

\begin{itemize}
\item Vita http:/\slash jbclements.wordpress.com

\item Social http:/\slash about.me\slash jbclements

\end{itemize}

\end{wsite}

\input{mpd-footer}

\end{document}
