\def\version{0.6.0}
\def\change{Hemingway}
\input{mpd-header}
\def\mytitle{The Eternal Purpose of God}
\def\module{1}
\def\translation{}
\def\isbn{978-1-907191-03-9}
\def\colour{false}
\input{mpd-document-en}

\chapter{Covenant and Scripture}
\label{covenantandscripture}

\begin{chapsynopsis}
\begin{center}

\textbf{The biblical covenants are key to faithfully interpreting God's historical and eternal purpose}. 

Appreciating the significance of God's covenants is essential to a holistic interpretation of Scripture.

\end{center}
\end{chapsynopsis}
\begin{topics}

\begin{enumerate}
\item Scripture in perspective

\item Characteristics of covenant

\item Characteristics of God's covenants

\item Unfolding covenantal purpose

\end{enumerate}

\end{topics}

\section{Scripture in perspective}
\label{scriptureinperspective}

\begin{quote}

\textbf{\emph{The books and stories of the Bible are bound together by a series of covenants. Covenant relationships, established with human beings, are crucial to God's redemption of his creation.}}
\end{quote}

Appreciating the significance of covenant is essential to a faithful, holistic understanding of Scripture. This may be illustrated by contrasting \emph{historical} and \emph{covenantal} perspectives of Scripture.

\subsection{Scripture in historical perspective}
\label{scriptureinhistoricalperspective}

Traditional perspectives of Scripture review the principal events and stories encountered in the books of the Old Testament in the historical order in which they took place. These events are then interpreted, looking at their significance for Christian faith. Principal events include:

\begin{itemize}
\item \emph{Creation}

\item Knowledge of good and evil, \emph{death} and murder.

\item A catastrophic \emph{flood} and the confusion of \emph{languages}.

\item The \emph{promises} of land and descendants, to a man named Abraham.

\item An \emph{exodus} from Egypt and formation of a new \emph{nation}, Israel.

\item Israel's journey into the \emph{land} promised to Abraham.

\item Israel's experiences, ruled over by \emph{judges, priests, prophets, kings}.

\item Israel and Judah's \emph{exile} from the land.

\end{itemize}

Figure 1 represents these Scriptural events as a biblical timeline. In these historical events, several significant biblical themes are evident. God's calling and blessing; human rebellion, judgement and disaster; return, forgiveness, deliverance. These themes run through the accounts of Adam, Noah, Abraham, Moses and David and Israel. They illustrate the importance of faithfulness and the dangers of disobedience towards God's commands.

\begin{figure}[htbp]
\centering
\includegraphics[width=296pt,height=69pt]{Figure1:Scriptureinhistoricalperspective.png}
\caption{Figure 1 : Scriptural events in historical perspective}
\label{figure1:scriptureinhistoricalperspective.png}
\end{figure}

\subsection{Scripture in covenantal perspective}
\label{scriptureincovenantalperspective}

Another theme is evident within these historical narratives. A profound theme that flows throughout Scripture: \emph{a series of divine covenants}. Discerning this pattern requires a re-examination of the biblical stories:\footnote{Blessing. Genesis 1:27–31. Curse. Genesis 3:17–19. Noah. Genesis 9:9–16. Abraham. Genesis 12:1–3, 15:18, 17:1–22, 22:15. Isaac. Genesis 26:1–5,24, 28:13–15. Israel. Exodus 19:4–6. David. 2 Samuel 7:12–16, 23:5; Psalm 89:3}

\begin{itemize}
\item The \emph{blessing} of creation.

\item A \emph{curse} placed upon the ground.

\item A \emph{covenant} established with \emph{Noah}.

\item A covenant established with \emph{Abraham}.

\item The covenant renewed with \emph{Isaac and Jacob}.

\item A covenant established with \emph{Israel}, after the exodus from Egypt

\item A covenant made with \emph{David}.

\end{itemize}

Adding these covenantal events to the timeline creates a covenantal perspective of Scripture (Figure 2). In this illustration, historical events are like trees growing upwards. The covenants are like roots growing downwards. Metaphorically, this emphasises the significance of the covenants.

\begin{figure}[htbp]
\centering
\includegraphics[width=299pt,height=179pt]{Figure2:AcovenantalperspectiveofScripture.png}
\caption{Figure 2 : A covenantal perspective of Scripture}
\label{figure2:acovenantalperspectiveofscripture.png}
\end{figure}

\begin{reflection}

Read: Genesis 9:9–11, Exodus 2:23--24

\begin{itemize}
\item Why did God choose to make covenants with human beings?

\item Testament is an alternative word for covenant. The Christian Bible consists of the Old and New Testaments. What does this suggest about the significance of covenant to Scripture?

\item What significance do God's covenants have to you and your community?

\end{itemize}

\end{reflection}

\section{Characteristics of covenant}
\label{characteristicsofcovenant}

\begin{quote}

\textbf{\emph{Covenants bind people together. A covenant is a binding obligation, forming a solemn relationship between two parties. A sacrificial meal and spoken oaths confirm the covenant relationship.}}
\end{quote}

The establishment of a covenant represented the formation of a solemn commitment. The obligation generally bound together two groups of families, tribes or nations. The Hebrew word for both secular and divine covenants is \emph{berîyth}. It occurs over 250 times in the Old Testament and always refers to a formal, binding arrangement between two parties.

\subsection{Historical covenant}
\label{historicalcovenant}

Covenants represented a form of treaty or alliance. They could be negotiated and established between two equal peoples. Or imposed by a stronger nation, or \emph{suzerain}, upon a subjugated nation. In this case the weaker nation became a servant nation, or \emph{vassal}, of the stronger nation.

\begin{itemize}
\item The suzerain provided protection and reward for faithful service.

\item The vassal gave allegiance, service and tax to the suzerain.\footnote{2 Kings 24:1}

\end{itemize}

A covenant was usually established by a solemn ceremony. This was called \emph{cutting covenant}. This ceremony involved sacrificing an animal and sprinkling its blood. Afterwards the two tribes or nations would seal and celebrate the covenant by sharing a meal.

\subsection{Terms}
\label{terms}

Terms accompanied the formation of covenants. They could refer to: 

\begin{itemize}
\item trade, food, water or other resources;

\item skill-sharing;

\item land, routes, territory;

\item Taxes (known as \emph{tribute});\footnote{1 Kings 4:21, Ezra 4:20}

\item ceremony, tradition;

\item protection, allegiance, peace or war.

\end{itemize}

The terms prescribed behaviours constituting the \emph{upholding} or the \emph{violating} of the covenant relationship. For a conquered nation, terms could be benign and generous, or oppressive and dominating.

\subsection{Oaths and invocations}
\label{oathsandinvocations}

Covenant partners swore an \emph{oath} to uphold the terms. Swearing an oath indicated serious intent to be faithful to the covenant.\footnote{Numbers 30:2; Deuteronomy 23:21; Ecclesiastes 5:4} To emphasise this seriousness, invocations were also spoken. Invocations called upon divine forces to reward faithfulness and punish unfaithfulness. Invocations thus consisted of:

\begin{itemize}
\item \emph{blessings} (rewards, bounty) for upholding the covenant

\item \emph{curses} (sanctions, punishments) for violating the covenant.

\end{itemize}

\begin{reflection}

Read: Genesis 26:26–31, Exodus 34:12

\begin{itemize}
\item What is the purpose of a covenant?

\item What are the essential elements of a covenant?

\item How is covenant understood in your culture? How does this compare with the covenants encountered in Scripture?

\end{itemize}

\end{reflection}

\section{Characteristics of God's covenants}
\label{characteristicsofgodscovenants}

\begin{quote}

\textbf{\emph{The covenants that God forms with his people follow the pattern of historical covenants. The biblical covenants create a servant nation. The faithfulness of this covenant community is vital to God's historical purposes.}}
\end{quote}

God forms covenants with people of his choosing. Like a powerful human lord he designates covenant terms without negotiation. With each covenant, God sets apart a tribe, a nation, a people—\emph{a covenant community}.

\subsection{Faithfulness of the covenant community}
\label{faithfulnessofthecovenantcommunity}

The faithfulness of this covenant community is vital to his wider purposes. They are a \emph{servant community}—a community called to serve God's purposes. As such, God extends his favour and blessing to them. Yet, his blessings are not the goal, but a means to a greater end: the glorification of God's name.

God blesses the covenant community so that they may serve him. Whenever this principle becomes obscured, the covenant community risks frustrating God's purposes.

\subsection{Keeping covenant}
\label{keepingcovenant}

\emph{Keeping covenant} meant preserving the solemn relationship, by observing the covenant obligations. It requires submission and obedience to the covenant terms. This includes forgiveness and restitution, following breaches of the terms.

Yet keeping faith with God's covenant means more than upholding precise terms. Keeping covenant requires wholehearted participation in the \emph{vocation of the covenant community}. This is the reason that God calls a community into covenant with him. That they may become a \emph{faithful} servant community.

\subsection{Breaking covenant}
\label{breakingcovenant}

The relationship at the heart of the covenant is always greater than the covenant terms. Thus, the covenant relationship can be restored following occasional breaches. Restoration happens through:

\begin{enumerate}
\item recognition and confession of breaches

\item forgiveness and restitution

\item recommitment and covenant renewal.

\end{enumerate}

When a covenant community repeatedly, or flagrantly breaches the terms, there is a problem. This indicates a loss of the solemnity of the relationship. Continual \emph{faithlessness} leads to the punishments and curses incorporated in the covenant terms. Breaking covenant is significant because it frustrates God's wider purposes. \emph{Covenant faithlessness is the sign that the covenant community has abandoned its vocation. The servant community is no longer capable of serving God's purposes.}

\subsection{Certainty of divine covenant}
\label{certaintyofdivinecovenant}

God won't change his mind. He will do what he has purposed to do through his covenants and through his Covenant Community. Even when God's people break the covenant, God remains faithful to his purposes. Because his character is unchanging, his eternal purpose remains intact. When God makes a covenant he is committing himself to fulfilling his purposes through that covenant. This is why he \emph{swears an oath}. He does so to confirm \emph{the unchangeable character of his intentions}.\footnote{Hebrews 6.13--18}

\begin{reflection}

Read: Hebrews 6:13--18, Deuteronomy 30:19--20a

\begin{itemize}
\item What happens when God's covenant community neglect his purposes?

\item How are blessings and curses related?

\item How does Scripture challenge cultural ideas about curses?

\end{itemize}

\end{reflection}

\section{God's covenant purposes}
\label{godscovenantpurposes}

\begin{quote}

\textbf{\emph{The biblical covenants point towards a restoration of creation. They represent God's response to humanity's rebellion against his purposes.}}
\end{quote}

Central to God's restoration of creation is the reconciliation of human beings. Each covenant reveals a significant and vital aspect of God's unfolding plans and purposes. Each one refers to his collaboration with a chosen covenant community

\begin{description}

\item[God's covenant with Noah]

expresses God's commitment to his creation. Despite human evil, God will remember his covenant with creation.

God's covenant with Noah echoes the creation narrative of Adam. God reminds Noah that human beings are made \emph{in his image}. He instructs Noah's family to \emph{be fruitful and multiply, swarm over the earth and multiply on it}.\footnote{Genesis 1:26--28} God establishes an \emph{everlasting covenant} with Noah. He extends this covenant to Noah's descendants and to \emph{every living creature of any kind on the earth}. The rainbow becomes a sign of God's covenant promise to never again destroy all living things.\footnote{Genesis 6:18, 8:6--22, 9:8--16}

\item[God's covenant with Abraham]

reveals his commitment to bless the families of the earth. He will do this through forming a great nation from Abraham's descendants. God is committing himself to restoring his creation, marred by human rebellion. He will do it through a people set apart to serve him.

After the flood and the covenant with Noah, the wickedness of human society continues. In response to this degeneration, God makes a covenant with Abraham. It reveals God's plan for \emph{a great nation} that will \emph{bless all the families of the earth}.\footnote{Genesis 12:1--3, 15.18, 17.1--22, 22.15}

In speaking to Abraham about his intentions, God uses a significant phrase. \emph{You are to be a blessing} suggests both a promise and a command. Abraham is to impart his sense of divine purpose and blessing to other tribes to whom he relates. Because of his relationship with God, he is to anticipate being a blessing to others. Thus, the impact of the covenant is destined to expand in two, complementary directions:

\begin{itemize}
\item downwards to Abraham's descendants, blessed \emph{in Abraham};

\item outwards to the whole human family, blessed \emph{by Abraham}.

\end{itemize}

\item[God's covenant with Israel]

expresses his intention to use a chosen people. Through this covenant community, he will reveal his love and glory to the whole world. He selects a small, weak, unimportant nation to reveal his love, power and faithfulness. Through them, he intends to show his goodness towards the whole human race.

Abraham's descendants become trapped under the oppression of Egypt. God remembers his covenant with Abraham and delivers the people of Israel. He guides and provides for them as they travel through the Sinai wilderness. He brings them to the edge of the sanctuary of the land promised to Abraham. There, God forms a covenant with the people of Israel. He invites Israel to become \emph{his own treasure, a kingdom of priests} and \emph{a nation set apart}. These unique roles hint at Israel's calling to mediate God's blessing to other nations.\footnote{Exodus 1–18, especially 2:23–25 and 6:2--8, 19.4--6; Romans 9:4--9}

Central to Israel's covenantal calling is the \emph{Torah,\footnote{Exodus 20–31. Deuteronomy 4:5–8. Isaiah 42:6. Torah means instruction, or teaching. The historical New Testament rendering of Torah as ``law'' introduces a significant distortion. This has skewed Christian tradition towards an uncharitable and legalistic view of the Torah. The Jewish New Testament, by David H. Stern, redresses this distortion.}} itself centred upon the \emph{Ten Words,\footnote{Deuteronomy 5. Traditionally interpreted in Christianity as ``Ten Commandments''.}} given to Moses on Mount Sinai. Torah provides detailed instructions informing Israel how to live in covenant relationship with God. Torah forms the basis for Israel's vocation. Through faithfulness, Israel is to show love, devotion and allegiance to God.\footnote{Deuteronomy 4:5--8}

Through the covenant, God gives life and blessing to the people of Israel. They receive grace and goodness intended to overflow towards all nations and peoples. Through faithfulness to the covenant, they are called to show God's wisdom and understanding. They are called to be \emph{a light to the nations}.\footnote{Isaiah 42:6, 49:6, 51:4, Luke 2:32, Acts 13:47, 26:18} 

Israel must choose either faithful covenant relationship or unfaithful idolatry. It is a choice between life and death, the blessing and the curse. Between deliverance and disaster, inheritance and exile.\footnote{Deuteronomy 30:1--20} These vital choices reveal:

\begin{itemize}
\item God's \emph{kindness} towards those embracing his covenant, submitting to his government and available to serve his purposes.

\item God's \emph{severity} towards those rejecting his covenant, resisting his purposes and rebelling against his government.\footnote{Romans 11:22}

\end{itemize}

\item[God's covenant with David]

expresses his commitment to anoint one of David's descendants. A \emph{branch} of David's line will establish God's reign on earth. This hints at a future Anointed King—a Messiah—who will eternally establish God's kingdom.

The covenant with Israel required them to recognise God as their King. When they demand a human king, they are rejecting God as their king. Samuel, the prophet, identifies this deep tragedy, yet God reassures him..\footnote{1 Samuel 8:7--8} The king, Saul, disobeys God by offering sacrifices that only priests may offer. God directs Samuel to replace Saul with another king, \emph{David, a man after God's own heart}.\footnote{Acts 13:22; 1 Samuel 13:14} God makes a covenant with David and blesses his reign. He promises that one of his descendants will build a Temple for God's name. And that David's royal throne will continue \emph{eternally}\footnote{2 Samuel 7:12--16; also 1 Chronicles 17:11--14, Psalm 89:19--37}

Following David's death, the kingdom passes to his son, Solomon. Solomon begins his reign with great wisdom. He shows great faithfulness towards God and the covenant. This includes the lavish construction and dedication of the Jerusalem Temple. Unfortunately, Solomon's foreign wives and concubines lead him to worship other gods. His idolatry invites God's judgement and the nation is divided into two kingdoms. Within a century, both the kingdoms of Israel and Judah are exiled from the Land.

Another descendant will fulfil the original promise to David. The promise refers to a \emph{branch} of David's line.\footnote{Jeremiah 23:5--6} This suggests a future descendant who will bring deliverance and blessing to Israel.
\end{description}

\begin{reflection}

Read: Genesis 12:1--3, Exodus 19:3--6, Romans 11:22

\begin{itemize}
\item What does God most need from a covenant community?

\item How did Israel respond to the calling to be a light to the nations?

\item How was David \emph{a man after God's own heart}?

\end{itemize}

\end{reflection}

\ssection{Covenant and Scripture}

The topics of this study, respectively:

\begin{enumerate}
\item Explain how the books and stories of the Bible are bound together by a series of covenants

\item Explore the characteristics of historical covenants, typical of near-Eastern cultures of the biblical era.

\item Illuminate significant aspects of God's covenants. God forms covenants with people of his choosing, for his own purposes.

\item Reveal the purpose of the biblical covenants. They point towards a restoration of creation corrupted by rebellion against God's purposes.

\end{enumerate}

\emph{Figure 3 : Covenantal Promises} updates the biblical panorama being constructed by this study. It incorporates elements from each of the patriarchal covenants and hints at how they are fulfilled in the Messiah and the promised new covenant.

\begin{figure}[htbp]
\centering
\includegraphics[width=299pt,height=218pt]{EP5c-messiah.png}
\caption{Figure 3 : Covenantal Promises}
\label{ep5c-messiah.png}
\end{figure}

\begin{quote}

Overarching the biblical panorama is the Abrahamic blessing. A great nation, blessed to be a blessing to all the families of the earth. God's concern extending towards every tribe, language and people.
\end{quote}

\vspace*{\fill}\begin{bonus}

\begin{itemize}
\item How do we \emph{contradict} or \emph{reflect} God's commitment to creation?

\item What is God's response to abuses of human power?

\item How does God call us to respond to peoples or places that have suffered abuse?

\end{itemize}

\end{bonus}

\chapter{The New, Messianic Covenant}
\label{thenewmessianiccovenant}

\begin{chapsynopsis}
\begin{center}

\textbf{The Messiah is the goal at which God's covenants aim}. Through the new covenant, Messiah Yeshua becomes the one, eternal mediator between God and humanity.

\end{center}
\end{chapsynopsis}
\begin{topics}

\begin{enumerate}
\item Renewal of the Covenant

\item Revelation of the Messiah

\item Identity of the Messiah

\item Greatness of the Messiah

\item Mediation of the Messiah

\end{enumerate}

\end{topics}

\section{Renewal of the Covenant}
\label{renewalofthecovenant}

\begin{quote}

\textbf{\emph{The historical background of the New Testament is one of expectation. Of a promised Messiah who will bring national deliverance.}}
\end{quote}

Biblical prophecies predict a new covenant and a godly Messiah. How these two realities merge is less clear. In the first century, Israel is living under Roman rule. The popular expectation is of a glorious king who will shepherd the people of Israel, as David once did. A leader who will cleanse the Land from pagan oppression and produce national prosperity.

\subsection{National renewal}
\label{nationalrenewal}

Israel's exile from the promised land represents God's judgement. It is a painful reminder of their idolatrous breach of the covenant. A national renewal would indicate the ending of Israel's political, geographical and spiritual exile.

In time, led by Ezra, Nehemiah and Malachi, a remnant of Jews begin a slow return to Israel. A symbolic, albeit partial rebuilding of the Jerusalem Temple follows. Yet it becomes clear that this restoration and return from exile is incomplete. It does not represent the longed-for sign of God's grace, forgiveness and \emph{covenantal renewal}.\footnote{Haggai 2:1--9}

\subsection{Spiritual renewal}
\label{spiritualrenewal}

The biblical prophets declare that God's concern is not for national, economic renewal. Nor a military victory over pagan oppressors. It is for a \emph{spiritual renewal} that addresses the covenant community's repeated faithlessness.

\begin{description}

\item[Jeremiah]

speaks of a new covenant that will provide an inner cleansing from sin. It will lead to a new intimate knowledge of the Lord. His Torah will no longer be disregarded. Instead, it will be \emph{written upon the hearts} of his covenant people.\footnote{Jeremiah 31:31--34. Torah: ``instruction'' or ``teaching,'' revealed to Moses; the first five books of the Bible; especially the Ten Words, Exodus 20:1--17}

\item[Ezekiel]

prophecies about a \emph{good shepherd} who will challenge the leadership of Israel. This David-like shepherd will call forth a renewed covenant people from amongst Israel.\footnote{Ezekiel 34:16--23}

\item[Isaiah]

reveals that Israel's messiah will not be a prosperous, military ruler. Instead, he will be a \emph{despised} and marginalised prophet. One who lives a priestly, intercessory life of sacrificial obedience and suffering service.\footnote{Isaiah 53:11ff} 

This messiah will operate under the anointing of the Spirit. He will restore justice to the poor and inclusion to those on the margins of the covenant community. He will provide a mediatory, atoning sacrifice that renews the covenant community of Israel. He will be \emph{a light to the Gentiles}.\footnote{Isaiah 49:6}

\item[Malachi]

writes of \emph{Messenger of the Covenant suddenly coming to his temple}.This represents the true sign that Israel's sins are forgiven. The sign that the exile is coming to an end{\ldots}\footnote{Malachi 3:1--4}
\end{description}

\begin{reflection}

Read: Jeremiah 31:31--34, Isaiah 49:6, Haggai 2:1--9

\begin{itemize}
\item What was the significance of the return of Jews to the promised Land, after exile in Babylon?

\item What kind of Messiah were the Jews expecting to arise and lead them? Why do you think that was?

\item Were the Jews anticipating being a light to the Gentiles?

\end{itemize}

\end{reflection}

\section{Revelation of the Messiah}
\label{revelationofthemessiah}

\begin{quote}

\textbf{\emph{The Messiah Yeshua is the goal at which the biblical covenants aim. The New Testament gospels establishes Yeshua's identity, significance and purpose as the Jewish Messiah.⁠}}
\end{quote}

The gospels of Matthew, Mark, Luke and John provide detailed accounts of Yeshua's life. They describe his work, signs, miracles, teaching, discipleship and, finally, death, resurrection and ascension.\footnote{``Jesus Christ'' is a Greek rendering of Hebrew Yeshua Moshiach.} \emph{Table 1 : Scriptural prophecies of the Messiah} demonstrates how Jesus fulfilled numerous biblical prophecies. Many others could be added.

\begin{figure}[htbp]
\centering
\includegraphics[width=294pt,height=390pt]{Table1-Prophets,GospelsandMessiah.png}
\caption{Table 1 : Scriptural Prophecies of the Messiah}
\label{table1-prophetsgospelsandmessiah.png}
\end{figure}

\subsection{Revealing the Messiah}
\label{revealingthemessiah}

The gospel writers do not write as not as impartial observers or secular historians. They write as committed members of an emerging new covenant community. Their intent is to establish Jesus' true identity, significance and purpose as the Jewish Messiah.⁠\footnote{Romans 9:4–5, 10:1–4. John 20:30–31. ``Messiah appears over 380 times in the NT. It is a continual reminder of its claim that Yeshua is the promised Moshiach for whom the Jewish people have yearned. The English word Christ does not point to Yeshua's fulfilment of the Jewish hopes and biblical prophecy''—David Stern, The Jewish New Testament Commentary. (P.2). JNT Publications, Clarksville. Maryland. 1992.}

The gospels reveal the Messiah's identity by drawing upon a wide range of biblical imagery rooted in Israel's covenant history. They use hints, stories, signs, events, parables, illustrations, metaphors and, above all, prophecies. These biblical signposts confirm and illuminate how Jesus fulfils the covenant promises and prophecies. In this way, they prove that Jesus is the promised One. He is the Messiah and instigator of \emph{a new covenant with the house of Israel and Judah}.\footnote{Hebrews 7:22, 8:6--13}

\begin{reflection}

Select and read three or more passages from Table 1.

\begin{itemize}
\item Why is it significant that Yeshua fulfilled messianic prophecies?

\item How was Yeshua recognised as the promised Messiah?

\item Do you believe that Yeshua is God's Messiah? Why?

\end{itemize}

\end{reflection}

\section{Vocation of the Messiah}
\label{vocationofthemessiah}

\begin{quote}

\textbf{\emph{The vocation of the Messiah is rooted in the roles of Israel's prophets, priests and kings. Messiah means anointed or poured on. Anointing oil was used to set apart the anointed one to serve God wholly.}}
\end{quote}

Only priests and kings could be anointed with the holy anointing oil. Prophets poured anointing oil over the heads of Israel's kings.\footnote{Exodus 28--30, 40; 1 Samuel 9:16, 15:1; 1 Kings 1:34; Psalm 2:2; Daniel 9:24} The anointing represented the placing of God's Spirit upon these leaders. It was a sign of their investiture with authority to lead and govern God's people. They are not to lead in their human strength and wisdom, but in the strength and wisdom provided by the Spirit.

Yeshua is not anointed with oil by a prophet. He is anointed with the Holy Spirit, after he undergoes the baptism of the prophet John. The Holy Spirit descends upon him \emph{like a dove}, as a voice from heaven declares: \emph{This is my son, in whom I am well pleased}.\footnote{Matthew 3:11–17; Luke 3:21–22, 4:14–20. Mark 1:7–12; Acts 10:38}

As the Messiah, Jesus fulfils three roles on behalf of Israel: prophet, king and high priest. Each role mediates a complementary aspect of God's authority.\footnote{Mediation, in a biblical context, is equivalent to intercession. Job 1:5, 42:10. Isaiah 53.}

\begin{description}

\item[Israel's prophets]

mediate God's authority in several vital ways:
\end{description}

\begin{itemize}
\item interceding, on behalf of God's purposes for his people

\item entering God's \emph{heavenly council}, to hear and receive God's Word

\item delivering God's word to his people, including judgement

\item anointing Israel's kings and challenging them to covenant faithfulness

\end{itemize}

Abraham is the first prophet identified in scripture. Moses is the greatest prophet in Israel's history. He receives the Torah. Moses' intercedes when God is planning to destroy the children of Israel and God hears. Other significant prophets include: Samuel, Elijah, Daniel, Isaiah and Jeremiah.\footnote{Genesis 20:7,17; Exodus 32:11–14, 33:1–23; Deuteronomy 18:15; 1 Samuel 7:3–8, 12:19: 1 Kings 17:17–24, 18:1–40; 2 Kings 4:17–37, 19:4,20, 20:9–11; James 5:17; Jeremiah 7:16, 11:14, 14:11, 23:18,22, 37:3; Daniel 6:10–11, 9:3–19; 2 Chronicles 32:20}

\begin{description}

\item[Israel's kings]

mediate God's authority as his co-rulers or princes. To Israel, the king represents God's government. To God, the king represents Israel. Israel's kings are even described as \emph{seated upon God's throne}. Kings David and Solomon are especially recognised for their wisdom, government and intercession.\footnote{2 Samuel 24:17; 1 Kings 8:22–53; 2 Chronicles 30:18–20; 1 Chronicles 28:5--7, 29:23}

\item[Israel's priests]

mediate God's authority by interceding ritually, on behalf of the people. Their responsibility is to approach God's presence, to receive grace, mercy, forgiveness of sin. To restore the covenant relationship on behalf of Israel. Once a year, the high priest enters the \emph{holiest place} to make atonement for the nation's sin.\footnote{Joel 2:17. Malachi 1:9. Ezra 9:5–15. Exodus 26:31–33, 36:35–36. Matthew 27:51}
\end{description}

\begin{reflection}

Read: Matthew 3:16, Matthew 27:50--51

\begin{itemize}
\item How do prophets, kings and priests mediate God's authority?

\item Towards which type of authority do you most readily relate?

\item How would you describe the vocation of the Messiah?

\end{itemize}

\end{reflection}

\section{Greatness of the Messiah}
\label{greatnessofthemessiah}

\begin{quote}

\textbf{\emph{Jesus is more that simply another prophet, priest or king. The gospels identify Jesus as a prophet ``like Moses,'' a ``branch of David'' and a ``priest of the order of Malchi-Tzedek.''}} 
\end{quote}

Ultimately, the New Testament reveals how the Messiah exceeds each of the Patriarchs, Abraham, Moses and David.

\subsubsection{A prophet like Moses}
\label{aprophetlikemoses}

Early on, Jesus is identified as a prophet, both by his disciples and the crowds. As prophet, Jesus announces the arrival of \emph{the kingdom of God}.\footnote{This means that God is about to act powerfully in the history of Israel.} Jesus calls the \emph{lost sheep of the house of Israel} to repentance and renewed faithfulness. He calls them to \emph{enter} the kingdom of God.\footnote{Matthew 13:57, 21:11,46; Luke 7:16; Matthew 4:17, 7:21}

Later on, Jesus is specifically identified as the \emph{prophet like Moses}. Moses is considered Israel's greatest prophet. Like Moses, Jesus addresses Israel's most potent symbols of faith: leadership, temple and covenant.\footnote{Matthew 10.6, 15:24; Luke 24:19: John 1:17–18}

\begin{description}

\item[Leadership]

— the \emph{seat of Moses} is a reference to the priestly leadership of Israel.\footnote{Matthew 23:1–39. ``Woe to you'' is a form of curse. There is ample precedent for such harsh denunciations in the history of Jewish prophets—though always in a covenantal context.} The Pharisees and Torah-teachers exercised their authority by claiming to interpret and follow the Torah of Moses. 

Jesus does not challenge their right to occupy their offices. He holds them responsible for the ongoing corruption of the covenant relationship. He accuses them of going through the motions of Torah-obedience with \emph{uncircumcised hearts}. He condemns them for neglecting the \emph{greater commandments} of humility, mercy and justice, whilst making covenant faithfulness harder for others.\footnote{Leviticus 26:41, Deuteronomy 10:16, 30:6, Jeremiah 4:4, 9:26, Ezekiel 44:7,9; Matthew 23:23--4}

\item[Temple]

— the temple represents the heart of the covenant with Israel. It is the meeting place with the \emph{Sh'kinah}.\footnote{Presence, or glory of God. 1 John 1:14, hebrews 1:3, Matthew 17:6, Luke 2:9, 1 Peter 4:13} Jesus prophesies the calamity of the destruction of the Jerusalem temple. Then he predicts that God would then raise it up again \emph{in three days}. This is a metaphorical reference to his resurrection, after three days in the tomb.\footnote{John 2:19--22}

The Messiah replaces the temple, as the Way to God.\footnote{John 2:13--22, 4:21, 6:1--4, 14:6; Matthew 27:51, Hebrews 9:3--9, 10:19--22} Under the first covenant, the high priests entered the Holiest Place once a year. The New Covenant opens the way to all who call upon the Lord to continually experience God's presence. Both Peter and Paul compare members of the Messianic Community to \emph{living stones}. Together they form a \emph{living temple} in which the Holy Spirit dwells.\footnote{1 Peter 2:4--5, 1 Corinthians 3:16--17, 6:19}

\item[Covenant]

— Moses mediated the covenant with Israel. Jesus identity as \emph{the prophet like Moses} hints at covenant renewal.\footnote{Deuteronomy 18:15--19; John 1:21; Acts 3:22--23, 7:37} Prior his death, Jesus affirms that he is making \emph{a new covenant} with his disciples. He does this by linking the Passover meal with his own death and the new covenant prophesied by Jeremiah.\footnote{Matthew 26:17--29; Mark 14:12--25; Luke 22:1--20}
\end{description}

\begin{reflection}

Read: Luke 1:68--75, Hebrews 7:22--25, 1 John 2:1b--2

\begin{itemize}
\item How was Jesus similar to and yet greater than Moses? 

\item What was Yeshua's message to Israel's leaders? 

\item How might such a warning apply today?

\end{itemize}

\end{reflection}

\subsubsection{A branch of David}
\label{abranchofdavid}

Messiah means the \emph{Special One} promised by God. A king descended from David and raised up by God to govern his covenant community. Jesus is a direct descendant of David. For a time, he is reluctant to confirm his identity as the Messiah. He instructs people not to speak about him or what he has done.⁠ The revelation and realisation that he is the Messiah is foundational to the vocation of the apostles. The accusation that he \emph{claims to be the Messiah} is central to his arrest and death.\footnote{Luke 1; Matthew 1:1–17, Luke 1:69; Matthew 8:4, John 6:15; compare John 4:25--26; John 1:49, 11:27; Matthew 16:13--20, Luke 22:66--23:42} The gospels confirm Jesus' identity through a range of messianic titles:

\begin{description}

\item[Son of God]

— a title synonymous with Messiah.\footnote{E.g. Matthew 16:16, 26:63; John 20:31. To first-century Jews, ``Son of God'' does not imply divinity. It means a godly, righteous person or the Special One, the Messiah, sent by God. This contrasts with Christian tradition, where ``Son of God'' implies divinity, i.e. ``God the Son,'' a member of the Trinity of Father, Son and Holy Spirit.} Before his birth, Mary identifies Jesus as one who \emph{will be called Son of God}. Later, a voice from heaven declares that he is God's \emph{only begotten{\ldots}beloved Son}.\footnote{Luke 1:31--35; Matthew 3:17; Luke 3:22; John 1:14,18}

\item[Son of David]

— a recognised title of the Messiah, frequently applied to Jesus. It relates to God's covenant promise to raise up a \emph{branch} of David (a descendant) to deliver his people. Jesus demonstrates that the Messiah is also \emph{David's Lord}—and thus greater than David.\footnote{Matthew 12:23, 15:22, 21:9; Acts 13:23; 2 Samuel 7:12--13; Isaiah 11:1; Jeremiah 23:5--6; Ezekiel 37:24; Psalms 89:3--4, 35--36, 132:11; Luke 1:69; Romans 1:4; Mark 12:35--37}

\item[Son of Man]

— an obscure messianic title, used by Jesus to identify himself. This may be because its meaning would be clear only to Torah-teachers and Pharisees. Jesus links it to a series of dramatic events soon to take place. These events culminate in his \emph{coming on the clouds} and \emph{being seated at God's right hand.}\footnote{Matthew 8:20, 9:6, 10:23, 11:19, Luke 9:22; Daniel 7:13--14; Luke 22:46--70}

\item[Good Shepherd]

— a title used by Jesus to identify himself to the people. As Messiah, he is sent \emph{to the lost sheep of the house of Israel}. Yet the gospels repeatedly hint at the overflow of salvation from Israel to the gentiles. In particular, Jesus speaks of \emph{other sheep{\ldots}not from this pen}.\footnote{Matthew 15:24, John 10:11--14, 16, 27; Ezekiel 34:23, 37:24; Hebrews 13:20; 1 Peter 5:4; Psalm 23}

\item[King of the Jews]

— used by non-Jews. The Pharisees use this title to incite the Roman authorities against Jesus.\footnote{Unlike his paranoid father, Herod Antipas is un-threatened by Jesus and untroubled by the messianic accusation (cf. Luke 2 and 23:8--12, 15)} Ironically, Pilate orders it to be placed on a notice atop the stake on which Jesus is crucified.\footnote{Matthew 27:37, Mark 15:26; Luke 23:3; John 19:19, Luke 23:38}
\end{description}

\begin{reflection}

\begin{itemize}
\item Why did Jesus hide his identity as the Messiah?

\item Was the Messiah a threat to the ruling authorities? Why?

\item Which of the Messiah's titles do you find most significant? Why?

\end{itemize}

\end{reflection}

\subsubsection{Priest of the order of Malchi-Tzedek}
\label{priestoftheorderofmalchi-tzedek}

Seated at the Father's right hand, he intercedes for his people in the true, heavenly tabernacle. As high priest, Yeshua guarantees the better promises of the new covenant. In particular, he makes purification for sins, sits down at the right hand of God and becomes an eternal advocate for human beings. These three realities are the culmination of the Messiah's vocation. They form the primary, enduring realities of the new covenant.

\begin{description}

\item[Purification for sins]

John the Baptist identifies Jesus as \emph{the Lamb of God, who takes away the sin of the world}. As a priest, Jesus \emph{offered one sacrifice, once and for all, by offering up himself. Through the eternal Spirit (he) offered himself to God as a sacrifice without blemish}.\footnote{John 1:29; 1 Corinthians 5:6--8; 1 John 2:1--2; Hebrews 7:27, 9:14; Numbers 6:14}

\item[Seated at God's right hand]

— Jesus identifies himself with \emph{the Son of Man{\ldots}at the right hand of the Power on high}. This represents a powerful claim to be the Messiah. The depiction of the Messiah at the right hand of God is definitive. Being sat at God's right hand represents the Messiah's enthronement as God's co-ruler. Identifying himself as co-existent with God is considered blasphemous. Jesus uses the Torah to powerfully refute this accusation.\footnote{Matthew 22:41--46; Luke 20:42; John 10:31--36, 58--59; Leviticus 24:15--16}

\item[Eternal, heavenly advocate]

— after his death and resurrection, Jesus enters the true, heavenly tabernacle. He enters to present the sacrifice of his blood. He does so as a sinless and eternal high priest. In this way, Jesus mediates the new \emph{eternal covenant}.\footnote{Hebrews 8:2–6, 9:11–24, 13:20, cf. 7:22, 85–13, 10:15–18}

Jesus' priesthood is of a different order to the Levitical priests. Jesus comes from the tribe of Judah. Using a \emph{midrash},\footnote{An allegorical application of a text. An Hebraic form of eisegesis—reading one's own thoughts into the text. As contrasted with exegesis—extracting from the text what it says plainly. The implication is that God uses the human spirit to uncover truths beyond the immediate text.} the writer of the book of Hebrews identifies him as a high priest. Moreover, an eternal priest \emph{of the order of Malki-Tzedek}.\footnote{From Malki (King) of Tzedek (Righteousness); he is also King of Salem (peace; shalom). Genesis 14:18, Hebrews 5:6–10, 7:1--21, Psalm 110:4}
\end{description}

\subsubsection{Greatness of the Messiah's mediation}
\label{greatnessofthemessiahsmediation}

The writer of Hebrews demonstrates that Yeshua \emph{deserves more honour than Moses}. He now shows how the Messiah is greater even than Abraham, Father of the Jewish nation. He does this by identifying the Messiah as an eternal priest, \emph{to be compared with Malki-Tzedek.} \footnote{Hebrews 3:3, 5:6--10, 6:20; John 1:17} 

\emph{Malki-Tzedek, priest of El 'Elyon} receives a tithe from Abraham and blesses him.According to scripture, \emph{the greater blesses the lesser}. Thus \emph{Malki-Tzedek} is greater than Abraham. And hence, the Messiah is greater than Abraham. And guarantor of a better covenant.\footnote{Genesis 14:18--20; Hebrews 7:7, 7:22, 8:6}

The writer also shows how the Messiah \emph{became a priest by virtue of the power of an indestructible life}. As a result, Jesus has an eternal intercessory ministry. He lives forever to \emph{advocate} on behalf of human beings who come to the Father through him.\footnote{Mediator, intercessor; one pleading on someone else's behalf. Hebrews 7:24–25; 1 John 2:2; Acts 2:36, 3:22–23; 1 Corinthians 15:23–28; Ephesians 1:20–22; Romans 8:26–27,34; Hebrews 2:17, 4:15, 7:3,25, 9:24}

\begin{reflection}

\begin{itemize}
\item What is the significance of being at the Right Hand of God?

\item What is a mediator? What power does an intercessor hold?

\item Why is it significant that Jesus lives forever?

\end{itemize}

\end{reflection}

\section{Mediator of an eternal covenant}
\label{mediatorofaneternalcovenant}

\begin{quote}

\textbf{\emph{As prophet, king, and eternal high priest Jesus becomes the centre of an emerging new covenant community and mediator of an eternal covenant.}}
\end{quote}

There is a progression in the disciples's awareness of the Messiah. He is rabbi, prophet, then Messiah. Finally, he is resurrected Lord and eternal high priest. In all these ways, Jesus mediates between God the Father and human beings.

\begin{description}

\item[As prophet]

Jesus calls the covenant community of Israel to repent and to enter the kingdom of Heaven. The kingdom arriving in the person of the Messiah, the new David.

\item[As messiah]

he forms the centre of a new, messianic covenant community. He is the anointed Good Shepherd, faithfully fulfilling God's purposes. He gathers both Jews and Gentiles to him. All who \emph{hear his voice} are called to follow and give their allegiance to him. He forms them into a new Messianic Covenant Community, giving his spirit to those who trust in and follow him.

\item[As eternal high priest]

the resurrected Yeshua sits besides the Father. He has compassion upon his brethren. He provides purification for sin and eternal advocacy for those who come to the Father through him.
\end{description}

Abraham, Moses and David represent the patriarchal roots of the covenantal, Jewish faith. In his holiness, obedience and submission towards the Father, the Messiah exceeds the Patriarchs. As priest, prophet and king, Yeshua completely fulfils the role of the Messiah. He:

\begin{itemize}
\item establishes a new, \emph{eternal} covenant

\item intercedes for the covenant community of Israel and the new, messianic covenant community of Jews and Gentiles

\item bears the world's sin, at the \emph{earthly} Passover of Calvary\footnote{Jesus is crucified at Calvary, following the annual Passover. Passover requires a sacrificial lamb. Jesus is ``the Lamb of God.'' John 1:29, 36. Exodus 12.}

\item advocates eternally in the true, \emph{heavenly} tabernacle.

\end{itemize}

Thus, he fulfils God's eternal purpose, planned from before creation. He becomes eternally the one mediator between God and human beings.\footnote{Hebrews 9:24, Romans 8:34, Ephesians 3:11; 1 Timothy 2:5--6}

\begin{reflection}

Read: Luke 2:25--32, Ephesians 3:11, 1 Timothy 2:5--6

\begin{itemize}
\item How is Yeshua different to priests, prophets, kings before him? 

\item Why does humanity need a mediator between them and God?

\item How do you respond to Jesus as an \emph{eternal high priest}?

\end{itemize}

\end{reflection}

\ssection{The New, Messianic Covenant}

The topics of this study, respectively:

\begin{enumerate}
\item Summarise the historical, covenantal background to the New Testament. It focusses on biblical prophecies of a new covenant and a priestly messiah.

\item Reveal how the Messiah Yeshua represents the goal at which the biblical covenants aim.

\item Examine the vocation of Yeshua the Messiah. A vocation rooted in the biblical covenants and messianic prophecies.

\item Explore Jesus' roles as Israel's prophet, king and high priest. Through these roles he becomes the centre of an emerging, new covenant, messianic community

\end{enumerate}

\emph{Figure 4, Messiah Yeshua and New Covenant} updates the timeline being constructed by this study. It illustrates how:

\begin{itemize}
\item the history of Israel provides the appropriate perspective for interpreting the new covenant

\item the Messiah and his mission form the true purpose and goal of Israel's covenantal history

\item the Messiah fulfils the covenantal calling of Abraham, Israel and David.

\end{itemize}

\begin{figure}[htbp]
\centering
\includegraphics[width=299pt,height=218pt]{EP8-messiah-covenant.png}
\caption{Figure 4: Messiah Yeshua and the New Covenant}
\label{ep8-messiah-covenant.png}
\end{figure}

\begin{quote}

From learned rabbi, to messianic prophet, to resurrected lord, to eternal high priest. The Messiah is is exalted to sit eternally at the Right Hand of God the Father. He has divine authority in heaven and on earth. He becomes the perfect mediator on behalf of people from every tribe, language and nation.
\end{quote}

\vspace*{\fill}\begin{bonus}

\begin{itemize}
\item What is the significance of covenant within Scripture?

\item Why do the people of Israel repeatedly need a deliverer?

\item Why was there a need for a new covenant?

\item How is the new covenant related to the former covenants?

\item How does Jesus compare to the Patriarchs of Israel?

\end{itemize}

\end{bonus}

\chapter{The Messianic Covenant Community}
\label{themessianiccovenantcommunity}

\begin{chapsynopsis}
\begin{center}

\textbf{Pentecost ushers in a new era of Spirit-led Messianic Community}. The Holy Spirit transforms the new covenant community into an anointed, charismatic community, empowered to serve God's mission amongst the nations.

\end{center}
\end{chapsynopsis}
\begin{topics}

\begin{enumerate}
\item Pentecost: Torah and Spirit

\item Pentecost: Messianic Community

\item Pentecost: Light to the Gentiles

\item Pentecost: A New Humanity

\end{enumerate}

\end{topics}

\section{Pentecost: Torah and Spirit}
\label{pentecost:torahandspirit}

\begin{quote}

\textbf{\emph{There are significant parallels between the Torah and the Holy Spirit. These parallels illuminate the uniqueness and the power of the new covenant.}}
\end{quote}

Pentecost signals the outpouring of the Holy Spirit on God's people.\footnote{Pentecost. From Greek, pente, meaning fifty. Occurs fifty days after the second day of Passover. Also known as Shavuot. One of three annual festivals in which every Jewish male makes a pilgrimage to Jerusalem. (Shavuot, Pesach and Sukkoth). John 2:13, 7:2–4; Leviticus 23:33–43; Numbers 29; Deuteronomy 16} It is the fulfilment of prophetic promises relating to the new covenant. The Holy Spirit fulfils God's purpose for the Torah by ``writing it'' on the hearts of his people. The Spirit accomplishes what the Torah could not do because of the weakness of human nature.

\subsection{New covenant promised}
\label{newcovenantpromised}

God is angry with the people of Israel when they \emph{harden their hearts against him}. He declares that he will make a new covenant with Israel. This new covenant will differ from the former one, under which God's people rebelled.\footnote{Hebrews 3:7--19, Psalm 95:7--11, Numbers 11--16, Jeremiah 31:31--34, Hebrew 8:7--13} It will bring about a new intimacy with God, characterised in these ways:

\begin{itemize}
\item God will put his Torah in the minds of his people

\item He will write his Torah upon their hearts

\item All will know him, from the least to the greatest

\item He will be merciful {\ldots} and remember their sins no more.

\end{itemize}

\subsection{Torah fulfilled in the new covenant}
\label{torahfulfilledinthenewcovenant}

The Messiah brings Torah to completion—to its intended goal. Indeed, the Messiah is the goal at which Torah aims. As well as prophet, priest and king, Jesus is the \emph{passover lamb} and the \emph{manna that comes down from heaven}. He fulfils the messianic prophecies and institutes the new covenant\footnote{Lamb: John 1:29,35, 1 Corinthians 5:7, 1 Peter 1:19, Exodus 12:5, Revelation 5:6f.; Acts 8:32. Manna: John 6:30–71. Fulfilment: Matthew 5:17--18, Romans 10:4. Greek, telos, often translated ``end'' (implying termination) is more appropriately translated ``goal, purpose, consummation.'' Understanding Torah's goal as the Messiah accords unity to New and Old Testaments and continuity to outworking of God's covenantal purposes; see Romans 3:31}.

\subsection{Another Counsellor}
\label{anothercounsellor}

After his resurrection, Jesus returns to the Father. He does not leave his disciples alone. He sends them another \emph{convicting counsellor}. This Counsellor, the \emph{Spirit of Truth}, will lead them \emph{into all the truth}. He will tell them about things that will happen in the future. He will prepare them for troubles, suffering and responsibilities lying ahead.\footnote{John 14:15--17. Frequently translated as comforter, the more appropriate sense is ``convict-er''—the Spirit who convicts God's people of God's truth. He leads them into truth. There is no comfort from the Spirit if we evade God's truth and reality.}

Jesus instructs the disciples to wait in Jerusalem, for this gift of the Holy Spirit. In this way, the disciples will receive power to be witnesses of the Messiah \emph{to the ends of the earth}. Forty days later, the Holy Spirit descends upon the disciples.\footnote{John 16:7--15, Acts 1:4--5, 8}

\subsection{Torah and Spirit}
\label{torahandspirit}

The outpouring of the Spirit coincides with the feast of Pentecost. Pentecost is a celebration of the giving of Torah to the Jewish people. Thus there is a significant parallel between the Torah and the Spirit. 

When God gave the Torah to the children of Israel, he descended upon the mountain of Horeb. It \emph{blazed with fire to the heart of heaven, with darkness, clouds and thick mist}. God spoke to the people, out of the fire, \emph{proclaiming his covenant} to them. He instructed them to obey the Ten Words, which he wrote on two stone tablets.\footnote{Deuteronomy 4:7--14, Exodus 20:1--17—Jews refer to the Ten Words: the first word being a proclamation, rather than a commandment.}

The disciples also experience a great sound and tongues of fire, when the Holy Spirit comes upon them. This echoes the fire that blazed upon Horeb. The twelve disciples represent the twelve tribes of Israel. This confirms that the Spirit is being poured out upon the whole body of God's people. 

A single tongue of fire upon each disciple signifies the gentleness and personal nature of relationship with God. In the new covenant, this is a vital aspect of the Holy Spirit's work amongst God's people. \footnote{Acts 2:2--4; Matthew 19:23–30, Revelation 21:10–14}

The outpouring of the Holy Spirit thus echoes the giving of Torah. Both mark a renewal of covenant. One led to the birth of the nation of Israel. The other gives birth to the Messianic Community. \emph{Table 2} displays multiple parallels between Torah and the Holy Spirit.\footnote{Shekinah: Divine Presence, the manifest glory of God present with human beings. Matthew 17:6, Luke 2:9, John 1:14, Hebrews 1:3, 1 Peter 4:13, 2 Peter 1:17, Revelation 7:15.}

\begin{figure}[htbp]
\centering
\includegraphics[width=300pt,height=473pt]{Table2-Torah,Spirit.png}
\label{table2-torahspirit.png}
\end{figure}

\begin{reflection}

Read: Hebrews 8:6--13, Matthew 5:17, Acts 1:8.

\begin{itemize}
\item How does the Holy Spirit fulfil the purpose of Torah?

\item How does the Spirit communicate the Messiah's heart and mind to God's people?

\item Have you received the Holy Spirit? Are you experiencing his presence?

\end{itemize}

\end{reflection}

\section{Pentecost: Messianic Community}
\label{pentecost:messianiccommunity}

\begin{quote}

\textbf{\emph{The Messianic Community is formed by the Spirit. He writes Torah on the hearts of God's people. He creates a new spirit within them. He forms them into a living temple. He forms them into the body of the Messiah.}}
\end{quote}

The outpouring of the Spirit echoes the original giving of Torah. Torah formed Israel into a nation. Similarly, the powerful work of the Spirit forms the new-covenant Messianic Community. The weakness of the first covenant was the human nature. In the new covenant, the Spirit writes Torah upon human hearts. The Spirit leads to a new intimate, personal relationship with God.\footnote{Romans 8:3--4, Ephesians 3:14--19}

\subsection{Formation of the Messianic Community}
\label{formationofthemessianiccommunity}

The New Testament describes the formation of the Messianic Community in three related categories: (a) spiritual life, (b) new creation, (c) temple and body. Together they define the essence of the Messianic Community. Some characteristics are rooted in Israel's identity; others are new. This is important as the community transitions from being ethnically defined, to becoming intercultural. As it transitions from being Torah-centric, to Spirit-centred.

\subsection{Spiritual life}
\label{spirituallife}

\begin{itemize}
\item \emph{Torah of the Spirit} — the Pharisees and Torah-teachers reduced the covenant to a legalistic interpretation of God's commandments. The new covenant transforms Torah through the Spirit's indwelling power. It establishes a new, living way of freedom, grace and liberty. In this covenant, God calls his people into a new life dominated by the Spirit of God.\footnote{Romans 8:2, Galatians 5:1, 6:2, 1 Corinthians 9:20--22, 2 Corinthians 3:17}

\item \emph{A new spirit} — God's people receive a new heart, a new \emph{breath} of the Spirit.\footnote{Greek, ruach. In Scripture, spirit, or breath, and heart are synonymous. Ezekiel 11:19, 36:26–27.} Members of the Messiah's body receive the communication of the Spirit of God.

\item \emph{A people after God's heart} — God is forming a people who will be \emph{after his own heart}. A people sharing his heart, faithful to his concerns, his purposes, his priorities. A people bearing his compassion, strength and joy within them.\footnote{Acts 13.22, Psalm 89:20, 1 Samuel 13:14.}

\item \emph{A pilgrim people} — God's people are a people on a pilgrimage. The cultures and societies in which they live cannot wholly define them. They become known as \emph{Followers of the Way}.\footnote{Acts 9:1–2, 18:25–26, 19:9,23, 22:4, 24:14,22.}

\end{itemize}

\subsection{New creation}
\label{newcreation}

\begin{itemize}
\item \emph{United with the Messiah} — through baptism into the Messiah, members of the Messianic Community are united with God. This baptism is made real by the power and presence of the Spirit of God..\footnote{Romans 8:9–11, Ephesians 1:4--11, 2:10--22, 3:6,12, 4:17--32, 5:8, 6:1--10, 2 Corinthians 1:21–22, Colossians 1:22, 2:6–7.}

\item \emph{Adopted into God's family} — members of the Messianic Community are children of God. They become joint-heirs with the Messiah.\footnote{Romans 8:14–17, 2 Corinthians 3:17–19, Galatians 3:29, Ephesians 2:19, Colossians 3:10, 2 Peter 1:41}

\item \emph{A new nature} — through receiving God's promises, the Messianic Community shares God's nature. They become alive to God and freed from slavery to sin. They are being recreated in the image of the Creator.\footnote{Romans 6:3–11, 7:5–6, 2 Corinthians 5:17, 2 Peter 1:4, Colossians 2:9–10, 3:10}

\item \emph{First-fruits of a new creation} — the resurrection of the Messiah is the greatest sign of the renewal of creation. The calling and sending of the Messianic Community is an extension of this renewal.\footnote{James 1:18, Romans 8:19–23,29, 1 Corinthians 15:20--23.}

\end{itemize}

\subsection{Temple and body}
\label{templeandbody}

\begin{itemize}
\item \emph{Body of the Messiah} — the new covenant, messianic community represents the \emph{earthly} body of the Messiah. He in turn is the \emph{heavenly} head of the body.\footnote{Ephesians 5:30. Colossians 2:19. 2 Corinthians 3:17–19}

\item \emph{Tabernacled} — the Word became a human being. The Presence of God \emph{tabernacled} amongst men. This happened so that human beings could know God the Father, through the Son, by the Spirit.\footnote{John 1:14. Typically translated ``lived amongst us,'' the original Greek is ``tabernacled''—an allusion to the Tabernacle that the Israelites constructed in the wilderness, precursor to Jerusalem Temple—Exodus 25:9. Hebrews 1:3; 2 Peter 1:17, Matthew 17:6. Philippians 2:6–8.}

\item \emph{Temple} — the Messianic Community is the temple of the Holy Spirit. This temple is made from living stones. This means human beings, united in allegiance and service to God.\footnote{1 Corinthians 3:16, 6:19, 2 Corinthians 6:16. Ephesians 2:21–22, 1 Peter 2:5. John 2:19, 10:18}

\end{itemize}

\begin{reflection}

Read: 1 Corinthians 3:16, 1 Peter 2:5

\begin{itemize}
\item Describe the work that the Holy Spirit is doing amongst God's people. What metaphors and symbols have you used?

\item New covenant realities are consistently represented by symbols and metaphors. What might this suggest about how we communicate our faith to others?

\item The Messianic Community is described by various images or metaphors. Which are most meaningful to you? Which are the most surprising?

\end{itemize}

\end{reflection}

\section{Pentecost: light to the Gentiles}
\label{pentecost:lighttothegentiles}

\begin{quote}

\textbf{\emph{The experience of the Spirit at Pentecost redefines the covenant vocation of Israel. It marks the beginning of the incorporation of Gentiles into the covenant community.}}
\end{quote}

\subsection{Every nation under heaven}
\label{everynationunderheaven}

The feast of Shavuot required Jewish adult males to make a pilgrimage to Jerusalem. When the disciples receive the Spirit \emph{religious Jews from every nation under heaven} are present. These international Jews recognise the disciples' miraculous speech, which takes place in many languages.

Peter tells the crowd that this extraordinary occurrence is a sign from God. And that God has made the man they had crucified, Jesus of Nazareth, \emph{both Lord and Messiah!}. As they are \emph{stung in their hearts} by this, Peter calls them to \emph{Turn from sin, return to God and be baptised.} He then say that God's promise \emph{is for you, for your children and for those far away — as many as Adonai our God may call!} This hints at the incorporation of people from many nations into the covenant community.\footnote{Acts 2:5, 37--39}

\subsection{Jewish and Gentile Identity}
\label{jewishandgentileidentity}

Yet it is not yet wholly clear that the Good News is about to overflow from Israel towards the Gentiles. First, a significant tension must be resolved. It requires discerning between the \emph{cultural identity} and \emph{covenant vocation} of the Jews. Another experience of the apostle Peter plays a significant part in this vital shift.

A god-fearing Roman Centurion asks Peter to visit him. Ordinarily this would have been out of the question. From a cultural perspective, it just wasn't done. Jews didn't associate closely with gentile peoples. Yet, because the Spirit instructs him to, Peter responds to the request.

Hearing of their hunger to hear about the Messiah, Peter realises how the vision relates to them. He recognises that he must no longer view any nation or people as unclean. That \emph{God does not play favourites}. Rather \emph{whoever fears him and does what is right is acceptable to him, no matter what people he belongs to}. 

Peter tells the centurion and his associates about Jesus life, crucifixion and resurrection, in fulfilment of what the Prophets wrote. As he is doing this, the Spirit falls upon and empowers the Gentiles to speak in tongues.

Peter's experience with the Gentile believers creates a ``spiritual earthquake'' amongst the Jewish believers. Until this point, the Messianic Community is entirely Jewish. Peter's experience shatters their misconceptions about what God is doing through his Messiah. They begin to recognise how his purposes extend beyond the Jewish people..\footnote{Acts 13:44--49, 10:28--9, 11:18--26}

\subsection{Gentiles incorporated into the covenant community}
\label{gentilesincorporatedintothecovenantcommunity}

A persecution scatters the Jewish believers beyond Jerusalem. They begin sharing the Good News with Gentiles and this raises an important question. How should Gentiles be incorporated into the covenant community? Since Abraham, Jewish males enter the covenant community through undergoing circumcision. To join the covenant community, must Gentiles undergo circumcision? Underlying this question is the greater issue of the how Gentiles relate to the Torah.

The apostles call an assembly to resolve the issue. They recognise the significance of the outpouring of the Spirit upon the Gentiles. It demonstrates that God is \emph{cleansing their hearts by trust}. They decide to place no heavy demands upon the Gentiles in relation to Torah. This opens the way for a full incorporation of Gentiles into the Messianic Community.\footnote{Luke 1:59 \& 2:21, Acts 15:5--31, 15:9--10. How can Gentiles integrate into the covenant community, without adopting Jewish culture? This question is at the heart of the New Testament. Jesus rebutted the presumption that Jewish ethnicity and faithfulness were synonymous, highlighting the faithfulness of Gentiles, such as the Syrophoenician woman and the Roman Centurion (Mark 7:24, Luke 7:1--5).}

\begin{reflection}

Read: Acts 11:11--18, Acts 10:34

\begin{itemize}
\item Which nations and peoples are acceptable to God?

\item Why are Jews surprised the spirit is given to Gentiles?

\item How can we discern God at work in people different to us?

\end{itemize}

\end{reflection}

\section{Pentecost: a new humanity}
\label{pentecost:anewhumanity}

\begin{quote}

\textbf{\emph{The incorporation of Gentiles into the Messianic Community is highly significant. It leads to identifying the emergence of nothing less than a new humanity.}}
\end{quote}

The Gentiles receive the Good News and enter the Messianic Community. They do so without circumcision, through baptism, as do Jewish believers. Yet questions remain. The apostle, Paul, takes up these deeper issues. He demonstrates the extraordinary reality of what has taken place. He does this by showing how far away from God were the Gentiles. And how, through the Messiah, they have now \emph{been brought near to God}.

\subsection{Once far off{\ldots}now brought near}
\label{oncefaroff...nowbroughtnear}

Paul first illustrates the former plight of the Gentiles. Outside the covenant, the Gentiles lacked hope of reconciliation with the Creator. They were:

\begin{itemize}
\item estranged from the national life of Israel,

\item foreigners to the covenants embodying God's promise,

\item without hope, without God.

\end{itemize}

Yet now, through the Messiah's death, from far away, the Gentiles are miraculously \emph{brought near to God}. They are brought into the covenant relationship, alongside the Jews. Yet without conversion to Jewish culture. How does such a dramatic spiritual relocation take place? To illustrate this extraordinary exchange of the Gentiles' status, Paul draws upon the unique, metaphorical image of the \emph{m'chitzah}.

\begin{description}

\item[m'chitzah]

a dividing wall or partition separating people into two groups. In particular, a 1.5 metre high stone partition in the Jewish temple built by Solomon. This partition separated the inner Temple courts from the Court of Gentiles. Only Jews could enter the inner courts.\footnote{Acts 21:28}
\end{description}

\subsection{Breaking down of the m'chitzah}
\label{breakingdownofthemchitzah}

The m'chitzah symbolically represents the covenantal separation between Jews and Gentiles. It embodies the denial of Gentile access to the heart of the Temple. On one side of the m'chitzah are the inner courts. The inner courts are adjacent to the Ark of the Covenant. The Ark represents the place where God's shekinah presence dwelt amongst his people.

The m'chitzah provides a profound image. On the other side of the m'chitzah is the court of the Gentiles. A physical barrier to inclusion. A persistent reminder of Gentile exclusion from God's covenant, promise and presence. This is why Paul selects the \emph{breaking down of the m'chitzah} to bring home the extraordinary truth he has understood. 

It evokes a dramatic picture of Gentile admission to the covenant. It implies a new freedom for them to approach God, on an equal footing with Jews. It symbolise the binding together of Jew and Gentile in the new covenant. A binding so significant that Paul refers to it as \emph{a new humanity}.\footnote{Ephesians 2:11–19}

\subsection{Shalom to those near and far}
\label{shalomtothosenearandfar}

The removal of this barrier points towards a greater reality. Something more profound even than access to the inner courts of the Temple. Not only is the m'chitzah removed. The veil that separated the Holiest Place from the inner courts has also been removed. In the Messiah, both Jews and Gentiles have \emph{shalom}. Members of this new humanity have \emph{access in one Spirit to the Father.\footnote{Shalom, (Hebrew): peace, tranquillity, safety, well-being, welfare, health, contentment, success, comfort, wholeness, integrity. Matthew 27.51, Ephesians 2:11–22.}} The Gentiles are no longer foreigners and strangers, excluded from covenants. Now they are:

\begin{itemize}
\item fellow-citizens with God's people

\item members of God's family

\item built on the foundation of the apostles and prophets and the Cornerstone of the Messiah

\item joined to the commonwealth of Israel

\item incorporated into the covenant community.

\end{itemize}

The result is a new covenant community. A single new humanity made up of Jews and Gentiles, united in the Messiah. A mystical body of the Messiah. A Messianic Community formed by the Spirit.

\begin{reflection}

Read: Ephesians 2:11–19, Acts 13:47--49

\begin{itemize}
\item Why were the Gentiles formerly without hope? How does this effect how we think about followers of other religions?

\item How do things change for Gentiles who unite with the Messiah?

\item Why does the New Testament talk about a new humanity?

\end{itemize}

\end{reflection}

\ssection{The New, Messianic Covenant Community}

The topics of this study, respectively:

\begin{enumerate}
\item identify significant parallels between the Torah and the Holy Spirit. These parallels illuminate the uniqueness and the power of the new covenant.

\item illustrate how the Messianic Community is formed by the Spirit. He writes Torah on the hearts of God's people. He creates a new spirit within them. He forms them into a living temple. He forms them into the body of the Messiah.

\item reveal how the Pentecostal experience of the Spirit redefines the covenant vocation of Israel. It marks the beginning of the incorporation of Gentiles into the covenant community.

\item conclude that the incorporation of Gentiles into the Messianic Community is highly significant. It leads to identifying the emergence of nothing less than a new humanity.

\end{enumerate}

\emph{Figure 5} updates the biblical panorama being constructed by Module 1. It illustrates how the Spirit's outpouring at Pentecost forms the body of the Messiah. It demonstrates the continuity of God's purpose, as the covenant is renewed. It shows how the making of disciples amongst all nations is a fulfilment of the blessing of Abraham.

\begin{figure}[htbp]
\centering
\includegraphics[width=299pt,height=218pt]{EP9-body.png}
\caption{Figure 5 : Pentecost, New Covenant and Body of the Messiah}
\label{ep9-body.png}
\end{figure}

\begin{quote}

The Father is committed is to creating a faithful covenant community. A people experiencing shalom through union with his Messiah. He pours out his Spirit to form the Messianic Community. A single new humanity of Jew and Gentile together. The mystical body of Messiah. A movement of disciples reaching out to all nations with the Good News of God's reign through his Messiah.
\end{quote}

\vspace*{\fill}\begin{bonus}

\begin{itemize}
\item What are the advantages of the new covenant, over the former?

\item What does Pentecost signify?

\item How can a body of people be a temple for the Spirit?

\item What is significant about Gentiles entering the new covenant community?

\item What behaviour ought to characterise the Messianic Community?

\end{itemize}

\end{bonus}

\chapter{Eternal Purpose}
\label{eternalpurpose}

\begin{chapsynopsis}
\begin{center}

\textbf{God's eternal purpose is fulfilled through his Messiah, Yeshua.} In him, Jews and Gentiles are united, in fellowship and covenant relationship with God. This body of people is anointed with the Spirit of Yeshua and sent forth as a great nation blessed to be a blessing to the nations of the world.

\end{center}
\end{chapsynopsis}
\begin{topics}

\begin{enumerate}
\item Heirs of Abraham

\item Inheriting the blessing

\item God's eternal purpose

\item Summary

\end{enumerate}

\end{topics}

\section{Heirs of Abraham}
\label{heirsofabraham}

\begin{quote}

\textbf{\emph{Through uniting with the Jewish Messiah, Yeshua, the Gentiles share in the covenant. They become joint heirs of Abraham.}}
\end{quote}

When the Gentiles receive the Holy Spirit, it amazes Peter and Jesus' other disciples. Was their surprise justified? Paul demonstrates that Scripture foresaw that God would \emph{justify the Gentiles by faith}. Paul explains that \emph{the Good News was announced in advance to Abraham}. It happened when God said to Abraham \emph{in connection with you, all peoples will be blessed}. Paul equates the promise given to Abraham with the Good News of the Messiah. Thus, he shows how the Good News was in God's heart and mind, centuries before, when he set Abraham apart. The \emph{new covenant} fulfils and completes the covenant with Abraham.

When God spoke to Abraham, it is his response to the human rebellion overwhelming his creation. Throughout the following eras, God cultivates a people to serve his covenant purposes. Israel is the nation chosen for this purpose. Yet Israel fails to uphold the covenant. Repeatedly, they become became insular, parochial and self-serving. In their unfaithfulness they frustrate God's purposes. Even so, \emph{in the fulness of time,} Israel gives birth to the Messiah.\footnote{Galatians 4:4--7} Through the Messiah God inaugurates a new covenant. God's fullest and greatest response to humanity's rebellion comes to fruition. He creates a \emph{new humanity}—a messianic community, formed of Jews and Gentiles together.

\subsection{God's secret plan}
\label{godssecretplan}

Paul describes this unfolding reality as nothing less than \emph{God's secret plan}. A plan kept secret for generations, now revealed God's apostles and prophets. This plan unites Jews and Gentiles in the Messiah. One Messianic Community demonstrating the \emph{many-sided wisdom of God}. Paul describes the completion of this secret plan as \emph{God's eternal purpose}.\footnote{Ephesians 3:3--11, Colossians 1:26--27, Hebrews 1:1–3}

This confirms the great significance of the Gentiles joining the covenant community. Through the Good News they are united with the Messiah. God's promise, through Abraham, to bless the nations is fulfilled. Moreover, in the Messiah, the Gentiles now become joint heirs with the Jews. They become heirs of the promise given to Abraham.\footnote{Galatians 3:7--9, 26--29}

\subsection{Heirs with Abraham}
\label{heirswithabraham}

God's promise to Abraham was of \emph{a great people, blessed to be a blessing to the nations of the world}. Those who receive the promises of God are blessed so that they may \emph{be a blessing}. In this way, they form part of God's response towards his creation.

Through the Messiah, the Gentiles now \emph{share} in God's blessing. On behalf of the Messiah, the Gentiles now \emph{bear} this blessing. Like the people of Israel before them, the Gentiles are not only called to be recipients of blessing. They also become those through whom the blessing flows outwards to other peoples. God's blessing is a profound gift. It is also a profound responsibility.

\begin{reflection}

Read: Hebrews 1:1–3, Galatians 3:7--9, 26–29, Romans 9:8

\begin{itemize}
\item What was God's secret plan? Why was it kept a secret?

\item How is the promise to Abraham related to the Good News?

\item Who may inherit the promise given to Abraham? Have you?

\end{itemize}

\end{reflection}

\section{Inheriting the blessing}
\label{inheritingtheblessing}

\begin{quote}

\textbf{\emph{A true inheritance alongside the Messiah means acting in covenant partnership with God. It means becoming God's co-workers.}}
\end{quote}

Jews and Gentiles are united in the Messiah, forming a new humanity. God forms this Messianic Community and calls them into covenant partnership with him. He empowers them for service by the Holy Spirit. This represents the fulfilment of the covenant promise first given to Abraham.

\subsection{New covenant blessing}
\label{newcovenantblessing}

The original promise to Abraham had two ``horizons.'' The horizon of Abraham's \emph{descendants} and the horizon of \emph{all the families of the earth}. The first are blessed \emph{in} Abraham. The second are blessed \emph{by} Abraham. So too with the new covenant. 

As families, tribes and peoples join the Messianic Community, they become \emph{joint heirs}. Heirs with the Messiah and heirs of the promise given to Abraham. United with the Messiah they are blessed, as Abraham was. And like Abraham they also receive the call to \emph{be a blessing to all the families of the earth}.

The Jews are the first people chosen to receive God's covenant and its blessings. They form the principal covenant community or nation. At Pentecost, God's blessing begins to overflow to the Gentiles. The Gentiles receive the new covenant blessing. Blessed, they share the responsibility of being a blessing to the families of the earth.

\subsection{Every family, nation, people}
\label{everyfamilynationpeople}

This is why the incorporation of Gentiles into the covenant community is so significant. It demonstrates how God's blessings spreads out. Each nation that receives the blessing of the Good News, is called to serve God's purposes in the earth. The Good News is always overflowing to other families, peoples, tribes and nations.

Thus, each nation receiving the \emph{blessing} of the Good News receives also the \emph{responsibility} of the Good News. They become a part of God's new covenant community. A part of God's servant nation. God's delivers his people from idolatry so that they may serve him and his eternal purpose.

He invests his blessing in the the body of Messiah amongst each tribe, people or nation. Each family of the earth. He does this so that they can reach out to other families and peoples and be a blessing to them in the power of the Spirit. The blessing of the Good News is given to the people of that tribe or nation. Yet the blessing is not intended to remain within that tribe or nation. It is intended to overflow to other tribes and peoples. This is the responsibility of the Good News.

\subsection{Failing to inherit the blessing}
\label{failingtoinherittheblessing}

In other words, God's promise to Abraham was not for his own benefit or that of his descendants alone. Abraham and his descendants are blessed for a specific purpose. They are to show God's covenant faithfulness to all nations. God's blessing is intended to overflow to those peoples who recognise God's goodness.

When Israel failed to serve the God's purposes, the blessing is stifled. Israel suffers God's judgement and the blessing of the nations cannot take place. Yet even Israel's failure does not persuade God to abandon his eternal covenant. He disciplines Israel, sometimes severely, calling them back to faithful covenant service. Above all though, he promises a new covenant that would not be like the old one. A new covenant that will produce the intimacy, faithfulness and fruitfulness he desired.

The New Covenant Community, the Church, has also become insular, self-serving, even idolatrous. For significant periods it too has failed in its vocation to be a blessing to the nations of the world. Yet again and again the blessing of God has flowed out towards the nations of the world. When the covenant community fails, God's purposes abide.

\subsection{The responsibility of the blessing}
\label{theresponsibilityoftheblessing}

Today, God's blessing remains on those united with the Messiah through their faith. They are a people blessed to be a blessing to all the nations of the world. With the blessing comes the responsibility. To receive the blessing without being a blessing to others is not an option. It represents faithfulness to the purpose of the New Covenant. It represents a rejection of the terms of the covenant. Receiving the Messiah and joining the covenant community brings significant responsibility. A true inheritance alongside the Messiah means acting in covenant partnership with God. \emph{It means becoming God's co-workers.}

\begin{reflection}

Read: Romans 8:16--17, 1 Corinthians 3:1, 2 Corinthians 6:1

\begin{itemize}
\item What is the significance of the two ``horizons'' relating to the Abrahamic covenant?

\item Why are the families of the earth significant to God?

\item How is the blessing of God a responsibility? How does this effect you?

\end{itemize}

\end{reflection}

\section{God's eternal purpose}
\label{godseternalpurpose}

\begin{quote}

\textbf{\emph{Through human history, as empires rise and fall, God's commitment abides. In the Messiah and the Messianic Community his eternal purpose arrives at its fullness.}}
\end{quote}

Throughout history, God has worked out his purposes in partnership with human communities. He does this through establishing a series of covenants, with Abraham and his descendants. With each covenant, God makes clear his commitment. \emph{Table 3} summarises these covenant commitments and the purpose established with each one.

\begin{figure}[htbp]
\centering
\includegraphics[width=299pt,height=234pt]{Table3-covenants.png}
\label{table3-covenants.png}
\end{figure}

\begin{reflection}

Read: Ephesians 3:4--11

\begin{itemize}
\item Why was God's plan kept hidden from earlier generations?

\item What is God's eternal purpose? How is it related to the kingdom of God and the Messianic Community?

\item What is your response to God's eternal purpose?

\end{itemize}

\end{reflection}

\ssection{Eternal Purpose}

Through his covenants God reveals his eternal purpose accomplished in the Messiah Yeshua. This eternal purpose comes to fruition in the new, messianic covenant community. A community of Jews and Gentiles united together. A community reconciled to God's purposes. Experiencing forgiveness of sin and a profound intimacy mediated by his Spirit. Brought near to him in shalom. 

\begin{itemize}
\item Those united with the Messiah are joint-heirs with him of the promise given to Abraham: \emph{a great nation{\ldots}blessed to be a blessing.}

\item This Messianic Community is anointed by the Spirit. They are a \emph{sign and a foretaste of God's kingdom}.

\item They are a community of disciples. They are sent in the power of the Spirit, \emph{to make disciples amongst all the nations of the world.}

\end{itemize}

Figure 6 updates the biblical panorama constructed by Module 1.

\begin{figure}[htbp]
\centering
\includegraphics[width=299pt,height=207pt]{EP10-kingdom.png}
\caption{Figure 6 : God's eternal purpose}
\label{ep10-kingdom.png}
\end{figure}

\begin{quote}

The apostles and prophets form the foundation stones of the living temple that God is building. Through reconciliation with God, disciples are made amongst the nations. God's people are a sign and symbol of the kingdom of God. They point towards the ancient hope of a renewal of heaven and earth.\footnote{2 Corinthians 5:11–21; Isaiah 65:17, 66:22, 2 Peter 3:13, Revelation 21:1, 27; Matthew 16:18, 1 Corinthians 12:28, Galatians 2:9, Ephesians 2:20, 4:11–3, Revelation 21:14.}
\end{quote}

\input{mpd-footer}

\end{document}
