\def\change{Hemingway}
\def\version{0.6.0}
\input{mpd-header}
\def\mytitle{Commissioning}
\def\module{2}
\def\translation{}
\def\isbn{978-1-907191-04-6}
\def\colour{false}
\input{mpd-document-en}

\chapter{Commissioning}
\label{commissioning}

\begin{chapsynopsis}
\begin{center}

\textbf{The Messianic Community has been commissioned to work alongside the Messiah in his mission}

\end{center}
\end{chapsynopsis}
\begin{topics}

\begin{enumerate}
\item Military commissioning

\item God's commissioning

\item Commissioned as disciples

\end{enumerate}

\end{topics}

\section{Military commissioning}
\label{militarycommissioning}

\begin{quote}

\textbf{\emph{A soldier must pass through three distinct stages: calling, training, commissioning.}}
\end{quote}

\emph{Calling} draws them into military service. \emph{Training} equips them for dutiful and skilled service. \emph{Commissioning} sends them into active, vocational service. The preparation of a military soldier provides a useful metaphor. It highlights important aspects of how the Messiah prepares us to be his disciples.

\subsection{Calling}
\label{calling}

A soldier's life starts when they perceive a \emph{calling} to a life of military service. A calling is the awareness that an occupation is a desirable, compelling, or appropriate. There may be many reasons why a person enlists in an army, but at some point they sense a personal calling towards it.

\subsection{Training}
\label{training}

A period of intense \emph{basic training} tests a soldier's calling. This basic training lays a foundation that equips them for a life of military service.

\begin{itemize}
\item Trials and hardships test trainees in many and varied ways. Including discipline, teamwork, communication, competence, obedience, initiative, determination, loyalty and resolve.

\item Each soldier's capabilities and characteristics are either affirmed, enhanced or exposed as inadequate. At the end, training officers assess whether each soldier has satisfactorily completed basic training.

\item \emph{Failure} means a soldier must repeat basic training. They must do so until either they pass, or leave military service altogether.

\item \emph{Success} means a soldier is equipped and ready for active service. Training officers deem them capable, ready and trusted to fulfil their duty. This includes general soldiering duties \emph{and} specific vocational responsibilities. For example: infantryman, driver, engineer, medic, officer, chef and so on.

\end{itemize}

\subsection{Commissioning}
\label{commissioning}

A soldier's commissioning represents the beginning of their active service. Superior officers appoint them to join a specific regiment, according to their vocation. A commissioning ceremony is overseen by the head of the army. By tradition, this is usually a nation's king or queen. This provides a vivid illustration of the calling to ``serve the crown,'' as a loyal subject.

Commissioned soldiers continue to lead disciplined lives and to experience demanding trials and tests. Yet there are significant differences to the training phase. Failure is no longer an option. The entire military apparatus supports commissioned soldiers, providing every opportunity for success. In this new phase of service, significant achievements receive the reward of enhanced responsibility.

\begin{reflection}

Read: Deuteronomy 3:28

\begin{itemize}
\item How does military soldiering differ from civilian life?

\item How helpful is the metaphor of soldiering in your context? 

\item What other lifestyles involve calling, training and selection?

\end{itemize}

\end{reflection}

\section{God's commissioning}
\label{godscommissioning}

\begin{quote}

\textbf{\emph{God's commissioning of a covenant community begins with Abraham. It continues with the people of Israel. God unites them with his mission through covenant. The new covenant unites the messianic community with the Messiah's mission.}}
\end{quote}

There is a significant linguistic relationship between ``mission,'' ``sending'' and ``commissioning.'' It illuminates the true meaning of mission.

\subsection{The sending of God}
\label{thesendingofgod}

The English word, mission⁠ was originally used only to refer to God's sending of himself.\footnote{Missio (Latin) means ``sending.'' Missio Dei (the mission or sending of God) is a theological concept, similar to ``God's eternal purpose'' (Module 1).} His entering into the world, to restore the world. To redeem it from the effects of human wickedness and idolatry. To restore it from chaos and evil. To rescue it from oppression and injustice. Scripture identifies the mission or sending of God, in three particular ways.

\begin{itemize}
\item \emph{The shekinah}. The pillar of cloud and fire during the exodus from Egypt. The cloud of glory filling the Tabernacle of Meeting. The fire and glorious presence at Solomon's dedication of the temple.\footnote{Shekinah (Hebrew): the ``glorious presence of God dwelling amongst his people''. Exodus 13:17–22, Exodus 40, 2 Chronicles 7:1–3 (cf. 6:18).}

\item \emph{The Messiah}. Sent to Israel as \emph{lamb of God, apostle, prophet, king} and \emph{high priest⁠}. He is ultimately identified as \emph{the radiance of the Shekinah. The very expression of God's essence{\ldots}the visible image of the invisible God}. In every way, Jesus the Messiah reflects the reality that God the Father is a missional god.\footnote{See Study2, Topic 3—Vocation of the Messiah. John 1:29, Hebrews 3:1 (4:14–16). Radiance of shekinah. Luke 2:9, John 1:14, 2 Peter 1:17, Matthew 17:6, Colossians 1:15,19, Hebrews 1:1–3.}

\item \emph{The Holy Spirit}. The Breath⁠ of Yeshua, sent by the Messiah, as the Father sent Yeshua. The Holy Spirit empowers and \emph{sends} the Messianic Community. He anoints it to do the works of God and to become a living temple.\footnote{John 6.28, 14:12–17 and 16:7–11. Ruach (Hebrew): breath, or spirit. Of Yeshua. Acts 16:7. Temple. 1 Corinthians 3:16–17; 1 Peter 2:4–5.}

\end{itemize}

\subsection{The sending of God's people}
\label{thesendingofgodspeople}

God expands his mission through calling and sending his people. This expansion happens through the biblical covenants. God's people unite with God and his mission. 

\begin{itemize}
\item When``co-'' is joined to another word, it means to join, unite or share in that thing. Thus ``co-mission'' means to join, or unite with a particular mission or purpose. From this comes the concept of commissioning. It means entrust a responsibility for that mission to another.

\item God commissions the covenant community of Israel. He unites them with his mission to reconcile and restore people to his purposes. This commissioning begins with Abraham and continues with the covenant community of Israel.

\end{itemize}

\subsection{Joining the Messiah's mission}
\label{joiningthemessiahsmission}

Through the Messiah and the new covenant, God renews and expands his commissioning.

\begin{itemize}
\item \emph{The Father sends the Son}: to do the works of God and to destroy the works of evil. This is the Messiah's mission.

\item \emph{The Messiah commissions his disciples}: ``As the Father sent me, so I now send you{\ldots}receive the Holy Spirit.'' Joining the Messiah's mission means joining him in doing the works of God. 

\item \emph{The Spirit commissions the messianic community}: to be a blessing to all the peoples of the world. Towards the world, for the sake of the world.

\end{itemize}

\begin{reflection}

Read: Exodus 13:17–22, John 1:14, 14:12–14, 16:7–15

\begin{itemize}
\item What is significant about God sending himself into the world?

\item What does it mean to be sent as the Messiah was sent?

\item Have you been commissioned by the Messiah? Explain.

\end{itemize}

\end{reflection}

\section{Commissioned as disciples}
\label{commissionedasdisciples}

\begin{quote}

\textbf{\emph{The military commissioning process provides a metaphor for the spiritual commissioning of messianic disciples}}. 
\end{quote}

Like soldiers, disciples must respond to a call to serve. Soldiers must please their overseeing officers. Disciples must learn to please the One who enlists them: the Lord, the Messiah.

\subsection{Called to serve}
\label{calledtoserve}

\emph{Military commissioning} (\autoref{militarycommissioning}), explored three principal stages of becoming a soldier. The process begins with hearing a call to service. Messianic discipleship likewise starts by discerning a calling to serve. God calls the messianic community to serve his eternal purpose. As we join this community, we discern how God is calling us to serve him and his mission.

\subsection{Many are called, few are chosen}
\label{manyarecalledfewarechosen}

Soldiers must endure and pass a period of basic training. Only then are they commissioned and sent into active service. Jesus told his disciples: \emph{Many are called, but few are chosen.\footnote{Matthew 22:1--14}} This suggests that discerning the call to serve God's mission is only a starting point. Progressing from calling to commissioning requires yielding to the demands of vocational service. Putting aside our personal desires to learn both basic spiritual disciplines and specialised skills. If we please the Lord, we will receive his commissioning.

\subsection{Coworkers, chosen, appointed and anointed}
\label{coworkerschosenappointedandanointed}

Calling, training and commissioning is how the Messiah prepares his disciples for useful service. It is how we become co-workers with him in his mission. It is how we are chosen by him. How we are appointed and anointed to work alongside him, in his mission to the world.

\begin{reflection}

Read: 2 Timothy 2:3--4 and 20--21

\begin{itemize}
\item What messianic disciplines are comparable to basic training?

\item How might a messianic disciple fail basic spiritual training?

\item How can someone who has failed be restored to useful service?

\end{itemize}

\end{reflection}

\ssection{The Commissioning of Messianic Community}

The topics of this study, respectively:

\begin{enumerate}
\item explore the calling, training and commissioning of soldiers;

\item explain how \emph{commissioning} means joining together in mission, illustrating how the messianic community joins with the Messiah's mission;

\item compare military service with messianic discipleship, highlighting similarities of obedience, faithfulness, duty and reward.

\end{enumerate}

\begin{quote}

\textbf{The Messianic Community has been commissioned to work alongside the Messiah in his mission}. A covenant community of disciples, called, equipped and chosen—commissioned—to serve God's eternal purpose.
\end{quote}

\vspace*{\fill}\begin{bonus}

\begin{enumerate}
\item What responsibility is implied by the giving and receiving of a commission?

\item How do the \emph{duties} of a commission differ from the \emph{disciplines} of basic training?

\end{enumerate}

\end{bonus}

\chapter{Strategies}
\label{strategies}

\begin{chapsynopsis}
\begin{center}

\textbf{Scripture reveals a cycle of five strategies empowering the mission of the Messianic community.} The cycle may be summarised: Pray, Reconcile, Disciple, Teach, Send.

\end{center}
\end{chapsynopsis}
\begin{topics}

\begin{enumerate}
\item The strategy of prayer

\item The strategy of reconciliation

\item The strategy of discipleship

\item The strategy of teaching

\item The strategy of sending

\end{enumerate}

\end{topics}

\section{Prayer}
\label{prayer}

\begin{quote}

\textbf{\emph{The foundational strategy of messianic mission is prayer.}}
\end{quote}

\begin{figure}[htbp]
\centering
\includegraphics[width=95pt,height=40pt]{pray.png}
\label{pray.png}
\end{figure}

There are many forms of prayer. Strategic prayer is a forward-thinking, planned priority, focussed upon a clear purpose. It represents a fundamental commitment to sharing in God's kingdom and mission.

\subsection{A plentiful harvest}
\label{aplentifulharvest}

When Jesus proclaimed the kingdom of God in Israel, he compared a ripe, abundant harvest of crops to crowds of distressed and dejected people, who were to him \emph{like sheep without a shepherd.} He says to his disciples:

\begin{quote}

The harvest is indeed plentiful, but the labourers are few. So pray to the Lord of the harvest to force and thrust out labourers into his harvest.\footnote{Matthew 9:35--38 ABV}
\end{quote}

By speaking of \emph{the Lord of the harvest} Jesus affirms how significant the harvest is to the Lord. As a crop of maize is crucial to an African farmer, a harvest of people is crucial to the Lord.

\subsection{A harvesting problem}
\label{aharvestingproblem}

Although the harvest is plentiful, Jesus identifies a problem. There is a shortage of workers ready and willing to gather it in. A shortage of trained disciples. A shortage of people commissioned and sent into service. 

How does Jesus teach his disciples to respond to this challenge? He points them towards the most fundamental strategy of mission. \emph{Pray to the Lord of the harvest{\ldots} to force and thrust out labourers into the harvest field.}

Focussed, intentional prayer is hence the start point of strategic mission. There is a need to intercede for workers, loyal to Lord of the Harvest. To pray for disciples to be empowered by the Spirit. And to pray for them to sent out into effective service of God's eternal purpose.

\begin{reflection}

Read: Matthew 9:35–38, Ephesians 6:18

\begin{enumerate}
\item What makes prayer strategic?

\item What hinders strategic prayer?

\item Why are ``the labourers'' few? Is this always true?

\end{enumerate}

\end{reflection}

\section{Reconciliation}
\label{reconciliation}

\begin{quote}

\textbf{\emph{The second strategic phase of messianic mission is reconciliation with God, through his Messiah.}}
\end{quote}

\begin{figure}[htbp]
\centering
\includegraphics[width=178pt,height=40pt]{reconcile.png}
\label{reconcile.png}
\end{figure}

The Messianic Community is called to be God's agent of reconciliation in the earth. Before we can participate in reconciling others, we must ourselves be reconciled. Unless reconciled to God, we cannot reconcile others.

\subsection{An end to hostility}
\label{anendtohostility}

Reconciliation means the end of hostility. We cannot find peace with God, if we live with hostility towards others. We are not reconciled with God if we harbour anger towards others. Loving God and loving our neighbour go hand-in-hand. If we would be reconciled to God, we must first be reconciled to our neighbour. To anyone who has something against us.\footnote{Matthew 22:36–40; 1 John 4:20; Matthew 5:23–24; Luke 10:25–37}

\subsection{A holistic process}
\label{aholisticprocess}

Reconciliation is not partial. It is a holistic process of being reconciled to God and his purposes. It is not only spiritual; it is practical. We experience peace with God when we submit our entire being to him. This includes how we live our daily lives, in our homes and our workplaces. It includes how we treat both those we know and those who are strangers to us. We cannot pick and choose which areas of our lives that we will submit to God. Reconciliation includes:

\begin{itemize}
\item receiving God's forgiveness for our wrongdoing;

\item forgiving others for wrongs inflicted on us;

\item being reconciled with those who have something against us;

\item deliverance from dominant sinful behaviour;

\item cleansing from spiritual and practical impurity;

\item renunciation of idols and idolatry;

\item wholehearted devotion to the Messiah;

\item loving our neighbour and our enemies;

\item serving God's eternal purpose.

\end{itemize}

\begin{reflection}

Read: 2 Corinthians 5:19--21, Colossians 1:22

\begin{enumerate}
\item What does it mean to be reconciled to God and his purposes?

\item What issues hinder your household from experiencing reconciliation and peace with God?

\item Are you reconciled to your neighbour? If not, what will you do?

\end{enumerate}

\end{reflection}

\section{Discipleship}
\label{discipleship}

\begin{quote}

\textbf{\emph{The third strategy of messianic mission is the formation of disciples.}}
\end{quote}

\begin{figure}[htbp]
\centering
\includegraphics[width=261pt,height=40pt]{disciple.png}
\label{disciple.png}
\end{figure}

The formation of disciples is the heart of the messianic community. The messianic commission calls for the formation of disciples from people of all nations.

\subsection{Discipleship deals with our hearts}
\label{discipleshipdealswithourhearts}

Messianic discipleship is learning to serve the Messiah, with a whole heart. The heart is the centre of our being, the seat of our motivation, our willpower, our commitment. The heart is where we accept the discipline of serving God's purposes, rather than our own. It is the place where God's spirit touches and renews our spirit.

Unless the Spirit transforms our heart, we remain only religious converts. Undisciplined individuals engaging in religious activity. Processing religious words with our natural mind, whilst our heart remains unchanged. Demonstrating no real change to our allegiances, our loyalties, lifestyle, character or will.

\subsection{Continual conversion}
\label{continualconversion}

The spiritual transformation of our hearts is not a singular, instantaneous event. It is a continual process of encounter, challenge, testing and yielding to the Messiah. Our hearts are first opened and then enlarged by the demands of mission and faithfulness. 

As we are being formed into disciples, we face a choice. We may resist and harden our hearts towards God. Or we may yield and make more of our hearts available to him. When we yield, the Spirit of God transforms our hearts. He leads us towards a new, disciplined life, as co-workers of the Messiah.

\subsection{Hearts and minds}
\label{heartsandminds}

Discipleship prepares us for faithful, vocational service. It is not learned in a classroom. It takes place in the contexts of community, work and daily life. Discipleship must precede exposure to concentrated biblical teaching, which is for proven, committed disciples.

\begin{reflection}

Read: Matthew 28:19--20, 1 Corinthians 3:9

\begin{enumerate}
\item Be honest: to what is your heart devoted?

\item What compromises your whole-hearted commitment to serving God's purposes?

\item What are the true marks of a disciple?

\end{enumerate}

\end{reflection}

\section{Teaching}
\label{teaching}

\begin{quote}

\textbf{\emph{The fourth strategic phase of messianic mission is teaching.}}
\end{quote}

\begin{figure}[htbp]
\centering
\includegraphics[width=285pt,height=33pt]{teach.png}
\label{teach.png}
\end{figure}

Scripture refers to \emph{solid food}, or \emph{strong meat}, as a metaphor for challenging teaching. The writer of \emph{Hebrews} chastises his readers that they should by now be teaching others. Yet instead they still need milk—a metaphor for elementary teaching.\footnote{The translation ``strong meat'' captures well the sense of maturity required to ingest and digest challenging scriptural teaching. Hebrews 6:1--3} Strong meat is for spiritual, mature, committed disciples. For them, Scripture is a source of insight, conviction, trust, knowledge, wisdom and understanding. It develops and deepens appreciation of messianic life, spirituality and vocational service.

The apostle Paul refers to the word of God in its fullness or whole counsel of God.\footnote{Colossians 1:25; also Acts 20:27 ff.} This fullness incorporates three complementary strands of biblical teaching. These strands are: pastoral-evangelistic, prophetic and apostolic.

\subsubsection{Pastoral-evangelistic teaching}
\label{pastoral-evangelisticteaching}

Pastoral-evangelistic teaching explores the accumulated wisdom, knowledge, understanding and traditions of messianic community.\footnote{1 Timothy 3:15} It applies this knowledge to:

\begin{itemize}
\item Call disciples into \emph{the Way} of the Messiah\footnote{Acts 9:2; 18:25--26; 19:9,23; 22:4; 24:14,22; cf. John 14:4--6}

\item Systematic study and interpretation of scripture

\item Integrating church history and cultural identity.

\end{itemize}

Pastoral-evangelistic teaching has dominated Western Christianity—with mixed results. Systematic doctrine has produced strong, rooted and active messianic communities. Yet the dogmatic defence of denominational tradition has bred aggression and violence. And the justification of corrupt political power.\footnote{Christian belief has even been used to justify the tyranny of anti-Semitism, colonialism, racism and racial apartheid.} Pastoral-evangelistic teaching needs balancing with prophetic insight and apostolic, missionary vision.

\subsubsection{Prophetic teaching}
\label{propheticteaching}

Prophetic teaching interprets historical, contextual signs of the times. It identifies kairos moments. It identifies God's mission to a specific context, at a specific time. It calls upon messianic communities to respond in light of God's ``here and now'' word. It critiques organisations, structures and sub-cultures, insisting they must authentically serve God's mission.\footnote{Matthew 16:3; see also 1 Chronicles 12:32. Kairos (Greek) means ``appointed time,'' of God's choosing, relating to a particular event. Contrasts with ``chronos,'' chronological time.}

Historically, prophetic teaching has tended to represent a threat to the status quo. This leads to the marginalisation of prophetic movements and the deepening of mainstream inertia. Equally, sectarian prophetic movements have become extreme in their views and practices. Prophetic insight needs balancing with pastoral-evangelistic teaching and apostolic vision.

\subsubsection{Apostolic teaching}
\label{apostolicteaching}

Apostolic teaching identifies God's global mission. It takes responsibility for the things that are on God's heart. It emerges from the heart of practitioners, doers—those who have themselves already acted. It does not only speak about God's mission: it demonstrates how God is at work. It reflects the pioneer spirit of apostolic mission. It links local communities with global, inter-cultural mission.

Apostolic teaching relates God's eternal purpose to human cultures. On the one hand, offering and relating the word of life to those outside the messianic community. On the other, critiquing and challenging human culture, in the light of God's Word.\footnote{Philippians 2:16}

\subsection{Spiritual revelation}
\label{spiritualrevelation}

The spirit of wisdom and revelation is essential to messianic teaching. Revelation is a gift of the Spirit, opening our understanding to spiritual truths. The Spirit expands our human understanding to incorporate spiritual truths. Truths neither obtained by, nor received by the action of our natural, rational minds. Truths revealed instead to our spirit.\footnote{Ephesians 1:16--19; Luke 24:45--47}

A vital key to experiencing wisdom and revelation is the desire to do God's will. Jesus promised that Anyone who chooses to do the will of God will find out whether my teaching comes from God. And that Everyone who asks receives; the one who seeks finds and to the one who knocks, the door will be opened.\footnote{John 7:17, Luke 11:10}

\begin{reflection}

Read: 1 Corinthians 2:13--14, Hebrews 5:12–14, John 7:17

\begin{enumerate}
\item What teaching you would describe: as ``milk,'' as ``elementary,'' as ``strong meat''? Give examples of each.

\item How have you responded to God's ``strong meat''?

\item How do we experience spiritual revelation?

\end{enumerate}

\end{reflection}

\section{Sending}
\label{sending}

\begin{quote}

\textbf{\emph{The fifth strategy of messianic mission is sending.}}
\end{quote}

\begin{figure}[htbp]
\centering
\includegraphics[width=292pt,height=27pt]{send.png}
\label{send.png}
\end{figure}

\emph{Sending} represents a culmination of the four strategies that have preceded it. God sends the Messianic Community towards the peoples and cultures of the world. He sends them \emph{inter-culturally} and \emph{intra-culturally}.

\subsection{Inter-cultural contexts}
\label{inter-culturalcontexts}

Intercultural mission is the sending of the Messianic Community to different cultural, ethnic and geographical contexts. It calls for:

\begin{itemize}
\item Formation and support of apostolic missionary teams and agencies

\item Specialist intercultural training, experience and understanding

\item Translation of messianic identity and vocation into culturally-appropriate practise

\end{itemize}

\subsection{Intra-cultural contexts}
\label{intra-culturalcontexts}

Intra-cultural mission is the sending of the Messianic Community towards its indigenous cultural context. It calls for:

\begin{itemize}
\item Cooperating with the Messiah's purposes wherever we are, whatever our vocational role .

\item Participation in industry, commerce, arts, sports, media, health, civil and other government services.

\item Seeking God's blessing in the vocational contexts of home, family, community and workplace.

\end{itemize}

\subsection{In harmony with the Spirit}
\label{inharmonywiththespirit}

Whether inter- or intra-culturally, God is the one who equips, empowers and sends workers, by his Spirit. When a particular community or workers are sent into a context, that sending must be in harmony with the Holy Spirit. When we send people, our role is to uphold the Lord's sending.

\begin{reflection}

Read: Acts 13:1--4, Romans 10:14--15, Galatians 2:8

\begin{enumerate}
\item How does God ``send'' the whole Messianic Community?

\item How does \emph{inter-cultural} differ from \emph{intra-cultural} sending?

\item Are we sent once, regularly, or continuously?

\end{enumerate}

\end{reflection}

\ssection{Strategies of the Messianic Commission}

This study identifies five foundational, missional strategies: 

\begin{enumerate}
\item strategic prayer

\item holistic reconciliation with God

\item formation of disciples

\item teaching the spiritually mature

\item sending people into mission.

\end{enumerate}

Each generation of disciples carries a responsibility to renew the whole process. This generational impetus is integral to the entire strategic process. It generates a cyclical process capable of establishing an expanding, missional movement.

\begin{figure}[htbp]
\centering
\includegraphics[width=273pt,height=253pt]{cycle.png}
\label{cycle.png}
\end{figure}

\textbf{Scripture reveals a cycle of five strategies empowering the mission of the Messianic community.} The cycle may be summarised: Pray, Reconcile, Disciple, Teach, Send.

\vspace*{\fill}\begin{bonus}

\begin{enumerate}
\item Which strategies are \emph{most-effectively} pursued in your context?

\item Which strategies are \emph{least-effectively} pursued in your context?

\item Towards which strategy do you feel called to contribute most actively?

\end{enumerate}

\end{bonus}

\chapter{Structures}
\label{structures}

\begin{chapsynopsis}
\begin{center}

\textbf{The Messianic Community has a God-ordained structure that uniquely equips it to fulfil the Messianic Commission}

\end{center}
\end{chapsynopsis}
\begin{topics}

\begin{enumerate}
\item Pastoral, evangelistic communities

\item Teaching and training centres

\item Apostolic missionary teams

\item Prophets, priests, mediators

\item Structures and strategies

\end{enumerate}

\end{topics}

\section{Pastoral, evangelistic communities}
\label{pastoralevangelisticcommunities}

\begin{quote}

\textbf{\emph{Pastoral evangelistic communities express a dual pastoral and evangelistic character.}}
\end{quote}

\begin{figure}[htbp]
\centering
\includegraphics[width=125pt,height=125pt]{pastoral.png}
\label{pastoral.png}
\end{figure}

\begin{description}

\item[Pastoral]

from \emph{pastor}, meaning shepherd. Reflecting the selfless, humble, affirming, protective, overseeing discipline, guidance and care of the Good Shepherd, Jesus.\footnote{John 10.11–16; Psalm 23:1, 80:1; Isaiah 40:11; Ezekiel 34:12, 23, 37:24; 1 Peter 2:25, 5:4.}

\item[Evangelistic]

from \emph{evangel} (Greek, \emph{euangelion}), announcement of Good News. Reflecting Jesus' Lordship over the Messianic Community, people everywhere and the spiritual powers influencing society.\footnote{Mark 1:1Acts 5:31, 10:36–42, Hebrews 1}
\end{description}

\emph{Pastors} and \emph{evangelists} equip messianic communities to express a pastoral, evangelistic character. Their role is to inspire, facilitate, catalyse. To equip whole communities—of all ages, abilities and types. To enable them support one another with hospitality and pastoral care. And to share faith with others, beyond the Messianic Community. Communities express pastoral-evangelistic character through \emph{faithfulness}, \emph{hospitality} and \emph{celebration.}

\subsection{Community faithfulness}
\label{communityfaithfulness}

Community faithfulness reflects a unity of trust in and faithfulness towards the Messiah. An expression of confidence in the Good News. Believing and living in the light of what God has done for people everywhere, through his Messiah.

Communities express faithfulness through love. Through sharing each other's concerns and burdens. This is a vital aspect of trusting in and proclaiming faithfulness to the Messiah. Love-in-action provides a living demonstration of the Messiah's victory over human self-centredness.\footnote{Galatians 5:6, John 13:35, 1 John 3:14 \& 4:20}

\subsection{Community hospitality}
\label{communityhospitality}

The root meaning of hospitality is to host others. Hospitality implies friendliness, kindness, warmth, welcoming, care, openness, acceptance and concern. It is especially critical towards those who are foreigners, strangers and outsiders.⁠\footnote{Deuteronomy 10:19; cf. 1 Peter 2:11--12}

\begin{itemize}
\item Being an inviting, hospitable community, requires more than inviting people to events. On a practical level, it means opening our hearts and our homes to one another. It means expressing faith and love in a way that invites outsiders to feel included.⁠\footnote{Colossians 4:6; Philippians 2:4}

\item The ultimate invitation is to join the Messianic Community. To give the Messiah wholehearted allegiance—through celebration, service and faithfulness.

\end{itemize}

\subsection{Community celebration}
\label{communitycelebration}

Community celebrations express honour, gratitude and commitment towards God. They commemorate his goodness and love towards his creation.

\begin{itemize}
\item In the Torah, covenant feasts⁠ incorporate prophetic signs. These signs point towards both what God has provided for his people and his call to faithful service.\footnote{Torah: the first five books of the Bible; the covenant foundation of Israel.}

\item The annual Passover is the most significant Hebraic celebration. The new covenant fulfils the Passover. The sharing of bread and wine represents the body and blood of the Messiah. It speaks of the Passover \emph{Lamb of God}, who sacrificed his life to serve God's eternal purpose.\footnote{Exodus 12; John 1:29; 1 Corinthians 5:7; 1 Peter 1:19--21; Mark 14:12}

\item When we share broken bread with one another, we celebrate the Messiah's sacrifice. It also reminds us of the devotional service to which the messianic community is called.

\end{itemize}

\subsection{Two characteristics, one community}
\label{twocharacteristicsonecommunity}

The pastoral and evangelistic characteristics of messianic communities are inseparable. A faithful, celebratory community will also be hospitable, open and welcoming towards outsiders. Such a community is both a practical embodiment and a living proclamation of the Good News.⁠\footnote{1 Peter 2:12}

Pastoral, evangelistic communities thus express both the shepherding of Jesus and the kingdom of God. Whole communities working towards a shared goal of \textbf{discipling people into faithful allegiance to God's Messiah}. Facilitating and encouraging significant spiritual and practical expressions of faithfulness. In homes, workplaces, organisations and community arenas.

\begin{figure}[htbp]
\centering
\includegraphics[width=150pt,height=150pt]{allegiance.png}
\label{allegiance.png}
\end{figure}

\begin{reflection}

Read: Galatians 5:6, Ephesians 4:11–13

\begin{enumerate}
\item How do we experience and express allegiance to families, tribes, sports teams, nations?

\item How is being in alliance with the Messiah significant?

\item With whom are you in alliance? How do you express it?

\end{enumerate}

\end{reflection}

\section{Teaching and training centres}
\label{teachingandtrainingcentres}

\begin{quote}

\textbf{\emph{Teaching and training centres complement the formation of disciples in pastoral, evangelistic communities.}}
\end{quote}

\begin{figure}[htbp]
\centering
\includegraphics[width=125pt,height=125pt]{teaching.png}
\label{teaching.png}
\end{figure}

Their function is to equip mature messianic disciples for effective, missional, vocational service. This requires exploring the tension between culture and faith. Researching, understanding, presenting and explaining the content of God's Word. Enabling messianic communities to become \emph{pillars and foundations of truth}.\footnote{1 Timothy 3:15} Relating God's word and call to cultures shaped by other spiritual and moral values.

The full scope of messianic mission incorporates both intra-cultural and inter-cultural contexts. Each of which provides different challenges.

\subsection{Intra cultural contexts}
\label{intraculturalcontexts}

Intra-cultural training equips disciples to live amongst their own people. In the midst of their own culture.

\begin{itemize}
\item Equipping disciples with a messianic worldview. A way of understanding and relating to the world with a scriptural, messianic perspective.

\item Edifying—encouraging, strengthening and correcting—the practices and self-understanding of messianic communities.

\item \emph{Examples:} Bible schools, conferences, seminars, workshops.

\end{itemize}

\subsection{Intercultural contexts}
\label{interculturalcontexts}

Intercultural teaching and training equips disciples to live amongst people of a different culture.

\begin{itemize}
\item Preparing disciples with spiritual confidence and practical resources to operate in foreign contexts.

\item Researching, presenting, explaining, understanding the worldviews of people from other cultures and religions.

\item \emph{Examples:} Scripture translation, language learning, cross-cultural training, missionary trips.

\end{itemize}

\subsection{Teachers and trainers}
\label{teachersandtrainers}

Messianic teachers and trainers are critical to the formation of mature disciples. To providing a faithful understanding of Scripture. To equipping them for works of service.

\begin{itemize}
\item Training—emphasises learning from the experience of others. It prepares learners to be responsive, effective and accountable in \emph{specific, predictable} contexts.

\item Teaching—emphasises the value of knowledge and understanding. It equips learners to analyse, internalise and, later, utilise knowledge in \emph{multiple, variable} contexts.

\end{itemize}

Teaching and training centres complement the formation of disciples in pastoral, evangelistic communities, by \textbf{equipping mature disciples with biblical truth}. Enabling them to appropriately expresses allegiance to the Messiah, in a range of cultural contexts.

\begin{figure}[htbp]
\centering
\includegraphics[width=150pt,height=150pt]{truth.png}
\label{truth.png}
\end{figure}

\begin{reflection}

Read: Acts 19:8--11, 1 Timothy 3:15

\begin{itemize}
\item Who benefits most from biblical teaching?

\item How does culture effect our understanding of mission?

\item What is your attitude to Scripturally-based teaching?

\end{itemize}

\end{reflection}

\section{Apostolic missionary teams}
\label{apostolicmissionaryteams}

\begin{quote}

\textbf{\emph{Two significant metaphors illustrate the function of apostles: architect and ambassador.}}
\end{quote}

\begin{figure}[htbp]
\centering
\includegraphics[width=149pt,height=150pt]{apostolic.png}
\label{apostolic.png}
\end{figure}

\begin{description}

\item[Architects]

design and supervise the construction of buildings that others inhabit and use.

\item[Ambassadors]

international diplomats, or emissaries, acting as official representative to a foreign country.
\end{description}

\subsection{Architect : laying a messianic foundation}
\label{architect:layingamessianicfoundation}

Paul compares the apostolic missionary to a skilful architect. A master builder laying a messianic foundation. Apostles and apostolic missionary teams are responsible for establishing messianic structures. Including pastoral, evangelistic communities (\autoref{pastoralevangelisticcommunities}) and messianic teaching and training centres (\autoref{teachingandtrainingcentres}). Once established, these messianic structures become the responsibility of local, contextual leaders. The apostle then moves on to other fields of work.\footnote{I Corinthians 3:10--15}

Because it lays a foundation on which others build, its quality is crucial. It is critical to the future of the messianic community in pioneering context. Apostolic work is especially critical in inter-cultural contexts.

\subsection{Ambassador : acting as the Messiah's representative}
\label{ambassador:actingasthemessiahsrepresentative}

Paul compares to apostolic responsibility to that of ambassadors of the Messiah. Apostles and their teams represent the Messiah amongst people of other nations or cultures. They cross geographical and cultural boundaries to pioneer messianic community. Especially amongst people and places where there are no, few or waning gospel communities.\footnote{2 Corinthians 5:20--21}

Living and working inter-culturally, in non-native contexts, places significant extra demands upon workers. In particular, due to differences of language, climate and food. As well as political, economic and bureaucratic systems, customs, social expectations and religious sensibilities.

\subsection{First in the church}
\label{firstinthechurch}

Ambassadors and architects are types of pioneers. Their work lays a foundation or creates a bridge for others. Likewise, apostles lay a foundation for the building up of messianic communities. For this reason the apostolic ministry is considered first in the Messianic Community.\footnote{1 Corinthians 12:28}

Paul, though, identifies his apostolic, missionary service not with pre-eminence, but with being last. With being on display at the end of a humiliating procession. Paul portrays his trials and tribulations as evidence of his apostolic ministry. To further the message of the Messiah, apostles risk significant humiliation. They endure many difficult challenges.\footnote{1 Corinthians 4:9–13—possibly a reference to an act of ritual humiliation carried out by Romans, as they subjugated conquered peoples.}

Apostolic workers exhibit sacrificial dedication, suffering, humility and deep-seated reliance upon the Holy Spirit. In these ways, they represent a profound example to the whole Messianic Community. They embody the Community's shared calling to faithful, sacrificial service.\footnote{2 Corinthians 4:7--12, 6:3--10}

The characteristic role of apostolic missionary teams is to lay a messianic foundation by \textbf{skilfully, resolutely, purposefully, sacrificially pioneering in the power of the Spirit}. Planting and establishing messianic structures, in new geographical and cultural contexts.

\begin{figure}[htbp]
\centering
\includegraphics[width=150pt,height=150pt]{power.png}
\label{power.png}
\end{figure}

\begin{reflection}

Read: 1 Corinthians 3:10--13, 2 Corinthians 5:20

\begin{itemize}
\item In what ways are apostles first? In which ways, last?

\item Which qualifications—gifts, talents, characteristics—are most needed by apostolic missionary workers?

\item What is your most significant experience of inter-cultural work?

\end{itemize}

\end{reflection}

\section{Prophetic, priestly mediation}
\label{propheticpriestlymediation}

\begin{quote}

\textbf{\emph{The prophetic ministry is a mediatory role, operating between God and the Messianic Community.}}
\end{quote}

\subsection{Heart of Messianic Community}
\label{heartofmessianiccommunity}

Topics 1–3 explored three messianic structures: pastoral-evangelistic communities, teaching centres, apostolic teams. Each structure relates to the gifts of pastor, evangelist, teacher, apostle. So: how, or where, does the prophetic ministry function?

In topics 1, 2 and 3, each messianic structure is represented by a circle. Overlaying the three circles creates a new meta-structure, formed by the three parts. It represents the structure of the temple of the Holy Spirit. At the centre is a region where all three circles overlap. This central region is not a structure, like the other three. It is formed by the interaction of the three structures. This region represents the heart of messianic community. Metaphorically speaking, it is where the prophetic gift operates.

\begin{figure}[htbp]
\centering
\includegraphics[width=224pt,height=235pt]{prophetic.png}
\label{prophetic.png}
\end{figure}

The human heart is central and vital, interacting with and influencing the whole body. Although hidden from sight, the heart's strong, unrelenting rhythm keeps the body alive. At the same time, the body holds the heart. They exist in a symbiotic relationship. Without the other, neither lives. This provides a profound metaphor for the prophetic ministry. It illustrates why, like the heart, the prophetic role is vital to the functioning of a healthy body.

Paul ascribes the significance of the prophet as second only to the apostle.\footnote{1 Corinthians 12:27--31} The prophetic gift has a profound capacity to spearhead spiritual renewal and cultural impact. Yet the prophetic role and ministry is often under-appreciated.

\subsection{God's heartbeat}
\label{godsheartbeat}

The role of the prophet is to listen for and discern God's heartbeat. The heartbeat of the prophet must be the heartbeat of God. This implies significant spiritual intimacy. It is this that enables prophets to:

\begin{itemize}
\item walk with God, discerning the thoughts, feelings and intentions upon his heart

\item share in the intercessory ministry of the Spirit

\item discern and share God's \emph{kairos} word for a particular context.\footnote{\emph{Kairos} (Greek) refers to a particularly opportune, favourable, suitable or appropriate moment--e.g. see John 7:6--8, 12:23; Luke 21:13; Mark 13:4; Acts 1:6--7; 1 Timothy 2:6; \emph{kairos} contrasts with \emph{chronos}, which refers to fixed, measurable units of time.}

\end{itemize}

Because of this need for spiritual intimacy prophets are often sensitive personalities. They may be poets, artists, writers, visionaries. Or other kind of imaginative, inventive or creative individuals. This sensitivity combines well with the hidden, non-structural mode of functioning. This can lead to prophetic voices being dismissed, reacted to and under appreciated. When this happens, it impairs the functioning of the whole body—like having a weak heart.

\subsection{Dual mediatory role}
\label{dualmediatoryrole}

The prophet has a dual role of both priestly and prophetic mediation. \emph{Priestly mediation} involves speaking with God, as a representative of the Messianic Community (intercession).
Prophetic mediation involves speaking with the Messianic Community, as a representative of God (prophecy). 

\begin{description}

\item[Priestly mediation]

requires purity of heart and faithfulness of life. It is this that enables practitioners to offer effective intercession on behalf of others. Priestly mediation includes sharing the intercession of the Holy Spirit. Upholding the Messianic Community's calling to be \emph{a house of prayer for all nations}.\footnote{James 5:13--20; Psalm 24:3--4. Romans 8:26--27. Matthew 21:13, c.f. Isaiah 56:7}

\item[Prophetic mediation]

requires hearing \emph{what the Spirit is saying to the Messianic Community}.\footnote{Revelation 2:7,11,17,29,3:13,22; also Matthew 11:15, Mark 4:9, Luke 8:8 etc.} With this comes responsibility to share discernment, direction and insight. The prophet must interpret contextual, historical challenges, opportunities and responsibilities. And exhort, teach and rebuke messianic communities that are failing in their vocational calling.
\end{description}

The prophetic role is a mediatory role. It requires a sensitive, intimate, faithful walk with God. This enables prophets and those gifted with prophetic insight to \textbf{pick up God's heartbeat for the peoples of the world.} Interceding with the Spirit for the purposes of God. Hearing and revealing what the Spirit is saying to the Messianic Community.

\begin{figure}[htbp]
\centering
\includegraphics[width=283pt,height=291pt]{heartbeat2.png}
\label{heartbeat2.png}
\end{figure}

\begin{reflection}

Read: Romans 8:26–27; Read: Ephesians 2:19--22

\begin{itemize}
\item How do you respond to the idea that it is possible to discern what is upon God's heart?

\item How might the \emph{hidden} (non-structural) nature of prophetic service effect people called to function in it?

\item What is your experience of prophetic ministry?

\end{itemize}

\end{reflection}

\ssection{Strategies of the Messianic Commission}

\emph{Strategies (\autoref{strategies})}, explored five foundational messianic strategies: pray, reach, disciple, teach, send. Those five strategies correspond with the messianic structures explored in this study. The table, \emph{Strategy, Structure, Function}, tabulates these strategies and structures and their function.

\begin{figure}[htbp]
\centering
\includegraphics[width=298pt,height=98pt]{Table1.png}
\caption{Strategy, Structure, Function}
\label{table1.png}
\end{figure}

These correspondences illustrate how different parts of the body contribute to its functioning. They allow the Messianic Community to function according to God's strategic plan and purpose.

\begin{figure}[htbp]
\centering
\includegraphics[width=224pt,height=194pt]{interactions2.png}
\caption{Structural interactions}
\label{interactions2.png}
\end{figure}

The figure, \emph{Structural interactions}, illustrates how the various messianic structures can interact with the others. Hence, apostolic missionary teams planting pastoral-evangelistic communities, establishing teaching centres. Pastoral-evangelistic communities sending apostolic missionary teams. And partnering with teaching and training centres. As these structures interact an effective missional, discipleship movement is established.

This concludes \emph{Study 3, Structures}, which examines the scripturally-based structures of messianic community:

\begin{enumerate}
\item Pastoral-evangelistic communities

\item Teaching and training centres

\item Apostolic missionary teams

\item Prophetic, mediatory ministry

\end{enumerate}

In summary, the study reveals that \textbf{the Messianic Community has a God-ordained structure that uniquely equips it to fulfil the Messianic Commission.}

\vspace*{\fill}\begin{bonus}

\begin{itemize}
\item Which \emph{structural interactions} are most significant? Why?

\item Which are presently least evident in your contexts?

\item How does prophetic ministry fit with the structural interactions?

\end{itemize}

\end{bonus}

\chapter{Expansion}
\label{expansion}

\begin{chapsynopsis}
\begin{center}

\textbf{The Messianic Community is intended to be a continually multiplying movement of disciples, constantly expanding into all the world}

\end{center}
\end{chapsynopsis}
\begin{topics}

\begin{enumerate}
\item Honey bees

\item Maize plant

\item Commercial organisations

\item Missional movements

\end{enumerate}

\end{topics}

\section{Honey bees}
\label{honeybees}

A honey bee colony usually contains between 2,000 and 60,000 bees. A bee colony is a kind of \emph{collective organism}. Individual bees cannot survive for long outside of a colony. There is no hierarchy or leadership, instead there are several defined \emph{roles}: 

\begin{itemize}
\item a single fertile \emph{queen bee}

\item a few thousand fertile, male \emph{drone bees}

\item several thousand infertile, female \emph{worker bees}.

\end{itemize}

\begin{figure}[htbp]
\centering
\includegraphics[width=276pt,height=60pt]{division.png}
\label{division.png}
\end{figure}

Bee colonies exhibit growth through an organic, steady, cyclical \emph{increase} in bee numbers. This leads to swarming and a \textbf{division} of the colony. The swarm then \emph{expands} into a different geographical region.

\subsection{Increase}
\label{increase}

A colony grows as worker bees raise thousands of new bees, born to the queen. Usually, around a queen's second springtime, a colony will prepare to \emph{swarm}. In readiness for swarming, worker bees begin preparing new virgin queen bees. One of them will take over the existing hive, by killing all the others, after the old queen leaves with the swarm.

\subsection{Division}
\label{division}

When it is time for the swarm to leave the hive, \emph{scout} bees find a suitable place for the swarm to gather. They report this location to the colony. Shortly afterwards, the most vigorous worker bees swarm around the queen bee. The swarm, of about 60\% of the colony, then leaves the hive altogether and moves to the scouted location.

\begin{figure}[htbp]
\centering
\includegraphics[width=99pt,height=135pt]{swarm-bw.png}
\caption{A swarm of bees}
\label{swarm-bw.png}
\end{figure}

\begin{description}

\item[Risks]

— swarming divides the colony, weakening both groups.

\item[Benefits]

— bees produce abundant honey, from plant pollen, to feed the growing colony. Honey provides a rich food source for animals and a harvestable crop for humans.
\end{description}

\subsection{Expansion}
\label{expansion}

Scout bees must swiftly identify a suitable, permanent hive location, so a swarm can form a new colony. This step is critical since swarming bees can survive on the honey in their stomachs for only 2--3 days. Once the swarm is settled in a new hive, the cycle of growth begins again. The old queen may not live long and must immediately begin repopulating the colony. In particular, she must produce new virgin queens, ready to take on her role.

\begin{reflection}

Read: Acts 1:8; 2:42--47

\begin{itemize}
\item What is a collective organism?

\item What can we learn from bee colonies as a metaphor for messianic communities?

\item When is \emph{division} a worthwhile strategy?

\end{itemize}

\end{reflection}

\section{Maize plant}
\label{maizeplant}

The maize plant is a cereal crop. It usually grows two or more metres high. Maize is grown all over the world, feeding millions of people every day. Its shallow roots allow rapid reproduction. They also make it intolerant of poor soil. And susceptible to drought and prone to uprooting by severe winds.

\begin{figure}[htbp]
\centering
\includegraphics[width=274pt,height=60pt]{multiplication.png}
\label{multiplication.png}
\end{figure}

Organic \emph{increase} in plant growth produces up to a thousand-fold \textbf{multiplication} of seeds. Often twice a year, allowing rapid crop \emph{expansion}, depending upon climate and soil conditions.

\subsection{Increase}
\label{increase}

Maize plants grow from seeds sown into the ground. Growth begins when the seed's hard, outer shell breaks open. This allows the soft, inner kernel to access the moisture and nutrients within the soil. It immediately sprouts roots and a single stem that moves towards the surface of the soil.

Once through the surface, the plants leaves begin \emph{photosynthesising} sunlight (turning light into growth). Meanwhile the roots continue drawing on soil nutrients and moisture. In arid locations the most significant growth factor is a sufficiency of rainfall.

\subsection{Multiplication}
\label{multiplication}

Each maize plant produces several ears, each containing 600--800 seeds. Thus, a single season of growth can produce thousands of seeds from a single plant.

\begin{figure}[htbp]
\centering
\includegraphics[width=237pt,height=205pt]{maize.png}
\label{maize.png}
\end{figure}

\begin{description}

\item[Risks]

— shallow roots make maize susceptible to poor soils, drought and severe winds.

\item[Benefits]

— a valuable, versatile crop, used in a variety of foodstuffs.
\end{description}

\subsection{Expansion}
\label{expansion}

Maize plants expand into new areas through redistribution of their seeds. To be effective, this process must be carried out by farmers. This is due to the cultivation of the plant, over hundreds of years.

\begin{reflection}

Read: Acts 5:42--6:7; Acts 9:31

\begin{itemize}
\item Why are maize plants dependent on farmers for reproduction?

\item What can we learn from maize plants as a metaphor for messianic communities?

\item What are the risks of rapid multiplying messianic communities?

\end{itemize}

\end{reflection}

\section{Commercial organisations}
\label{commercialorganisations}

Commercial organisations, including many charitable enterprises, seek to achieve \emph{economic growth}. They use profit and loss accounts as the primary measure of success and failure. 

Rewards associated with commercial growth typically produce competitive, demanding environments. This can lead to extraordinary performance, but also to a loss of human dignity.

\begin{figure}[htbp]
\centering
\includegraphics[width=276pt,height=60pt]{duplication.png}
\label{duplication.png}
\end{figure}

Commercial growth is marked by economic \emph{increase}. This leads to structural \textbf{duplication} and geographical \emph{expansion} into new territory.

\subsection{Increase}
\label{increase}

Increase relates to the measurement of a range of economic factors. This may include: income, profit, employee numbers, management hierarchy, production capacity, market size and share. At some point, most organisations reach a limit to their growth in one or other of these areas. Further growth is sought by duplicating shops, offices or factories managed by the organisation.

\subsection{Duplication}
\label{duplication}

Duplication means reproducing a copy of an existing, successful enterprise, in a new location. This duplicates the original, proven concept, increases systemic production and establishes an identifiable brand. Buildings, budgets, payrolls and competition play a significant role in decision-making processes. A centralised, hierarchical structure governs management decisions.

\begin{figure}[htbp]
\centering
\includegraphics[width=137pt,height=140pt]{commerce.png}
\label{commerce.png}
\end{figure}

\begin{description}

\item[Risks]

— duplication may overlook, ignore or suppress local insight and initiative. Duplication may ignore contextual differences between original and new structures.

\item[Benefits]

— successful duplication leads to increased profits, asset ownership and economic strength.
\end{description}

\subsection{Expansion}
\label{expansion}

A central, organisational strategy determines the location, form, speed and costs of expansion. New markets are sought in either geographical regions or new product ranges. New initiatives that fail to meet management expectations may be closed down. Closures usually take place without reference to local, contextual concerns.

\begin{reflection}

Read: Acts 19:18--20; Acts 16:5

\begin{itemize}
\item What benefits and risks are associated with the expansion of commercial organisations?

\item What can we learn from commercial organisations in relation to messianic community?

\item What can commercial leaders learn from messianic leaders? What about vice-versa?

\end{itemize}

\end{reflection}

\section{Missional movements}
\label{missionalmovements}

A movement is a groups of people and organisations sharing a common focus of action. A missional movement is a group sharing a dedication to serving God's eternal purpose. It includes congregations, denominations, bible schools, missionary agencies and other agencies. Yet it is not limited to them, because:

\begin{enumerate}
\item the Spirit is always bringing to life new structures and initiatives within the Messianic Community 

\item not all organisations continue to faithfully pursue God's mission.

\end{enumerate}

Identifying with a regional missional movement lifts our vision beyond personal and organisational horizons. It allows us to recognise how the Messiah is orchestrating a movement through his Spirit.

\begin{figure}[htbp]
\centering
\includegraphics[width=274pt,height=60pt]{reproduction.png}
\label{reproduction.png}
\end{figure}

Growth is measured by an \emph{increase} in the formation of disciples. This leads to a faithful \textbf{reproduction} of the life of the Spirit in those disciples. This leads to an \emph{expansion} in social and cultural spheres of influence.

\subsection{Increase}
\label{increase}

Missional movements increase through forming messianic disciples who make other messianic disciples. Who make other disciples. And so on—as explored in \emph{Commissioning (\autoref{commissioning})}, \emph{Strategies (\autoref{strategies})} and \emph{Structures (\autoref{structures})}.

\subsection{Reproduction}
\label{reproduction}

Messianic disciples are people who unite with the Messiah to \emph{embody} missional faithfulness. This implies a faithful reproduction of the life of Jesus and the heart of the Father. In a community of people, through the power of his Spirit.

Within missional movements, as in nature, not all reproduction succeeds. Some individuals and groups fail to mature. Others reach maturity, but don't reproduce. Some start slowly, others rapidly. Some groups evolve a different identity, purpose or form to their original contexts.

\begin{description}

\item[Risks]

— rapid numerical growth may lead to overlooking the importance of authentic spiritual maturity. Authentic discipleship is a slow process, usually costly to achieve. It may also be difficult to measure and assess.

\item[Benefits]

— members exhibit a wide variety of charisma, talent and contextual influence. Reproduction leads to transformation amongst families, households, communities, societies, cultures, people groups and nations.
\end{description}

\subsection{Expansion}
\label{expansion}

The Holy Spirit governs the wild, haphazard growth of a movement. It is the Spirit who takes hold of a people and implants missional vision. It is the Spirit who directs the application of our missional energy. It is the Spirit who enables a movement to impact communities, cultures and societies.⁠

The Holy Spirit works to establish communities manifesting God's heart. He does this throughout the world, wherever he finds willing hearts and available hands. The dynamics and disciplines of the Maize Plant Discipleship syllabus illustrate how he works. How he equips and move us onwards, towards this goal.

\begin{reflection}

Read: John 3:8, Acts 4:31--33

\begin{itemize}
\item Considering both the \emph{risks} and \emph{benefits} of missional movements, are they a worthwhile investment?

\item What are you prepared to invest in establishing a disciple-forming, missional movement?

\item How does a missional movement achieve significant momentum?

\end{itemize}

\end{reflection}

\ssection{The Expansion of Messianic Community}

This concludes Study 4, which explored four examples of organic or organisational reproduction:

\begin{itemize}
\item the division of bee colonies

\item the multiplication of maize plants

\item the duplication of commercial organisations

\item the Spirit-led reproduction of missional movements;

\end{itemize}

The study highlighted a diverse range of characteristics associated with these various type of increase and expansion, as well as the associated risks and benefits of each. It explored how each type of growth could contribute towards understanding healthy growth patterns of Messianic Community.

\begin{quote}

In summary, the study suggested that \textbf{the Messianic Community is intended to be a continually multiplying missional movement, constantly expanding into all the world}.
\end{quote}

\vspace*{\fill}
\begin{bonus}

\begin{enumerate}
\item Which of the four types of expansion appeals most to you personally?

\item Which provides the greatest challenge to the growth of Messianic Community in your context?

\item What is needed when missional momentum slows?

\end{enumerate}

\end{bonus}

\input{mpd-footer}

\end{document}
