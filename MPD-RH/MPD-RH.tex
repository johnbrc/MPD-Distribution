\def\version{0.6.0}
\def\change{Hemingway.}
\input{mpd-header}
\def\mytitle{Syllabus Handbook}
\def\module{rh}
\def\translation{}
\def\isbn{978-1-907191-02-2}
\def\colour{false}
\input{mpd-document-sh}

\chapter{The Process}
\label{theprocess}

\begin{quote}

Maize Plant Discipleship is based on an open, reflective, group learning process. As students relate Scripture to context, they learn to hear what the Spirit is saying to them.
\end{quote}

\section{Learning groups}
\label{learninggroups}

\begin{quote}

Together, we experience and learn quite differently to studying alone.
\end{quote}

There are many reasons to bring together a group of people to learn together. Some people would point to Jesus' gathering of twelve disciples. For most people, groups represent a natural and lively place in which to learn. They bring together people with different experience, talent, capability and perspective.

When we share our lives, we learn together and \emph{learning groups} mirror this reality. Together, we experience and learn quite differently to when we study alone. Reflective group discussion provides a stimulating forum for learning through exploration, listening and discovery.

\subsection{Learning, not teaching}
\label{learningnotteaching}

\begin{quote}

A decision to learn something new always begins with the student.
\end{quote}

Maize Plant Discipleship is a \emph{learning}, rather than a teaching resource. Learning depends on many factors, including desire, temperament, experience, talent, time, energy, environment and so on. Most of these factors depend on the student, rather than the teacher or teaching content.

Thus, discipleship is a type of learning that must be called out of students or disciples. This happens through the guidance and direction of facilitators, mentors, educators and teachers. They come alongside motivated learners to guide and ``call out'' the learning taking place.

\section{Facilitating Learning Groups}
\label{facilitatinglearninggroups}

\subsection{Facilitating openness}
\label{facilitatingopenness}

\begin{quote}

A facilitator helps a group to open up: to the message of Scripture, to one another and to the leadership of the Holy Spirit.
\end{quote}

Fostering an environment of learning and discovery is essential. Debate and discussion must be lively, yet relaxed and uncompetitive. An ideal environment will allow strong and diverse views to be expressed. Yet it will do so without creating either conflict or conformity. In this way, all become comfortable in contributing views, questions and burdens.

It is especially challenging to foster openness in cultures where conformity is highly valued. Where authority flows downwards. Thus, facilitators should generally contribute to discussions as a regular group member. They should never dominate or belittle the views of others.

\begin{itemize}
\item Encourage all to contribute, especially quieter members, women, youth and elders.

\item People sometimes reconsider their responses and want to return to earlier discussions. Allow discussion to ebb and flow.

\item If discussion becomes harsh or factious, quieten the group. Invite someone with a harmonious or gentle spirit to summarise—not resolve—the tension. \emph{Then move on.}

\end{itemize}

\subsection{Facilitating Spirit-led discipleship}
\label{facilitatingspirit-leddiscipleship}

\begin{quote}

The goal of discipleship is not to establish shared dogmatic belief. Nor conformity to the convictions of a leader or church tradition. Nor to every aspect of Maize Plant Discipleship. The goal of discipleship is conformity to the Spirit of the Messiah and obedience to the will of the Father.
\end{quote}

Discipleship is not the replication of information, from teacher, or textbook, straight into students. We may have learned this way in school; Spirit-led discipleship is different. Discipleship \emph{transforms}, as well as informs.

Through teaching, reflection and discussion, God's Spirit speaks to our hearts. He leads, warns, directs, encourages, educates, challenges and exhorts. As each person differs in gift, personality and development, at any time, each person may be learning something different.

\subsection{Who can facilitate?}
\label{whocanfacilitate}

\begin{quote}

A facilitator must be comfortable with authentic, reflective, exploratory group discussion.
\end{quote}

A facilitator needs to be someone called to help others become faithful Christian disciples. This must motivate them to be humble, patient, warm, flexible, open. They must be secure enough to allow others to explore their personal boundaries. Boundaries of vocation, experience and creativity. And to allow each individual to do so at their own pace.

A facilitator does not hold a position of authority over people. They simply facilitate the gathering of people into groups, for learning and discussion. Hence, a facilitator:

\begin{itemize}
\item may be a lay-leader;

\item may be relatively young;

\item may be a woman;

\item need not have gone to bible college;

\item need not be an established church leader;

\item need not be an experienced mentor.

\end{itemize}

Established leaders, mentors or disciplers may facilitate. But they must allow \emph{authentic, reflective, exploratory} group discussion.

\section{Practicalities}
\label{practicalities}

\begin{quote}

Appropriate, practical planning will maximise the effectiveness of group learning.
\end{quote}

\subsection{Group size}
\label{groupsize}

Maize Plant Discipleship is ideal for learning groups of 5–10 people. Small enough to allow group members to grow together with a degree of intimacy. Large enough for group members to explore discipleship commitments at their own pace.

\subsection{Involving others}
\label{involvingothers}

A facilitator may delegate responsibility for hosting, presenting the teaching, or moderating group discussions. This avoids one person carrying too much. Group members should share these responsibilities, according to their talent and capacity. In this way, all will gain some experience of the facilitating role.

The person responsible for presenting a study should read it through, in advance. They should take time to reflect upon the teaching and its lessons. If anything is unfamiliar or unclear, invite discussion about it. Encourage group members to share their perspectives, rather than searching for one ``right'' answer.

\subsection{Location and timetable}
\label{locationandtimetable}

Learning groups may meet in any appropriate location that comfortably accommodates them. This could be a large room in someone's home, or a communal building, such as a church.

\begin{itemize}
\item \emph{Arrange seating} to create intimate, practical spaces for discussion, so that everyone can see and hear each other.

\item \emph{Experiment and learn what works for your group.} Consider using different locations, maybe meeting outside. 

\item \emph{The syllabus} of 15 modules contains 60 studies and about 240 topics in total. Consider availability when planning a timetable. Are learners affected by agricultural seasons, or academic terms?

\end{itemize}

\subsection{Adaption}
\label{adaption}

Where appropriate, adapt the teaching and the method of presentation. Aim to create a helpful learning environment. Take account of the capabilities of each particular discipleship group. Thus:

\begin{itemize}
\item \emph{Ensure literacy is a helpful servant}, not a hard task-master, especially to oral learners. As far as possible, keep things simple and lighthearted.

\item \emph{Ensure everyone present views topical illustrations}. Reproduce illustrations, using whatever medium is available. If you have an artist in the group, give them responsibility for this task.

\item \emph{Invite creative people to contribute}, perhaps by interpreting or celebrating topical subjects in song, art or drama.

\item \emph{Mother tongue is the natural language of the heart}. Encourage its use for Scripture reading and memorisation and group discussions, in particular.

\end{itemize}

\subsection{Other learning forums}
\label{otherlearningforums}

The Maize Plant Discipleship syllabus can facilitate various forms of guided learning:

\begin{itemize}
\item \emph{Theological education}. Students may gain valuable experience for their own vocational contexts by forming reflective learning groups within or alongside classroom contexts. Discussion questions may form a basis for short essays or written answers. 

\item \emph{Personal study}. Incorporate reflective learning by submitting learning to the oversight of a mentor. Or by sharing it with a fellow student, for reflection, discussion and critical consideration.

\item \emph{Congregational teaching}. After hearing a study, congregants could divide into discussion groups. Or learning groups could meet on a separate occasion, to discuss the study and pray together.

\end{itemize}

\vspace*{\fill}
\begin {reflection}

\begin{itemize}
\item How do Maize Plant Discipleship learning groups differ from other learning forums, (e.g. schools)?

\item What are the essential characteristics of a good facilitator?

\end{itemize}

\end {reflection}

\chapter{The Learning Cycle}
\label{thelearningcycle}

\begin{quote}

Maize Plant Discipleship Studies provide a reflective learning process. Reflective learning allows practice and theory to inform one another.
\end{quote}

\section{Reflective Learning}
\label{reflectivelearning}

Reflective learning incorporates four elements: hearing, reflection, discussion, action.

\begin{itemize}
\item \emph{Hearing} the perspectives of others

\item \emph{Reflection} upon old and new ideas

\item \emph{Discussion} and dialogue with others.

\item \emph{Action} in the light of renewed understanding.

\end{itemize}

The Maize Plant Discipleship learning process incorporates these basic reflective learning elements. It combines them with scriptural insights to produce a six-stage process: hear, retain, open, share, pray, act.

\section{1 — Hear what the Spirit is saying}
\label{1—hearwhatthespiritissaying}

\begin{quote}

Disciples of the Messiah seek more than human ideas and knowledge. They seek to understand spiritual words and truths, taught by the Spirit of God.
\end{quote}

\begin{description}

\item[I Corinthians 2:12–13]

Now we have not received the spirit of the world, but the Spirit of God. Thus we are able to understand the things God has so freely given us. These are the things we are talking about when we avoid a manner of speaking that human wisdom would dictate. Instead we use a manner of speaking taught by the Spirit. We explain the things of the Spirit to people who have the Spirit.
\end{description}

\textbf{We listen to God's message with our mind, but we also listen with our heart}. We listen to hear what the Spirit is saying to God's people. We listen not to become puffed up with intellectual knowledge, but to learn how to better love God and neighbour. 

\begin{figure}[htbp]
\centering
\includegraphics[width=108pt,height=108pt]{HEAR@2x.png}
\label{hear2x.png}
\end{figure}

\section{2 — Retain God's message inwardly}
\label{2—retaingodsmessageinwardly}

\begin{quote}

It is not enough to only hear God's message. We must listen with the intent to learn and follow. 
\end{quote}

\textbf{We must learn to hold onto his words, so that they can live within us and bear fruit.} We must allow God's message to settle in our spirit. There it will shape our convictions and renew our hope. The foolish person is one who hear but do not retain the God's word in their hearts. They do not bear good fruit.

\begin{description}

\item[Matthew 13:18–23, Luke 8:11–15]

The seed sown on rocky ground are those who, when they hear the word, accept it with joy. But these have no root — they go on trusting for a while, but when a time of testing comes, they fall away. As for what fell in the midst of thorns, these are the ones who hear the message. But it is choked by the worries of the world and the deceitfulness of wealth, so that their fruit never matures. The seed on good soil stands for those who hear the message and hold onto it with a good, receptive heart. And by persevering they bring forth a harvest.
\end{description}

Think about how we receive and retain food. We chew it, enjoying the taste, swallowing, digesting, inwardly retaining its vitality and goodness. It's the same with God's word. We must chew it over, meditating and reflecting upon its meaning and application to our lives. Both as individuals and as communities.

\begin{figure}[htbp]
\centering
\includegraphics[width=108pt,height=108pt]{RETAIN@2x.png}
\label{retain2x.png}
\end{figure}

\section{3 — Open hearts to others}
\label{3—openheartstoothers}

\begin{quote}

Group discussion provides an opportunity to discover others' experience and perspective. This requires listening with our heart, as well as our head. 
\end{quote}

\textbf{Our aim is to hear what others are sharing from their hearts}. Not to win a debate or an argument. When we listen to others' stories, we affirm their experience. And, by seeing the world through their eyes, we increase our own knowledge of the world.

Explore practical, vocational applications of what is being learned. Vocation incorporates all the responsibilities towards which God calls us. Including families, workplaces, community roles and involvements, artistic talents and sporting abilities.

Consider how biblical teachings relate to cultural and social contexts. Encourage people to express themselves using ``mother tongue'' phrases and the ``sweet talk'' of proverbs.

\begin{figure}[htbp]
\centering
\includegraphics[width=108pt,height=108pt]{OPEN@2x.png}
\label{open2x.png}
\end{figure}

\section{4 — Share broken bread together}
\label{4—sharebrokenbreadtogether}

\begin{quote}

Breaking bread is symbolic of the new covenant. It provides a profound way for groups to proclaim a shared devotion to the Messiah. 
\end{quote}

\textbf{Sharing food together deepens fellowship and celebrates God's provision}. In modern Christianity, breaking and sharing bread is usually ceremonial and liturgical. For the early messianic communities it was simply a shared meal. A meal based on the Passover feast, shared by Jesus and his disciples, before his execution.

Consider sharing a simple meal together. Recognise the meal as a form of breaking bread. Make it a celebration of the Messiah's faithfulness. If a meal is not possible, share some bread together. Treat it as a symbolic act of hospitality and commitment to membership of the Messiah's body.

\begin{figure}[htbp]
\centering
\includegraphics[width=108pt,height=108pt]{SHARE@2x.png}
\label{share2x.png}
\end{figure}

\section{5 — Pray for God's will to be done}
\label{5—prayforgodswilltobedone}

\begin{quote}

Prayer is a natural response to learning and breaking bread together. Allow what is being learned to infuse prayer with fresh confidence about God's will and purpose. 
\end{quote}

\begin{itemize}
\item \textbf{Pray for personal and vocational concerns}. 

Encourage group members to share the challenges they are facing in their homes, workplaces or other arenas.

\item \textbf{Speak God's blessings}. 

Don't curse trials and temptations. Allow the Holy Spirit to reveal how God wants to work in the midst of suffering and challenges. speak specific blessings, rooted in Scripture. 

\item \textbf{Pray for the gospel}. 

Intercede for neighbourhoods, networks and communities, for local and national governments and cultural institutions. Pray for deep impact and transformation of individuals, cultures and societies. Turn your focus to the nations. Pray for Africa, Europe, Asia and the Americas. Pray for unreached people groups.

\item \textbf{Pray for the Maize Plant Discipleship project}. 

Pray that the syllabus is used it to bless and edify the Messianic Community, across the world!

\end{itemize}

\begin{figure}[htbp]
\centering
\includegraphics[width=108pt,height=108pt]{PRAY@2x.png}
\label{pray2x.png}
\end{figure}

\pagebreak 

\section{6 — Act in the light of God's message}
\label{6—actinthelightofgodsmessage}

\begin{quote}

The purpose of gathering to hear God's message is not only to hear, but to act on it. We deceive ourselves when we listen to God's word, yet do not do what it says.
\end{quote}

\textbf{Renewed action is the goal of learning and reflecting together.} Unless learning leads to renewed action, it is in vain. Actions renewed through reflection in the light of God's message to his people. This leads to personal and communal transformation in the way we live, work and serve .

\begin{description}

\item[James 1:22–25]

Don't I beg you, only hear the message, but put it into practice. Otherwise you are merely deluding yourselves.

The man who simply hears and does nothing about it is like a man catching the reflection of his own face in a mirror. He sees himself, it is true. Yet he goes on with whatever he was doing without the slightest recollection of what sort of person he saw in the mirror. 

But the man who looks into the perfect mirror of God's law, the law of liberty (or freedom), and makes a habit of so doing, is not the man who sees and forgets. He puts that law into practice and he wins true happiness. (JBP)
\end{description}

Transformation is the goal of discipleship. As we are personally transformed we influence our homes, workplaces and social networks. Together, we begin to fulfil our corporate vocation: a Messianic Community blessed to be a blessing to the families of the earth!

\begin{figure}[htbp]
\centering
\includegraphics[width=108pt,height=108pt]{ACT@2x.png}
\label{act2x.png}
\end{figure}

\section{Maize Plant Discipleship learning cycle}
\label{maizeplantdiscipleshiplearningcycle}

As it is repeated, the learning process becomes a cycle. The aim is to facilitate, not control the learning taking place. Allow the process to direct, but not limit growth and dynamism. Where appropriate, adapt it.

\begin{figure}[htbp]
\centering
\includegraphics[width=247pt,height=275pt]{mpd-learning-cycle@2x.png}
\label{mpd-learning-cycle2x.png}
\end{figure}

\begin {reflection}

\begin{enumerate}
\item How does the Maize Plant Discipleship learning process differ from secular reflective learning? Why is this significant?

\item Which element(s) of the Maize Plant Discipleship learning process are most vital?

\item Which elements might sometimes be omitted, intentionally or otherwise? 

\end{enumerate}

\end {reflection}

\chapter{The Syllabus}
\label{thesyllabus}

\begin{quote}

The Syllabus consists of a Resource Handbook and 15 Module Handbooks. Each module consists of 4 studies; each study contains 3--5 topics. Each topic is accompanied by scripture ratings and three reflective discussion questions. A total of \ensuremath{\sim}240 topics and over 600 reflective questions.
\end{quote}

\section{The Maize Plant Metaphor}
\label{themaizeplantmetaphor}

Jesus referred to his mission using the metaphor of a seed.\footnote{John 12:24} Seeds enter the ground and die. Yet from their `death' comes a harvest. The Maize Plant Discipleship Syllabus builds on this metaphor, relating to life, death, growth and sustenance: 

\begin{summary}

\textbf{Good seed, sown in good soil, stimulated by sunshine and refreshed by rainfall produces dynamic growth and a good harvest.}

\end{summary}

This metaphor provides the basic structure of the syllabus, which is divided into two main parts: \emph{Roots and Fruits} and \emph{Sunlight and Rainfall}.

\begin{figure}[htbp]
\centering
\includegraphics[width=297pt,height=442pt]{mp-metaphor-applied@2x.png}
\label{mp-metaphor-applied2x.png}
\end{figure}

Like maize plants, messianic communities need to be in good ground. Good ground provides life-giving spiritual nutrients and the living water of the Spirit. God's light provides energy for spiritual growth. Strong, scriptural roots anchor messianic communities against the destructive winds of false teaching. They sustain them amidst the withering heat of trials, temptations and vocational responsibility.

\subsection{Roots and fruits (Messianic dynamics)}
\label{rootsandfruitsmessianicdynamics}

\emph{Modules 1 to 8} examine eight messianic dynamics. These dynamics relate to the roots, fruits and growth of the Messianic Community.⁠.\footnote{\emph{Dynamic} derives from a Greek word, \emph{dunamis}, meaning power and refers to forces stimulating change within a process or system, such as a plant or a body.}

\subsubsection{1 — Eternal Purpose}
\label{1—eternalpurpose}

\begin{quote}

The root of the syllabus. The Messiah represents the fullness of Gods eternal purpose. The Seed that grows in the soil of the Israel. A Seed that produces a rich harvest of people from the earth.
\end{quote}

\subsubsection{2 — Commissioning}
\label{2—commissioning}

\begin{quote}

The Messianic Community is commissioned by God. This is how they join the Messiah's mission towards the world.
\end{quote}

\subsubsection{3 — Nations}
\label{3—nations}

\begin{quote}

The Messianic Community has a unique and special vocation: to bless the nations of the world.
\end{quote}

\subsubsection{4 — Jews}
\label{4—jews}

\begin{quote}

The Messianic Community has a special responsibility towards the Jewish people.
\end{quote}

\subsubsection{5 — Membership}
\label{5—membership}

\begin{quote}

The body of the Messiah is called to fully express his life. This requires progressive corporate growth into spiritual maturity.
\end{quote}

\subsubsection{6 — Revival}
\label{6—revival}

\begin{quote}

The pathway towards a fruitful harvest is to walk in practical and spiritual faithfulness.
\end{quote}

\subsubsection{7 — Intercession}
\label{7—intercession}

\begin{quote}

The Messianic Community has a priestly calling to be \emph{a house of prayer for all nations}.
\end{quote}

\subsubsection{8 — Transformation}
\label{8—transformation}

\begin{quote}

The Messianic Community is called to work amongst and on behalf of the nations of the world. It's goal is to establish cultural transformations that signal the presence of God's kingdom.
\end{quote}

\subsection{Sunlight and rainfall (Messianic disciplines)}
\label{sunlightandrainfallmessianicdisciplines}

\emph{Modules 9 to 15} examine seven key disciplines that feed the growth of the Messianic Community. These disciplines relate to how messianic disciples walk in the light and receive the living water of God's Spirit.

\subsubsection{9 — Maturity}
\label{9—maturity}

\begin{quote}

The spiritual growth of Messianic disciples and communities incorporates three distinct phases.
\end{quote}

\subsubsection{10 — Service}
\label{10—service}

\begin{quote}

An enduring life of service relies upon nurturing essential motivations, virtues and disciplines.
\end{quote}

\subsubsection{11 — Vocation}
\label{11—vocation}

\begin{quote}

A deepening relationship with the Messiah is the foundation of personal vocation.
\end{quote}

\subsubsection{12 — Economics}
\label{12—economics}

\begin{quote}

Scriptural perspectives upon economic faithfulness, wealth and poverty are radically different to those of the world in general.
\end{quote}

\subsubsection{13 — Leadership}
\label{13—leadership}

\begin{quote}

Faithful messianic leadership requires the nurture of essential motivations, qualities and disciplines.
\end{quote}

\subsubsection{14 — Faith}
\label{14—faith}

\begin{quote}

Seeing with eyes of faith enables us to endure times of testing and purification. We learn to embrace challenge as an opportunity for experiencing God's faithfulness.
\end{quote}

\subsubsection{15 — Overcoming}
\label{15—overcoming}

\begin{quote}

Overcoming in the power of the Spirit leads to confrontation with idolatrous, cultural strongholds.
\end{quote}

\section{Significant Terms}
\label{significantterms}

These definitions explain how these important terms are used in the Syllabus.

\begin{description}

\item[messiah]

— a mediator or saviour, acting with God's authority to deliver a people from the grip of their enemies. Or acting to govern over and keep them safe (experiencing \emph{shalom}). In the biblical history of Israel, deliverance came through a variety of mediators, such as \emph{prophets, priests, judges} and \emph{kings}. 

The root meaning of messiah is \emph{anointed} or \emph{poured on}. It refers to the anointing oil poured onto Israel's kings and priests. The anointing oil is symbolic of the pouring out, or placing of God's Spirit upon a leader. This happened as prophets invested them with authority to govern.\footnote{E.g. Exodus 30:22--25} Thus God's priestly, kingly and prophetic authority is inherent in the concept of messiah. 

The New Testament identifies Jesus as the Jewish Messiah.\footnote{\emph{Jesus Christ} is the Greek rendering of \emph{Yeshua Moshiach}} He fulfils three mediatory roles on behalf of God's people: prophet, priest and king.\footnote{Matthew 3:13--17; Mark 1:9--11; Luke 3:21--22 \& 4:16--19; John 1:32--34} After ascending to the \emph{Right Hand of God}, he becomes the \emph{one mediator between God and humanity}.\footnote{Hebrews 8:1--2; 1 Timothy 2:5. See Syllabus, Module 1}

\item[messianic]

— of or relating to the Messiah. Primarily used in the syllabus to refer to \emph{Messianic Community} or \emph{messianic communities}.

\item[Messianic Community]

(capitalised) — the universal and historical body of people belonging to the Messiah. The New Testament refers to this community as the body of Messiah. This emphasises both organic community and the rulership of the Messiah. Equivalent to \emph{worldwide Christian community} or \emph{Church}.

\item[messianic communities]

(un-capitalised) — localised congregations of Messianic Community. The term emphasises local communities living under the lordship of the Messiah. It also emphasises the link to the whole body of the Messiah, the Messianic Community. Equivalent to \emph{congregations.}

\item[vocation]

— A calling, life's work, mission, purpose, function. A profession, occupation, career, job, employment, trade, craft, business, line of work, métier. In the Syllabus, vocation and vocational refers to both personal and communal calling. These flow from the scriptural call to serve God's eternal purpose, in union with the Messiah.
\end{description}

Vocation is thus an umbrella term incorporating and dignifying all forms of work and ministry. It looks beyond traditional divides of laity and clergy, male and female. It signals that God calls people to serve him in their homes, workplaces and communities.

\vspace*{\fill}
\begin {reflection}

\begin{itemize}
\item How does the seed metaphor feature in the ministry of Jesus?

\item What is meant by ``messianic dynamics''?

\item How does the metaphor of maize relate to the significance of messianic community in your context?

\end{itemize}

\end {reflection}

\chapter{The Philosophy}
\label{thephilosophy}

\begin{quote}

The life of the Messiah is reproduced in his people through a dynamic, generational process empowered by the Holy Spirit.
\end{quote}

\section{What is Messianic Discipleship?}
\label{whatismessianicdiscipleship}

The apostle, Paul, explains the basic principles of messianic discipleship to his disciple, Timothy:

\begin{description}

\item[2 Timothy 1:13--2:2]

Follow the pattern of the sound teachings you have heard from me, with trust and the love which is yours in the Messiah Yeshua. Keep safe the great treasure that has been entrusted to you, with the help of the Holy Spirit, who lives in us{\ldots} 

Be empowered by the grace that comes from the Messiah Yeshua. And the things you heard from me, which were supported by many witnesses, these commit to faithful people, such as will be competent to teach others.
\end{description}

This passage of Scripture describes the three foundational components of messianic discipleship:

\begin{enumerate}
\item \textbf{The great treasure of knowing the Messiah, Jesus}

A real, personal, experiential knowledge of the Messiah is more than human knowledge or philosophy. It is \emph{a great treasure}, a divine relationship, founded on sound teaching, trust and love.

\item \textbf{The vitality of the Holy Spirit}

The Holy Spirit provides an intimate source of divine help to messianic disciples. The Spirit mediates the power of the Gospel. He safeguards the presence of the Messiah amongst his people.\footnote{John 16:7--15}

\item \textbf{The necessity of generational formation}

The necessity of generational formation. Paul imparts the reality of the Messiah to his disciple, Timothy. He instructs him to keep this treasure safe by passing it on to competent, faithful people. This is generational formation in action.

\end{enumerate}

\begin{figure}[htbp]
\centering
\includegraphics[width=267pt,height=41pt]{generational@2x.png}
\caption{Generational discipleship}
\label{generational2x.png}
\end{figure}

Generational, messianic discipleship is how treasure is kept safe in the kingdom of God. The metaphor of seed and harvest illustrates this principle.

\subsection{Seed and harvest}
\label{seedandharvest}

\begin{quote}

The sacrificial, disciplined giving of ourselves is like sowing precious seed into the ground. Instead of consuming it as food, seed is ``sacrificed'' for the sake of a future harvest.
\end{quote}

In agricultural contexts, seeds are a form of wealth. They are a type of treasure. Yet seed is generally stored for only a short time before it is used. Whatever is not required for food, for daily bread, must be sown to produce another harvest.⁠\footnote{2 Corinthians 9:6--12}

In a similar way, God supplies spiritual life to us. This is what Paul refers to as the treasure of knowing the Messiah. Experiencing the fruit of the Holy Spirit.\footnote{Galatians 5:22} Becoming alive to the God's presence. This is the spiritual equivalent of receiving daily bread..

Yet this personal aspect of knowing the Messiah is not the whole purpose of our relationship with him. God calls the Messianic Community to him in order for us \emph{to become his servant community}. He asks us to give our lives to serving his purposes. This requires genuine sacrifice and discipline—which is what it means to be a disciple.

\subsection{Sharing treasure}
\label{sharingtreasure}

\begin{quote}

Only in sacrificially sharing our spiritual treasure do we discover and realise our vocation. And then, in due time, we reap a harvest of faithfulness.⁠
\end{quote}

Discipline and sacrifice are amongst the most significant secrets to living a messianic life. They are secrets that many people hardly discover at all. Let alone realise as life-giving principles.

Yet the illustration of seed demonstrates that there is no other way to a rich harvest. Keeping our knowledge of and communion with God to ourselves is not an option. We must share this treasure both within and beyond our own communities.

Even so, we should never waste our spiritual treasure by casting it carelessly away. Although some seed inevitably falls upon unreceptive ground, a farmer never intentionally wastes seed. Likewise, spiritual treasure is precious. We should beware of squandering it on ``poor soil.'' Our greatest investment should be in those who recognise the worth of this treasure. They do this when they make room for its transformative power.

People who treasure the seed that is the Word of God are those whom Jesus refers to as good soil.\footnote{Matthew 13:1--23} They are those who hear the message and understand it. They go on to produce a harvest, thirty, sixty or a hundred times what was sown. 

Faithful disciples, transformed through a personal knowledge and experience of the Messiah, sharing their treasure with other faithful people, who share it with other faithful people and so on and so on. Each person's faithfulness contributing momentum to a movement of messianic discipleship.

\section{Discipleship Movements}
\label{discipleshipmovements}

\begin{quote}

Historical dynamism is linked directly to functioning as a generational discipling movement. A movement spreading openly across social, ethnic, linguistic, geographical and cultural boundaries.
\end{quote}

The formation of faithful disciples was at the heart of Jesus' life and work. Today, two millennia of history bears testimony to the significance of this strategy. His once-tiny group of disciples has spawned a worldwide movement of Christian people.

This community has developed far beyond its origins as an obscure Jewish sect. Today it is an international, inter-cultural, multi-ethnic community. A Messianic Community existing, in some form or another, in every part of the world. As it has spread, its message and way of living has impacted innumerable peoples and cultures.

\begin{pause}

The book of Acts testifies to the dynamic growth of the early messianic movement. From its beginnings in Jerusalem, it expands rapidly across the ancient world. From Israel, into Asia Minor, across Greece and finally to Rome, the seat of imperial power.

Witness how the growth of the early messianic community took place, by examining the context of these passages:

\begin{itemize}
\item Acts 2:46--47, 6:7, 9:31, 12:24, 16:5, 19:20, 28:30--31

\end{itemize}

\end{pause}

\subsection{Loss of generational momentum}
\label{lossofgenerationalmomentum}

History also demonstrates that momentum does not always continue. Many messianic movements began well, yet are now only historical footnotes. Some have been obliterated by fatal levels of persecution. Others continue institutionally, yet without any sense of spiritual renewal. They lack both generational momentum and the confidence to challenge cultural and social idols.

In such cases, the church has ceased to be a movement. Invariably, it has ceased making disciples. Empty traditions have taken the place of a living, reproducing body of people. It may remain dogmatic about its creeds. Yet it no longer exhibits an authentic zeal to serve God's purpose amongst the nations. It has ceased persuading either itself or others to forsake human idols and wholeheartedly follow the Messiah.

\subsection{Establishing generational momentum}
\label{establishinggenerationalmomentum}

\begin{quote}

Visionary, generational discipleship must be at the heart of spirituality and practical activity. Disciples must be invited, formed and sent forth as part of a world-facing movement.
\end{quote}

Establishing the momentum of generational discipleship requires constant visionary vigilance. It calls for a persistent dedication to personal, social and cultural transformation. A critical engagement with society combined with the call to turn towards the living God.

This requires more than the maintenance of congregational activity. It requires more than learning to care for one another, within local communities. It requires a practical and spiritual preparation of disciples equipped to serve God's purposes. Equipped to work amongst a world of oppressed, pained, fearful, idol-bound populations.

Such a reality begins by gripping people's hearts with a vision of faithfulness towards God. A faithfulness that leads to personal and social renewal. To the transformation of individuals, families, marriages, partnerships, communities, organisations, structures, workplaces and working practices. Especially where there is systemic injustice. 

This kind of transformation is only possible through the formation of faithful, persistent disciples. Disciples committed to living and working interdependently together. Disciples active in forming other, faithful disciples.

\section{Anointed Community}
\label{anointedcommunity}

\begin{quote}

The messianic community is a charismatic people. Disciples called into covenant relationship with the Father. United with the Son. Sent into the world. To bless the nations. In the power of the Spirit.
\end{quote}

To make possible this high calling, messianic discipleship provides a unique ingredient. One that no other philosophy, ideology or faith provides. The dynamic of the indwelling Spirit of the Messiah.

\begin{itemize}
\item Through the Spirit, the covenant community is transformed into a \emph{charismatic community}. A group of people endowed with spiritual gifts profoundly shaped to liberate human beings from idolatry and the allegiances and falsehoods that compete against the knowledge of God.⁠\footnote{2 Corinthians 10:3--5}

\item This charismatic community is brought under God's authority by being baptised into the Messiah. Through its baptism in the Spirit, it is anointed with the fragrant presence of the Holy Spirit.

\item It is a body learning to walk in the footsteps of Jesus. Learning to exercise its God-appointed role, under the direction of the Spirit of God.⁠\footnote{Romans 6:3--4; Galatians 3:26--29; Hebrews 6:4}

\end{itemize}

\begin{pause}

\begin{description}

\item[charismatic]

— from the Greek, \emph{charism}, meaning \emph{gift}; the \emph{charisma} of the Messianic Community derives from its anointing with the Spirit of the Messiah.

\item[messianic]

— essentially means \emph{anointed to bring deliverance}; the Messianic Community is anointed with the Spirit to mediate on behalf of God's people and the nations, bringing deliverance from evil and new life in the Messiah.
\end{description}

\end{pause}

\subsection{Dying to live}
\label{dyingtolive}

God's intention is that this messianic, charismatic, covenant community co-works in partnership with him. Using the strength, the power, the spiritual life, the anointing that he provides.

Too often, the power of the anointed-life-of-Christ-within seems to elude us. It seems out of our reach. Beyond our grasp. Indeed, it is not something to be grasped at all.

The only pathway to anointed, messianic life is through dying. Yielding ourselves to God the Father, through unity with the Messiah, by the power of the Spirit. That is the message of the cross. As we \emph{die to self}, we become \emph{alive to God}.\footnote{Romans 6:4--13}

\subsection{The heart of discipleship}
\label{theheartofdiscipleship}

\begin{quote}

As we embrace a practical form of discipleship, dying daily to ourselves, we become alive to God. We are equipped to serve his eternal purpose. That is the heart of Maize Plant Discipleship.
\end{quote}

Thus we end as we began. With the foundational principle of transformative discipleship. The seed sown into the ground to produce a rich harvest.

\begin{quote}

I tell you that unless a grain of wheat that falls to the ground dies, it stays just a grain. But if it dies, it produces a big harvest.\footnote{John 12:24}
\end{quote}

This life-giving spiritual reality represents the heart of Jesus' life, mission, ministry and death. And this same principle forms the foundation and wellspring of Maize Plant Discipleship.

\begin{center}\rule{3in}{0.4pt}\end{center}



\vspace*{\fill}
\begin {reflection}


\begin{itemize}
\item How is discipleship valued in your context?

\item How faithfully is it practised?

\item If there is a gap between what is believed and valued and what is practised, why do you think that is?

\end{itemize}

\end {reflection}

\chapter{The Africa Factor}
\label{theafricafactor}

\begin{quote}

``You are touching something that is not already existing. If we talk about evangelism, it may well be a new way of approaching evangelism, but we already have many methods of evangelism. But (a series of books focusing on) discipleship is {\ldots} really an innovative thing'' — a Burkinabé theological educator, examining a prototype Maize Plant Discipleship booklet.
\end{quote}

\section{African contexts}
\label{africancontexts}

An important philosophy guided the development of Maize Plant Discipleship: \emph{African voices should determine the theology of practical relevance to African contexts.} 

Maize Plant Discipleship is authored by an outsider to Africa. Yet its development is wholly a response to African contexts. The missional energy, insight and cultural perspective of African people has been crucial and elemental. In Romans 4:17, Paul reveals that Abraham's faith called into being things that did not exist. In essence, the faith of African leaders and learners called Maize Plant Discipleship into being.

\emph{Assemblée Evangélique de Pentecôte} and \emph{Mouvement des Jeunes Serviteurs de Dieu} were especially significant. Between 2003 and 2010, they regularly organised leadership training conferences, in Léo and Ouagadougou. This collaboration facilitated the repeated testing and continual refinement of the Maize Plant Discipleship Syllabus. It culminated in doctoral research, conducted by the author of Maize Plant Discipleship, amongst Burkinabé leaders and learners, in 2010.

\section{Doctoral Research}
\label{doctoralresearch}

Survey questionnaires, focus groups and personal interviews gathered insights from over seventy participants. A significant number of participants held denominational responsibility for leadership training and discipleship.

The research explored multiple themes relating to leadership training and intercultural dynamics. Data analysis of participant responses revealed critical findings relating particularly to:

\begin{itemize}
\item Discipleship

\item Theology

\item Literature.

\end{itemize}

\subsection{Discipleship}
\label{discipleship}

Research participants identified strongly with the concept of discipleship. In particular, they identified a need to embrace fresh, holistic, generational discipleship practices. Hence, Maize Plant Discipleship focuses upon:

\begin{enumerate}
\item Awakening or strengthening contextual ownership of the call to serve God's eternal purpose

\item Promoting lifelong commitment to missional action and disciplines, including the generational formation of disciples

\item Encouraging personal, communal and cultural transformation.

\end{enumerate}

\subsection{Theology}
\label{theology}

Participants consistently validated the theological content of a prototype learning resource. Accordingly, Maize Plant Discipleship incorporates:

\begin{enumerate}
\item A holistic worldview, communal orientation, charismatic spirituality

\item A historical, covenantal, missionary interpretation of Scripture

\item A biblical theology incorporating: (i) discipleship; (ii) suffering and overcoming; (iii) spiritual revival, intercessory prayer and spiritual power; (iv) poverty and prosperity; (v) personal and corporate vocation; (vi) Christ-centred servant-leadership and (vii) cultural transformation.

\end{enumerate}

\subsection{Literature}
\label{literature}

Research participants repeatedly identified a need for appropriate literature. In Burkina Faso, leaders and learners frequent two cultural worlds: orality and literacy. Maize Plant Discipleship resources seek to bridge between these two cultures. Hence, they:

\begin{enumerate}
\item Focus on practical discipleship formation

\item Eschew academic, philosophical language

\item Follow a thematic, modular structure

\item Aim at cohort group use

\item Incorporate reflection, group discussion and memorisation

\item Include graphical and metaphorical illustrations

\item Are economic to reproduce and distribute

\item Are licensed for non-commercial, vernacular adaption, translation and republication

\end{enumerate}

\vspace*{\fill}
\begin {reflection}

\begin{itemize}
\item Why should African Christians decide what is appropriate to African contexts?

\item Has this happened historically in your contexts? If not, why might that be? If so, what has changed as a result?

\item What missiological issues are important to you and others in your context?

\end{itemize}

\end {reflection}

\chapter{The Licence}
\label{thelicence}

\begin{quote}

A Creative Commons licence permits republication of all Maize Plant Discipleship Resources.
\end{quote}

\begin{figure}[htbp]
\centering
\includegraphics[width=250pt,height=53pt]{cc@3x.png}
\label{cc3x.png}
\end{figure}

What follows is a simplified summary of—and not a substitute for—the licence, which may be accessed at: 

\begin{itemize}
\item \href{http://creativecommons.org/licenses/by-nc-sa/4.0/}{http:/\slash creativecommons.org\slash licenses\slash by-nc-sa\slash 4.0\slash } and

\item \href{http://maizeplantdiscipleship.info}{http:/\slash maizeplantdiscipleship.info}

\end{itemize}

High-quality PDF files for republication and plain-text files for translation are available at:

\begin{itemize}
\item \href{http://johnbrc.github.io/MPD-Distribution/}{http:/\slash johnbrc.github.io\slash MPD-Distribution\slash }

\end{itemize}

\section{The Licence}
\label{thelicence}

Maize Plant Discipleship by John B. Clements is licensed under a \textbf{\href{http://creativecommons.org/licenses/by-nc-sa/4.0/}{Creative Commons Attribution-ShareAlike 4.0 International License.}\footnote{\href{http://creativecommons.org/licenses/by-nc-sa/4.0/}{http:/\slash creativecommons.org\slash licenses\slash by-nc-sa\slash 4.0\slash }}}

\begin{itemize}
\item Based on a work at \href{http://johnbrc.github.io/MPD-Distribution/}{http:/\slash johnbrc.github.io\slash MPD-Distribution\slash }.

\item The licence applies to all handbooks belonging to the Maize Plant Discipleship Syllabus (\autoref{thesyllabus}).

\item Please take care to understand and observe the terms of the licence.

\end{itemize}

\begin{summary}

You are free to:

\begin{description}

\item[Share]

— copy and re-distribute the material in any medium or format

\item[Adapt]

— remix, transform and build upon the material
\end{description}

The licensor cannot revoke these freedoms as long as you follow the licence terms:

\begin{description}

\item[Attribution]

— You must give appropriate credit, provide a link to the licence, and indicate if changes were made. You may do so in any reasonable manner, but not in any way that suggests the licensor endorses you or your use.

\item[NonCommercial]

— You may not use the material for commercial purposes (i.e. primarily intended for commercial advantage or monetary compensation.)

\item[ShareAlike]

— If you remix, transform, or build upon the material, you must distribute your contributions under the same licence as the original.
\end{description}

\textbf{No additional restrictions} — You may not apply legal terms or technological measures that legally restrict others from doing anything the licence permits.

\end{summary}

\input{mpd-footer}

\end{document}
