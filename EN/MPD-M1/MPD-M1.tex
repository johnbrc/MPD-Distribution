\def\version{0.1.1 Author's draft}
\def\change{first draft, ch1}
\input{mpd-header}
\def\mytitle{The Eternal Purpose of God}
\def\module{1}
\def\translation{}
\def\isbn{}
\def\colour{true}
\input{mpd-document-en}
\chapter{Covenant and Scripture}
\label{covenantandscripture}

\begin{synopsis}

\textbf{The biblical covenants are key to faithfully understanding
God's historical and eternal purposes.}

Appreciating the significance of God's covenants is essential
to a holistic interpretation of Scripture.

\end{synopsis}\begin{topics}

\begin{enumerate}
\item Scripture in perspective

\item Characteristics of covenant

\item Characteristics of God's covenants

\item Unfolding covenantal purpose

\end{enumerate}

\end{topics}

\osection{Scripture}\bible

Read these passages (memorise \textbf{bold}):

\begin{itemize}
\item Genesis 1:27?32

\item Genesis 3:17?20

\item Genesis 9:9?16

\item \textbf{Genesis 12:1--3}, 15.18, 17.1--22, 22.15

\item Genesis 26:1?5,24 \& 28:13--15

\item \textbf{Exodus 19.4?6}

\item 2 Samuel 7.12--16 \& 23:5

\end{itemize}

Keep a marker in the passages, to refer to them throughout the study.

\section{Scripture in perspective}
\label{scriptureinperspective}

This topic explains how the books and stories within the Bible are bound together by a series of covenants. Appreciating the significance of covenant is essential to a faithful, holistic understanding of Scripture. This relevance of covenant may be illustrated by contrasting \emph{historical} and \emph{covenantal} perspectives of Scripture.

\subsection{Scripture in historical perspective}
\label{scriptureinhistoricalperspective}

Traditional perspectives of Scripture review the principal events and stories encountered in the books of the Old Testament in the historical order in which they took place. These events are then interpreted, looking at their significance for Christian faith. Principal events include:

\begin{itemize}
\item \emph{Creation}, the knowledge of good and evil, the awfulness of \emph{death} and murder.\footnote{Genesis 1--4}

\item A catastrophic \emph{flood} and the confusion of \emph{languages}.\footnote{Genesis 9--11}

\item The \emph{promises} of land and descendants, to a man named Abraham.\footnote{Genesis 12--22}

\item An \emph{exodus} from Egypt and formation of a new \emph{nation}, Israel.\footnote{Exodus-Deuteronomy}

\item Israel's journey into the \emph{land} promised to Abraham.\footnote{Joshua}

\item Israel's experiences, ruled over by \emph{judges, priests, prophets, kings}.\footnote{Judges, Samuel}

\item Israel and Judah's \emph{exile} from the land.\footnote{Kings, Chronicles}

\end{itemize}

Figure 1 represents these events on a timeline of biblical history.

\begin{figure}[htbp]
\centering
\includegraphics[width=296pt,height=69pt]{EP1history.png}
\caption{Figure 1 : Scripture in historical perspective}
\label{ep1biblicalhistory.png}
\end{figure}



In these historical events, several significant biblical themes are evident. For example, God's calling and blessing; human rebellion, judgement and disaster; return, forgiveness, deliverance. These themes run through the accounts of Adam, Noah, Abraham, Moses and David. And the history of the people of Israel. They illustrate the importance of faithfulness and the dangers of disobedience towards God's commands.

\subsection{Scripture in covenantal perspective}
\label{scriptureincovenantalperspective}

Another theme is evident within these historical narratives. A profound theme that flows throughout Scripture: \emph{a series of divine covenants}. Discerning this pattern requires a re-examination of the biblical stories:

\begin{itemize}
\item The \emph{blessing} of creation.\footnote{Genesis 1:27?32}

\item A \emph{curse} placed upon the ground.\footnote{Genesis 3:17?20}

\item A \emph{covenant} established with Noah.\footnote{Genesis 9:9?16}

\item A \emph{covenant} established with Abraham.\footnote{Genesis 12:1--3 \& 15.18, 17.1--22, 22.15}

\item The \emph{covenant} renewed with Isaac and Jacob.\footnote{Genesis 26:1?5,24 \& 28:13--15}

\item A \emph{covenant} established with Israel, after the exodus from Egypt.\footnote{Exodus 19.4?6}

\item A covenant made with David.\footnote{2 Samuel 7.12--16 \& 23:5; see also Psalm 89.3}

\end{itemize}

Adding these covenantal events to the timeline creates \emph{a covenantal perspective of Scripture} (Figure 2).

\begin{figure}[htbp]
\centering
\includegraphics[width=299pt,height=179pt]{EP2covenants.png}
\caption{Figure 2 : A covenantal perspective of Scripture}
\label{ep2covenantalhistory.png}
\end{figure}



In this illustration, the historical events are like trees growing upwards, whilst the covenants are like roots growing downwards. Metaphorically, this demonstrates how essential the covenants are to biblical history.

\begin{summary}

\textbf{Covenant relationships, established with human beings, are crucial to God's redemption of his creation}

\end{summary}

\begin{questions}

\begin{enumerate}
\item Why did God chose to make covenants with human beings?

\item The Christian Bible consists of the Old and New Testaments. Testament is an alternative word for covenant. What does this suggest about the significance of covenant to Scripture?

\item What significance do God's covenants have to you and your community?

\end{enumerate}

\end{questions}

\section{Characteristics of covenant}
\label{characteristicsofcovenant}

This topic explores the characteristics of historical covenants. Covenants were typical to the near-Eastern cultures of the biblical era. This is the cultural background of the biblical covenants. 

Covenants were a form of treaty or alliance between two peoples. The establishment of a covenant signified the formation of a solemn commitment. The obligation bound together individuals, families, tribes and nations. They might be imposed or negotiated. They could be established between two equal peoples. Or by a stronger nation, conqueror or deliverer.

\begin{itemize}
\item A conquering nation (the \emph{suzerain}) offered protection, land and rewards for faithful service.

\item A subjugated tribe became a servant-nation (a \emph{vassal}) of the suzerain. They pledged allegiance, military service and payment of taxes.

\end{itemize}

\subsection{Establishing covenant}
\label{establishingcovenant}

The Hebrew word for both secular and divine covenants is \emph{ber�yth}. It occurs over 250 times in the Old Testament and always refers to a formal, binding arrangement between two parties. 

\begin{itemize}
\item A covenant was usually established by a solemn ceremony.

\item The blood of a sacrificial animal would be sprinkled. The two tribes would then share a celebratory meal.

\item This was called cutting covenant, after the knife cut that killed the sacrificial animal.

\end{itemize}

\subsection{Terms}
\label{terms}

Terms accompanied the formation of covenants. The terms prescribed behaviours constituting the \emph{upholding} or the \emph{violating} of the covenant relationship. For a conquered nation, terms could be benign and generous, or oppressive and dominating. They could refer to: 

\begin{itemize}
\item trade, food, water or other resources;

\item skill-sharing;

\item land, routes, territory;

\item taxes;

\item ceremony, tradition;

\item protection, allegiance, peace or war.

\end{itemize}

\subsection{Oaths}
\label{oaths}

Covenant partners swore an \emph{oath} to uphold the terms. These oaths were pronounced to the accompaniment of invocations consisting of: 

\begin{itemize}
\item \emph{blessings} (rewards, bounty) for upholding the covenant

\item \emph{curses} (sanctions, punishments) for violating the covenant.

\end{itemize}

\begin{summary}

\textbf{Covenant refers to a binding obligation forming a solemn relationship between two parties}. 

A sacrificial meal and spoken oaths confirm the covenant relationship: blessings and curses prescribe rewards and sanctions for covenant faithfulness and violation respectively.

\end{summary}

\begin{discuss}[\currentsectiontitle]

\begin{itemize}
\item How is covenant understood within your culture? 

\item How does this compare with covenants encountered in Scripture?

\end{itemize}

\end{discuss}

\section{Characteristics of God's covenants}
\label{characteristicsofgodscovenants}

This topic illuminates significant aspects of God's covenants. Like a powerful human lord, or chief, God makes covenant relationship with a people of his choice, designating covenant terms without negotiation.

\subsection{The calling of a covenant community}
\label{thecallingofacovenantcommunity}

Each covenant that God forms sets apart a family, a tribe, a nation, a people?\emph{a covenant community}. The faithfulness of this covenant community is vital to the service and success of God's wider purposes for his creation. They are a \emph{servant community}: a community called to serve God's purposes.

As such a community, they are promised God's favour and blessing. Yet these blessings are a \emph{means to an end}: the covenant community is blessed by God \emph{in order that} they can serve his purposes effectively, bringing glory to his name. When this underlying reality is obscured, the covenant community risks significantly frustrating God's purposes.

\subsection{Keeping covenant}
\label{keepingcovenant}

\emph{Keeping covenant} means faithfully preserving the solemn relationship, by observing the covenant obligations. It requires submission and obedience to the covenant terms: including forgiveness and restitution, following breaches. Yet it means \emph{more} than simply upholding precise terms. 

Being faithful to God's covenant requires recognising, accepting and co-operating with the \emph{vocation of the covenant community}. This is the reason that God calls a community into covenant with him: that they may become a \emph{faithful} servant community, called to serve his eternal purpose.

\subsection{Breaking covenant}
\label{breakingcovenant}

Since the covenant relationship is always greater than the covenant terms (which point towards the relationship), occasional breaches can be repaired and relationship restored. Restoration happens through: 

\begin{enumerate}
\item a recognition of breaches \emph{(confession)}

\item restitution \emph{(forgiveness)}

\item recommitment \emph{(renewal)}.

\end{enumerate}

When many breaches are made, without any restoration, it indicates that the solemnity of the relationship has been lost. The consequences of continual \emph{faithlessness} towards the covenant relationship leads to the punishments and curses incorporated within the covenant.

\emph{Why is the breaking of covenant such a serious matter?} Not because God is a harsh judge, but because it frustrates his purposes for the covenant. When his servant community fails to uphold the covenant relationship they cannot faithfully fulfil their covenant vocation.

\subsection{Certainty of divine covenant}
\label{certaintyofdivinecovenant}

Even if God's people break the covenant, \emph{God remains faithful} to his covenant purposes and upholds his sworn obligation. Because his character is unchanging, his eternal purpose remains intact. When God makes a covenant he is committing himself to fulfilling his divine purpose through that covenant. 

In Hebrews 6.13--18, the writer illustrates how God's purpose in swearing an oath is to \emph{demonstrate the unchangeable character of his intentions}. God won't change his mind. He will do what he has purposed and promised to do through his covenants and through his Covenant Community.

\begin{discuss}[\currentsectiontitle]

\begin{itemize}
\item In God's covenants, how are blessings and curses related?

\item How might this effect ideas about sicknesses and other problems sometimes associated with curses?

\end{itemize}

\end{discuss}

\section{God's covenantal purpose}
\label{godscovenantalpurpose}

This topic reveals how the biblical covenant points towards the \emph{restoration} of a creation corrupted by rebellion against God's purposes. 

The \emph{reconciliation} of human beings, into faithful relationship with their Creator, is central to this restoration. Furthermore, each of the four \emph{patriarchal} covenants reveals a significant and vital aspect of God's unfolding plans and purposes?in collaboration with his covenant community.

\subsection{Covenant with Noah}
\label{covenantwithnoah}

God's covenant with Noah begins by reminding Noah that human beings are \emph{made {\ldots} in his image}. There is an echo of the original creation blessing given to Adam, as Noah's family are instructed to \emph{be fruitful and multiply, swarm over the earth and multiply on it}.\footnote{Genesis 1:26--28}

\begin{itemize}
\item God establishes this covenant with Noah and \emph{his descendants, all the inhabitants of the earth, every living creature {\ldots} with the earth}.

\item The sign of \emph{a rainbow} serves as perpetual reminder of God's covenant promise to never again destroy all living things.\footnote{Genesis 6:18, 8:6--22, 9:8--16}

\end{itemize}

\begin{summary}

\textbf{God's covenant with Noah reveals and expresses God's commitment to uphold his creation}. In spite of man's evil, God will not abandon his covenant with his creation.

\end{summary}

\subsection{Covenant with Abraham}
\label{covenantwithabraham}

God's covenant with Abraham represents God's response towards the rebellion, degeneration and wickedness of human society, recorded in Genesis chapters 1--11. It reveals God's plan for \emph{a great nation} that will bless all \emph{the families of the earth}.\footnote{Genesis 12:1--3, 15.18, 17.1--22, 22.15}

The covenant thus contains both a promise and a subtle command: \emph{be a blessing}. This suggests Abraham is intended to convey his sense of promise, protection, blessing and divine purpose to the tribes and peoples with whom he relates. Because of his covenant with God, he is to anticipate being a blessing to others. The force of this covenant is thus destined to grow in two complementary directions: 

\begin{itemize}
\item \textbf{Downwards} towards Abraham's descendants, who are to be blessed \emph{in Abraham}.

\item \textbf{Outwards} towards the whole human family, who are to be blessed \emph{by Abraham}.

\end{itemize}

\begin{summary}

\textbf{God's covenant with Abraham reveals and expresses God's commitment to bless all the families of the earth through a great nation.} God is committed to restoring his creation, marred by human rebellion, in and through a faithful covenant community.

\end{summary}

\subsection{Covenant with Israel}
\label{covenantwithisrael}

A covenant is formed with the nation of Israel, against the backdrop of a powerful deliverance and exodus from the oppression of Egypt's Pharaoh. Abraham's descendants, the \emph{sons of Israel,} acknowledge God as their powerful and faithful Deliver, who has brought them to the point of entering the sanctuary of the land promised to their ancestors.

\begin{itemize}
\item Israel is offered the opportunity to become God's own \emph{treasure} and to become a \emph{kingdom of priests,} hinting at how Israel is called to mediate God's blessing to other nations.\footnote{Exodus 6:2--8, 19.4--6, Romans 9:4--9}

\end{itemize}

\subsubsection{Torah?pathway to covenant faithfulness}
\label{torahpathwaytocovenantfaithfulness}

Central to Israel's calling is the \emph{Torah} (meaning \emph{Teaching}), given to Moses on Mount Sinai and centred upon the \emph{Ten Commandments}.\footnote{Deuteronomy 5}

\begin{itemize}
\item Torah provides a set of clear, detailed instructions informing Israel how to live, in covenant relationship with God.

\item Torah thus forms the basis for Israel's vocation: through faithfulness to the covenant, Israel is called to demonstrate love, devotion and allegiance to God.\footnote{Deuteronomy 4:5--8}

\item Through covenantal faithfulness, they are to demonstrate God's wisdom and understanding and thus become \emph{a light to the nations}.\footnote{Isaiah 42:6}

\end{itemize}

\subsubsection{Life or death, blessing or curse, service or exile}
\label{lifeordeathblessingorcurseserviceorexile}

Israel's choice of faithful covenant relationship or unfaithful idolatry, represent a choice between \emph{life or death}, \emph{blessing or curse}, deliverance or disaster, inheritance or exile,\footnote{Deuteronomy 30:1--20} revealing God's: 

\begin{itemize}
\item \emph{Kindness} towards those embracing his covenant, submitting to his government and available to serve his purposes; 

\item \emph{Severity} towards those rejecting his covenant, resisting his purposes, rebelling against his goodness, disobeying his teachings.\footnote{Romans 11:22} 

\end{itemize}

Through their covenant with Adonai, the people of Israel become recipients of life, blessing, grace and goodness from God that will eventually overflow towards all nations and peoples.

\begin{summary}

\textbf{God's covenant with Israel reveals and expresses God's commitment to use a chosen people?a covenant community?to reveal his love and glory to the whole world}. God chooses an unimportant nation to demonstrate through them his goodness, kindness and covenant-faithfulness towards a world of people that has rebelled against his purpose for Creation.

\end{summary}

\subsection{Covenant with David}
\label{covenantwithdavid}

The covenant with Israel called upon them to recognise God as their King. The prophet Samuel thus recognises a deep tragedy unfolding when, in demanding a mortal king to rule over them, Israel rejects God's kingship.\footnote{1 Samuel 8:7--8} 

\begin{itemize}
\item After Saul, Israel's first king, falters, Samuel is directed to anoint \emph{David}, \emph{a man after God's own heart}.\footnote{Acts 13:22; cf. 1 Samuel 13:14} 

\item God makes a covenant with David, promising that one of his descendants will build a Temple for God's name and that David's royal throne will continue \emph{eternally}.\footnote{2 Samuel 7:12--16; also 1 Chronicles 17:11--14, Psalm 89:19--37}

\end{itemize}

Following David's death, rulership passes to his son, Solomon, who begins his reign with great wisdom and overt expressions of covenant faithfulness towards God?including the lavish construction and dedication of the Jerusalem Temple. 

Solomon's many non-Hebrew wives and concubines lead him to worship other gods. His idolatry invites God's judgement and results in a national division: into the Northern Kingdom of Israel and the Southern Kingdom of Judah and, ultimately, the exile of both from the Land.

The covenant promise to David must be fulfilled by another descendant?a \emph{branch} of David's line\footnote{Jeremiah 23:5--6}: a Redeemer, a King, a Messiah, who will rise in the future to bring deliverance and blessing to Israel.

\begin{summary}

\textbf{God's covenant with David reveals and expresses God's commitment to choose and anoint one of David's descendants to eternally establish God's kingdom reign on earth}. God hints at a future Anointed King?a Messiah?who will establish the Kingdom of God eternally.

\end{summary}

\begin{discuss}[\currentsectiontitle]

\begin{itemize}
\item Why does God need a covenant community and what does he most need from them?

\item David was shepherd, warrior and king. Which role most faithfully expressed him as \emph{a man after God's own heart}?

\end{itemize}

\end{discuss}

\ssection{Covenant and Scripture}

This concludes Study 1, which:

\begin{itemize}
\item explained how historical, biblical events are linked by a series of covenants

\item explored the nature of covenant

\item illuminated significant aspects of God's covenants

\item revealed how the patriarchal covenants point towards God's unfolding plans and purposes, both for his covenant community and the nations of the world.

\end{itemize}

The diagram below updates the biblical panorama, constructed by this study, incorporating scriptural elements from each of the patriarchal covenants:

\begin{itemize}
\item The first covenant with the house of Israel

\item A light to the gentiles

\item A great nation, a royal priesthood

\item A house of prayer for all peoples

\end{itemize}

\begin{figure}[htbp]
\centering
\includegraphics[width=299pt,height=210pt]{EP4light.png}
\label{ep4light.png}
\end{figure}



\begin{summary}

\textbf{Overarching the biblical panorama is the Abrahamic blessing: a great nation, blessed to be a blessing to all the families of the earth.} God's universal concern for every tribe, language and nation stretches out, beyond the era of Abraham and Israel, forming a foundational framework for a new, messianic covenant{\ldots}soon to be revealed

\end{summary}

\vspace*{\fill}
\begin{bonus}

\begin{itemize}
\item Who are \emph{all the families of the earth}?

\item Africa, the place and its peoples, has known many tragedies and abuses of human power: what is God's response?

\item In the way we live, how do we both reflect and contradict God's commitment to creation?

\item To reflect God's commitment to creation, what might we do: More often? Less often? For the first time? For the last time?

\end{itemize}

\end{bonus}

\chapter{The New, Messianic Covenant}
\label{thenewmessianiccovenant}

\begin{synopsis}

\textbf{The revelation of Messiah is the goal of God's covenants with his chosen people.} 

Through the New Covenant, the Messiah became the One Mediator between God and humanity

\end{synopsis}\begin{topics}

\begin{enumerate}
\item Renewal of the Covenant

\item Revelation of the Messiah

\item Identity of the Messiah

\item Mediation of the Messiah

\end{enumerate}

\pagebreak 

\end{topics}\osection{Terms}\begin{description}

\item[Torah]

? \emph{Instruction} or \emph{Teaching}, as revealed to Moses and recorded in the first five books of the Bible, especially the Ten Words.\footnote{Exodus 20:1--17}

\item[Tanakh]

? Hebrew Bible (Old Testament); TaNaKh is an acronym formed by the first letter of the three traditional subdivisions of the Hebrew Bible: \textbf{T}orah, \textbf{N}evi'im (\emph{Prophets}) and \textbf{K}etuvim (\emph{Writings})

\item[Advocate]

? a person who pleads on someone else's behalf; a mediator, an intercessor
\end{description}

\osection{Scripture}\bible

Read these passages aloud; memorise the \textbf{bold} passages.

\begin{itemize}
\item Haggai 2:1--9

\item \textbf{Jeremiah 31:31--34}

\item Ezekiel 34:16--23

\item Isaiah 53:11ff

\item Malachi 3:1--4

\item \textbf{Romans 9:4--5}

\item Hebrews 9:15

\end{itemize}

\section{The renewal of the Covenant}
\label{therenewalofthecovenant}

THIS TOPIC SUMMARISES the historical, covenantal background to the New Testament, focussing upon biblical prophecies of a new covenant and a priestly messiah.

\subsection{National renewal}
\label{nationalrenewal}

Israel's lengthy exile from the promised Land symbolises a period of spiritual darkness: a reminder of their idolatrous breach of the covenant and a painful indicator of God's ongoing judgement. 

Even when a remnant of Jews, led by Ezra, Nehemiah and Malachi, begins a slow, partial return to Israel, followed by a rebuilding of the Jerusalem Temple, it becomes clear that this restoration and return from exile is practically and spiritually incomplete and that it does not represent the definitive, hoped-for sign of God's grace, forgiveness and \emph{covenantal renewal}.\footnote{Haggai 2:1--9}

\subsection{Spiritual renewal}
\label{spiritualrenewal}

The popular expectation is of a glorious, messianic king who will righteously shepherd the people of Israel, cleanse the Land from pagan oppression and lead the nation towards prosperity. According to the prophets, though, God's concern is not with national, economic renewal, but a \emph{spiritual renewal} that addresses the covenant community's repeated faithlessness. 

\begin{itemize}
\item Jeremiah speaks of a new covenant which provides an inner cleansing from sin and a new intimate knowledge of the Lord and his \emph{Torah}, \emph{written upon the hearts} of his covenant people.\footnote{Jeremiah 31:31--34}

\item Ezekiel prophecies about a renewal in terms of a \emph{good shepherd} who will will challenge the leadership of Israel, in order to call forth a renewed covenant people from amongst Israel.\footnote{Ezekiel 34:16--23}

\end{itemize}

According to Isaiah,\footnote{Isaiah 53:11ff} Israel's messiah will not be a prosperous, military ruler, but a \emph{despised} and marginalised prophet, who lives a priestly, intercessory life of sacrificial obedience and suffering service, under the anointing of the Spirit: restoring justice to the poor and marginalised of the covenant community; providing a mediatory, atoning sacrifice, which renews the covenant community of Israel and provides \emph{a light to the Gentiles}.\footnote{Isaiah 49:6}

Thus, rather than a triumphal, military victory over pagan enemies, the true sign that Israel's sins have been forgiven and that her spiritual exile is over will be the Lord, \emph{suddenly coming to his temple}, in the pristine form of his Messiah, the Messenger of the (New) Covenant.\footnote{Malachi 3:1--4}

\begin{discuss}[\currentsectiontitle]

\begin{enumerate}
\item What was the significance of the return of Jews to the promised Land, after their exile in Babylon?

\item What kind of Messiah were the Jews expecting to arise and lead them and why?

\end{enumerate}

\end{discuss}

\section{The revelation of the Messiah}
\label{therevelationofthemessiah}

This topic reveals how the Messiah, Jesus (Yeshua), represents the goal at which the biblical covenants aim.\footnote{Romans 9:4--5 and 10:4}

\subsection{Revealing the Messiah}
\label{revealingthemessiah}

The New Testament Gospels provide historical accounts of Jesus' life, work, signs, miracles, teaching, discipleship and, finally, death, resurrection and ascension. Matthew, Mark, Luke and John write not as impartial observers or secular historians, but as committed members of an emerging \emph{new covenant community}. Their intent is to establish Jesus' true identity, significance and purpose as the Jewish Messiah.\footnote{John 20:31} 

They achieve this by drawing upon a wide range of biblical imagery, hints, stories, signs, events, parables and metaphors and, above all, prophecies that are deeply rooted in Israel's Patriarchal, covenant framework. They use these biblical signposts and illustrations to confirm and illuminate precisely how Jesus fulfils the ancient, covenant promises and biblical prophecies. In doing so, they strongly affirm that Jesus is the promised One: the Messiah and the guarantor of \emph{a new covenant with the house of Israel and Judah}.\footnote{Hebrews 7:22, 8:6--13}

\subsubsection{Messianic prophecies fulfilled in the Gospels}
\label{messianicpropheciesfulfilledinthegospels}

The following table of Scripture references illustrates some of the most significant ways in which the New Testament Scriptures reveal that Jesus fulfilled Old Testament Scriptures and prophecies, as the Jewish Messiah.\footnote{If groups do not have sufficient time or resources to go through all the Scripture references, learning groups should select and discuss those elements of most interest to them.}

\begin{table}[htbp]
\begin{minipage}{\linewidth}
\setlength{\tymax}{0.5\linewidth}
\centering
\small
\begin{tabulary}{\textwidth}{@{}LLL@{}} \toprule
Prophets wrote&\textbf{Messiah would be}&Gospels affirm\\
\midrule
Genesis 12.1--3, 17.19, 21.12, 28.14&\textbf{the seed of Abraham, Isaac and Jacob}&Matthew 1.2, Luke 3.34\\

\midrule
Deuteronomy 18.15,18&\textbf{a prophet like Moses}&Acts 3.20--22, Hebrews 3.1--2 ff.\\

\midrule
Samuel 7.11--16; Isaiah 9.6; Genesis 12.1--3&\textbf{a descendant of David and heir to his throne}&Matthew 1.16, 22.40, Acts 2.30\\

\midrule
Isaiah 40.3--6; Malachi 3.1&\textbf{preceded by one who would announce him}&Matthew 3.1--3, Luke 1.7, 3.2--6\\

\midrule
Psalm 2.7; 2 Samuel 7.12--16 (Proverbs 30.4)&\textbf{the Messiah, the Son of God}&Matthew 3.17; Luke 1.32\\

\midrule
Isaiah 11.2, 61.1; Psalm 45.7&\textbf{anointed with the Spirit of God}&Matthew 3.16, John 3.34\\

\midrule
Isaiah 42.1--4&\textbf{tender, compassionate, unostentatious}&Matthew 12.15--20\\

\midrule
Psalm 110:4; Zechariah 6:13&\textbf{a priest and mediator}&Hebrews 5:5--6; 6:20; 7:15--17, Hebrews 7:25--8:2; Romans 8:34\\

\midrule
Isaiah 61.1--2; 35.5--6; 42.18&\textbf{a source of healing and restoration to the oppressed}&Luke 4.18--19; Matthew 11.5, throughout gospels\\

\midrule
Isaiah 53.7--12&\textbf{one whose death atoned for Israel's sins}&Matthew 27.38; Mark 10.45; John 1.20; Acts 8.30--35\\

\midrule
Isaiah 53.9--10; Psalm 2.7, 16.10&\textbf{raised from the dead}&Matthew 28.1--20, Acts 2.23--36, 13.33--37\\

\midrule
Psalm 16:11; 68:18; 110:1&\textbf{seated at the Father's right hand}&Luke 24:51; Acts 1:9--11; 7:55, Hebrews 1:3\\

\midrule
Isaiah 11:10; 42:1,4; 49:1,12&\textbf{accepted by Gentiles}&Matthew 12:21; Romans 9:30; 10:20; 11:11; 15:10\\

\midrule
Isaiah 28:16; Psalm 118:22--23&\textbf{the cornerstone of God's messianic community}&Matthew 21:42; Ephesians 2:20, 1 Peter 2:5--7\\

\bottomrule

\end{tabulary}
\end{minipage}
\end{table}


\pagebreak 

\subsection{Messiah: goal of the covenants}
\label{messiah:goalofthecovenants}

The figure below, \emph{Messiah and New Covenant}, updates the biblical, historical timeline being constructed by this module, illustrating how: 

\begin{itemize}
\item the covenantal history of Israel provides the most appropriate perspective for encountering the New Testament narratives of the Messiah and the Kingdom of God

\item the Messiah and his mission form the true \emph{purpose, goal and climax} of the narratives and prophetic writings about Israel.

\end{itemize}

\begin{figure}[htbp]
\centering
\includegraphics[width=299pt,height=210pt]{EP4messiah.png}
\caption{Messiah and New Covenant}
\label{ep4messiah.png}
\end{figure}



\begin{discuss}[\currentsectiontitle]

\begin{itemize}
\item Why was there a need for a \emph{new covenant}?

\item What is the relationship of the new covenant with the biblical covenants that preceded it?

\end{itemize}

\end{discuss}

\section{The identity of the Messiah}
\label{theidentityofthemessiah}

THIS TOPIC EXAMINES the identity of Yeshua, the Messiah, which is rooted in the biblical covenants and the Jewish Tanakh.

\subsection{Comprehending the role of the Messiah}
\label{comprehendingtheroleofthemessiah}

The root meaning of the word, \emph{messiah}, is \emph{anointed} or \emph{poured on}. Anointing oil was poured onto Israel's kings, as they were invested with their authority.\footnote{Exodus 30:22--25} This anointing represents the placing of God's Spirit upon these leaders.

Withal, Jesus is not anointed by another human being. At his baptism by the prophet John, he is anointed directly with the Holy Spirit, at the same time as a voice from heaven declares: \emph{This is my son, in whom I am well pleased}. Following this, the Gospels and apostolic epistles reveal a series of identities and roles relating to Jesus,\footnote{Also, initially, \emph{rabbi}?a recognised religious office, yet with no messianic overtones and thus no threat to Israel's established leadership.} most notably prophet, king and high priest. 

The Old Testament records Israel's prophets, priests and kings holding different, complementary roles, each responsible for \emph{mediating} a particular aspect of God's authority.\begin{description}

\item[Israel's prophets]

mediate God's authority, in several distinct ways:
\end{description}

\begin{itemize}
\item interceding before God, on behalf of his purposes for his people. 

\item entering the \emph{heavenly council}, to hear and receive God's Word

\item delivering God's word to his people, announcing \emph{the coming of his kingdom}?including judgement upon faithlessness

\item anointing Israel's kings, challenging them to covenant faithfulness

\end{itemize}

Abraham is the first prophet identified in scripture; he intercedes and God hears his prayer. Moses is the greatest prophet in Israel's history; he receive the Ten Commandments and the Torah. Remarkably, Scripture records Moses' intercession changes God's mind, when God is planning to destroy the children of Israel.\footnote{Deuteronomy 33} God also works especially powerfully through the intercession of prophets Samuel, Elijah, Daniel, Isaiah and Jeremiah.\begin{description}

\item[Israel's kings]

mediate God's authority as his earthly co-ruler. To Israel, the king represents God's government. To God, the king represents Israel. They are so closely identified with God's authority that they are described as \emph{seated upon God's throne}.\footnote{1 Chronicles 28:5 \& 29:23} Kings David and Solomon are especially recognised for their skill in governing the people and for their prayerful intercession on behalf of Israel.

\item[Israel's priests]

mediate God's authority interceding ritually before God, on behalf of the nation and individuals. Their role and responsibility is to approach God's throne, to obtain grace, mercy, forgiveness of sin and a restoration of covenant relationship. In particular, once a year, the high priest enters the \emph{holiest place} to make atonement for the nation's sin.\footnote{Exodus 26:31--33, 36:35--36 (Matthew 27:51)}
\end{description}

\begin{discuss}[\currentsectiontitle]

\begin{itemize}
\item How does the mediation of God's authority differ between prophets, kings and priests?

\item Towards which type of authority do you personally relate most vitally?

\end{itemize}

\end{discuss}

Each role of prophet, priest, king provides a vital form of mediation and important insight into the identity, role and work of the Messiah. In the New Testament, Jesus is identified as:

\begin{enumerate}
\item \emph{A prophet like Moses}

\item \emph{A branch of David}

\item \emph{An eternal high priest}

\end{enumerate}

\osection{A prophet like Moses}

Within the gospels, \emph{prophet} is the principal role with which Jesus is popularly identified.\footnote{Matthew 21:11, Luke 24:19} As prophet, Jesus:

\begin{itemize}
\item \emph{Announces} the arrival of \emph{the kingdom of God}?declaring that God is about to act decisively in the history of Israel.\footnote{Mark 1:14--15} 

\item \emph{Calls} the \emph{lost sheep of the house of Israel} to repentance and a renewed covenant faithfulness.\footnote{Matthew 10:6, 15:24}

\end{itemize}

Eventually, Jesus is identified as the promised one, who is \emph{a Prophet like Moses}.\footnote{Deuteronomy 18:15--19, John 1:21, 6:14, 7:40, Acts 3:23, 7:37} Like Moses, Jesus addresses three of the most potent symbols of Israel's faith: leadership, temple and covenant.\begin{description}

\item[Leadership]

? \emph{the seat of Moses} is a metaphorical reference to the priestly leadership of Israel, who exercised their authority by virtue of the Torah of Moses, which they claimed to interpret and follow.\footnote{Matthew 23:1--39} 

At the time of Jesus, \emph{Pharisees} and \emph{Sadducees} exercise this position. Jesus does not challenge their holding this important office, rather he holds them responsible for the ongoing corruption of the covenant relationship: criticising them for publicly going through the motions of Torah-obedience, yet with \emph{uncircumcised hearts}, neglecting the \emph{greater commandments} of humility, mercy and justice, whilst making covenant faithfulness harder for others.\footnote{Jesus denounces these leaders, with a form of curse: \emph{Woe to you}{\ldots} (Matthew 23). There is ample precedent for such harsh denunciations in the history of Jewish prophets. \emph{Note}: the cursing of the fig tree (Matthew 17) is a metaphor, acted out by Jesus as a prophetic warning directed against the corrupted, incumbent leadership?\emph{not} the Jewish people in general. Moreover, the gospels reveal that amongst them some individuals were \emph{close to the kingdom}; some became disciples (Mark 12:28--34).}

\item[Temple]

? Jesus prophesies the calamity of the destruction of the Jerusalem temple, which symbolically represented the heart of the first covenant with Israel, predicting that God would then raise it up again, \emph{in three days}?a metaphorical reference to Jesus resurrection. 

In time, the disciples came to understand that, as far as access to God is concerned, the temple had been displaced in its significance by the Messiah.\footnote{John 2:13--22, 4:21, also 6:1--4; see also Matthew 27:51, Hebrews 9:3--9, 10:19--22} Following Pentecost, the temple came to be identified with the Messianic Community, who were \emph{living stones} being made into up a \emph{living temple} in which the Holy Spirit could dwell.

\item[Covenant]

? being \emph{the Prophet like Moses} hints at a renewal of covenant because Moses inaugurated the first covenant with Israel.\footnote{Deuteronomy 18:15--19; John 1:21; Acts 3:22--23 \& 7:37} Prior his death, Jesus clearly affirms to his disciples this reality of a new covenant.
\end{description}

\begin{bonus}

\begin{itemize}
\item What is the relationship between prophets, kingdom and covenant?

\item What faults does Jesus expose in Israel's leaders?

\end{itemize}

\end{bonus}

\osection{A branch of David}

In the context of first-century Judaism, \emph{messiah} meant \emph{king}: the special One, promised by God, descended from David, appointed and anointed by God to govern his covenant community.

For a significant time, Jesus is highly reluctant to confirm his identity as the Messiah, frequently instructing people not to speak about him or what he had done\footnote{Matthew 8:4, John 6:15; though compare John 4:25--26}. Recognising him as Messiah is foundational to the apostolic vocation.\footnote{John 1:49, 11:27; Matthew 16:16} Ultimately it forms the accusation against him that leads to his death.\footnote{Luke 22:66--23:42} 

Jesus' identity as the Messiah is confirmed by several different titles: \emph{Son of Man}, \emph{Son of God}, \emph{Son of David}, \emph{Good Shepherd}, \emph{King of the Jews}.\begin{description}

\item[Son of God ?]

Before his birth, Jesus is identified, to his mother, Mary, as one who will be called \emph{Son of God}.\footnote{Luke 1:31--35} Later, a voice from heaven declares that he is God's \emph{only begotten} {\ldots} \emph{beloved Son}.\footnote{Matthew 3:17; Luke 3:22; John 1:14,18} \emph{Son of God} and \emph{Messiah} are essentially synonymous.\footnote{E.g. Matthew 16:16, 26:63; John 20:31. In first-century Judaism, \emph{Son of God} is not a title that implies divinity, being simply used to refer to a godly, righteous person or the \emph{special one,} the Messiah, sent by God. In contrast, Christian tradition usually posits \emph{Son of God} as a direct reference to Jesus' divinity: i.e. equivalent to \emph{God, the Son}, a member of the \emph{Trinity} of Father, Son and Holy Spirit. The eternal divinity of Messiah is \emph{hinted} at within the Synoptic Gospels (Matthew, Mark, Luke), yet practically never the central idea. Thus Luke ends his gospel (ch.24:44--48) with Jesus encouraging his disciples to be witnesses, \emph{not of his divinity}, but rather that: \emph{Everything written about me in the Torah, the Prophets and the Psalms had to be fulfilled: the Messiah is to suffer and to rise from the dead on the third day.} The pre-existence of the Messiah is communicated by John the evangelist (John 1:1ff, 17:5). Paul's epistles affirm the Messiah's divine glory (Hebrew: \emph{Sh'kinah}), universal significance, eternal existence and exaltation to God's Right Hand.}

\item[Son of David ?]

a name by which Jesus is frequently identified\footnote{e.g. Matthew 12:23, 15:22, 21:9 etc}, a recognised title of the Messiah, relating to God's covenant promise to raise up a deliverer from King David descendants.\footnote{Acts 13:23; 2 Samuel 7:12--13; Isaiah 11:1; Jeremiah 23:5--6; Ezekiel 37:24; Psalms 89:3--4,35--36, 132:11; Luke 1:32--33, 69ff.; Romans 1:4} Jesus demonstrates that the Messiah is not only the descendant of David, but also David's \emph{Lord} and thus, greater than David.

\item[Son of Man ?]

a messianic title by which Jesus identifies himself\footnote{E.g. Matthew 8:20, 9:6, 10:23, 11:19, Luke 9:22}. It is apparently based on rabbinic interpretations of Daniel 7:13--14, thus possibly intentionally obscure to those other than Torah-teachers, Pharisees etc\footnote{Luke 22:46--70}.

\item[Good Shepherd ?]

Jesus identifies himself as the \emph{good shepherd}, a messianic reference from the Old Testament, synonymous with Son of David\footnote{Matthew 15:24, Ezekiel 34:23, 37:24. See Hebrews 13:20; 1 Peter 5:4; Psalm 23}. Although he declares himself sent only \emph{to the lost sheep of the house of Israel}, the gospels repeatedly hint at the salvation of the Gentiles?a long-anticipated aspect of the Messiah's work\footnote{John 10:11--14, 16, 27}.

\item[King of the Jews ?]

A title used by non-Jews\footnote{Matthew 27:37, Mark 15:26; Luke 23:3; John 19:19}. Jesus represented a threat to the Torah-teachers and the Pharisees, who eventually plotted to kill him, using this accusation to incite the Roman authorities and the crowds against him.\footnote{By contrast, Herod Antipas (Luke 23:8--12, 15, unlike his paranoid father, Herod the Great, Luke 2) is un-threatened by Jesus and untroubled by the messianic accusation.}
\end{description}

\begin{bonus}

\begin{itemize}
\item Why did Jesus hide his identity as the Messiah?

\item Which title of the Messiah do you find most significant? Why?

\end{itemize}

\end{bonus}

\osection{An eternal high priest}

The final aspect of the Messiah's role is as Israel's true \emph{high priest}, guaranteeing the better promises of the new, \emph{eternal covenant}.

As a high priest, the Messiah: 

\begin{itemize}
\item makes purification for sins

\item sits down at the right hand of God

\item lives forever, interceding for human beings.

\end{itemize}

These three realities represent the heart of the New Covenant and the fulfilment of the Messiah's vocation.\begin{description}

\item[Purification for sins]

John the Baptist identifies Jesus as \emph{the Lamb of God, who takes away the sin of the world}.\footnote{John 1:29; see 1 Corinthians 5:6--8} As a priest, Jesus \emph{offered one sacrifice, once and for all, by offering up himself {\ldots} the Messiah {\ldots} through the eternal Spirit, offered himself to God as a sacrifice without blemish}.\footnote{Hebrews 7:27, 9:14, cf. Leviticus, Numbers}

\item[Seated at God's right hand]

Jesus exerts his claim to be the Messiah most potently when he identifies himself with \emph{the Son of Man} {\ldots} \emph{at the right hand of the Power on high}.\footnote{Matthew 22:41--46; Mark 12:36; Luke 20:42*} 

Being invited to sit at God's right hand constituted the Messiah's enthronement as God's vice-regent, or co-ruling Prince.\footnote{Matthew 28:18} The idea of the Messiah seated at the right hand of God is found throughout scripture, forming the most definitive image of the Messiah and expression of his rule and authority.

Yet, in the traditional framework of Jewish teaching, for a man to identify himself as divine, co-existent with or exalted beside God was considered blasphemous and punishable by death. When challenged by Pharisees about this, Jesus skilfully demonstrates how this is not applicable.

\item[Eternal, heavenly advocate]

After his death and resurrection, Jesus enters the true, heavenly tabernacle, in order to offer to Adonai the sacrifice of his blood. In this way, the Messiah mediates the new, \emph{eternal covenant} as a sinless and eternal high priest. 

Jesus' priesthood is of a different order to the Levitical priests. Jesus comes from the tribe of Judah, not Levi and, through a \emph{midrash}, he is identified as a high priest, \emph{of the order of Malki-Tzedek}.\footnote{Hebrews} The significance attached to this is two-fold:

\begin{itemize}
\item He is therefore greater than Abraham

\item He lives forever, interceding eternally

\end{itemize}

\item[Greater than Abraham]

Having already demonstrated that \emph{Yeshua deserves more honour than Moses},\footnote{Hebrews 3:3}, the writer of Hebrews now demonstrates that the Messiah is greater than Abraham, Father of the Jewish nation. This is achieved by introducing \emph{Malki-Tzedek\footnote{From \emph{Malki} (King) of \emph{Tzedek} (Righteousness); he is also King of \emph{Shalem} (peace)}, priest of }El 'Elyon\emph{, who receives a tithe from Abraham and blesses him.\footnote{Genesis} Because }the greater blesses the lesser*,\footnote{Hebrews 7:7} the writer of Hebrews establishes that Melchizedek is greater than Abraham.

\item[Interceding eternally]

The writer of Hebrews uses Malchi-Tzedek's lineage to draw attention to the reality that the Messiah \emph{became a priest by virtue of the power of an indestructible life}\footnote{Hebrews 7:16}. Consequently, he has an \emph{eternal intercessory ministry}, advocating on behalf of human beings who come to the Father through him.\footnote{Hebrews 7:24--25; 1 John 2:2; Romans 8:34}
\end{description}

\begin{bonus}

\begin{itemize}
\item What does it mean to be \emph{seated at the Right Hand of the Power}?

\item Why is it vital that Jesus lives forever?

\end{itemize}

\end{bonus}

\section{The mediation of the Messiah}
\label{themediationofthemessiah}

This topic explores Jesus' roles as Israel's prophet, king and high priest and, thus, centre of an emerging, new covenant, messianic community.\begin{description}

\item[As prophet]

Jesus calls the covenant community of Israel to repent and to enter fully into the Kingdom of Heaven, which was arriving in the person of the Messiah, the new David, King of the Jews.

\item[As king]

he forms the centre of a new-covenant community: the anointed \emph{One Shepherd,} whom God is using to bring about his eternal purpose. He gathers not only the Jews, but also of \emph{other sheep{\ldots}not from this pen,}\footnote{John 10:16}:the Gentiles, \emph{all the families of the earth,} whom God promised to bless in Abraham

As Shepherd-King, he calls to all those who \emph{hear his voice}: \emph{Turn towards God; trust in me, follow and give your allegiance to me; submit to me as your Messiah, your Lord}.\footnote{Matthew 23:10} Amongst those who hear, follow and give allegiance to him he inaugurates a new, Messianic Covenant Community, centred around their resurrected, exalted Lord.

\item[As high priest]

the Messiah completely fulfils the \emph{new covenant} prophesied by Jeremiah, through which an anointed covenant community may experience the \emph{Sh'khinah} glory of God.
\end{description}

\subsection{The Eternal Covenant}
\label{theeternalcovenant}

As prophet, priest and king, Yeshua the Messiah is presented by the New Testament writers as greater even than the patriarchal roots of the covenantal, Jewish faith: Abraham, Moses and David. The New Testament carefully demonstrates that Jesus the Messiah is not only linked to these Patriarchs, but that he exceeds each one in holiness, obedience and submission towards God the Father.

Together,
 they establish that the Messiah has established the New \emph{Eternal} Covenant: at the earthly \emph{Pesach} of Calvary \emph{and} in the heavenly Tabernacle of the Power on High. Through it, Jesus has become the One Mediator between God and human beings:

\begin{quote}

\emph{God, our deliverer {\ldots} wants all humanity to be delivered and come to a full knowledge of truth. For God is one and there is but \textbf{one Mediator between God and humanity}, Yeshua the Messiah, himself human, who gave himself as a ransom on behalf of all, thus providing testimony to God's purpose at just the right time.}\footnote{I Timothy 2:3--6}
\end{quote}

\begin{discuss}[\currentsectiontitle]

\begin{itemize}
\item How do you respond to Jesus as an eternal high priest?

\item Why does humanity need a mediator between them and God?

\end{itemize}

\end{discuss}

\ssection{The New, Messianic Covenant}

This concludes Study 2, \emph{The New Messianic Covenant}, which:

\begin{itemize}
\item summarised the historical, covenantal background to the New Testament, focussing upon biblical prophecies of a new covenant and a priestly messiah

\item revealed how the Messiah, Jesus (Yeshua), represents the goal at which the biblical covenants aim

\item examined the identity of Yeshua, the Messiah, which is rooted in the biblical covenants and the Jewish Tanakhh

\item explored Jesus' roles as Israel's prophet, king and high priest and, thus, centre of an emerging, new covenant, messianic community

\end{itemize}

In summary, the study revealed how:

\begin{summary}

The Messiah's identity and vocation transitioned from being recognised, amongst his native Jewish people, as learned rabbi, to divinely-anointed prophet, to transcendent, priestly Messiah:
\textbf{Seated eternally at the Right Hand of God the Father, with complete authority in heaven and on earth, making him a perfect mediator on behalf of people from every ethnicity, tribe, language and nation}

\end{summary}

\chapter{The Messianic Covenant Community}
\label{themessianiccovenantcommunity}

\begin{synopsis}

\textbf{The outpouring of the Holy Spirit, at the feast of Pentecost, inaugurates a new covenant era of Spirit-led Messianic Community}

\end{synopsis}\begin{topics}

\begin{enumerate}
\item Pentecost: Torah and Spirit

\item Pentecost: Messianic Community

\item Pentecost: Light to the Gentiles

\item Pentecost: A New Humanity

\end{enumerate}

\pagebreak 

\end{topics}\osection{Terms}\begin{description}

\item[Pentecost]

? arriving fifty days after the second day of Passover, \emph{Pentecost} derives directly from the Greek word, \emph{pente,} meaning fifty; also known as \emph{Shavuot}?one of three annual Jewish festivals in which every Jewish male makes a pilgrimage to Jerusalem
\footnote{John 2:13, 7:2--4; Leviticus 23:33--43; Numbers 29; Deuteronomy 16}
\end{description}

\osection{Scripture}\bible

Read these passages aloud; memorise the \textbf{bold} passages.

\begin{itemize}
\item Matthew 5:17--18

\item Acts 1:4--5, \textbf{1:8}, 24:14

\item Romans 7:14, 8:1--9

\item 2 Corinthians 1:21--22, 3:17--19

\item Colossians 3:10

\item Hebrews 10:16

\item James 1:25

\end{itemize}

\section{Pentecost: Torah and Spirit}
\label{pentecost:torahandspirit}

THIS TOPIC EXPLORES significant parallels between \emph{Torah} and the Holy Spirit, including how key messianic prophecies are fulfilled by the Spirit working in and through the Messianic Covenant Community.

\subsection{New covenant promised}
\label{newcovenantpromised}

Under the first covenant with Israel, God becomes displeased when the people harden their hearts against him.\footnote{Hebrews 3:7ff.} In time, he declares through the prophet Jeremiah that he will make a new covenant with Israel?as discussed in The New, Messianic Covenant (\autoref{thenewmessianiccovenant}).\footnote{Hebrew 8:7--13, Jeremiah 31:31--34} This covenant will different from the first. It will bring about a new intimacy with God, characterised in these ways:

\begin{itemize}
\item God will put his Torah in the minds of his people

\item He will write his Torah upon their hearts

\item All will know him, from the least to the greatest

\item He will be merciful {\ldots} and remember their sins no more.

\end{itemize}

\subsection{Torah brought to completion}
\label{torahbroughttocompletion}

The Messiah, therefore, does not come to abolish Torah. He comes to bring it to completion?to bring it to its intended goal.\footnote{Matthew 5:17--18} Indeed, the Messiah is the goal at which Torah aimed!\footnote{Romans 10:4. Greek, \emph{telos}, typically translated \emph{end}, meaning \emph{goal, purpose, consummation}?not \emph{termination.} Hence, CJB: \emph{the goal at which the Torah aims is the Messiah.} Understanding Torah's goal as the Messiah accords \emph{unity} to New and Old Testaments and \emph{continuity} to outworking of God's covenantal purposes; see also Romans 3:31} This accords with the earlier studies: \emph{Covenant and Scripture} (\autoref{covenantandscripture}) and \emph{The New, Messianic Covenant} (\autoref{thenewmessianiccovenant}), which revealed how Jesus fulfilled the messianic prophecies and covenant promises, through his three mediatory roles of prophet, priest and king. He further fulfilled Torah as the Passover lamb and as the heavenly \emph{manna}.\footnote{John 6:30--71}, as well as through his teaching, which amplified and revised the Torah.\footnote{Matthew 5--6}

\subsection{Another Counsellor}
\label{anothercounsellor}

After his resurrection, Jesus explains that he must go to the Father, yet he will not leave his disciples alone: he will send them another \emph{comforting Counsellor}.\footnote{John 14:15--17} This Counsellor, the Spirit of Truth, will lead them \emph{into all the truth}, telling them about things that will happen in the future and preparing them for the troubles, suffering and responsibilities lying ahead.\footnote{John 16:7--15} They are instructed to wait in Jerusalem, for this immersion in the Holy Spirit.\footnote{Acts 1:4--5} In this way, the disciples will receive power to be witnesses of the Messiah \emph{to the ends of the earth}.\footnote{Acts 1:8} Significantly, when it eventually takes place, this event coincides with the feast of \emph{Shavuot}, also known as Pentecost. Shavuot celebrates the giving of the Torah to the Jewish people and therein lies the significance of the outpouring of God's Spirit at that time.

\subsection{Torah and Spirit}
\label{torahandspirit}

When the Torah was originally given to the children of Israel, God descended upon the mountain of Horeb, which \emph{blazed with fire to the heart of heaven, with darkness, clouds and thick mist}. He spoke to the people, out of the fire, \emph{proclaiming his covenant} to them and instructing them to obey the Ten Words,\footnote{Exodus 20:1--17?Jews refer to the Ten Words: the first \emph{Word} being a proclamation about God, rather than a commandment.} which he wrote on two stone tablets.\footnote{Deuteronomy 4:7--14} 

The great sound and the tongues of fire which are experienced by the disciples, when the Holy Spirit is sent upon them,\footnote{Acts 2:2--4} provides an echo of the fire that blazed upon Horeb. At the same time, the singular nature of a tongue of fire upon each disciple signifies the gentleness and personal nature that is characteristic of the relationship with the Holy Spirit, under the new covenant. 

Table 2 compares this and other significant characteristics relating to \emph{Torah} and the first covenant and the Holy \emph{Spirit} and the New Covenant.

\begin{table}[htbp]
\begin{minipage}{\linewidth}
\setlength{\tymax}{0.5\linewidth}
\centering
\small
\caption{Table 2, Parallels between Torah and Spirit}
\label{table2parallelsbetweentorahandspirit}
\begin{tabulary}{\textwidth}{@{}RCL@{}} \toprule
\textbf{Torah} ? First Covenant&Scripture references&\textbf{Spirit} ? New Covenant\\
\midrule
teaches God's truth&Psalm 119; John 14.26; 15.26; 16.13&teaches God's truth\\

\midrule
given to Israel at Shavuot&Exodus 19.1 \& 34.22; Acts 2:1&given to Messianic Community at Shavuot\\

\midrule
Sh'khinah manifests in heavenly fire, smoke&Exodus 19.16--19, 24.9--11; Acts 2:2--3&Sh'khinah manifests in wind, tongues of fire\\

\midrule
teaching, instruction, commandments of God, written on tablets of stone&Exodus 31.16, Deuteronomy 4.11--14, Deuteronomy. 6.6; 1 Corinthians 2.15--16&Torah, mind of Messiah imparted, written on human hearts\\

\midrule
declares people guilty, brings death, fading glory&Leviticus 26:14--38; Romans 5:12--21; 2 Corinthians 3.2--18&declares people innocent, brings life, liberty, unfading glory\\

\midrule
lacks power to transform human nature&Jeremiah 31.32; Romans 8:3--10&transforms human nature, cleansing hearts\\

\bottomrule

\end{tabulary}
\end{minipage}
\end{table}


\begin{discuss}

\begin{itemize}
\item What are the differences between the \emph{end} (termination) of something and the \emph{goal} (completion, fulfilment) of its purpose?

\item How does the Holy Spirit communicate the heart and mind of the Messiah to God's people?

\end{itemize}

\end{discuss}

\section{Pentecost: Messianic Community}
\label{pentecost:messianiccommunity}

This topic examines how, as the people of Israel were formed by Torah, the nascent Messianic Community is formed by the Spirit, who \emph{writes Torah} on their hearts.

\subsection{Spirit poured out}
\label{spiritpouredout}

The Spirit of Truth is poured out,\footnote{John 14--16} leading to establishment of a new intimate, personal relationship with God, writing Torah upon the hearts of God's people,\footnote{Romans 8:3--4} as described in Jeremiah's prophecy. The reality of the Spirit's work amongst the Messianic Community is described by New Testament writers in a number of significant ways, which may be broadly grouped in these three categories:

\begin{itemize}
\item Spiritual life

\item New creation

\item Temple and body

\end{itemize}

\subsection{Spiritual life}
\label{spirituallife}

\begin{itemize}
\item \emph{Torah of the Spirit}: the first covenant had became reduced to a legalistic following of the letter of God's commandments in the Torah. The new covenant establishes a new, living way of freedom, grace and liberty, in which God's people are called into a new life dominated by the Spirit of God.\footnote{Galatians 5:1; 2 Corinthians 1:24, 3:17}

\item \emph{A new spirit}: God's people are being given a new heart, a new breath (Greek \emph{ruach}, means breath, spirit) of the Spirit.

\item \emph{A people after God's heart}: God is forming a people who will be ``after his own heart'' (Acts 13 cf David): a people sharing his heart, being faithful to his concerns, his purposes, his priorities, bearing his compassion, strength and joy within them.

\end{itemize}

\subsection{New creation}
\label{newcreation}

\begin{itemize}
\item \emph{A uniting with God's Messiah}, by the power and presence of the Spirit of God.\footnote{2 Corinthians 1:21--22}

\item \emph{Sharing in God's nature}, being \emph{recreated} in the image of the Creator.\footnote{2 Peter 1:41; Colossians 3:10; 2 Corinthians 3:17--19}

\item \emph{First-fruits of a new creation}. The Messiah is also called first-fruits, evoking imagery of the festivals{\ldots}

\end{itemize}

\subsection{Temple}
\label{temple}

\begin{itemize}
\item \emph{The priestly body of the Messiah}: a new covenant, messianic community representing the earthly body of the Messiah, who in turn represents their heavenly \emph{Head}

\item \emph{Incarnation}: the Word being made flesh, the Presence of God being able to \emph{tabernacle}, to dwell amongst men.

\item \emph{A temple of living stones}: in which God himself dwells.\footnote{1 Peter 2:4--5; Ephesians 2:21--22}

\end{itemize}

\begin{discuss}

\begin{itemize}
\item Which descriptions or symbols of the Spirit's outpouring are most familiar to you? Which are least familiar?

\item What is the significance of these images

\end{itemize}

\end{discuss}

\section{Pentecost: light to the Gentiles}
\label{pentecost:lighttothegentiles}

This topic examines the impact of Pentecost upon the vocation of the covenant community of Israel.

\subsection{Every nation under heaven}
\label{everynationunderheaven}

The feast of Shavuot required Jewish adult males to make a pilgrimage to Jerusalem. Thus, when the Spirit was poured out on the disciples, they were surrounded by \emph{religious Jews from every nation under heaven}.\footnote{Acts 2:5} As they received the Spirit, the disciples began to speak in different languages, identifiable by the international Jews! 

Peter addresses the gathered crowd, using the book of Joel, to confirm that this extraordinary occurrence is a sign from Adonai to the nation of Israel, confirming that God has made the man they had crucified, Jesus of Nazareth, \emph{both Lord and Messiah!}. As they are \emph{stung in their hearts} by this message, Peter calls them to \emph{Turn from sin, return to God and be baptised{\ldots}}, reminding them that God's promise \emph{is for you, for your children and for those far away ? as many as Adonai our God may call!}\footnote{Acts 2:37--39} That day, about three thousand are added to the disciples group.

\subsection{Jewish and Gentile Identity}
\label{jewishandgentileidentity}

Although these Jews will soon carry news of this event internationally, it is not yet wholly apparent that the Good News about Jesus is intended, according to God's purposes, to overflow from Israel towards the Gentiles.\footnote{Acts 13:44--49} For this to become clear a significant tension between Jewish \emph{cultural identity} and Jewish \emph{covenantal vocation} must first be resolved. 

The apostle, Peter, after experiencing a vision from God, travels to share the message about Jesus as Lord with a Roman centurion and his associates. Peter explains that \emph{for a man who is a Jew to have close association with someone who belongs to another (gentile) people, or to come and visit him, is something that just isn't done}, yet through the vision he has experienced he now understands that 

\begin{quote}

\emph{God has shown me not to call any person common or unclean{\ldots} that God does not play favourites, but whoever fears him and does what is right is acceptable to him, no matter what people he belongs to}\footnote{Acts 10:28--9} 
\end{quote}

As Peter declares the significance of the events about Jesus to the gentile believers, the Spirit falls upon and empowers them to speak in tongues! Unsurprisingly, these events produce considerable consternation amongst the Jewish believers. Before long though, believers scattered by persecution are sharing the Good News about Jesus widely, amongst Gentiles.\footnote{Acts 11:18--26}

\subsection{The incorporation of Gentiles into the covenant community}
\label{theincorporationofgentilesintothecovenantcommunity}

The unanticipated spreading of the Good News amongst the Gentiles brings a highly significant question to the fore: how are gentiles to be formally incorporated into the covenant community? Ever since Abraham, Jewish males have been incorporated into the covenant community of Israel by undergoing a circumcision.\footnote{Luke 1:59 \& 2:21} The immediate question arises, therefore, whether Gentiles must undergo circumcision in order to ratify their membership of the covenant community. Underlying this question, however, is the greater issue of the how the Gentiles are to relate to Torah.\footnote{Acts 15:5} 

At an assembly called to resolve the issue,\footnote{Acts 15:6--31} the apostles and elders effectively recognise how the outpouring of the Holy Spirit upon the Gentiles is an indication that their hearts are \emph{being cleansed by trust} and that no heavy demands, relating to Torah should be placed upon them, at that time.\footnote{Acts 15:9--10} Ultimately, deeper questions about the relationship of Jews and Gentiles, within the covenant community, will be resolved by the apostle, Paul.

\begin{discuss}

\begin{itemize}
\item Which people are acceptable to God?

\item Who is able to receive God's Spirit?

\end{itemize}

\end{discuss}

\section{Pentecost: a new humanity}
\label{pentecost:anewhumanity}

This topic explores the implications of the incorporation of Gentiles into the Messianic Community.

\subsection{Once far off{\ldots}}
\label{oncefaroff}

\begin{quote}

Remember your former state: you Gentiles by birth {\ldots} at that time had no Messiah. You were estranged from the national life of Israel. You were foreigners to the covenants embodying God's promise. You were in this world without hope and without God.
\end{quote}

Paul, writing to disciples in Ephesus, confirms precisely what has been made clear by our study of covenantal history:

\begin{itemize}
\item the covenants embody God's promise;

\item Gentiles did not share Israel's covenant vocation;

\item Gentiles formerly lacked specific hope of reconciliation with the Creator

\end{itemize}

\subsection{Now brought near}
\label{nowbroughtnear}

\begin{quote}

Now, you who were once far off have been brought near through the shedding of Messiah's blood.\footnote{Ephesians 2:13}
\end{quote}

Gentiles?formerly without a Messiah, without hope, estranged from and outside the reach of God's covenant faithfulness\footnote{The covenant faithfulness of God?that is, God's faithfulness to his intent and purpose and promise made to and through the covenant community?is usually translated by the phrase \emph{righteousness}. Righteousness, however, is typically understood as a moral characteristic. Yet God's \emph{covenant faithfulness} actually refers specifically to the certainty that he will fulfil?that he will be faithful to?the purposes that he has set forth through his Covenant, principally with the Patriarch Abraham, with subsequent covenants serving to confirm and amplify his intentions. Understanding this distinction provides a critical hermeneutical principle for a faithful interpretation of Scripture.}?are now, through the Messiah's death, brought near to God. They are brought into the covenant relationship, alongside the Jews?yet without conversion to Jewish \emph{culture}.\footnote{For the Jews, the relationship between their culture and the covenant itself was so close that, to use a proverb, they ``could not see the wood for the trees.'' The great struggle of the New Testament is to make plain how gentile peoples can be incorporated within the covenant and covenant community, without becoming Jewish?whilst also upholding and maintaining the dignity and uniqueness of the calling of Israel, in the wake of the rejection and death of the Messiah, at the hands of Jewish leaders, as well as Gentile rulers.} How is this made possible?

\begin{quote}

He himself is our shalom?he has made us both one and has broken down the \emph{m'chitzah} which divided us, by destroying in his own body the enmity occasioned by the Torah, with it's commands set forth in the form of ordinances.\footnote{Ephesians 2:14--15a}
\end{quote}

Paul is explaining the enmity between Jews and Gentiles. To explain the significance of this he refers to a \emph{m'chitzah}. A m'chitzah is a dividing wall or partition which separates people into two groups.\footnote{To this day, in Orthodox Jewish synagogues, a partition separates men from women.} In particular, it is a 1.5m high stone partition in the Jewish temple built by Solomon. This partition separated the inner Temple courts, where only Jews could enter, from the Court of Gentiles.\footnote{E.g. Acts 21.28} The m'chitzah is thus a highly symbolic barrier, representing the denial of gentile interaction with the heart of the temple, which itself embodied the covenant.

Accordingly, the \emph{breaking down of the m'chitzah} represents a profound statement of gentile admission to the covenant: a new freedom to approach God, on an equal footing with Jews, binding Jew and Gentile together equally within new covenant:

\begin{quote}

He did this in order to create in union with himself from the two groups a single new humanity and thus make shalom.\footnote{Ephesians 2.15b}
\end{quote}

\subsection{Shalom to those near and far}
\label{shalomtothosenearandfar}

\begin{quote}

when he came he announced as Good News shalom to you far off and shalom to those nearby, news that through him we both have access in one Spirit to the Father
\end{quote}

shalom
: peace, tranquillity, safety, well-being, welfare, health, contentment, success, comfort, wholeness, integrity

Note reference to ``access in one Spirit to the Father,'' which may hint at dividing wall (m'chitzah) also referring to removal of veil separating Holiest Place, access beyond which is now open to both Jews, Gentiles (no longer only High Priest)?Mt. 27.51

\begin{quote}

so then you are no longer foreigners and strangers; on the contrary, you are fellow-citizens with God's people and members of God's family{\ldots}
\end{quote}

Paul emphasises Gentiles new position : since both have access by same Spirit, to same Father-God, Gentiles no longer foreigners and strangers (excluded from covenants), also fellow-citizens?equal in status?moreover, members of God's family{\ldots}

\begin{quote}

you have been built on the foundation of the emissaries (apostles, missionaries) and the prophets, with the cornerstone being Jesus the Messiah himself.\footnote{Ephesians 2.17--19}
\end{quote}

{\ldots} a family built on common foundations, namely the Rock of Jesus himself!

\subsection{new state of gentiles}
\label{newstateofgentiles}

{\ldots} through being united with Messiah, gentiles become

\begin{itemize}
\item fellow citizens of God's family

\item joined to house, commonwealth of Israel

\item grafted into vine of Israel

\item incorporated into covenant community

\end{itemize}

{\ldots} resulting in

\begin{itemize}
\item messianic new covenant community

\item a single new humanity of Jews, Gentiles

\item mystical body of messiah

\end{itemize}

\subsection{Sharing faith of Abraham}
\label{sharingfaithofabraham}

\begin{quote}

the Tanakh, foreseeing that God would consider the Gentiles righteous when they live by trusting and being faithful, told the Good News to Abraham in advance, by saying, \emph{in connection with you, all nations (``Goyim'', gentiles) will be blessed}.\footnote{Galatians 3.8}
\end{quote}

Paul confirms conclusions from biblical history: God considering Gentiles as righteous (i.e. faithful members of covenant community) was foreseen by Scriptures

\begin{itemize}
\item when Good News, told in advance to Abraham

\item when God said, \emph{in connection with you, all peoples will be blessed}

\end{itemize}

Thus, Paul is explaining that the \emph{Good News}?the announcement of the arrival of the kingdom, of the reign of God arriving within, through life, death, resurrection of Messiah?was within God's heart and mind, when the covenant was made with Abraham{\ldots}

Thus we understand that from the beginning, God was making a response to the rebellion and death of creation,\footnote{Genesis 1--11} This response that began with the covenant with Abraham, is brought to fulfilment within, through Messiah and the people of God, both Jew and Gentile together, united in him. This confirms God's eternal purpose ? creation's glorious renewal through the Messiah (for whom, through whom{\ldots}) ? has been within God's heart from beginning of covenant history.

\subsection{Gentiles receive blessing of Abraham}
\label{gentilesreceiveblessingofabraham}

\begin{quote}

Jesus did this so that in union with him the Gentiles might receive the blessing announced to Abraham, so that through trusting and being faithful we might receive what was promised, namely, the Spirit{\ldots}
in union with the Messiah, you are all children of God through this trusting faithfulness; if you belong to the Messiah, you are seed of Abraham and heirs according to the promise?Galatians 3:9,14,26--29
\end{quote}

\subsection{Gentiles united spiritually to Messiah}
\label{gentilesunitedspirituallytomessiah}

Becoming spiritual \emph{seed of Abraham{\ldots}heirs according to promise}, through trusting, being faithful, receiving the blessing of Abraham: 

\begin{quote}

I have blessed you{\ldots} those who bless you, I will bless, those who curse you I will curse; you are to be a blessing to all families of earth.\footnote{Genesis 12.1--3}
\end{quote}

Thus, through Messiah, the gentile nations begin to share in the covenant purpose : a great people, blessed to be a blessing to all families of earth.

\begin{discuss}

- 

-

\end{discuss}

\ssection{The New, Messianic Covenant Community}

This concludes Study 3, \emph{The New, Messianic Covenant Community}, which:

\begin{itemize}
\item explored significant parallels between Torah and the Holy Spirit

\item examined how the Messianic Community is formed by the Spirit

\item examined the impact of Pentecost upon the vocation of the covenant community of Israel.

\item explored the implications of the incorporation of Gentiles into the Messianic Community.

\end{itemize}

In summary, the study revealed:

\begin{summary}

\textbf{God's commitment to create a faithful covenant community, experiencing \emph{shalom}, through receiving the forgiveness of sin and the gift of the Holy Spirit}

God pours out his Holy Spirit to create an anointed, messianic covenant community, a single new humanity?the mystical body of Messiah?comprising Jew and Gentile together, reaching out to all nations with Good News of God's reign through his Messiah

\end{summary}

\vspace*{\fill}
\begin{bonus}

-

\end{bonus}

\chapter{The Eternal Purpose of God}
\label{theeternalpurposeofgod}

\begin{synopsis}

\begin{center}\rule{3in}{0.4pt}\end{center}


\end{synopsis}\begin{topics}

\begin{enumerate}
\item Covenant partnership

\item New covenant glory (part A)

\item New covenant glory (part B)

\item God's eternal purpose

\end{enumerate}

\pagebreak 

\end{topics}\osection{Terms}

\osection{Scripture}\bible

Read these passages aloud; memorise the \textbf{bold} passages.

-

\section{Access by one Spirit}
\label{accessbyonespirit}

The Holy Spirit:

\begin{itemize}
\item leads into truth

\item unites disciples with Messiah

\item provides access to the Father

\item sends out disciples, from `temple', as generation of priestly people, mediation God's blessing to the peoples and families of the earth, in union with and in the name of Jesus, God's Son.

\end{itemize}

\section{Covenant partnership}
\label{covenantpartnership}

THIS TOPIC EXAMINES the ultimate purpose of the new covenant: covenant partnership between God and a people, blessed to be a blessing to the families and people of the earth.

\subsection{Inheriting the blessing}
\label{inheritingtheblessing}

Adonai invites those who are willing to trust in and follow his Messiah, to receive the fresh `breath' of his Spirit, to become \textbf{covenant-partners,} pursuing his eternal purpose. Gentile followers of the Messiah, by being united spiritually to the Messiah have become the spiritual ``\emph{seed of Abraham}'' and, as a consequence, heirs of the blessing: 

\begin{quote}

If you belong to the Messiah, you are seed of Abraham and heirs according to the promise.\footnote{Galatians 3:26--29}
\end{quote}

Accordingly, those united with the Messiah are not only those who themselves are `blessed,' \emph{but also} those who are called ``\emph{to be a blessing{\ldots}to all the families of the earth},'' receiving not only the blessing of first horizon of the covenant promise?the blessing of God towards them?but the also receiving the injunction of the second horizon?the injunction to be a blessing to all the families of the earth:

\begin{quote}

I will make of you a great nation. I will bless you and I will make your name great and \textbf{you are to be a blessing}. I will bless those who bless you, but I will curse anyone who curses you; and \textbf{by you all families of earth will be blessed}.\footnote{Genesis 12:1--3}
\end{quote}

What this means is that the body of the Messiah?the Messianic Community, the \emph{ekklesia}, the Church?has inherited the Abrahamic blessing and are, therefore, themselves, like Abraham \emph{avinu} before them, \textbf{blessed to be a blessing{\ldots}to all the nations of the world}. God has invested his blessing in those who follow the Messiah, in order that they reach out to other families, peoples and nations to bless them.

\subsection{Failing to inherit the blessing}
\label{failingtoinherittheblessing}

Adonai's covenant promise to bless Abraham was not simply for his own benefit or that of his descendants; Abraham and his descendants were blessed \emph{for a specific purpose}: to demonstrate God's covenant faithfulness and for his blessing to spread out to all nations. Whenever Israel became insular and parochial and failed in serving the God's purposes, Israel suffered God's judgement. Yet God could not abandon his eternal covenant?instead, he disciplined them, sometimes severely, in order to return them towards faithful covenant service.

In two thousand years, the Messianic, Christian covenant community?the Church?has too often become insular, parochial, self-serving, even idolatrous and, for long periods, largely failed in its vocation to be a blessing to the nations of the world. Oftentimes, other forces?military, legal, political and religious?entered into the moral vacuum, further short-circuiting the blessing that God intended to come through his covenant community. Nevertheless, the blessing of God has continually flowed out towards all nations and in the twentieth century, if numerical growth is any measure, it has reached surging proportions.

\subsection{The responsibility of the blessing}
\label{theresponsibilityoftheblessing}

Today, the Abrahamic blessing of God remains upon those who are united with the Messiah through their faith: they are \textbf{blessed to be a blessing to all the nations of the world}. To receive the blessing without seeking to be a blessing to peoples outside of God's covenant grace and goodness represents faithlessness towards the new covenant into which followers of the Messiah have been brought and, effectively, a rejection of the covenant terms. 

Thus it is that we find Paul expressing the radical determinant that those confessing allegiance to the Messiah cannot claim a true inheritance alongside the Messiah \emph{unless }they also act in covenant partnership with God?sharing practically in the *``fellowship of his suffering''\footnote{Philippians 3:10}:

\begin{quote}

The Spirit himself bears witness with our own spirits that we are children of God and if we are children, then we are also heirs, heirs of God and joint-heirs with the Messiah?\textbf{provided we are suffering with him} in order to be glorified with him.\footnote{Romans 8:17}
\end{quote}

Thus, all those united with the Messiah?Messianic Jews and Gentile Christians?are called into a form of `covenant community partnership' with God: \emph{called to serve the eternal purpose of God.}

\textbf{God's `new covenant' with the house of Israel expresses God's commitment to create a faithful messianic covenant community{\ldots}}

A community receiving and imparting the `shalom' of the kingdom through forgiveness of sin; a single 'new humanity'? a mystical 'body of the Messiah'?comprising Jew and Gentile, united in reaching out to all nations with the Good News of God's reign through his Son

\begin{discuss}[\currentsectiontitle]

- 

-

\end{discuss}

\section{New covenant glory}
\label{newcovenantglory}

This topic examines

From the genesis of creation, throughout the ages and the various biblical covenants, to the present and into the future, God has been and remains committed to his eternal purpose. He has never been, nor ever will be, diverted from it, until all is fulfilled. In the new covenant he has been faithful to this eternal purpose, revealed progressively through the covenants and the prophetic writings that bear witness to it:

\subsection{Table 2}
\label{table2}

God's covenant commitment
with
{\ldots}therefore, through the covenant
to uphold his creation
Noah
God will not abandon his commitment to his creation, in spite of man's evil
to bless all the families of the earth through `a great nation'
Abraham
God is committed to restoring his creation, marred by human rebellion, using a covenant community to bring human beings back into right relationship with him
to use a covenant community to reveal his love and glory to the whole world
Israel
God chooses an unimportant nation, to demonstrate his loving, covenant faithfulness towards a world of people which has rebelled against his eternal purpose for Creation
to anoint one of David's descendants to eternally establish God's kingdom
David
God hints at a coming Anointed One?a Messiah?a coming King who will establish the Kingdom of God eternally, on earth
to create a renewed, faithful covenant community, blessing all nations through God's reign through his Messiah
Israel, Messiah Yeshua
God establishes a Spirit-anointed, covenant-keeping community of Jew and Gentile together, receiving and imparting the `shalom' of the kingdom

\subsection{The covenants reveal God's eternal purpose}
\label{thecovenantsrevealgodseternalpurpose}

Clearly, throughout `covenant history', God has been resolutely working out his purposes in covenant partnership with human communities:

\begin{itemize}
\item The first covenant recognised his commitment to creation. He would not `begin again,' but work through this creation.

\item The second and third covenant revealed and established his method: a covenant nation through which all other nations would ultimately be blessed.

\item The fourth and fifth covenants revealed an Anointed King?a king, in some mysterious way, who was actually the Eternal One, God himself?acting to bring reconciliation, redemption and righteousness to a renewed covenant-community, blessed?by the outpouring of the Holy Breath of God?to be a blessing to all the nations of the earth.

\end{itemize}

Together the covenants reveal{\ldots}

\begin{quote}

God's eternal purpose accomplished in the Messiah Yeshua.\footnote{Ephesians 3:11}
\end{quote}

\subsection{Bringing many sons to glory}
\label{bringingmanysonstoglory}

\begin{quote}

Those who love God and are called in accordance with his purpose{\ldots}he determined in advance would be conformed to the pattern of his Son, so that he might be the firstborn among many brothers; and those whom he thus determined in advance, he also called; and those whom he called, he also caused to be considered righteous; and those whom he caused to be considered righteous, he also glorified!\footnote{Romans 8:29--30}
\end{quote}

Paul, writing to the Roman believers, arrives at one of the summits of his epochal arguments, speaking specifically about those called in accordance with God's eternal purpose. For those thus called, he writes, the process of their conformance to the pattern of the Son, the Messiah, comes to completion with their glorification. The writer of Hebrews makes essentially the same point when he declares:

\begin{quote}

In bringing many sons to glory, it was fitting that God, the Creator and Preserver of everything should bring the Captain of their salvation to the goal through suffering.\footnote{Hebrews 2:9--10}
\end{quote}

The Messiah was brought by God, through suffering, to his goal. Which goal? The goal of ``bringing many sons to glory''{\ldots}``that he might be the firstborn among many brothers.'' Thus, God's eternal purpose may be understood as bringing many `sons' to glory.

What does it mean to be brought to glory? The New Testament principally presents the following four distinct, though overlapping, aspects of `glory':

\begin{itemize}
\item The glory of the Messiah

\item The glory of knowing the Messiah

\item The glory of suffering with the Messsiah

\item The glory of resurrection with the Messiah

\end{itemize}

\subsection{The glory of the Messiah}
\label{thegloryofthemessiah}

`Glory' typically refers to ``brilliant, radiant beauty, splendour'' or ``high honour.'' The Hebrew word, Sh'khinah, refers uniquely to the glorious presence of God. In the old covenant schema, the Sh'khinah presence dwelt in the Holiest place, only entered by the high priest, once a year. In the new covenant schema, the Sh'khinah presence was no longer confined to the temple, but located absolutely within the Messiah:

\begin{quote}

{\ldots}his Son, to whom he has given ownership of everything and through whom he has created the universe. This Son is the radiance of the Sh'khinah, the very expression of God's essence, upholding all that exists by his powerful word.\footnote{Hebrews 1:1--3}
\end{quote}

The New Testament, however, also presents the Messiah as a relatively inglorious figure, ``emptying'' himself and ``taking the form of a slave,'' in order to obey God, in a way that resonates with the curious Messianic prophesy of Isaiah, ``he was not well-formed or handsome{\ldots}his appearance did not attract us{\ldots}he was despised; we did not value him.''\footnote{Isaiah 53:2--3}

Thus, the New Testament presents a more subtle form of glory. One that does not physically overwhelm, as when angels met prophets like Daniel, Ezekiel and John, causing them to fall on their faces, but rather which invites trust, confidence, love and self-surrender and, ultimately, transformation into the likeness of the character of the Messiah.

\begin{quote}

The Word became a human being and lived with us and we saw his Sh'khinah, the Sh'khinah of the Father's only Son, full of grace and truth.\footnote{John 1:14}
\end{quote}

\subsection{The glory of knowing the Messiah}
\label{thegloryofknowingthemessiah}

Such transformation takes place mysteriously within the hearts of those in covenant relationship with the Messiah.

\begin{quote}

God who{\ldots}said, ``Let light shine out of darkness''{\ldots}has made his light shine in our hearts, the light of the knowledge of God's glory shining in the face of the Messiah Yeshua.\footnote{2 Corinthians 4:6}
\end{quote}

Those who respond to this glory of the grace, truth and humility of the Messiah are thus, themselves ``{\ldots}changed into his image from one degree of glory to the next.'' However, this transformation does not take place in a spiritual vacuum, but by following in the footsteps of the Messiah, who ``learned obedience through his sufferings.''\footnote{Hebrews 5:8}

\subsection{The glory of suffering with the Messiah}
\label{thegloryofsufferingwiththemessiah}

Peter was one of three disciples who ascended a mountain with Jesus and saw him transformed, exhibiting brilliant, radiant splendour and actually heard the Father speak:

\begin{quote}

We saw his majesty with our own eyes{\ldots}we were there when he received honour and glory from God the Father and the voice came to him from the grandeur of the Sh'khinah, saying `This is my son, whom I love; I am well pleased with him!'\footnote{2 Peter 1:17; cf. Matthew 17:6}
\end{quote}

Yet, Peter also saw Jesus completely abased, humbled to the point of a despicable death on an execution stake?and was almost certainly changed more by the latter than the former. Peter thus presents this aspect of God's `glory' to his readers: the willingness to be completely humbled and to suffer in the pursuit of serving God's eternal purpose. 

If you are being insulted because you bear the name of the Messiah, how blessed are you! For the Spirit of the Sh'khinah, that is, the Spirit of God, is resting on you{\ldots}if anyone suffers for being Messianic, let him not be ashamed; but let him bring glory to God by the way he bears his name.\footnote{1 Peter 4:12--16}

Peter carefully subverts the traditional notion of the Sh'khinah presence, transforming it to include the resting of the Spirit of God upon those who suffer for following the Messiah. 

The essence of this subversive message is, in reality, the core of the Good News: the immense paradox that true life, eternal life, real, restored, abundant human life comes through embracing death. Not primarily the final death, but a life of `sharing in the fellowship of the Messiah's sufferings'. Thus, Paul writes of his own apostolic experience:

We have all kinds of troubles, but we are not crushed; we are perplexed, yet not in despair; persecuted, yet not abandoned; knocked down, yet not destroyed. We always carry in our bodies the dying of Yeshua, so that the life of Yeshua may be manifested in our bodies too. For we who are alive are always being handed over to death for Yeshua's sake, so that Yeshua's life also might be manifested in our mortal bodies. Thus, death is at work in us but life in you.\footnote{2 Corinthians 4:8--12}

Paul also declares ``I die every day,''\footnote{1 Corinthians 15:31} in speaking of the surrendering of his life to serve God, in spite of abasements, hardships, persecutions, hunger, danger {\ldots} It was because of him that I gave up everything and regard it all as garbage in order to gain the Messiah and be found in union with him{\ldots}I gave it all up in order to know him{\ldots}to know the power of his resurrection{\ldots}the fellowship of sharing in his sufferings as I am being conformed to his death.\footnote{Philippians 3:8--11}

Paradoxically, Paul explains, those who suffer with the Messiah are being transformed into his likeness through an experience of `dying to live.' They are highly honoured?blessed?yet willing to lay aside honour, wealth, acclaim in order to reach out to others: to be a blessing to the all the ethnos of the world. The glory manifested in those who so live, invisible in this life, will prove of immeasurable worth in the age to come:

\begin{quote}

Our light and transient troubles are achieving for us an everlasting glory whose weight is beyond description \footnote{2 Corinthians 3:17--4:17}
\end{quote}

Effectively, within the new covenant matrix, for those ``with an ear to hear,''\footnote{Revelation 2:7,11,17,29; 3:6,13,22; 13:9} a new subtle, form of honour is ascribed to those willing to ``take up their cross'' in following the Messiah along the narrow road of living an intercessory life on behalf of others:

\begin{quote}

By his knowing pain and sacrifice my righteous servant makes many righteous; it is for their sins that he suffers. Therefore, I will assign him a share with the great{\ldots}for having exposed himself to death{\ldots}interceding for the offenders.\footnote{Isaiah 53:10--12}
\end{quote}

Those willing to travel this road will share with the Messiah in his glory:

\begin{quote}

We are also heirs, heirs of God and joint-heirs with the Messiah?provided we are suffering with him in order to be glorified with him. I don't think the sufferings we are going through are even worth comparing with the glory that will be revealed to us in the future.\footnote{Romans 8:17--18}
\end{quote}

Thus we understand that suffering means sharing in the intercessory life and spirit of Jesus, the Messiah?this is true covenant faithfulness.

\subsection{The glory of resurrection with the Messiah}
\label{thegloryofresurrectionwiththemessiah}

The New Testament presents the Messiah's glorification-through-resurrection as the `first-fruits' of a new creation, which ultimately in the olam-haba, the age to come, will be consummated in a regeneration of creation that effectively marks a reuniting of God's glory with the whole of creation:

\begin{quote}

The creation waits eagerly for the sons of God to be revealed; for the creation was made subject to frustration{\ldots}but it was given a reliable hope that it too would be set free from its bondage to decay and would enjoy the freedom accompanying the glory that Gods' children will have.\footnote{Romans 8:19--23} 
\end{quote}

In the Regeneration, the Sh'khinah presence will dwell continuously amongst human beings, his glory covering and reigning over the earth, filling it with his shalom. 

\begin{quote}

We, following along with his promise, wait for new heavens and a new earth, in which righteousness will be at home{\ldots}\footnote{2 Peter 3:13, citing Isaiah 65:17 \& 66:22} I saw a new heaven and a new earth{\ldots}I heard a loud voice from the throne say, See! God's Sh'khinah is with mankind and he will live with them. They will be his peoples and he himself, God-with-them, will be their God.\footnote{Revelation 21:1--3?evoking Leviticus 26:11, Isaiah 7:14, 8:8; Jeremiah 31:34; Ezekiel 37:27}
\end{quote}

At that time, God's covenant people will be glorified?restored, through their resurrection from the dead, to the absolute fullness of life and humanity, as manifested by the Messiah. Not an ethereal escape to a celestial paradise dislocated from this earthly creation, but inhabiting a glorious spiritual body fitted for life within a renewed creation:

\begin{quote}

There are heavenly bodies and there are earthly bodies, but the beauty and glory of the heavenly bodies is of one kind, while the beauty and glory of earthly bodies is a different kind. So it is with the resurrection of the dead{\ldots}As surely as there is a physical body, there is also a spiritual body.\footnote{1 Corinthians 15:40, 44 (Amplified Version)}
\end{quote}

This spiritual body is not simply about being alive forever. Rather about liberation from the curse of corruption, death, disease, mortality. Receiving the fullness of life, of true humanity, united with and formed into the character, the image of the Messiah; having the capacity to fully live, on earth, sharing in the olam' haba?the world, the age to come?as God intends human beings to live: Created in unhindered fellowship with Creator?this is the eternal purpose of God.

\begin{discuss}[\currentsectiontitle]

- 

-

\end{discuss}

\section{God's eternal purpose}
\label{godseternalpurpose}

This topic examines

\subsection{Completing the big picture}
\label{completingthebigpicture}

In conclusion, the final elements can be included in the schematic biblical panorama constructed by this thesis: the incorporation of the messianic gentiles, the great commission and new covenant `ministry of reconciliation' and the final culmination of the Messiah's return and the resurrection of the dead in the renewed creation completing God's eternal purpose: to bring many people to his glory.

\subsubsection{Fig 6: God's eternal purpose}
\label{fig6:godseternalpurpose}

\begin{figure}[htbp]
\centering
\includegraphics[width=324pt,height=228pt]{EP7eternalpurpose.png}
\caption{God's eternal purpose}
\label{ep7eternalpurpose.png}
\end{figure}


\subsubsection{Getting on board}
\label{gettingonboard}

\begin{figure}[htbp]
\centering
\includegraphics[width=325pt,height=126pt]{EP8locomotive.png}
\caption{Getting on board?}
\label{ep8locomotive.png}
\end{figure}


\begin{discuss}[\currentsectiontitle]

- 

-

\end{discuss}

\ssection{The New, Messianic Covenant Community}

This concludes Study 4, \emph{The New Messianic Covenant Community}, which:

- 

In summary, the study revealed how:

\begin{summary}

\begin{center}\rule{3in}{0.4pt}\end{center}


\end{summary}

\vspace*{\fill}
\begin{bonus}

-

\end{bonus}

\input{mpd-footer}

\end{document}