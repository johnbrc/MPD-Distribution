\def\version{1.0 --- First Edition}
\def\change{}
\input{mpd-header}
\def\mytitle{Syllabus Handbook}
\def\module{fh}
\def\translation{}
\def\isbn{978-1-907191-02-2}
\def\colour{false}
\input{mpd-document-fh}
\chapter{The Facilitator Role}
\label{thefacilitatorrole}

THIS CHAPTER PROVIDES practical insights and suggestions for facilitating Maize Plant Discipleship learning groups.

\section{What is Maize Plant Discipleship?}
\label{whatismaizeplantdiscipleship}

Maize Plant Discipleship is a biblical learning resource, designed to be practical, relevant and accessible for use in African and other Majority World contexts. It has been derived and road-tested in collaboration with Africans and formulated in response to contextual doctoral research conducted in Burkina Faso, for publication as a syllabus of short, modular handbooks:

\begin{itemize}
\item equipping facilitators to guide small learning groups

\item incorporating reflective learning and group discussions

\item suitable for formal and informal modes of study

\item easily replicable, in terms of republication and training.

\end{itemize}

Its goal is to facilitate biblical learning that continuously moves outwards, drawing whole communities into patterns of scripturally-based discipleship, in living dialogue with contextual culture.

\section{Maize Plant Discipleship Learning Groups}
\label{maizeplantdiscipleshiplearninggroups}

Maize Plant Discipleship is intended to be an open, reflective, group learning process, in which leaders and learners alike participate together in discovering what the Spirit is saying, as Scripture is studied and related to contextual \emph{signs of the times.}

\subsection{Why learning groups?}
\label{whylearninggroups}

There are many reasons to bring together a group of people to learn together. Some people would point to Jesus' gathering of twelve disciples. For most people, groups represent a natural and lively place in which to learn. They bring together people with different experiences, gifts, capacities and perspectives. 

When we share our lives, we learn together and \emph{learning groups} mirror this reality. Together, we experience and learn quite differently to when we study alone. Reflective discussion with others, in particular, provides a highly stimulating forum for learning through exploration, listening and discovery. 

For further discussion about reflective learning, see \emph{The Learning Process} (\autoref{thelearningprocess}).

\subsection{Learning, not teaching}
\label{learningnotteaching}

Maize Plant Discipleship is principally a \emph{learning}, rather than a teaching resource. Learning depends on many factors, in addition to the presentation of topical information, most of which relate to the student, rather than the teacher and teaching. Such factors include desire, temperament, experience, talent, time, energy, environment and so forth. The decision to learn something new must therefore always begin with the student themselves. 

Accordingly, discipleship should be recognised as a type of learning that is \emph{called out} of students or disciples, under the guidance and direction of a facilitator, mentor, educator or teacher. People in those roles come alongside \emph{motivated learners} to assist, encourage, facilitate and call out the learning taking place within those being discipled.

\section{Facilitating Maize Plant Discipleship Learning Groups}
\label{facilitatingmaizeplantdiscipleshiplearninggroups}

A Maize Plant Discipleship Facilitator can help a group open up to the message of Scripture, to one another and, above all, to the leadership of the Holy Spirit. This section explores how.

\subsection{Facilitating openness}
\label{facilitatingopenness}

Fostering an environment of learning and discovery, where debate and discussion are lively, yet relaxed and uncompetitive, is essential. An ideal environment will allow strong and diverse views to be expressed, yet without creating either conflict or conformity, so that all present feel comfortable to contribute their views, questions and burdens. 

It can be especially challenging to foster openness in cultures where authority traditionally flows downwards and conformity is highly valued. Thus, facilitators should typically contribute to discussions as regular group members, never dominating or belittling the views of others.

\begin{itemize}
\item Allow discussion to ebb and flow, as people consider their responses and return to earlier discussions. Encourage others to contribute, especially quieter members, women, youth and elders.

\item If discussion becomes harsh or factious, quieten the group, then invite a member with a harmonious or gentle spirit to summarise (not resolve) the tension, then move on.

\end{itemize}

\subsection{Facilitating Spirit-led discipleship}
\label{facilitatingspirit-leddiscipleship}

Discipleship cannot be reduced to a mere replication of information, from teacher, or textbooks, into students. We may have been taught this way in school; Spirit-led discipleship is different. It \emph{tranforms}, as well as informs.

The intention is that through teaching, reflection and discussion, God's Spirit is able to speak to, lead, warn, direct, encourage, educate, challenge and exhort us, personally and corporately. Since each person differs in gift, personality and development, at any time, each person may be learning something different from the Spirit. 

\begin {pause}\begin{description}

\item[The goal of discipleship]

is not to establish shared dogmatic belief or conformity to the convictions of a leader, mentor or facilitator, nor to church traditions, certainly not to every aspect of Maize Plant Discipleship.

\item[The goal of discipleship]

is conformity to the Spirit of the Messiah and obedience to the will of the Father.
\end{description}

\end {pause}

\subsection{Who can facilitate?}
\label{whocanfacilitate}

A facilitator needs to be someone who senses a calling to help others become faithful Christian disciples. This must motivate them to be humble, patient, warm, flexible, open and secure enough to allow others to explore personal boundaries of vocational understanding, experience and creativity, at their own pace.

A facilitator does not hold a position of authority over people. They simply facilitate the gathering of people into groups, for learning and discussion. Accordingly, a facilitator:

\begin{itemize}
\item may be a lay-leader;

\item may be relatively young;

\item may be a woman;

\item need not have gone to bible college;

\item need not be an established church leader;

\item need not be an experienced mentor.

\end{itemize}

Of course, Maize Plant Discipleship can be facilitated by established leaders, mentors or disciplers—providing they are willing and comfortable to facilitate group discussions that are genuinely \emph{reflective and exploratory}.

\section{Practicalities}
\label{practicalities}

Facilitating a discipleship group will be most effective when the following practicalities are considered in advance and appropriate planning takes place. 

\subsection{Group size}
\label{groupsize}

Maize Plant Discipleship is ideal for learning groups of 8--10 people. This is small enough to allow group members to grow together with a degree of intimacy and large enough for group members to explore discipleship commitments at their own pace.

\begin {pause}\begin{description}

\item[More than ten people?]

Think about helping others to facilitate additional learning groups. What problems might you face?
\end{description}

\end {pause}

\subsection{Involving others}
\label{involvingothers}

Although a facilitator is responsible for convening gatherings, they may delegate responsibility for hosting, presenting the teaching, or moderating group discussions. Ideally, over time, all group members will carry some responsibility, according to their gift and capability. This avoids one person carrying too much and gives everyone some experience of the responsibilities of the facilitating role.

The person responsible for presenting a topical study should read through it carefully, in advance, reflecting upon the teaching and its lessons. If anything is unfamiliar or unclear, invite discussion about that area of the study, encouraging group members to bring forward their perspectives.

\begin {pause}\begin{description}

\item[Replication is an important facet of discipleship]

— \emph{see The Philosophy (\autoref{thephilosophy})} — it may not be necessary for a group member to complete the entire syllabus before branching out to facilitate another learning group; be led by the Spirit.
\end{description}

\end {pause}

\subsection{Location and timetable}
\label{locationandtimetable}

Meeting together can take place in any appropriate location that can comfortably accommodate a learning group. For example, a large room in someone's home, or a communal building, such as a church.

\begin {pause}\begin{description}

\item[Experiment and learn what works for your group]

— consider using different locations, maybe even meeting outside sometimes. Arrange seating to create intimate, practical spaces for discussion, so that everyone can see and hear each other.

\item[The syllabus incorporates about 64 studies]

— consider the group's availability when planning a timetable: are members affected by agricultural seasons, or academic terms?
\end{description}

\end {pause}

\subsection{Adaption}
\label{adaption}

Be prepared to adapt the teaching and the method of presentation, in order to create a helpful and culturally-appropriate learning environment. Take account of the capabilities of each particular discipleship group. 

\begin {pause}\begin{description}

\item[Ensure literature is a helpful servant]

— not a hard task-master, especially to oral learners. As far as possible, keep things simple and lighthearted.

\item[Invite creative people to contribute]

— by interpreting or celebrating topical subjects in song, art or drama.
\end{description}

\end {pause}

\subsection{Other learning forums}
\label{otherlearningforums}

Studying the treasury of Scripture references contained in the footnotes and using discussion questions as a basis for written answers or even short essays enables Maize Plant Discipleship handbooks to facilitate other forms of guided learning:

\begin{itemize}
\item \emph{Theological education} — students may gain highly valuable experience for their own vocational contexts by forming reflective learning groups within or alongside classroom contexts. 

\item \emph{Personal study} — incorporate reflective learning by submitting to the oversight of a mentor, or sharing with a fellow student, for reflection, discussion and critical consideration.

\item \emph{Congregational teaching} — after hearing a study, congregants could divide into discussion groups — or learning groups could meet on a separate occasion, to discuss the study and pray together.

\end{itemize}

\vspace*{\fill}
\begin {questions}

\begin{itemize}
\item How do Maize Plant Discipleship learning groups differ from other learning forums, such as schools?

\item What are the essential characteristics of a good facilitator?

\end{itemize}

\end {questions}

\chapter{The Learning Process}
\label{thelearningprocess}

THIS CHAPTER PRESCRIBES the reflective learning process intended to underlie presentation of the Maize Plant Discipleship syllabus.

\section{Reflective Learning}
\label{reflectivelearning}

Maize Plant Discipleship modules are structured to provide a \emph{reflective learning process}. Reflective learning minimally incorporates these components:

\begin{itemize}
\item \textbf{Hear} — about the experiences and perspectives of others.

\item \textbf{Reflect} — upon ideas and concepts, old and new.

\item \textbf{Discuss} — sharpen understanding, in dialogue with others.

\item \textbf{Act} — integrate learning into vocational practices.

\end{itemize}

In this way, practice is allowed to inform theory and theory to inform practice. As this learning process is repeated, it forms a cycle of hearing, reflection, discussion, action{\ldots} hearing, reflection, discussion, action and so on, which can be illustrated figuratively:

\begin{figure}[htbp]
\centering
\includegraphics[width=229pt,height=230pt]{reflective-learning.png}
\label{reflective-learning.png}
\end{figure}



\subsection{Adapting the learning process}
\label{adaptingthelearningprocess}

The \emph{Maize Plant Discipleship Learning Process} draws on this basic reflective learning cycle and combines it with a number of practical and spiritual principles, inspired by the experience of the early messianic community,\footnote{Acts 2:42} arriving at at learning process consisting of these six elements:

\begin{enumerate}
\item \textbf{HEAR} {\ldots} what the Spirit is saying

\item \textbf{RETAIN} {\ldots} God's message inwardly

\item \textbf{OPEN} {\ldots} hearts to others

\item \textbf{SHARE} {\ldots} broken bread together

\item \textbf{PRAY} {\ldots} for God's kingdom to come

\item \textbf{ACT} {\ldots} in the light of God's message

\end{enumerate}

\pagebreak 

\section{1. HEAR what the Spirit is saying}
\label{hearwhatthespiritissaying}

\begin{figure}[htbp]
\centering
\includegraphics[width=108pt,height=108pt]{HEAR.png}
\label{hear.png}
\end{figure}


When we gather together as disciples of the Messiah, to hear biblical teaching, we are opening ourselves not simply to human ideas or wisdom, but to spiritual words and truths, taught by the Spirit of God.

\begin{quote}

Now we have not received the spirit that belongs to the world, but the Holy Spirit Who is from God, given to us that we might realise and comprehend and appreciate the gifts of divine favour and blessing so freely and lavishly bestowed on us by God. 

And we're setting these truths forth in words not taught by human wisdom but taught by the Holy Spirit, combining and interpreting spiritual truths with spiritual language to those who possess the Holy Spirit — \emph{1 Corinthians 2:12--13 AMP}
\end{quote}

\begin {pause}\begin{description}

\item[We listen in order to live more faithfully]

— this type of listening is called \emph{heeding}: listening with the intention to learn and obey, or follow.

\item[We listen with our mind, but also with our heart]

— in order to \emph{hear what the Spirit is saying to his people} (Revelation 2:29, 3:6,13,22; Matthew 11:15, Mark 4:9 etc)—never in order to become \emph{puffed up} by knowledge.
\end{description}

\end {pause}

\pagebreak 

\section{2. RETAIN God's message inwardly}
\label{retaingodsmessageinwardly}

\begin{figure}[htbp]
\centering
\includegraphics[width=108pt,height=108pt]{RETAIN.png}
\label{retain.png}
\end{figure}


It is not enough to only hear God's message: we must learn to \emph{retain} his word inwardly, where it can begin to \emph{dwell richly within us}—Colossians 3:16

\begin{quote}

The one who received the seed that fell on rocky places is the man who hears the word and at once receives it with joy. But since he has no root, he lasts only a short time{\ldots} The one who received the seed that fell among the thorns is the man who hears the word, but the worries of this life and the deceitfulness of wealth choke it, making it unfruitful. 

The seed on good soil stands for those with a noble and good heart, who hear the word and \emph{retain} it, and by persevering produce a crop {\ldots} yielding a hundred, sixty or thirty times what was sown — \emph{Matthew 13:18--23; Luke 8:15}
\end{quote}

\begin{pause}\begin{description}

\item[Think about how we receive and retain food]

— chewing it, enjoying the taste, swallowing, digesting, inwardly retaining its vitality and goodness.

\item[It's the same with God's word]

— we must \emph{chew it over,} meditating and reflecting upon its meaning and application to our lives, both as individuals and as communities, allowing it to settle in our spirit, where it can form and shape our convictions and renew our hope.
\end{description}

\end{pause}

\pagebreak 

\section{3. OPEN hearts to others}
\label{openheartstoothers}

\begin{figure}[htbp]
\centering
\includegraphics[width=108pt,height=108pt]{OPEN.png}
\label{open.png}
\end{figure}


Discussion and debate is an opportunity to open our hearts to the perspectives and experiences of those around us and those who see things differently to ourselves. 

\begin{itemize}
\item This requires listening with the heart, as well as the head, in order to appreciate what others are sharing, rather than to win an argument!

\item Discussion of practical, vocational applications of topical study is vital; think about how Maize Plant Discipleship teachings relate to the cultural contexts amongst which group members live.

\item Allow plenty of time for this aspect of Maize Plant Discipleship learning!

\end{itemize}

\begin{pause}\begin{description}

\item[Vocation is more than simply our job]

— it incorporates all the responsibilities towards which God calls us, including families, workplaces and practices, communities and networks.

\item[The ``sweet talk'' of proverbs]

may provide fresh insight and be helpful in discussing discipleship and biblical teaching with others, such as elders or non-believers.
\end{description}

\end{pause}

\pagebreak 

\section{4. SHARE broken bread together}
\label{sharebrokenbreadtogether}

\begin{figure}[htbp]
\centering
\includegraphics[width=108pt,height=108pt]{SHARE.png}
\label{share.png}
\end{figure}


The celebratory breaking of bread, in order to remember the sacrificial obedience of Jesus, is a significant symbol of the New Covenant and a profound way for discipleship groups to proclaim their shared devotion to the Messiah.

\begin{itemize}
\item In modern forms of Christianity, breaking and sharing bread is typically ceremonial \emph{(Eucharist, Holy Communion, Mass)}. Yet, the earliest messianic communities based it simply upon the Passover meal, like the one Jesus shared with his disciples, immediately prior to his arrest by the Jerusalem authorities.

\item Sharing food together is therefore both a vital part of human fellowship \emph{and} a practical way of celebrating and proclaiming God's covenantal provision and blessing.

\end{itemize}

\begin {pause}\begin{description}

\item[Consider incorporating a simple meal]

— perhaps once a month, into times of meeting together and prayerfully identifying it as a form of breaking bread.

\item[If a meal is not a practical option]

— consider sharing a small amount of bread together, as a symbolic act of hospitality and shared commitment to membership of the Messiah's body.
\end{description}

\end {pause}

\pagebreak 

\section{5. PRAY for God's will to be done}
\label{prayforgodswilltobedone}

\begin{figure}[htbp]
\centering
\includegraphics[width=108pt,height=108pt]{PRAY.png}
\label{pray.png}
\end{figure}


After a period of discussion, invite the group to pray together, including intercession on behalf of neighbours, networks and communities, local and national rulers and governors. 

\begin{itemize}
\item Allow the teaching to infuse prayer with fresh confidence concerning God's will and purpose, including personal and vocational concerns and challenges faced by group members. 

\item Allow the Holy Spirit to lead you in speaking blessings, rooted in Scripture, over one another's lives and over your community or nation, or with regards to a specific problem.

\item Expect the power of God to overcome all opposition, through the blessing of his overcoming life at work in and through his people!

\end{itemize}

\begin {pause}\begin{description}

\item[Pray for the gospel]

— to deeply impact and transform individuals, communities, cultures and societies throughout your nation, Africa, Europe, Asia and the Americas; pray for unreached people groups.

\item[Pray for the Maize Plant Discipleship project]

— that it will be used by God to edify, strengthen and bless the Messianic Community, within Africa {\ldots} and beyond!
\end{description}

\end {pause}

\pagebreak 

\section{6. ACT in the light of God's message}
\label{actinthelightofgodsmessage}

\begin{figure}[htbp]
\centering
\includegraphics[width=108pt,height=108pt]{ACT.png}
\label{act.png}
\end{figure}


The purpose of our gathering to hear God's message is not simply to hear it, but to act upon it. As the epistle of \emph{James} explains, we deceive ourselves when we listen to God's word, yet do not do what it says:

\begin{quote}

Don't deceive yourselves by only hearing what the Word says, but do it! For whoever hears the Word but doesn't do what it says is like someone who looks at his face in a mirror, who looks as himself, goes away and immediately forgets what he looks like. 

But if a person looks closely into the perfect \emph{Torah (Teaching)}, which gives freedom, and continues, becoming not a forgetful hearer but a doer of the work it requires, then he will be blessed in what he does — \emph{James 1:22--25}
\end{quote}

\begin {pause}\begin{description}

\item[The goal of discipleship]

— is to be transformed ourselves and thus a transformative influence, in our homes and amongst our workplaces and social networks.

\item[As we are transformed]

— as part of a growing, dynamic movement of disciples — we begin to fulfil our corporate vocation: \emph{to be a Messianic Community blessed to be a blessing to the families of the earth!}
\end{description}

\end {pause}

\pagebreak 

\section{Maize Plant Discipleship learning cycle}
\label{maizeplantdiscipleshiplearningcycle}

Combining together the six elements of the Maize Plant Discipleship learning process produces the Maize Plant Discipleship learning cycle.

\begin{figure}[htbp]
\centering
\includegraphics[width=256pt,height=284pt]{mpd-learning-cycle.png}
\label{mpd-learning-cycle.png}
\end{figure}



\begin {pause}\begin{description}

\item[A learning cycle is a tool]

— its purpose is to facilitate, rather than control. It may not be appropriate to incorporate every element, each time a group meets. \emph{Allow the cycle to stretch, but not to limit; where appropriate, adapt it}. For example, groups might cover multiple topics, repeating the first three elements of the cycle, before moving on to the final three.
\end{description}

\end {pause}

\vspace*{\fill}
\begin {questions}

\begin{itemize}
\item How does the Maize Plant Discipleship learning process differ from secular reflective learning? Why is this significant?

\item Consider each element of the Maize Plant Discipleship learning process. Which are most vital?

\item In practice, which elements might most easily be left out—intentionally or otherwise? Would this matter? Why?

\end{itemize}

\end {questions}

\chapter{The Syllabus}
\label{thesyllabus}

THIS CHAPTER INTRODUCES the metaphor of the maize plant and the sixteen modules of the Maize Plant Discipleship Syllabus.

\section{The Maize Plant Metaphor}
\label{themaizeplantmetaphor}

Jesus refers to his own mission using the metaphor of a seed that enters the ground and dies, in order to produce a large harvest.\footnote{John 12:24} 

\begin{quote}

I tell you that unless a grain of wheat that falls to the ground dies, it stays just a grain; but if it dies, it produces a big harvest 
\end{quote}

Maize is grown throughout sub-Saharan Africa and provides a similar, highly recognisable metaphor relating to the essential dynamics of life, death, sustenance and growth. Accordingly, the Maize Plant Discipleship Syllabus is symbolically structured to reflect the following fundamental metaphor:

\pagebreak 

\begin{summary}

\textbf{Good seed, sown in good soil, stimulated by sunshine and refreshed by rainfall produces dynamic growth, resulting in a good harvest.}

\end{summary}

\begin{figure}[htbp]
\centering
\includegraphics[width=190pt,height=270pt]{mpd-metaphor.png}
\label{mpd-metaphor.png}
\end{figure}



Like maize plants, messianic communities need to be rooted in good ground where they can draw on life-giving spiritual nutrients, be refreshed by the living water of the Spirit and be stimulated by revelatory light. Strong roots will anchor messianic communities against the destructive winds of false teaching and sustain them amidst the withering heat of trials, temptations and vocational responsibility. Based on this metaphor, the Maize Plant Discipleship Syllabus is divided into three parts:

\begin{enumerate}
\item Soil and roots (Module 1)

\item Dynamics of growth (Modules 2--8)

\item Sunlight and rainfall (Modules 9--15)

\end{enumerate}

\subsection{Soil and roots}
\label{soilandroots}

Metaphorically, the roots of the maize plant represent the biblical community of Israel. The soil in which the roots grow equates to the historical, cultural and geo-political contexts of Israel's covenant vocation (such as Egypt, Canaan, Babylon and the Roman occupation).

\subsubsection{Eternal Purpose -- module 1}
\label{eternalpurpose--module1}

\begin{itemize}
\item Module 1 forms the foundation of the syllabus, providing a panoramic overview of Scripture that reveals God's unchanging, eternal purpose. In this perspective, the Messiah is the Seed that grows out of the soil of the biblical, covenantal history of the people of Israel. A Seed that then dies, in order to produce a rich harvest of people—a Messianic Covenant Community—from amongst all the peoples of the earth.

\end{itemize}

\begin{figure}[htbp]
\centering
\includegraphics[width=180pt,height=124pt]{mp-roots.png}
\label{mp-roots.png}
\end{figure}



\begin {pause}\begin{description}

\item[Maize produce is effectively a type of daily bread]

for millions of African people; the Messianic Community is called, like the Messiah, to become a kind of \emph{life-giving bread} to the peoples of the world—see John 6 and 20:21
\end{description}

\end {pause}

\pagebreak 

\subsection{Dynamics of growth}
\label{dynamicsofgrowth}

Modules 2 to 8 explore seven messianic dynamics that represent the characteristic development, growth, structure, shape and fruit of the Messianic Community.\footnote{\emph{Dynamic} derives from a Greek word, \emph{dunamis}, meaning power and refers to forces stimulating change within a process or system, such as a plant or a body.}

\begin{figure}[htbp]
\centering
\includegraphics[width=255pt,height=373pt]{mp-dynamics.png}
\label{mp-dynamics.png}
\end{figure}



\subsubsection{Nations -- module 2}
\label{nations--module2}

\begin{itemize}
\item The historical development of the Messianic Community's vocational mission to bless the peoples of the world.

\end{itemize}

\subsubsection{Jews -- module 3}
\label{jews--module3}

\begin{itemize}
\item The special, biblical responsibility of the Messianic Community towards the Jewish people.

\end{itemize}

\subsubsection{Commissioning -- module 4}
\label{commissioning--module4}

\begin{itemize}
\item Strategic and structural dynamics of messianic commissioning and community growth.

\end{itemize}

\subsubsection{Membership -- module 5}
\label{membership--module5}

\begin{itemize}
\item Dynamics of commitment and spiritual maturity within the body of the Messiah.

\end{itemize}

\subsubsection{Revival -- module 6}
\label{revival--module6}

\begin{itemize}
\item The dynamics of revival leading to a spiritual harvest as we walk in practical, covenantal faithfulness towards God's revelation, wisdom and direction.

\end{itemize}

\subsubsection{Intercession -- module 7}
\label{intercession--module7}

\begin{itemize}
\item The priestly vocation of the Messianic Community: to be \emph{a house of prayer for all nations}.

\end{itemize}

\subsubsection{Transformation -- module 8}
\label{transformation--module8}

\begin{itemize}
\item The call to work amongst and on behalf of the nations, towards cultural transformations that signal the presence of God's kingdom.

\end{itemize}

\pagebreak 

\subsection{Sunlight and rainfall}
\label{sunlightandrainfall}

Modules 9 to 15 examine seven characteristic \emph{disciplines} that enable messianic communities to receive the revelatory \emph{light} and sustaining \emph{living water} of God's Spirit, without which we become spiritually weak and incapable of producing good fruit and a plentiful harvest.

\begin{figure}[htbp]
\centering
\includegraphics[width=280pt,height=398pt]{mp-disciplines.png}
\label{mp-disciplines.png}
\end{figure}



\subsubsection{Maturity -- module 9}
\label{maturity--module9}

\begin{itemize}
\item Three stages of encounter and growth in spiritual maturity of messianic disciples and communities.

\end{itemize}

\subsubsection{Service -- module 10}
\label{service--module10}

\begin{itemize}
\item Motivations, qualities and disciplines for living an enduring life of service.

\end{itemize}

\subsubsection{Vocation -- module 11}
\label{vocation--module11}

\begin{itemize}
\item Identifying and excelling in our personal vocation, through a deepening of our relationship with the Messiah.

\end{itemize}

\subsubsection{Economics -- module 12}
\label{economics--module12}

\begin{itemize}
\item A biblical perspective upon economic faithfulness, wealth and poverty—radically different to that of the world.

\end{itemize}

\subsubsection{Leadership -- module 13}
\label{leadership--module13}

\begin{itemize}
\item Qualifications, motivations and characteristics of faithful messianic leadership.

\end{itemize}

\subsubsection{Faith -- module 14}
\label{faith--module14}

\begin{itemize}
\item Seeing with eyes of faith enables us to endure times of testing and purification and to embrace challenge as an opportunity for experiencing God's faithfulness.

\end{itemize}

\subsubsection{Overcoming -- module 15}
\label{overcoming--module15}

\begin{itemize}
\item Confronting idolatrous, cultural strongholds in the power of the Spirit and discerning strategies that make room for a transformative encounter with God's overcoming power.

\end{itemize}

\pagebreak 

\section{Module Handbooks}
\label{modulehandbooks}

Each handbook contains one module of the syllabus, consisting of four interrelated \emph{studies}, each incorporating:

\begin{itemize}
\item Scripture readings

\item several topical sections

\item illustrations and summaries

\item topical discussion questions

\item occasional bonus questions.

\end{itemize}

\begin {pause}\begin{description}

\item[Illustrations]

— ensure everyone present is able to appreciate topical illustrations by sharing or passing handbooks around. Or reproduce illustrations, using whatever medium is available—\emph{if you have an artist in the group, give them responsibility for this task}.

\item[Mother-tongue is the natural language of the heart]

and its use may be encouraged and employed as appropriate, particularly during group discussions and especially for Scripture reading and memorisation.
\end{description}

\end {pause}

\section{Significant Terms}
\label{significantterms}

Important definitions provided below explain how these significant terms are used in the the Maize Plant Discipleship Syllabus.\begin{description}

\item[messiah]

— a mediator or saviour, acting with God's authority to deliver a people from the grip of their enemies or to govern over and keep them safe (experiencing \emph{shalom}). In the biblical history of Israel, deliverance came through a variety of mediators, such as \emph{prophets, priests, judges} and \emph{kings}. 

The root meaning of messiah is \emph{anointed} or \emph{poured on}, referring to the anointing oil poured onto Israel's kings and priests. Anointing oil symbolises the pouring out, or placing of God's Spirit upon a leader, as they were invested with their authority, usually by prophets.\footnote{E.g. Exodus 30:22--25} Thus the idea of God's priestly, kingly and prophetic authority is inherent in the concept of \emph{messiah}. 

The New Testament identifies Jesus as the Jewish Messiah,\footnote{\emph{Christ} is the Greek translation of \emph{Moshiach} (Hebrew, Messiah); Jesus Christ is the Greek rendering of \emph{Yeshua Moshiach}} spoken of by the prophets and anointed by the Spirit to fulfil three mediatory roles—prophet, priest and king—on behalf of God's people.\footnote{Matthew 3:13--17; Mark 1:9--11; Luke 3:21--22 \& 4:16--19; John 1:32--34} After his ascension to the \emph{Right Hand of God}, Jesus becomes the \emph{one mediator between God and humanity}.\footnote{Hebrews 8:1--2; 1 Timothy 2:5}

\item[messianic]

— relating to the Messiah; primarily used in the syllabus to refer to \emph{Messianic Community} and \emph{messianic communities}.

\item[Messianic Community]

(capitalised) — the whole, worldwide and historical body of people belonging to the Messiah. In the New Testament this community is referred to as \emph{the body of Messiah}. The reference is broadly equivalent to \emph{Worldwide Christian Community} or \emph{Church}. Those terms, however, are generally avoided because of their association with particular historical expressions of Christianity that are not inclusive.

\item[messianic communities]

(un-capitalised) — localised congregations of the Messianic Community. The term is used in preference to \emph{churches} in order to emphasise the biblical link to the whole body of the Messiah, the Messianic Community.

\item[vocation]

— a calling, life's work, mission, purpose, function; profession, occupation, career, job, employment, trade, craft, business, line of work, métier.

In the Maize Plant Discipleship syllabus, \emph{vocation} and \emph{vocational} refers to both personal and communal calling, both of which flow from the prophetic and scriptural call to serve God's eternal purpose, in union with the Messiah. Vocation thus provides an umbrella term that incorporates and dignifies all forms of work and ministry. It looks beyond traditional divides of laity and clergy, male and female, by pointing towards the reality that all followers of the Messiah are called \emph{to faithfully serve God's purposes, within homes, workplaces and communities}. 
\end{description}

\vspace*{\fill}
\begin {questions}

\begin{itemize}
\item What is the significance of the metaphor of seed to the ministry of Jesus?

\item How would you describe a \emph{dynamic} of growth?

\item Consider the various ways that maize is important in your context. What could this suggest about the importance of the messianic community to a social or cultural context?

\end{itemize}

\end {questions}

\chapter{The Philosophy}
\label{thephilosophy}

THIS CHAPTER EXPLORES the foundational biblical perspectives of messianic discipleship, upon which Maize Plant Discipleship is established.

\section{What is Messianic Discipleship?}
\label{whatismessianicdiscipleship}

Maize Plant Discipleship approaches messianic discipleship as a dynamic, generational process, empowered by the Holy Spirit, through which the life of the Messiah is reproduced within his people. Two crucial statements made by the apostle, Paul, in his second letter to his disciple, Timothy, reveal the essence of this process:\footnote{Statements separated only by Paul's expression of strong disappointment, directed at two disciples who failed to stand with him at a crucial time—in sharp contrasts with the loyalty of Onesiphorus.}

\begin{quote}

Keep safe the great treasure that has been entrusted to you, with the help of the Holy Spirit, who lives in us {\ldots} and the things you heard from me, which were supported by many witnesses, these commit to faithful people, such as will be competent to teach others---\emph{2 Timothy 1.14, 2.2}
\end{quote}

Taken together, these two Scriptures establish three key components of messianic discipleship:

\begin{enumerate}
\item \textbf{The great treasure of knowing the Messiah, Jesus}

The real, personal, experiential knowledge of the Messiah is more than human knowledge or philosophy: it is \emph{a great treasure}, a divine relationship, mediated by the Holy Spirit.

\item \textbf{The vitality of the Holy Spirit}

The Holy Spirit provides an intimate source of divine help to messianic disciples, mediating and helping to safeguard the reality of the Gospel and the presence of the Messiah amongst his people.\footnote{John 16:7--15}

\item \textbf{The necessity of generational formation}

Having received through Paul an impartation of the reality of the Messiah, Timothy is called upon by his mentor to safeguard this treasure by committing it to the stewardship and safekeeping of other faithful people. \emph{This is generational discipleship in action}.

\end{enumerate}

\begin{figure}[htbp]
\centering
\includegraphics[width=267pt,height=41pt]{generational.png}
\caption{Generational discipleship}
\label{generational.png}
\end{figure}



\begin{quote}

\textbf{Generational discipleship is how treasure is kept safe in the kingdom of God}. The reality of this principle is illustrated by the metaphor of seed and the harvests reaped from seed sown into good soil.
\end{quote}

\subsection{Seed and harvest}
\label{seedandharvest}

In agricultural contexts, seeds are a form of wealth. They are a type of treasure. Yet seed is generally stored for only a short time before being used. Whatever is not required for food, \emph{for daily bread}, must soon be sown to produce another harvest.\footnote{2 Corinthians 9:6--12}

In a similar way, God supplies spiritual life to us. This is what Paul refers to as the treasure of knowing the Messiah. Being alive to God and experiencing the grace of the Messiah and the love, joy, peace, patience, kindness, goodness, faithfulness, gentleness and self-control of the Holy Spirit,\footnote{Galatians 5:22} is the spiritual equivalent of receiving daily bread.

This personal aspect of knowing the Messiah, however, is not the whole purpose of our relationship with him. In fact, the Messianic New Covenant Community (the whole body of the Messiah's people) has been called to know God \emph{in order to become his servant community}. This means that we are called to give ourselves, our lives, to serve his purposes. This requires sacrifice and discipline—which is what it means to be a disciple.

\begin{quote}

\textbf{The sacrificial, disciplined giving of ourselves, in service to God, is the equivalent of taking precious seed that could be consumed by ourselves and instead sowing it within the ground, in order to produce a harvest.}
\end{quote}

\subsection{Sharing treasure}
\label{sharingtreasure}

Discipline and sacrifice are amongst the most significant secrets to living a truly messianic life. Sadly, they are secrets that many people never properly discover, let alone realise as life-giving principles. Yet the illustration provided by seeds demonstrates that hoarding the treasure of our knowledge, relationship and communion with God is not the way to a rich harvest. Only in sacrificially sharing our spiritual treasure, both within and beyond our own communities, do we discover and realise our vocation and, in due time, reap a harvest of faithfulness.\footnote{Matthew 10:38--39, 13:23; Galatians 6:6--10; Hebrews 12:11; James 3:18}

Nevertheless, spiritual treasure should never be wasted or cast away carelessly. Whilst some seed inevitably falls upon unreceptive ground, a farmer never intentionally wastes seed. Likewise, spiritual treasure is too precious to be deliberately squandered on people who spurn its value.\footnote{Matthew 7:6} Investment must be made with people who recognise the worth of this treasure and who make room for its transformative power. 

People who treasure the seed that is the Word of God are those whom Jesus, in the parable of the sower, refers to as \emph{good soil}:\footnote{Matthew 13:1--23} 

\begin{quote}

\textbf{Faithful disciples, transformed through a personal knowledge and experience of the Messiah, sharing their treasure with other faithful people, who share it with other faithful people and so on and so on, each person's faithfulness contributing momentum to a missional movement of messianic discipleship}.
\end{quote}

\section{Discipleship Movements}
\label{discipleshipmovements}

The formation of faithful disciples was at the heart of Jesus' life and work. Today the entire historical and now worldwide Messianic movement testifies to the significance of that small, core group of disciples, formed around Jesus.

From its first-century origins, as an obscure, tiny, Jewish sect, the Messianic Community has grown and developed across two millennia. Today it is an international, intercultural, multi-ethnic community that exists, in some form or another, in practically every nation of the world. The message it has proclaimed has impacted innumerable peoples, stories and cultures. 

\begin{quote}

\textbf{The historical dynamism of the Messianic Community can be traced directly to its functioning faithfully as a generational discipling movement, spreading out across social, ethnic, linguistic, geographical and cultural boundaries.}
\end{quote}

\begin{pause}\begin{description}

\item[The book of Acts provides a powerful illustration]

of the dynamic growth of the earliest messianic discipling movement, as it spread across the ancient world. 

From its beginnings in Jerusalem, it expands rapidly throughout Israel, into Asia Minor, across Greece and finally to Rome, the seat of imperial power.

\emph{Witness how this growth took place, by examining the context of these verses, in your own Bibles:}
\end{description}

\begin{itemize}
\item Acts 2:46--47, 6:7, 9:31, 12:24, 16:5, 19:20, 28:30--31

\end{itemize}

\end{pause}

\subsection{Generational momentum}
\label{generationalmomentum}

History, however, also teaches us that momentum, once gained, does not continue inevitably. Many messianic movements began well, yet are sadly now only historical footnotes, obliterated by fatal levels of persecution. Others continue institutionally, yet without any sense of spiritual renewal, generational momentum or confidence in challenging cultural and social idols.

In such cases, the church has ceased to be a \emph{movement}. Invariably, it has ceased making disciples, lacking the confidence to persuade either itself or others to forsake human idols and to become wholehearted devotees of the Messiah. Empty traditions have taken the place of a living, reproducing body of people. Even if it remains dogmatic about its creeds, it no longer exhibits authentic, biblical zeal for the call to serve God's purpose amongst the nations. 

\begin{quote}

\textbf{Maintaining the momentum of generational discipleship requires a persistent, critical engagement with society and culture, combined with a focus upon personal, social and cultural transformation, as people turn from idols, towards the living God.}
\end{quote}

\subsection{Establishing momentum}
\label{establishingmomentum}

Any messianic community seeking spiritual momentum needs to place visionary, generational discipleship at the heart of its spirituality and practical activity.

Disciples must be invited, formed and sent forth as part of a \emph{world-facing} movement. This requires much more than the maintenance of congregational activity. It requires more than learning to care spiritually for one another, within messianic communities. What it requires is nothing less than the practical, spiritual and visionary preparation of disciples, equipped to serve God's eternal purpose, amongst a world of confused, oppressed, pained, fear-filled, idol-bound populations.

\begin{quote}

\textbf{Establishing a transformational movement requires a vision capable of gripping people's hearts and forming them into faithful, persistent and determined disciples, committed to significant personal and social renewal, living and working interdependently and active in forming other, faithful disciples.}
\end{quote}

\section{Anointed Community}
\label{anointedcommunity}

To make possible the realisation of such a high calling, messianic discipleship provides a unique ingredient that no other philosophy, ideology or faith can provide: the dynamic of the indwelling Spirit of the Messiah. 

\begin{itemize}
\item Through the Spirit, the covenant community is transformed into a \emph{charismatic community}—a group of people endowed with spiritual gifts profoundly shaped to liberate human beings from idolatry and the allegiances and falsehoods that compete against the knowledge of God.\footnote{2 Corinthians 10:3--5}

\item This Charismatic Community is a body of people brought under God's authority, through baptism into the Messiah, and anointed with the fragrant presence of the Holy Spirit, through the baptism of the Spirit. 

\item It is a body learning to walk in the footsteps of Jesus: learning to exercise its God-appointed, mediatory, intercessory, messianic role, under the direction of the Spirit of God.\footnote{Romans 6:3--4; Galatians 3:26--29; Hebrews 6:4}

\end{itemize}

\begin{quote}

\textbf{This community of disciples
is a messianic, charismatic people
called into covenant relationship with the Father,
through being united with the Son,
sent into the world, to bless the nations,
in the power of the Spirit!}
\end{quote}

\begin{pause}\begin{description}

\item[Charismatic]

— from the Greek, \emph{charism}, meaning \emph{gift}; the \emph{charisma} of the Messianic Community derives from its anointing with the Spirit of the Messiah.

\item[Messianic]

— essentially means \emph{anointed to bring deliverance}; the Messianic Community is anointed with the Spirit to mediate on behalf of God's people and the nations, bringing deliverance from evil and new life in the Messiah.
\end{description}

\end{pause}

\subsection{Dying to live}
\label{dyingtolive}

God's intention is that this messianic, charismatic, covenant community co-works in partnership with him—using the strength, the power, the spiritual life, the anointing that he provides. 

Yet too often, the power of the anointed-life-of-Christ-within seems to elude us. It seems out of our reach. Beyond our grasp. Indeed, it is not something that can be grasped. 

Rather, the pathway to life is through dying. Yielding ourselves to God the Father, through unity with the Messiah, by the power of the Spirit. That is the message of the cross. As we \emph{die to self}, we become \emph{alive to God}.\footnote{Romans 6:4--13}

\subsection{The heart of discipleship}
\label{theheartofdiscipleship}

Thus we end, as we began, with the foundational principle of transformative discipleship: seed sown into the ground, in order to produce a rich harvest.

\begin{quote}

I tell you that unless a grain of wheat that falls to the ground dies, it stays just a grain; but if it dies, it produces a big harvest.\footnote{John 12:24}
\end{quote}

This life-giving spiritual reality represents the heart of Jesus' own life, mission, ministry and pain-filled death. And this same principle forms the foundation and wellspring of Maize Plant Discipleship:

\begin{quote}

\textbf{As we embrace a practical form of discipleship, incorporating a daily dying-to-self, we learn how to become truly alive-to-God and authentically equipped to serve his eternal purpose}
\end{quote}

\emph{That is the heart of Maize Plant Discipleship}.


\vspace*{\fill}
\pagebreak
\begin {questions}


\begin{itemize}
\item How is discipleship valued in your context?

\item How faithfully is it practised?

\item If there is a gap between what is believed and valued and what is practised, why do you think that is?

\end{itemize}

\end {questions}

\chapter{The Africa Factor}
\label{theafricafactor}

THIS CHAPTER EXPLAINS the development and potential of Maize Plant Discipleship as a resource for use in African contexts.

\section{African contexts}
\label{africancontexts}

Tite Tienou, a Malian who grew up in Burkina Faso, encapsulates a special and significant aspect of the philosophy that has guided the development of Maize Plant Discipleship: African voices should determine the theology that is of practical relevance to African contexts. Tienou states,\footnote{The Uphill Road: Indigenous African Christian Theologies, 1990}

\begin{quote}

Africanness and (theological) correctness should not be measured in either dissimilarity or similarity to the West. The way forward is to measure the Africanness of any theology purporting to be African by the degree to which it speaks to the needs of Africans in their total context. Quite naturally, the needs of African Christians should be taken seriously when determining these needs.
\end{quote}

Maize Plant Discipleship is authored by an outsider to Africa, yet its development is entirely a response to African contexts. At every stage, the missional energy and cultural perspective of African people has been crucial. Put simply: it is African pastors, students, missionaries and young people who have called Maize Plant Discipleship into being.

In particular, between 2003 and 2010, leaders from \emph{Assemblée Evangélique de Pentecôte} and \emph{Mouvement des Jeunes Serviteurs de Dieu} regularly organised leadership training conferences, in Léo and Ouagadougou. This collaboration called forth and facilitated the exposure and continual refinement of the teaching that now constitutes the Maize Plant Discipleship Syllabus (\autoref{thesyllabus}).

\begin {pause}\begin{description}

\item[Romans 4:17]

Abraham is described as one whose faith \emph{called into being things that did not exist}. In essence, Maize Plant Discipleship has been \emph{called into being} by the faith of Africans, acting in response to African contexts
\end{description}

\end {pause}

\section{Doctoral Research}
\label{doctoralresearch}

In its penultimate stage of development, doctoral research conducted amongst Burkinabé leaders and learners, by the author, led to Maize Plant Discipleship's formulation as a series of low-cost, modular handbooks, designed to be practical, relevant and accessible for use in African contexts.

\begin{itemize}
\item Prototypical Maize Plant Discipleship seminars formed an integral element of the research, which explored themes of discipleship, theology, leadership training, literature and intercultural dynamics.

\item Altogether, survey questionnaires, focus groups and personal interviews gathered insights from over 70 Burkinabé participants, of which a significant number held denominational responsibility for leadership training and discipleship. 

\end{itemize}

Analysis of participant responses revealed critical findings relating to: 

\begin{itemize}
\item Discipleship

\item Theology

\item Literature. 

\end{itemize}

\subsection{Discipleship}
\label{discipleship}

Research participants identified strongly with the concept of discipleship and, in particular, the need to freshly embrace holistic, generational discipleship practices. Accordingly, Maize Plant Discipleship focuses upon:

\begin{enumerate}
\item Awakening or strengthening contextual ownership of the call to serve God's eternal purpose

\item Promoting lifelong commitment to missional action and disciplines, including the generational formation of disciples

\item Encouraging personal, communal and cultural transformation.

\end{enumerate}

\subsection{Theology}
\label{theology}

Participants consistently validated the theological content of a prototype learning resource. Accordingly, Maize Plant Discipleship incorporates:

\begin{enumerate}
\item A holistic worldview, communal orientation, charismatic spirituality

\item A historical, covenantal, missionary interpretation of Scripture

\item Biblical theologies of discipleship, suffering and overcoming; spiritual revival, intercessory prayer and spiritual power; poverty and prosperity; personal and corporate vocation; Christ-centred servant-leadership and cultural transformation.

\end{enumerate}

\begin {pause}\begin{description}

\item[One Burkinabé theological educator]

— on examining a prototype Maize Plant Discipleship booklet, stated:
\end{description}

\begin{quote}

\emph{You are touching something that is not already existing. If we talk about evangelism, it may well be a new way of approaching evangelism, but we already have many methods of evangelism. But (a series of books focusing on) discipleship is {\ldots} really an innovative thing.}
\end{quote}

\end {pause}

\subsection{Literature}
\label{literature}

Research participants strongly expressed a desire for \emph{contextually appropriate literature}. In Burkina Faso, as in Africa and many parts of the Majority World in general, leaders and learners typically mediate between two cultural worlds of orality and literacy. Accordingly Maize Plant Discipleship handbooks are expressly formulated to cross the boundaries of orality and literacy:

\begin{enumerate}
\item Focussed on practical discipleship formation

\item Eschewing academic, philosophical language

\item Following a thematic, modular structure

\item Devised for use amongst cohort groups

\item Incorporating reflection, group discussion and memorisation

\item Including numerous graphical and metaphorical illustrations

\item Published according to a missional, non-commercial philosophy

\item Licensed for vernacular adaption, translation and re-publication

\item Designed for economical reproduction and distribution

\end{enumerate}

\vspace*{\fill}
\begin {questions}

\begin{itemize}
\item Why is it crucial that African Christians decide upon what is missiologically appropriate to African contexts?

\item Has this happened historically in your contexts? 

\begin{itemize}
\item If not, why might that be?

\item If so, what has changed as a result?

\end{itemize}

\item What missiological issues are important to you and others in your context? 

\end{itemize}

\end {questions}

\chapter{The Licence}
\label{thelicence}

THIS CHAPTER DESCRIBES the Maize Plant Discipleship licence.

In order to allow Maize Plant Discipleship handbooks to be translated, adapted and adopted for use in multiple and especially African and other Majority world contexts, a Creative Commons Licence permits republication of the entire syllabus. 

\emph{Please read the licence carefully.}

\begin{pause}\begin{description}

\item[Organising republication]

on behalf of a number of agencies, acting together, is likely to provide to be the most economic method, potentially enabling the cost of handbooks to be subsidised for those least able to pay.

\item[Feedback]

— the author welcomes hearing from from facilitators and students, with comments, feedback and suggestions of all kinds and would especially value culturally appropriate stories, proverbs and metaphors that may help to illuminate future editions of the resource!
\end{description}

\end {pause}

\pagebreak 

\section{The Licence}
\label{thelicence}

Maize Plant Discipleship by John B. Clements is licensed under a \textbf{Creative Commons Attribution-ShareAlike 4.0 International License}. Based on a work at \href{http://johnbrc.github.io/MPD-Distribution/}{http:/\slash johnbrc.github.io\slash MPD-Distribution\slash }. This licence applies to all handbooks belonging to the Maize Plant Discipleship Syllabus (\autoref{thesyllabus}). Please take care to understand the terms of the licence.

\begin{quote}

\emph{Note: What follows is a simplified summary of (and not a substitute for) the licence, which may be accessed at: \href{http://creativecommons.org/licenses/by-nc-sa/4.0/}{http:/\slash creativecommons.org\slash licenses\slash by-nc-sa\slash 4.0\slash } and \href{http://maizeplantdiscipleship.info}{http:/\slash maizeplantdiscipleship.info}}
\end{quote}

\begin{summary}

\textbf{You are free to}:

\begin{itemize}
\item \textbf{Share} — copy and re-distribute the material in any medium or format

\item \textbf{Adapt} — remix, transform and build upon the material 

\end{itemize}

The licensor cannot revoke these freedoms as long as you follow the licence terms:

\begin{itemize}
\item \textbf{Attribution} — You must give appropriate credit, provide a link to the licence, and indicate if changes were made. You may do so in any reasonable manner, but not in any way that suggests the licensor endorses you or your use.

\item \textbf{NonCommercial} — You may not use the material for commercial purposes (i.e. primarily intended for commercial advantage or monetary compensation.)

\item \textbf{ShareAlike} — If you remix, transform, or build upon the material, you must distribute your contributions under the same licence as the original.

\item \textbf{No additional restrictions} — You may not apply legal terms or technological measures that legally restrict others from doing anything the licence permits.

\end{itemize}

\end{summary}

\input{mpd-footer}

\end{document}
