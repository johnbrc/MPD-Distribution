\def\version{0.4.7 "Country Road"}
\def\change{finalising LaTex tweaks}
\input{mpd-header}
\def\mytitle{Maize Plant Discipleship Module 4}
\def\subtitle{Dynamics of Commissioning}
\def\myauthor{Dr John B Clements}
\def\twitter{@johnbrc, @mpdresource}
\def\email{clements.jb@gmail.com}
\def\web{http:/\slash maizeplantdiscipleship.wordpress.com}
\def\mycopyright{John B Clements, 2014.  Maize Plant Discipleship by John B Clements is licensed under a Creative Commons Attribution-NonCommercial-ShareAlike 4.0 International License. Based on a work at http:/\slash maizeplantdiscipleship.wordpress.com}
\def\keywords{discipleship, mission, messianic, community}
\def\latexmode{memoir}
\input{mpd-document}

\chapter*{Using this handbook}\label{Usingthishandbook}

This module, number 4, explores \emph{God's commissioning} of the Messianic Community, in four inter-related studies:

\begin{studies}

\begin{enumerate}
\item The \textbf{Commissioning} of Messianic Community

\item The \textbf{Strategies} of Messianic Community

\item The \textbf{Structures} of Messianic Community

\item The \textbf{Expansion} of Messianic Community

\end{enumerate}

\end{studies}

Maize Plant Discipleship handbooks facilitate reflective learning and group discussions, based on scripture readings and topical studies. They rely simply upon facilitators willing to co-ordinate small groups of learners.

Comprehensive guidelines on appropriately facilitating study groups and discussions are incorporated in the \emph{Maize Plant Discipleship Facilitators Handbook}—available from the same source as this handbook, or via the Internet.

\begin{wsite}

\begin{itemize}
\item http:/\slash maizeplantdiscipleship.wordpress.com\slash 

\item http:/\slash johnbrc.github.io\slash MPD-Distribution\slash 

\end{itemize}

\end{wsite}

\pagebreak 

\osection{Syllabus}

The Maize Plant Discipleship Syllabus incorporates a Facilitators Handbook and 16 modular Handbooks, themed to symbolically reflect the maize plant metaphor, by referring to the imagery of soil, roots, growth, sunlight and rainfall.

\begin{syllabus}

\begin{enumerate}
\item The Eternal Purpose of God

\item Dynamics of Vocation, The Nations

\item Dynamics of Vocation, The Jews

\item \textbf{Dynamics of Commissioning}

\item Dynamics of Body Membership

\item Dynamics of Revival

\item Dynamics of Truth

\item Dynamics of Intercession

\item Dynamics of Cultural Transformation

\item Disciplines of Spiritual Maturity

\item Disciplines of Running the Race

\item Disciplines of Pressing Towards our Vocation

\item Disciplines of Economic Faithfulness

\item Disciplines of Messianic Leadership

\item Disciplines of Living By Faith

\item Disciplines of Overcoming

\end{enumerate}

\end{syllabus}

\chapter{The Commissioning of Messianic Community}
\label{thecommissioningofmessianiccommunity}

\begin{synopsis}

\textbf{The Messianic Covenant Community has been commissioned to work alongside the Messiah in his mission}

\end{synopsis}\begin{topics}

\begin{enumerate}
\item Military commissioning

\item God's commissioning

\item Commissioned as disciples

\end{enumerate}

\pagebreak 

\end{topics}\osection{Terms used in this study}

\begin{description}

\item[Commission]

To charge with responsibility for a task or duty, as when a military officer is given a specific rank and responsibility, e.g. \emph{he was commissioned after attending the training academy}

To grant authority to undertake a task or function, as when an architect is authorised to build something, e.g. \emph{the architect was commissioned to manage the project}
\end{description}

\osection{Scripture}\bible

Read these passages aloud; memorise the \textbf{bold} passages.

\begin{itemize}
\item Numbers 27:23

\item \textbf{Matthew 22:14}

\item \textbf{1 Corinthians 9:17}

\item 2 Corinthians 2:17

\item Galatians 1:1

\item Colossians 1:25

\end{itemize}

\section{Military commissioning}
\label{militarycommissioning}

This topic examines the experiences of a military solider, in order to enlarge our understanding of what it means to be \emph{commissioned}. It highlights three principal stages of becoming a soldier: calling, training and commissioning.

\subsection{Calling}
\label{calling}

\textbf{A soldier's life starts} when they perceive a \emph{calling} to a life of military service. A calling is an awareness that a particular occupation represents a desirable, compelling, or appropriate vocation to pursue. There may be many reasons why someone enlists in an army, but at some point they sense a calling towards it.

\subsection{Training}
\label{training}

\textbf{A soldier's calling is tested} with a period of intensive, \emph{basic training}, which potentially equips them for a lifestyle of military service. 

Trainee soldiers experience trials and hardships intended to test their discipline, teamwork, communication, competence, obedience, initiative, determination, loyalty and resolve. During this time, each individual's capabilities and characteristics are either affirmed, enhanced or exposed as inadequate. At the end of this period, training officers assess whether each soldier has satisfactorily completed basic training. 

\begin{description}

\item[Failure]

means that a soldier will be required to undergo basic training again, until they pass satisfactorily—or they must leave military service altogether.

\item[Success]

means a soldier is equipped and authorised for \emph{active service}: capable, ready and trusted to fulfil their duties, including general soldiering \emph{and} personal vocational duties—such as those of infantry, driver, engineer, medic, officer, chef and so on.
\end{description}

\subsection{Commissioning}
\label{military}

\textbf{A soldier's commissioning} represents the beginning of their \emph{active service}, as a member of a regiment, or division of the army. Historically, soldiers are commissioned ceremonially by the national sovereign—the King or Queen—to \emph{serve the crown}, as a loyal subject.

In this new phase, soldiers continue to lead highly-disciplined lives and to experience demanding trials and tests. However, unlike the training phase, significant achievements are rewarded—generally by receiving enhanced or additional responsibility.

\begin{discuss}[\currentsectiontitle]

Use Discussion 1 now, or continue to \autoref{godscommissioning} and use Discussions 1--4 together at the end of this Study

\end{discuss}

\pagebreak 

\section{God's commissioning}
\label{godscommissioning}

This topic explores the origins of the word \emph{mission} and its connection with \emph{commissioning}, in order to understand how the messianic, new covenant community is united with God's mission.

\subsection{The sending of God}
\label{thesendingofgod}

The English word, \emph{mission} (from Latin \emph{missio}, meaning \emph{sending}) was originally used exclusively to refer to God's active sending of himself, into the world, to restore it from the effects of human wickedness, idolatry, chaos, spiritual darkness, oppression, injustice and evil.\footnote{\emph{Missio Dei}---an increasingly widespread theological concept; closely equivalent with \emph{God's eternal purpose} (Module 1).} We see this in three particular ways. 

\begin{description}

\item[The Old Testament reveals God's glorious Presence]

as a pillar of cloud and fire during the exodus from Egypt, within the Tabernacle of Meeting, above the Ark of the Covenant and at Solomon's dedication of the temple.\footnote{Hebrew, \emph{Shekinah} or Glorious Presence of God dwelling amongst his people—e.g. Exodus 13:17--14:29; Exodus 40; 2 Chronicles 7; also Luke 2:9, Hebrews 1:3; 1 John 1:14, 2 Peter 1:17; Matthew 17:6 etc;}

\item[The New Testament reveals the Messiah, Jesus]

as \emph{Lamb of God,} apostle and high priest and \emph{the sole expression of the glory (}Shekinah\emph{) of God—the perfect imprint and very image of God's nature\footnote{Hebrews 4:14--16 and 1:1--4; Colossians 1:15,19}}—in every way Jesus the Messiah reflects the reality that God the Father is a missional god.

\item[The New Testament reveals the Holy Spirit]

as the \emph{Breath\footnote{Hebrew: \emph{ruach}, means breath, or spirit}} of the Messiah—sent by the Messiah, as the Messiah was sent by the Father—the Holy Spirit empowers and \emph{sends} the messianic community, anointing us to do the works of God.\footnote{See John 6.28, 14:12--17 and 16:7--11} and to become his living temple.\footnote{1 Corinthians 3:16--17; 1 Peter 2:4--5}
\end{description}

\pagebreak 

\subsection{The sending of God's people}
\label{thesendingofgodspeople}

Having begun with God's \emph{sending} of himself, mission is enlarged through the biblical covenants, as God's people are united with God and his sending of himself into the world, to restore it.

\begin{description}

\item[Commission]

\emph{co} means joining, uniting or sharing, hence, \emph{commission}: to join, or unite with a particular mission.
The covenant community is \emph{commissioned} by God: united with him in his mission to reconcile and restore creation to himself, through the Messiah. This commissioning begins with the covenant community of Israel; it is renewed through the Messiah.
\end{description}

\subsection{Joining the Messiah's mission}
\label{joiningthemessiahsmission}

Jesus' \emph{mission} is the purpose for which he was sent into the world: to represent the Father and to do the works of God, forming and sending disciples, in his name.\footnote{John 16:5, for example} He then calls and commissions (sends) his disciples to complete his mission:

\begin{quote}

\emph{As the Father sent me, so I now send you}—John 20:21
\end{quote}

\begin{description}

\item[Messianic commission]

therefore, refers joining, being united with and sharing in the \emph{mission} of the Messiah: representing the Father, joining him in doing the works of God, forming and sending disciples in his name. The messianic community is called to be with him; to be prepared, set apart, blessed, anointed and \emph{sent} towards the world, for the sake of the world, \emph{commissioned} to bless the peoples of the world, in God's name.\footnote{See \emph{Module 1, The Eternal Purpose of God}.}
\end{description}

\begin{discuss}[\currentsectiontitle]

Use Discussion 2 now, or continue to \autoref{commissionedasdisciples} and use Discussions 1--4 together at the end of this Study

\end{discuss}

\pagebreak 

\section{Commissioned as disciples}
\label{commissionedasdisciples}

This topic explores parallels between soldiering and discipleship.

\subsection{The metaphor of a soldier}
\label{themetaphorofasoldier}

The apostle, Paul, uses the metaphor of a soldier to emphasise that disciples of Jesus Christ must endure discipline, hardship and suffering.\footnote{Modules 10--16 explore the disciplines, hardships and suffering required, as disciples of the Messiah progress in their calling to serve God's eternal purpose.}

\begin{quote}

Take your share of the hardships and suffering which you are called to endure as a good, first-class soldier of Jesus Christ. No soldier when in service gets entangled in the enterprises of civilian life; his aim is to satisfy and please the one who enlisted him—\emph{2 Timothy 2:3--4 ABV}
\end{quote}

\subsection{Called to serve}
\label{calledtoserve}

Military commissioning (\autoref{militarycommissioning}) explored how soldiering begins with discerning a calling to military service. Messianic discipleship similarly begins with a \emph{calling}. This happens as we personally or corporately discern a calling to serve God's mission, in some way—to serve God's eternal purpose.\footnote{See modules 1--3, incorporating \emph{The Eternal Purpose of God} and \emph{Dynamics of Vocation}.}

\subsection{Many are called, but few are chosen}
\label{manyarecalledbutfewarechosen}

Those who hear and respond to a calling to military service are only commissioned after satisfactorily completing basic training. Jesus' words, \emph{Many are called, but few are chosen}\footnote{Matthew 22:1--14} confirms that hearing the call to serve God's mission is only a start. 

Like soldiers, disciples of the Messiah need to learn basic disciplines and specialised skills. Like soldiers, progressing from calling to commissioning requires yielding ourselves fully to the demands of vocational service, in order to become disciplined, skilled, faithful. 

The process of \emph{calling, training and commissioning} is how we are prepared for useful service.\footnote{2 Timothy 2:20--21} It is how we become co-workers with the Messiah in his mission. This is what it means to be \emph{chosen} by the Messiah: to be trained and appointed to work alongside him, gathering a \emph{harvest of faithfulness,} in the power of his Spirit.\footnote{Hebrews 12:11}

\begin{discuss}[\currentsectiontitle]

Use Discussion 3 now, or continue to the Summary and then Discussions 1--4 to complete Study 1

\end{discuss}

\osection{Summary}

This concludes Study 1, \emph{The Commissioning of Messianic Community}, which:

\begin{itemize}
\item explored the metaphor of soldiering, including calling, training and commissioning;

\item explained that \emph{commissioning} essentially means joining together in mission, illustrating how the new-covenant community joins with the Messiah's mission;

\item compared military service with messianic discipleship, highlighting similarities relating to obedience, faithfulness and reward.

\end{itemize}

In summary, the study revealed that

\begin{summary}

\textbf{The Messianic Covenant Community has been commissioned to work alongside the Messiah in his mission: as a community of disciples, called, equipped and chosen to serve God's eternal purpose}

\end{summary}

\pagebreak 

\osection{Discussions}
\begin{disc}[relating to T1, Military commissioning]

\begin{itemize}
\item How do the demands of military soldiering differ from the challenges of civilian life?

\item How helpful is the metaphor of soldiering in your cultural context? And in your own life?

\end{itemize}

\end{disc}
\begin{disc}[relating to T2, God's commissioning]

\begin{itemize}
\item What is significant about God's sending of himself into the world?

\item What does it mean to be sent as the Messiah was sent?

\end{itemize}

\end{disc}
\begin{disc}[relating to T3, Commissioned as disciples]

\begin{itemize}
\item What messianic disciplines might be compared to basic military training?

\end{itemize}

\begin{quote}

How might a messianic disciple fail the equivalent of basic training?
\end{quote}

\end{disc}
\begin{disc}[relating to Study 1, as a whole]

\begin{itemize}
\item How does the \emph{duty} of a commission differ from the \emph{discipline} of basic training?

\item What is the significance of receiving a commission?

\end{itemize}

\end{disc}

\chapter{The Strategies of Messianic Community}
\label{thestrategiesofmessianiccommunity}

\begin{synopsis}

\textbf{Scripture reveals a series of five primary strategies empowering the mission of the Messianic community}

\end{synopsis}\begin{topics}

\begin{enumerate}
\item The strategy of prayer (\autoref{thestrategyofprayer})

\item The strategy of reconciliation (\autoref{thestrategyofreconciliation})

\item The strategy of discipleship (\autoref{thestrategyofdiscipleship})

\item The strategy of teaching (\autoref{thestrategyofteaching})

\item The strategy of sending (\autoref{thestrategyofsending})

\end{enumerate}

\pagebreak 

\end{topics}\osection{Terms used in this study}

\begin{description}

\item[Strategy]

a plan of action or policy designed to achieve a major or overall aim; military operations and movements in a war or battle.
\end{description}

\osection{Scripture}\bible

Read these passages aloud; memorise the \textbf{bold} passages.

\begin{itemize}
\item Matthew 9:35--38

\item Mark 16:15--18

\item Luke 24:44--49

\item \textbf{John 20:21--23}

\item \textbf{Matthew 28:18--20}

\item Acts 26:15--18

\item Hebrews 5:11--14

\end{itemize}

\section{The strategy of prayer}
\label{thestrategyofprayer}

The foundational strategy of messianic mission is prayer.

\begin{figure}[htbp]
\centering
\includegraphics[width=72pt,height=30pt]{pray.png}
\caption{}
\label{pray.png}
\end{figure}

\subsection{Strategic prayer}
\label{strategicprayer}

There are many kinds of prayer, much of which is not \emph{strategic}. Strategic prayer is prayer that is a forward-thinking, planned priority, focussed upon a clear purpose.

\subsection{A plentiful harvest}
\label{aplentifulharvest}

When Jesus proclaimed the Kingdom of God throughout Israel, he compared a ripe, abundant crop harvest to crowds of distressed and dejected people who were to him \emph{like sheep without a shepherd.} Then he explained to his disciples:

\begin{quote}

\emph{The harvest is indeed plentiful, but the labourers are few. So pray to the Lord of the harvest to force and thrust out labourers into his harvest}—Matthew 9:35--38 ABV\footnote{Amplified Bible Version amplifies meaning from original Hebrew and Greek}
\end{quote}

In this way, Jesus implies that the people represent a harvest that is ready to be gathered in to his kingdom, in the same way that a cereal crop is ready in due season. By speaking of \emph{the Lord of the harvest} he also affirms how important the harvest is to the Lord—just as a cereal crop, such as maize, is vitally important to a farmer, who depends upon the harvest for his food and for seed to sow.

\subsection{A harvesting problem}
\label{aharvestingproblem}

Although the harvest is ready, Jesus identifies a problem: a shortage of labourers ready and willing to gather the harvest, a shortage of trained disciples, ready to be commissioned into service.

How does Jesus teach his disciples to respond to this challenge? He points immediately towards the foundational strategy of mission, instructing them:

\begin{quote}

\textbf{\emph{Pray to the Lord of the harvest}}.
\end{quote}

\begin{discuss}[\currentsectiontitle]

Use Discussion 1 now, or continue to \autoref{thestrategyofreconciliation} and use Discussions 1--6 together at the end of this Study

\end{discuss}

\section{The strategy of reconciliation}
\label{thestrategyofreconciliation}

The second strategic step of messianic mission is reaching people, in order to encourage reconciliation with God, through the Messiah.

\begin{figure}[htbp]
\centering
\includegraphics[width=135pt,height=30pt]{pray-reach.png}
\caption{}
\label{pray-reach.png}
\end{figure}

\subsection{Reconciliation with God}
\label{reconciliationwithgod}

Reaching people implies a holistic process of reconciliation with God and his eternal purpose.\footnote{Acts 26:17--18} This process of reconciliation incorporates: 

\begin{itemize}
\item God's forgiveness and cleansing of our wrongdoings;

\item Forgiving others for wrongdoings inflicted upon us;

\item Deliverance from dominant sinful behaviour;

\item Cleansing from spiritual and practical impurity;

\item Renunciation of idols and idolatry;

\item Establishing a wholehearted devotion to the Messiah

\item Inviting the Messiah to rule in our daily lives.

\end{itemize}

Unless people experience the transformative power of being reconciled to God, through the Messiah, they cannot be liberated to serve him as disciples.

\subsection{Personal reconciliation}
\label{personalreconciliation}

Reaching others with a message of reconciliation challenges our own lifestyle and faithfulness. Failing to exhibit kingdom values and priorities corrupts our personal testimony and that of the Messianic Community as a whole. Before reconciling others, we ourselves must first be fully reconciled and submitted to God.

\begin{discuss}[\currentsectiontitle]

Use Discussion 2 now, or continue to \autoref{thestrategyofdiscipleship} and use Discussions 1--6 together at the end of this Study

\end{discuss}

\section{The strategy of discipleship}
\label{thestrategyofdiscipleship}

The third strategy is the formation of disciples.

\begin{figure}[htbp]
\centering
\includegraphics[width=199pt,height=30pt]{pray-reach-disciple.png}
\caption{}
\label{pray-reach-disciple.png}
\end{figure}

\subsection{Discipleship is at the heart of messianic community}
\label{discipleshipisattheheartofmessianiccommunity}

The forming of disciples is the very heart and centre of messianic community and mission. We are called and commissioned to make people from all nations into disciples, by immersing (baptising) them into the reality of God's life, through the Messiah, by the Spirit.\footnote{Matthew 28:19--20}

\subsection{Discipleship deals with our hearts}
\label{discipleshipdealswithourhearts}

Discipleship deals with something deeper than the mind: our \emph{hearts}—the centre of our being, the seat of our motivation, our willpower, our commitment. Through discipleship, we are challenged to become wholly aligned with God's eternal and vocational purposes and to serve him as our Lord and Master. As these new allegiances and practical priorities are adopted and applied, our heart becomes renewed.

\begin{itemize}
\item Unless our hearts are challenged and renewed in this way, we remain mere \emph{religious converts}—engaging in devotional, religious activity, whilst our will, character, allegiances, loyalties and lifestyle remain practically unchanged.

\item As we experience a process of formation into disciples of Jesus, we face the challenge of whether to make our whole heart available to God—or to shrink back from the demands of Messianic mission. 

\item Only as we allow our hearts to be transformed by the Holy Spirit do we begin the life of a disciplined co-worker of the Messiah\footnote{1 Corinthians 3:9}: one who has been authentically \emph{co-missioned} with him.

\end{itemize}

\subsection{Discipleship is the priority}
\label{discipleshipisthepriority}

Because discipleship deals with the heart, strategically and practically it should \emph{precede} concentrated biblical teaching (\autoref{thestrategyofteaching}), which is for committed disciples who have progressed beyond the elementary teachings.

\begin{discuss}[\currentsectiontitle]

Use Discussion 3 now, or continue to \autoref{thestrategyofteaching} and use Discussions 1--6 together at the end of this Study

\end{discuss}

\section{The strategy of teaching}
\label{thestrategyofteaching}

The fourth strategy is teaching.

\begin{figure}[htbp]
\centering
\includegraphics[width=262pt,height=29pt]{pray-reach-disciple-teach.png}
\caption{}
\label{pray-reach-disciple-teach.png}
\end{figure}

\subsection{Solid food}
\label{solidfood}

\begin{quote}

Though by this time you ought to be teachers, you need someone to teach you the elementary truths of God's word all over again. You need milk, not solid food! Anyone who lives on milk, being still an infant, is not acquainted with the teaching about righteousness. But solid food is for the mature, who by constant use have trained themselves to distinguish good from evil—Hebrews 5:12--14
\end{quote}

The writer of Hebrews talks about \emph{solid food}, or \emph{strong meat,\footnote{The KJV phrase, \emph{strong meat}, captures well the sense of maturity required to ingest and digest challenging scriptural teaching.}} as a metaphor for the challenging teaching of God's word. He chastises them that they should be teaching others by this time, yet instead they still need \emph{milk}—a metaphor for elementary teaching.\footnote{Hebrews 6:1--3}

Strong meat is for spiritually mature, committed disciples. For them, biblical teaching represents a fruitful source of insight, conviction, trust, knowledge, wisdom and understanding. It develops and deepens appreciation of messianic life, spirituality and vocational service.

\subsection{The whole counsel of God}
\label{thewholecounselofgod}

Paul refers to \emph{the word of God in its fullness} or \emph{the whole counsel of God.\footnote{Colossians 1:25; also Acts 20:27 ff.}} In this section, I propose that this fulness incorporates three complementary aspects of biblical teaching that need to be held together, in tension: pastoral-evangelistic, prophetic and apostolic.

\subsubsection{Pastoral-evangelistic teaching}
\label{pastoral-evangelisticteaching}

Pastoral-evangelistic teaching is directed towards appreciating and applying the accumulated wisdom, knowledge, understanding and traditions of messianic community.\footnote{1 Timothy 3:15} 

\begin{itemize}
\item Calling disciples into \emph{The Way\footnote{Acts 9:2; 18:25--26; 19:9,23; 22:4; 24:14,22; cf. John 14:4--6}} of the Messiah.

\item Studying and interpreting scripture.

\item Exploring church history and denominational identity.

\end{itemize}

Historically, pastoral-evangelistic teaching has dominated the Western Church, with dogmatic belief used vigorously, sometimes even violently, to protect denominational traditions and to justify political power and even the tyranny of anti-Semitism, colonialism, racism and apartheid. In the modern period, it also became unhelpfully shackled to western, academic standards.

Although academically correct, dogmatic belief can contribute to deep, strong traditions and active, close-knit communities, it can equally contribute to spiritual sterility, generational inertia and a lack of effective engagement with outsiders. Balance is needed with the spontaneity and cultural relevance of prophetic teaching and the intercultural outlook of apostolic teaching.

\subsubsection{Prophetic teaching}
\label{propheticteaching}

Prophetic teaching is directed towards interpreting the historical, contextual \emph{signs of the times\footnote{Matthew 16:3; see also 1 Chronicles 12:32}} that call for an appropriate response from messianic communities.

\begin{itemize}
\item Critiquing the parochial culture of messianic communities, organisations and structures.\footnote{We may think of culture as including the \emph{activities, institutions, knowledge, traditions, values, motivations and thought-processes} of a particular nation or people group.}

\item Responding to God's historical, contextual, missional purposes, amongst the nations.

\end{itemize}

Prophetic teaching has historically tended to represent a threat to the status quo of Christian tradition. This can lead to a marginalisation of fresh, prophetic movements and a deepening of mainstream inertia.

\subsubsection{Apostolic teaching}
\label{apostolicteaching}

Apostolic teaching is directed towards interpreting the responsibility of the messianic community to look beyond its own sphere of activity, towards other regions where there are different needs to be served.

\begin{itemize}
\item Offering and relating \emph{the word of life\footnote{Philippians 2:16}} to those outside of messianic community.

\item Appropriately critiquing human culture, in the light of God's words.

\item Preparing personnel for intercultural missionary activity.

\end{itemize}

The peculiar demands of intercultural ministry can mean that is does not rest easily alongside parochial concerns, with the effect that pastoral communities and apostolic, missionary agencies operate separately—to the detriment of both.

\subsection{Spiritual revelation}
\label{spiritualrevelation}

The spirit of \emph{revelation} plays a vital role in messianic teaching. The disciples experienced this when they met the risen Messiah.

\begin{quote}

He opened their minds, so that they could understand the Scriptures{\ldots}—Luke 24:45--47
\end{quote}

Revelation is a gift of the Spirit that opens our understanding to spiritual truths and realities. The Spirit of God expands our human understanding to incorporate spiritual truths that are not obtained by, or received within, our natural, rational minds. Instead, they are revealed to our our spirit.\footnote{1 Corinthians 2:6--16}

A vital key to experiencing revelation is the desire to do God's will: Jesus promised that \emph{Anyone who chooses to do the will of God will find out whether my teaching comes from God{\ldots}\footnote{John 7:17}} and \emph{{\ldots}Everyone who asks receives; the one who seeks finds and to the one who knocks, the door will be opened}.\footnote{Luke 11:10}

\begin{discuss}[\currentsectiontitle]

Use Discussion 4 now, or continue to \autoref{thestrategyofsending} and use Discussions 1--6 together at the end of this Study

\end{discuss}

\section{The strategy of sending}
\label{thestrategyofsending}

The fifth strategy is sending disciples.

\begin{figure}[htbp]
\centering
\includegraphics[width=292pt,height=26pt]{pray-reach-disciple-teach-send.png}
\caption{}
\label{pray-reach-disciple-teach-send.png}
\end{figure}

\subsection{Culmination of strategies}
\label{culminationofstrategies}

\emph{Sending} represents a culmination of the four strategies that have preceded it: prayer, reconciliation with God, formation of disciples and teaching about the kingdom. It refers to being sent to fulfil a missional, vocational calling. This sending refers to two types of context.

\begin{itemize}
\item Inter-cultural

\item Intra-cultural

\end{itemize}

\subsubsection{Inter-cultural contexts}
\label{inter-culturalcontexts}

Inter-cultural mission implies being sent to significantly different cultural, ethnic and geographical contexts. 

\begin{itemize}
\item Intercultural work requires specialist intercultural training, experience and understanding, because culture profoundly affects how messianic community and vocation is translated into practice.

\item Intercultural work is typically highly-demanding, because of various risks and hardships related to living in and engaging with a different culture.

\item Intercultural work is the vocation of \emph{apostolic missionary teams} (\autoref{apostolicmissionaryteams}).

\end{itemize}

\subsubsection{Intra-cultural contexts}
\label{intra-culturalcontexts}

Because God sends the whole messianic community towards all the peoples and cultures of the world, mission is not limited to crossing geographic, ethnic or intercultural barriers. \emph{Intra-cultural mission} identifies that the messianic community is called and \emph{sent} towards the people within its own original context.

\begin{itemize}
\item Intra-cultural sending implies co-operating with God's purposes and forming disciples \emph{wherever} we are working: living out our vocational responsibility in ways that influence and transform homes, families, communities, workplaces and institutions.

\item Whatever vocational role we occupy, each disciple should recognise how faithfulness to the Messiah's missional priorities creates opportunities to introduce messianic perspectives, biblical truths and spiritual power into our personal and vocational contexts.

\item This includes participation in \emph{worth-creating} sectors, such as: industry, commerce, arts, sports, media, health, civil and other government services, many of which have historically been overlooked or devalued by some forms of Christian spirituality.

\end{itemize}

\subsection{Sent by the Spirit}
\label{sentbythespirit}

Whether inter- or intra-culturally, God is the one who equips, empowers and sends workers, by his Spirit. When a particular community \emph{sends} workers into any context, it needs to do so \emph{in step with the Spirit}, upholding God's sending of those people.\footnote{See Acts 13:1--4, for an example of the Messianic Community and the Holy Spirit acting in harmony together, in sending Paul and Barnabas on a missionary journey.}

\begin{discuss}[\currentsectiontitle]

Use Discussion 5 now, or continue to the Summary and then Discussions 1--6 to complete Study 2

\end{discuss}

\osection{Summary}

This concludes Study 2, \emph{The Strategies of Messianic Community}, which explored five foundational, missional strategies:

\begin{figure}[htbp]
\centering
\includegraphics[width=175pt,height=179pt]{commissioncycle.png}
\caption{}
\label{commissioncycle.png}
\end{figure}

As each generation of disciples follow these strategies, a cyclical process begins to establish an expanding, missional movement.

In summary, the study revealed that

\begin{summary}

\textbf{Scripture reveals a cycle of five major strategies empowering the mission of the Messianic community: Pray, Reach, Disciple, Teach, Send}

\end{summary}

\pagebreak 

\osection{Discussions}

\begin{disc}[relating to T1, Prayer]

\begin{itemize}
\item What makes prayer strategic?

\item What hinders strategic prayer?

\end{itemize}

\end{disc}
\begin{disc}[relating to T2, Reconciliation]

\begin{itemize}
\item What does it means to be fully reconciled to God and his kingdom purposes?

\item What issues hinder you, your household or family, from personally experiencing peace with God?

\end{itemize}

\end{disc}
\begin{disc}[relating to T3, Discipleship]

\begin{itemize}
\item To what is your heart devoted? Be honest.

\item What compromises your expression of whole-hearted commitment to serving God's purposes?

\end{itemize}

\end{disc}
\begin{disc}[relating to Teaching]

\begin{itemize}
\item How have you responded to God's \emph{strong meat}?

\item How have your responses effected your life?

\end{itemize}

\end{disc}
\begin{disc}[relating to Sending]

\begin{itemize}
\item What does it mean to be sent—is it something that happens once, regularly, or continuously?

\item Discuss the role of the Holy Spirit in sending the whole Messianic Community.

\end{itemize}

\end{disc}

\chapter{The Structures of Messianic Community}
\label{thestructuresofmessianiccommunity}

\begin{synopsis}

\textbf{The messianic community has a God-ordained structure that uniquely equips it to fulfil the messianic commission}

\end{synopsis}\begin{topics}

\begin{enumerate}
\item Pastoral, evangelistic community (\autoref{pastoralevangelisticcommunity})

\item Teaching and training centres (\autoref{teachingandtrainingcentres})

\item Apostolic missionary teams (\autoref{apostolicmissionaryteams})

\item Prophets, priests, mediators (\autoref{prophetspriestsmediators})

\item Structures and strategies (\autoref{structuresandstrategies})

\end{enumerate}

\pagebreak 

\end{topics}\osection{Terms used in this study}

\begin{description}

\item[Structure]

referring to vocational communities structured (organised) according to their purpose and function. 

\emph{Note: In the following sections, various messianic structures are represented by circles—and a combination of overlapping circles.}

\item[Pastoral]

from \emph{pastor,} meaning shepherd—implying care, protection, provision, discipline and guidance—as a good shepherd with his sheep.

\item[Evangelistic]

from \emph{evangel,} meaning \emph{Good News}—the proclamation that the Messiah, Jesus, is Lord, especially of his covenant community.

\item[Prophetic]

from \emph{prophet}—those appointed by God, to speak to human beings on behalf of God \emph{and} to God, on behalf of human beings.

\item[Apostolic]

from \emph{apostle,} meaning \emph{sent one}; referring to those sent as intercultural, missionary pioneers.
\end{description}

\osection{Scripture}\bible

Read these passages aloud; memorise the \textbf{bold} passages.

\begin{itemize}
\item Exodus 26:30

\item 2 Corinthians 5:20--21

\item Acts 13:1--4

\item Acts 19:8--11

\item \textbf{Ephesians 4:11--13}

\item \textbf{Hebrews 8:5}

\end{itemize}

\pagebreak 

\section{Pastoral, evangelistic community}
\label{pastoralevangelisticcommunity}

\begin{figure}[htbp]
\centering
\includegraphics[width=125pt,height=125pt]{pastoral-evangelistic-community.png}
\caption{}
\label{pastoral-evangelistic-community.png}
\end{figure}

Pastoral evangelistic community is defined by its dual pastoral \emph{and} evangelistic character.\footnote{Pastoral, evangelistic community is expressed contextually as \emph{local church}.} 

\begin{itemize}
\item Being pastoral means reflecting the affirming, selfless, humble, protective, overseeing care of the Good Shepherd, Jesus.

\item Being evangelistic means reflecting the Lordship of Jesus over the community itself \emph{and} the spiritual and natural powers that influence and shape human society.

\end{itemize}

In practice, this pastoral-evangelistic character is demonstrated through being a community that is \emph{faithful}, \emph{hospitable} and \emph{celebratory}.

\subsection{Community faithfulness}
\label{communityfaithfulness}

Community faithfulness reflects a unity of trust in and faithfulness towards the Messiah: a shared expression of confidence in the \emph{evangel}—the Good News—of what God has done for all peoples, through his Messiah.

\begin{itemize}
\item Messianic faith is most powerfully demonstrated through love. Sharing one another's concerns and burdens is a vital aspect of trusting in and proclaiming our faithfulness to the Messiah. Love-in-action provides a profound, living demonstration of the Messiah's victory over human self-centredness.\footnote{Galatians 5:6}

\end{itemize}

\subsection{Community hospitality}
\label{communityhospitality}

Community hospitality offers friendliness, kindness, warmth, welcoming, care, openness, acceptance and concern, most especially for those who are aliens, strangers, outsiders.\footnote{Deuteronomy 10:19; cf. 1 Peter 2:11--12} The root meaning of hospitality is to host others.

\begin{itemize}
\item \emph{Being an inviting, hospitable community}, however, implies more than simply offering invitations to events. On a practical level, it means opening our hearts and our homes to one another. Above all, it means expressing messianic faith and love in a manner that \emph{invites the interest and involvement} of outsiders.\footnote{Colossians 4:6; Philippians 2:4} 

\item The ultimate invitation to outsiders is join the messianic community in giving the Messiah wholehearted allegiance, which incorporates worship, service and faithfulness.

\end{itemize}

\subsection{Community celebrations}
\label{communitycelebrations}

Community celebrations express honour, gratitude and commitment towards God \emph{and} the public commemoration of his goodness and love towards his creation.

\begin{itemize}
\item Covenant feasts, prescribed in the \emph{Torah,\footnote{The first five books of the Bible, accredited to Moses and forming the covenantal foundation of the nation of Israel.}} incorporate prophetic signs, pointing towards both the good things that God has done for his people \emph{and} his call to faithful service. The annual \emph{Passover} is the most significant Hebraic celebration. 

\item The new covenant, inaugurated by Jesus, fulfils the Passover: the sharing of bread and wine, representing the body and blood of the Messiah, speaks of the Passover \emph{Lamb of God}\footnote{John 1:29}, who sacrificed his life to serve God's eternal purpose.

\item Celebrating this sacrifice, by sharing \emph{daily bread} with one another, reminds us of the devoted, sacrificial service to which we are called and commissioned, as members of the messianic community.

\end{itemize}

\subsection{Dual characteristics, one community}
\label{dualcharacteristicsonecommunity}

Clearly, pastoral and evangelistic characteristics of messianic communities overlap each other. It is true to say that the two things cannot be separated: when a community is living a faithful, devoted, celebratory lifestyle that is hospitable, open, welcoming and inviting towards outsiders, its existence represents both a practical \emph{embodiment} and a living \emph{proclamation} of the Good News.\footnote{1 Peter 2:12}

\subsection{Pastors and evangelists}
\label{pastorsandevangelists}

Pastors and evangelists are responsible for equipping messianic communities to express their pastoral and evangelistic character.

\begin{itemize}
\item Pastors and evangelists are called to be inspirers, facilitators, catalysts and equippers, so that a \emph{whole community}, of all ages and abilities, is equipped to mutually support one another with hospitality and pastoral care \emph{and} to share their faith confidently with others, outside of the messianic community.

\end{itemize}

\subsection{Summary: Pastoral, evangelistic communities}
\label{summary:pastoralevangelisticcommunities}

Pastoral, evangelistic communities are called to be a fellowship of people learning to share their lives and values in ways that practically express both the \emph{Shepherding} and the \emph{Lordship} of Jesus, so that the whole community is working together towards a shared, primary goal of:

\begin{figure}[htbp]
\centering
\includegraphics[width=125pt,height=125pt]{disciplingallegiance.png}
\caption{}
\label{disciplingallegiance.png}
\end{figure}

\begin{summary}

\textbf{Discipling people into faithful allegiance to God's Messiah, by facilitating and encouraging deep, lasting spiritual and practical expressions of faithfulness and loyalty, in homes, workplaces and community arenas}

\end{summary}

\begin{discuss}[\currentsectiontitle]

Use Discussion 1 now, or continue to \autoref{teachingandtrainingcentres} and use Discussions 1--5 together at the end of this Study

\end{discuss}

\section{Teaching and training centres}
\label{teachingandtrainingcentres}

\begin{figure}[htbp]
\centering
\includegraphics[width=125pt,height=125pt]{teachingtrainingcentres.png}
\caption{}
\label{teachingtrainingcentres.png}
\end{figure}

Teaching and training centres supplement the formation of disciples taking place within pastoral, evangelistic community (\autoref{pastoralevangelisticcommunity}). Their function is to equip mature messianic disciples for missional, vocational service, in both \emph{intra-cultural} and \emph{inter-cultural} contexts.

\subsection{Intra cultural contexts}
\label{intraculturalcontexts}

Intra-cultural teaching and training equips messianic disciples and communities to live faithfully within their own culture, amongst their own people. 

\begin{itemize}
\item Researching, understanding, presenting and explaining the content of God's Word, in order to equip disciples with a \emph{messianic worldview}—a way of understanding and relating to the world with a biblical, messianic perspective.

\item Edifying—encouraging, strengthening and correcting—the practices and self-understanding of messianic communities, enabling them to become \emph{pillars and foundations of truth},\footnote{1 Timothy 3:15} in the context of cultures shaped by different spiritual and moral values.

\item Typical examples include: Bible schools, conferences, seminars, workshops.

\end{itemize}

\subsection{Intercultural contexts}
\label{interculturalcontexts}

Intercultural teaching and training equips disciples to live faithfully and effectively in relationship with people of a different culture. 

\begin{itemize}
\item Preparing and equipping disciples with spiritual confidence and practical resources to undertake \emph{apostolic missionary work} in non-native contexts.

\item Researching, presenting, explaining, understanding the \emph{worldviews} of people from other cultures and religions.

\item Typical examples include: Scripture translation, language learning, cross-cultural training, missionary trips.

\end{itemize}

\subsection{Different contexts, one calling}
\label{differentcontextsonecalling}

There are significant and important differences relating to intra- and inter-cultural contexts. However, only together do they represent the fulness of messianic mission: to serve God's eternal purpose, blessed to be a blessing to all the peoples of the world—both those near and those far away.

\subsection{Teachers and trainers}
\label{teachersandtrainers}

Messianic teachers and trainers are called to prepare mature disciples with an understanding of biblical truth that appropriately equips them for faithful works of service, in a variety of vocational contexts.\footnote{Ephesians 4:11--16}

\begin{itemize}
\item \emph{Training} tends to emphasise learning from the experience of others, encouraging learners to be responsive and accountable to overseers, in \emph{specific} contexts.

\item \emph{Teaching} tends to emphasise the value of knowledge and understanding, making learners responsible for evaluating, internalising and using knowledge, in \emph{multiple} contexts.

\end{itemize}

\subsection{Summary: Equipping with truth}
\label{summary:equippingwithtruth}

The characteristic role of messianic teaching and training centres is to supplement the formation of disciples taking place within pastoral, evangelistic communities, by:

\begin{figure}[htbp]
\centering
\includegraphics[width=125pt,height=125pt]{equippingtruth.png}
\caption{}
\label{equippingtruth.png}
\end{figure}

\begin{summary}

\textbf{Equipping mature disciples with biblical truth, enabling them to fulfil personal, vocational callings, in a manner that expresses faithful allegiance to the Messiah, in a range of cultural contexts}.

\end{summary}

\begin{discuss}[\currentsectiontitle]

Use Discussion 2 now, or continue to \autoref{apostolicmissionaryteams} and use Discussions 1--5 together at the end of this Study

\end{discuss}

\section{Apostolic missionary teams}
\label{apostolicmissionaryteams}

\begin{figure}[htbp]
\centering
\includegraphics[width=125pt,height=125pt]{apostolicmissionaryteams.png}
\caption{}
\label{apostolicmissionaryteams.png}
\end{figure}

New Testament missionary apostle, Paul, employs two significant metaphors to illustrate the function of apostles and their teams: \emph{architect} and \emph{ambassador}.

\subsection{Architect}
\label{architect}

Paul compares the apostolic missionary role to that of a \emph{skilful architect}, or master builder, laying a Messianic foundation.\footnote{I Corinthians 3:10--15} 

\begin{description}

\item[Architects]

or \emph{master-builders}, are responsible for both designing and supervising the construction of buildings—which others will inhabit and use.
\end{description}

Thus, apostles and apostolic missionary teams are responsible to pioneer the formation and establishment of pastoral, evangelistic communities (\autoref{pastoralevangelisticcommunity}) and messianic teaching and training centres (\autoref{teachingandtrainingcentres}). These messianic structures are then handed over to local, contextual leaders, while the apostle moves on to other fields of work.

Because apostolic work is laying a foundation upon which others will build, its quality is crucial to the future of the messianic community in those settings. Apostolic work is especially critical to intercultural contexts.

\subsection{Ambassador}
\label{ambassador}

Paul refers to the responsibility of representing God inter-culturally, amongst people of other nations or cultures, as that of \emph{ambassadors of the Messiah}.\footnote{2 Corinthians 5:20--21} 

\begin{description}

\item[Ambassadors]

are internationally-accredited diplomats, or emissaries, sent by a country as its official representatives to a foreign country.
\end{description}

Thus, apostles and apostolic missionary teams cross geographical and cultural boundaries in order to pioneer and build the foundational structures of messianic community—most particularly amongst people and places where there are no, few or waning gospel communities.

Living and working inter-culturally, in non-native contexts, places significant additional demands upon workers, because of differences encountered in a whole range of experiences, including:

\begin{itemize}
\item language, climate and food;

\item political, economic and bureaucratic systems;

\item customs, social expectations and religious sensibilities.

\end{itemize}

\subsection{First in the church}
\label{firstinthechurch}

Through the \emph{pioneering} roles of ambassador and architect, apostles lay a crucial foundation for the building of the whole messianic community. For this reason the apostolic ministry is considered \emph{first} in the Messianic Community.\footnote{1 Corinthians 12:28} 

Yet for all the reasons discussed, apostolic work is especially demanding for practitioners. Thus Paul, as one who experienced many trials and tribulations, identifies his apostolic, missionary service with being \emph{put on display at the end of the procession}.\footnote{2 Corinthians 4:7--12 \& 6:3--10; 1 Corinthians 4:9--13} 

\begin{itemize}
\item Paul is referring to the apostolic missionaries' need to persevere in the presence of all manner of difficult challenges, for the sake of furthering the message of the Messiah. 

\item Through such sacrificial dedication, endurance of suffering, embrace of humility and deep-seated reliance upon the power of the Holy Spirit, apostolic workers provide a profound example to the whole Messianic Community, of our \emph{shared calling} to faithful missional service.

\end{itemize}

\subsection{Summary: Pioneering with power}
\label{summary:pioneeringwithpower}

The characteristic role of \emph{apostolic missionary teams} is to lay a messianic foundation, by:

\begin{figure}[htbp]
\centering
\includegraphics[width=125pt,height=125pt]{pioneeringpower.png}
\caption{}
\label{pioneeringpower.png}
\end{figure}

\begin{summary}

\textbf{Skilfully, resolutely, purposefully, sacrificially pioneering in the power of the Spirit: planting and establishing both pastoral, evangelistic communities and teaching and training centres, in new geographical and cultural contexts}

\end{summary}

\begin{discuss}[\currentsectiontitle]

Use Discussion 3 now, or continue to \autoref{prophetspriestsmediators} and use Discussions 1--5 together at the end of this Study

\end{discuss}

\section{Prophets, priests, mediators}
\label{prophetspriestsmediators}

Of the five leadership gifts,\footnote{Ephesians 4:11--12} pastor, evangelist, teacher and apostle are each represented by one or other of the three principal structures examined in Topics 1--3. This section explores how the \emph{prophetic} ministry functions in relation to other messianic structures.

\subsection{Heart of Messianic Community}
\label{heartofmessianiccommunity}

Overlaying the three circles representing each of the three foundational messianic structures, creates a central region where all three overlap. I believe this region may be identified as representing the \emph{prophetic heart} of messianic community.

\begin{figure}[htbp]
\centering
\includegraphics[width=179pt,height=199pt]{propheticmediatoryrole.png}
\caption{}
\label{propheticmediatoryrole.png}
\end{figure}

The idea of a central heart, interacting with and influencing each of the other ministry structures thus provides a profound metaphor for the prophetic ministry. This suggest the prophetic role is vital, influential and central, even though it functions in a somewhat hidden, non-structural manner.

Interestingly, Paul describes the foundational significance of the prophet as second only to that of the apostle.\footnote{1 Corinthians 12:27--31} My observation is that the prophetic role is easily misunderstood, reacted to, dismissed and under-appreciated, perhaps partly because of its somewhat hidden way of functioning. Yet, when embraced, it has a profound capacity to spearhead deep, spiritual renewal.

\subsection{Dual mediatory role}
\label{dualmediatoryrole}

The prophetic ministry fulfils a dual mediatory role, on behalf of the messianic community, incorporating both \emph{priestly} and \emph{prophetic} mediation.

\begin{description}

\item[Priestly mediation]

involves \emph{speaking with God, on behalf of human beings}, based upon a purity of heart and faithfulness of life that is able to offer effective intercession on behalf of others.\footnote{James 5:13--20; Psalm 24:3--4; also \emph{Module 8: The Dynamic of Intercession}.}
\end{description}

\begin{itemize}
\item Sharing the intercession of the Holy Spirit, faithfully modelling, advocating and encouraging the vital mediatory work of \emph{intercessory prayer.\footnote{Romans 8:26--27}}

\item Facilitating, with others, a fulfilment of the Messianic Community's calling to be \emph{a house of prayer for all nations}.\footnote{Matthew 21:13, c.f. Isaiah 56:7}

\end{itemize}

\begin{description}

\item[Prophetic mediation]

involves \emph{speaking with human beings, on behalf of God}, based upon a particular capacity and responsibility for \emph{hearing what the Spirit is saying to the Messianic Community.}\footnote{Revelation 2:7,11,17,29,3:13,22; also Matthew 11:15, Mark 4:9, Luke 8:8 etc.} This includes:
\end{description}

\begin{itemize}
\item providing discernment, direction and insight, especially when messianic communities are embracing the challenges and responsibilities of particular contexts and historical events;

\item providing exhortation, clarification and even rebuke, especially when messianic communities are failing in their vocational calling to serve God's purposes.

\end{itemize}

\subsection{God's heartbeat}
\label{godsheartbeat}

The prophetic role may be thought of as \emph{listening to and discerning God's heart}.\footnote{See \emph{Module 8: The Dynamic of Intercession}.} This implies a significant level of spiritual intimacy, enabling prophets to:

\begin{itemize}
\item walk closely with God, in order to discern the thoughts, feelings and intentions upon his heart;

\item share in the intercessory ministry of the Spirit;

\item discern and share God's \emph{kairos\footnote{\emph{Kairos} is a Greek word, referring to a particularly opportune, favourable, suitable or appropriate moment---e.g. see John 7:6--8, 12:23; Luke 21:13; Mark 13:4; Acts 1:6--7; 1 Timothy 2:6; it contrasts with \emph{chronos}, referring to fixed, measurable units of time.}} word for a particular context.

\end{itemize}

\begin{description}

\item[Note]

\emph{Those gifted to walk prophetically with God are often sensitive personalities, frequently poets, artists, writers, visionaries and other kinds of imaginative, inventive or creative individuals. This factor, combined with it relatively obscure, non-structural mode of functioning, means prophetic gifts are frequently overlooked or misunderstood.}
\end{description}

\subsection{Summary: Picking up God's heartbeat}
\label{summary:pickingupgodsheartbeat}

The prophetic role is essentially a mediatory role, requiring a sensitive, intimate, faithful walk with God that enables prophets to:

\begin{figure}[htbp]
\centering
\includegraphics[width=200pt,height=233pt]{pickingupgodsheartbeat.png}
\caption{}
\label{pickingupgodsheartbeat.png}
\end{figure}

\begin{summary}

\textbf{Pick up God's heartbeat for the peoples of the world, interceding with the Spirit for the purposes of God, in order to hear and convey what the Spirit is saying to the messianic community}

\end{summary}

\begin{discuss}[\currentsectiontitle]

Use Discussion 4 now, or continue to \autoref{wholebodyworkingtogether} and use Discussions 1--5 together at the end of this Study

\end{discuss}

\section{Whole body working together}
\label{wholebodyworkingtogether}

Having explored the three foundational structures and the prophetic heart of messianic community, this topic explores various forms and modes of interaction between all of them.

\subsection{Structures and strategies}
\label{structuresandstrategies}

The Strategies of Messianic Community (\autoref{thestrategiesofmessianiccommunity}) illustrated and explored five foundational strategies: pray, reach, disciple, teach, send. These five strategies correspond closely with the messianic \emph{structures} explored in this study. This correspondence is displayed in the table, \emph{Messianic structure, strategy and function}.

\begin{table}[htbp]
\begin{minipage}{\linewidth}
\setlength{\tymax}{0.5\linewidth}
\centering
\small
\caption{Messianic structure, strategy and function}
\label{table1}
\begin{tabulary}{\textwidth}{@{}RCL@{}} \toprule
\textbf{Strategy}&\textbf{Messianic structure}&\textbf{Function}\\
\midrule
pray&prophetic mediatory role&picking up God's heartbeat\\

\midrule
reach, disciple&pastoral, evangelistic communities&discipling into allegiance\\

\midrule
teach&teaching and training centres&equipping with truth\\

\midrule
send&apostolic missionary teams&pioneering in power\\

\bottomrule

\end{tabulary}
\end{minipage}
\end{table}

\subsubsection{Working together}
\label{workingtogether}

The correspondence indicated in the table, \emph{Messianic structure, strategy and function}, does not imply that \emph{prayer}, for example, is solely the responsibility of people who are called specifically to prophetic and intercessory ministry. It simply illustrates how different \emph{parts} of the body contribute something significant to the functioning of the body as a \emph{whole}:

\begin{quote}

You are{\ldots}members of the household of God, built upon the foundation of the apostles and prophets, with Christ himself as the cornerstone. In him \emph{the whole structure} is joined together and grows into a holy temple in the Lord, in whom you also are being built spiritually into a dwelling place for God—\emph{Ephesians 2:19b--22 TAB}
\end{quote}

\subsection{Structural interactions}
\label{structuralinteractions}

As these messianic structures learn to function faithfully and to interact dynamically with one another in accordance with God's strategic design and purpose an effective missional, discipleship movement becomes established.

The figure and table, \emph{Structural interactions}, suggests some specific ways in which the different structural parts of the whole body interact and supply the other parts.

\begin{figure}[htbp]
\centering
\includegraphics[width=210pt,height=200pt]{structuralinteractions.png}
\caption{Structural interactions}
\label{structuralinteractions.png}
\end{figure}

\begin{table}[htbp]
\begin{minipage}{\linewidth}
\setlength{\tymax}{0.5\linewidth}
\centering
\small
\caption{Structural interactions}
\label{table2}
\begin{tabulary}{\textwidth}{@{}RCL@{}} \toprule
\textbf{Structure}&\textbf{Dynamic}&\textbf{Structure}\\
\midrule
pastoral, evangelistic communities&supplying&teaching and training centres\\
&sending&apostolic missionary teams\\
teaching and training centres&equipping&pastoral, evangelistic communities\\
&forming&apostolic missionary teams\\
apostolic missionary teams&planting&pastoral, evangelistic communities\\
&establishing&teaching and training centres\\
prophetic, intercessory teams&mediating&on behalf of messianic community\\
&sharing insight&with messianic community\\
messianic community&receiving&prophetic, intercessory teams\\

\bottomrule

\end{tabulary}
\end{minipage}
\end{table}

\begin{discuss}[\currentsectiontitle]

Use Discussion 5 now, or continue to the Summary and then Discussions 1--5 to complete Study 3

\end{discuss}

\osection{Summary}

This concludes the study, \emph{The Structures of Messianic Community}, which examined the foundational structures of messianic community, including their relation to the five foundational gifts—pastor, evangelist, teacher, apostle and prophet—and the five foundational strategies—pray, reach, disciple, teach, send.

In summary, the study revealed that:

\begin{summary}

\textbf{The messianic community has a God-ordained structure that uniquely equips it to fulfil the messianic commission}

\end{summary}

\osection{Discussions}

\begin{disc}[relating to T1, Pastoral-evangel'c community]

\begin{itemize}
\item How do people express and experience alliances and allegiance to families, tribes, sports teams, nation? 

\item How is being \emph{in alliance} with the Messiah significant?

\end{itemize}

\end{disc}
\begin{disc}[relating to T2, Teaching, training centres]

\begin{itemize}
\item How does context effect how mission is understood?

\item How are you passing on your experience and knowledge?

\end{itemize}

\end{disc}
\begin{disc}[relating to T3, Apostolic, missionary teams]

\begin{itemize}
\item What are your most significant experience of intercultural work or workers?

\item Which qualifications—gifts, talents, characteristics—are \emph{most} needed by apostolic missionary workers?

\end{itemize}

\end{disc}
\begin{disc}[relating to Prophets, priests, mediators]

\begin{itemize}
\item How does its \emph{non-structural} aspect affect the confidence of people called to function prophetically?

\item How might messianic communities make appropriate room for people fulfilling prophetic, mediatory roles?

\end{itemize}

\end{disc}
\begin{disc}[relating to Interaction]

\begin{itemize}
\item Which structural dynamic is most significant? Why?

\item Which is presently \emph{least evident} in your contexts?

\end{itemize}

\end{disc}

\chapter{The Expansion of Messianic Community}
\label{theexpansionofmessianiccommunity}

\begin{synopsis}

\textbf{The Messianic Community is intended to be a rapidly multiplying movement of disciples, constantly expanding into all the world}

\end{synopsis}\begin{topics}

\begin{enumerate}
\item Honey bees (\autoref{honeybees})

\item Maize plant (\autoref{maizeplant})

\item Commercial organisations (\autoref{commercialorganisations})

\item Missional movements (\autoref{missionalmovements})

\end{enumerate}

\pagebreak 

\end{topics}\osection{Terms used in this study}

\begin{description}

\item[Increase]

referring to numerical growth

\item[Expansion]

referring to geographical enlargement
\end{description}

\osection{Scripture}\bible

Read these passages aloud; memorise the \textbf{bold} passages. \emph{Notice how each passage summarises a significant period of missional activity, during which early messianic community established an expanding movement of disciples.}

\begin{itemize}
\item \textbf{Acts 1:8}

\item Acts 2:42--47

\item Acts 6:7

\item \textbf{Acts 9:31}

\item Acts 12.24

\item Acts 16.5

\item Acts 19.20

\end{itemize}

\section{Honey bees}
\label{honeybees}

A bee colony is a kind of \emph{collective organism}, because individual bees cannot survive very long outside of a colony and, although there are clearly defined \emph{roles}, there is no hierarchy or leadership. A bee colony may contain between 2,000 and 60,000 bees, including: 

\begin{itemize}
\item a single fertile \emph{queen bee}

\item a few thousand fertile male \emph{drone bees}

\item several thousand non-fertile female \emph{worker bees}

\end{itemize}

\begin{figure}[htbp]
\centering
\includegraphics[keepaspectratio,width=\textwidth,height=0.75\textheight]{swarm.png}
\caption{A swarm of bees}
\label{swarm.png}
\end{figure}

\subsection{Increase}
\label{increase-bees}

A colony grows as workers raise thousands of new bees, born to the queen. Usually, around a queen's second springtime, a colony will prepare to \emph{swarm}. In readiness for swarming, worker bees begin preparing new virgin queen bees, one of which will take over the existing hive, by killing all of the other virgin queens, after the old queen leaves with the swarm.

\subsection{Division}
\label{division}

When it is time for the swarm to leave the hive, \emph{scout bees} will find a suitable place for the swarm to gather initially and report this location to the colony. Shortly afterwards, about 6 out of every 10 worker bees in the colony—usually the the most vigorous ones—swarm around the queen bee. The swarm then leaves the hive altogether, moving directly to the scouted location.

\subsection{Expansion}
\label{expansion-bees}

Scout bees must swiftly identify a suitable permanent hive location, so the swarm can form a new colony. This step is critical because swarming bees can survive on the honey in their stomachs for only 2--3 days. Once a new hive is settled, the cycle of growth begins again. The old queen may not live long and must quickly start the process of repopulating the colony, including producing new virgin queens, ready to take over her role.

\begin{description}

\item[Harvest]

bees produce an abundance of honey, from plant pollen, which is used to feed the growing colony; honey is also a harvestable crop for both humans and wild animals, such as bears.

\item[Risks]

swarming divides the colony, temporarily weakening both groups.
\end{description}

\subsection{Summary: bees}
\label{summary:bees}

Organic, steady, cyclical \emph{increase} in numbers of bees, produces swarming and a eventual \emph{division} of the colony, which then \emph{expands} into a different area.

\begin{figure}[htbp]
\centering
\includegraphics[width=207pt,height=35pt]{division.png}
\caption{}
\label{division.png}
\end{figure}

\begin{summary}

Increase leads to \textbf{division}, leading to expansion

\end{summary}

\begin{discuss}[\currentsectiontitle]

Use Discussion 1 now, or continue to \autoref{maizeplant} and use Discussions 1--4 together at the end of this Study

\end{discuss}

\section{Maize plant}
\label{maizeplant}

The maize plant is a rapidly-reproducing cereal crop, with a leafy stalk, typically growing two or more metres high (\autoref{mp-fig}).

\begin{figure}[htbp]
\centering
\includegraphics[keepaspectratio,width=\textwidth,height=150pt]{mp-diagram.png}
\caption{Maize plant}
\label{mp-fig}
\end{figure}

\subsection{Increase}
\label{increase-maize}

Maize plants grow from seeds sown into the ground. Growth begins when the seed's hard, outer shell breaks open, allowing the soft, inner kernel to access the moisture and nutrients within the soil. It immediately sprouts roots and a single stem that moves upwards, towards the surface of the soil. 

Once through the surface, the plants leaves can begin photosynthesising sunlight, while its roots continue drawing on soil nutrients and moisture. In arid locations, typical of Africa, the most significant growth factor is a sufficiency of rainfall.

\subsection{Multiplication}
\label{multiplication}

Each maize plant produces a number of ears, each of which typically contains 600--800 seeds. Thus, in one season, a single plant may produce thousands of seeds.

\subsection{Expansion}
\label{expansion-maize}

Maize plants expand into new areas through redistribution of their seeds. Because of the way the plant has been cultivated, over hundreds of years, the intervention of farmers is required to effectively distribute seeds.

\begin{description}

\item[Harvest]

maize is a valuable crop, suitable for use in a variety of foodstuffs.

\item[Risks]

shallow roots make plants susceptible to poor soils, drought and severe winds.
\end{description}

\subsection{Summary: maize}
\label{summary:maize}

Organic \emph{increase}, as each plant grows, potentially producing a thousand-fold \emph{multiplication} of seeds, often twice a year, allowing rapid \emph{expansion} of crop, depending upon climate and soil conditions.

\begin{figure}[htbp]
\centering
\includegraphics[width=208pt,height=35pt]{multiplication.png}
\caption{}
\label{multiplication.png}
\end{figure}

\begin{summary}

\emph{Increase} leads to \emph{multiplication} leading to \emph{expansion}

\end{summary}

\begin{discuss}[\currentsectiontitle]

Use Discussion 2 now, or continue to \autoref{commercialorganisations} and use Discussions 1--4 together at the end of this Study

\end{discuss}

\section{Commercial organisations}
\label{commercialorganisations}

Commercial, business organisations (including most charitable enterprises) seek to achieve \emph{economic growth}, using profit and loss accounts as the primary indicator of success and failure. A large corporation, such as a mineral company, can often grow to employ hundreds of thousands of people, in multiple offices, in many different countries, with a budget larger than some nations.

\subsection{Increase}
\label{increase-orgs}

\begin{figure}[htbp]
\centering
\includegraphics[width=249pt,height=145pt]{Organisationalexpansion.png}
\caption{}
\label{Organisationalexpansion.png}
\end{figure}

Organisational growth is marked by an increase in commercial measurements such as turnover, profit, employees, management experience, production capacity, market size and share and so on. At some point, most organisations reach a limit to their growth in one or other of these areas. Further growth is (usually) sought by duplicating the number of outlets (shops, offices, factories etc.) owned or managed by the organisation.

\subsection{Duplication}
\label{duplication}

Duplication essentially reproduces, in a new location, a \emph{copy} of an existing, successful model. This builds upon proven characteristics of the original concept, increases production and establishes an identifiable brand. Buildings, budgets, payrolls and competition typically play a significant role in decision-making processes, usually controlled closely by a centralised, hierarchical, \emph{command-and-control} structure of management.

\subsection{Expansion}
\label{expansion-orgs}

Regional location, form, speed and costs of expansion are usually determined by a central, organisational strategy. If a new structure does not function according to expectations, it may be closed down—without reference to local, contextual concerns.

\begin{description}

\item[Harvest]

successful duplication leads to increased profits and more managers.

\item[Risks]

duplication typically ignores or suppresses local insight and initiative and the effect of contextual differences on the establishment of new structures.
\end{description}

\subsection{Summary: organisations}
\label{summary:organisations}

Unpredictable, inorganic growth measured by \emph{increase} in profits, personnel and management experience, leading to \emph{duplication} of structures (premises and management hierarchy) and geographical \emph{expansion} of the market being served.

\begin{figure}[htbp]
\centering
\includegraphics[width=208pt,height=35pt]{duplication.png}
\caption{}
\label{duplication.png}
\end{figure}

\begin{summary}

\emph{Increase} leads to \emph{duplication} leading to \emph{expansion}

\end{summary}

\begin{discuss}[\currentsectiontitle]

Use Discussion 3 now, or continue to \autoref{missionalmovements} and use Discussions 1--4 together at the end of this Study

\end{discuss}

\section{Missional movements}
\label{missionalmovements}

\begin{description}

\item[A people movement]

represents an informally-organised grouping of people and organisations, dedicated to achieving shared political, social, or artistic ideas, ideals and goals.

\item[A missional movement]

represents an informally-organised grouping of people and organisations, dedicated to serving God's eternal purpose.
\end{description}

\subsection{Increase}
\label{increase}

Missional movements grow through the forming of disciples who make disciples, who make disciples (see \emph{The Commissioning of Messianic Community} (\autoref{thecommissioningofmessianiccommunity}) and \emph{The Structures of Messianic Community} (\autoref{thestructuresofmessianiccommunity}). Disciples do more than simply \emph{believe} in missional values: they \emph{embody} core beliefs and values by realigning their lives, in order to affect their own contexts and futures — including making disciples{\ldots}who make disciples{\ldots}who make disciples and so on.

\subsection{Reproduction}
\label{reproduction}

Within missional movements, as in nature, not all reproduction succeeds. Some individuals and groups fail to mature. Others reach maturity, but don't reproduce. Some start slowly, others rapidly. Some groups evolve a different sense of identity, purpose or form to that of their originating contexts.

\subsection{Expansion}
\label{expansion}

\begin{quote}

The wind breathes where it wills; and though you hear its sound, yet you neither know where it comes from, nor where it is going. So it is with everyone born of the Spirit—\emph{John 3:8}
\end{quote}

Ultimately, the Holy Spirit is responsible for governing the wild, haphazard growth of a missional, discipling movement capable of impacting communities, regions and even nations, over time.

\begin{itemize}
\item It is the Holy Spirit who is able to direct how and where to establish our efforts and energies.\footnote{Witness the role of the Holy Spirit, throughout \emph{Acts} (see Scripture readings, [][Scripture-expansion])}

\item It is the Holy Spirit who takes hold of a community, implanting a missional vision larger than that of any individual or single organisation.

\end{itemize}

Throughout the world, within every people group, wherever willing hearts and available hands are found, the Holy Spirit is working to establish a community that manifests God's heart. \emph{The dynamics and disciplines that we are studying in this syllabus illustrate and demonstrate how he works amongst us, moving us onwards towards this goal.}

\begin{description}

\item[Harvest]

members exhibits wide variety of charisma, talent and contextual influence. As the Good News is accepted amongst people, it leads to gradual transformation amongst families, households, communities, societies, cultures and people groups.

\item[Risks]

missional movements are unpredictable; results are difficult to measure and assess accurately; groups that lose focus and faithfulness to foundational values, slow the pace of change and expansion—yet are difficult to identify and reform.
\end{description}

\subsection{Summary}
\label{summary}

An \emph{increase} in fruitfulness amongst missional movements is measured in terms of disciples, in whom their is a faithful \emph{reproduction} of the life of the Spirit, including the vision to produce disciples amongst people of all nations, leading to a cultural \emph{expansion} of the sphere of the Gospel's influence.

\begin{figure}[htbp]
\centering
\includegraphics[width=208pt,height=35pt]{reproduction.png}
\caption{}
\label{reproduction.png}
\end{figure}

\begin{summary}

\emph{Increase} leads to \emph{reproduction} leading to \emph{expansion}

\end{summary}

\begin{discuss}[\currentsectiontitle]

Use Discussion 4 now, or continue to Summary and then Discussions 1--4 to complete Study 4

\end{discuss}

\osection{Summary}

This concludes the study, \emph{The expansion of Messianic Community} (\autoref{theexpansionofmessianiccommunity}), which explored three examples of either organic or organisational reproduction, in order to compare and contrast them with the organised expansion of messianic, discipleship movements. 

The study highlighted characteristics associated with the growth, reproduction, relocation, harvest and risks of swarming bees, maize plants, commercial organisations and missional movements. 

In summary, the study revealed that:

\begin{summary}

\textbf{The Messianic Community is intended to be a rapidly multiplying movement of disciples, constantly expanding into all the world, in order to bless the peoples of the world.}

\end{summary}

\osection{Discussions}

Begin with the first topic; progress to the secondary question, as appropriate.

\begin{disc}[relating to T1, Honey bees]

\begin{quote}

\textbf{Discuss the collective nature of bee colonies, including the various types of bees, and the phenomena of swarming}.
\end{quote}

\begin{itemize}
\item If bee colonies are considered to be a metaphor for Christian community, what can we learn from their characteristics?

\end{itemize}

\end{disc}
\begin{disc}[relating to T2, Maize plant]

\begin{quote}

\textbf{Discuss how maize plants have become dependent on farmers for reproduction}.
\end{quote}

\begin{itemize}
\item If maize plants are considered to be a metaphor for Christian community, what can we learn from their characteristics?

\end{itemize}

\end{disc}
\begin{disc}[relating to T3, Commercial organisations]

\begin{quote}

\textbf{Discuss the duplication of commercial organisations with which you are familiar}.
\end{quote}

\begin{itemize}
\item If commercial organisations are considered to be a metaphor for messianic community, what can we learn from their characteristics?

\item What benefits and risks would you associate with following the principles of commercial organisations?

\end{itemize}

\end{disc}
\begin{disc}[relating to Missional movements]

\begin{quote}

\textbf{Discuss both the risks and benefits associated with establishing discipleship movements}.
\end{quote}

\begin{itemize}
\item On balance, are discipleship movements a worthwhile investment?

\item What are you prepared to invest, what are you prepared to renounce, in order to help the establishment of a discipleship movement?

\end{itemize}

\end{disc}

\chapter*{Author}\label{author}
\osection{Dr John B Clements}

John was awarded a Doctorate of Missiology (Contextual Missiology) by Fuller Theological Seminary School of Intercultural Studies, in 2013, for research that led to the development of Maize Plant Discipleship. He is presently working to establish the \emph{Ajori Crowther Centre for African Mission}, in Oxford.

\begin{figure}[htbp]
\centering
\includegraphics[width=150pt,height=111pt]{john-rhossilli.jpg}
\caption{}
\label{john-rhossilli.jpg}
\end{figure}

John is married to Sarah; they have four children and live in a delightful corner of West South Wales. John is an avid bird-watcher and casual photographer, pastimes that he combines with his enjoyment of countryside and coastal walking.

\begin{itemize}
\item Vita http:/\slash jbclements.wordpress.com

\item ACCAM http:/\slash ukcafricanmission.wordpress.com

\item Social http:/\slash about.me\slash jbclements

\end{itemize}

\input{mpd-footer}

\end{document}
